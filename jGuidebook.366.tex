\documentstyle[titlepage,longtable]{jarticle}
%% NetHack 3.6  Guidebook.tex $NHDT-Date: 1431192762 2015/12/16 17:32:42 $  $NHDT-Branch: master $:$NHDT-Revision: 1.60 $ */
%%+% we're still limping along in LaTeX 2.09 compatibility mode
%%-%\documentclass{article}
%%-%\usepackage{hyperref} % before longtable
%%-%% if hyperref isn't available, we can get by with this instead
%%-%%\RequirePackage[errorshow]{tracefnt} \DeclareSymbolFont{typewriter}{OT1}{cmtt}{m}{n}
%%-%\usepackage{longtable}
\textheight 220mm
\textwidth 160mm
\oddsidemargin 0mm
\evensidemargin 0mm
\topmargin 0mm

\newcommand{\nd}{\noindent}

\newcommand{\tb}[1]{\tt #1 \hfill}
\newcommand{\bb}[1]{\bf #1 \hfill}
\newcommand{\ib}[1]{\it #1 \hfill}

\newcommand{\blist}[1]
{\begin{list}{$\bullet$}
    {\leftmargin 30mm \topsep 2mm \partopsep 0mm \parsep 0mm \itemsep 1mm
     \labelwidth 28mm \labelsep 2mm
     #1}}

\newcommand{\elist}{\end{list}}

%% this will make \tt underscores look better, but requires that
%% math subscripts will never be used in this document
\catcode`\_=12

\begin{document}
%%
%% input file: guidebook.mn
%%
%%.ds h0 "
%%.ds h1 %.ds h2 \%
%%.ds f0 "

%@ "Guidebook.tex" Japanese Translation
%@
%@ Translated
%@   for {\it NetHack\/} 3.1.0 PL10 by Keizo Yamamoto
%@   for {\it NetHack\/} 3.1.3      by Kentaro Shirakata
%@   for J{\it NetHack\/} 1.0.5     by Issei Numata
%@   for {\it NetHack\/} 3.3.0      by Kentaro Shirakata
%@   for (J){\it NetHack\/} 3.4     by Kentaro Shirakata <argrath@ub32.org>
%@   for (J){\it NetHack\/} 3.6     by Kentaro Shirakata <argrath@ub32.org>
%@
%@WORD:	Dungeoneer		洞窟の主
%@WORD:	the Mazes of Menace	恐怖の迷宮
%@WORD:	dungeon			洞窟
%@WORD:	proficiency		技量
%%.mt
%\title{\LARGE A Guide to the Mazes of Menace:\\
%\Large Guidebook for {\it NetHack\/}}
\title{\LARGE 恐怖の迷宮への招待:\\
\Large {\it NetHack\/} ガイドブック}

%%.au
%\author{Original version - Eric S. Raymond\\
%(Edited and expanded for 3.6 by Mike Stephenson and others)}
\author{オリジナル - Eric S. Raymond\\
(Mike Stephenson らによるバージョン 3.6 のための校訂と拡充)}
%\date{March 8, 2020}
\date{2020 年 3 月 8 日}

\maketitle

%%.pg
%%.hn 1
%\section{Introduction}
\section{はじめに}

%%.pg

%Recently, you have begun to find yourself unfulfilled and distant
%in your daily occupation.  Strange dreams of prospecting, stealing,
%crusading, and combat have haunted you in your sleep for many months,
%but you aren't sure of the reason.  You wonder whether you have in
%fact been having those dreams all your life, and somehow managed to
%forget about them until now.  Some nights you awaken suddenly
%and cry out, terrified at the vivid recollection of the strange and
%powerful creatures that seem to be lurking behind every corner of the
%dungeon in your dream.  Could these details haunting your dreams be real?
%As each night passes, you feel the desire to enter the mysterious caverns
%near the ruins grow stronger.  Each morning, however, you quickly put
%the idea out of your head as you recall the tales of those who entered
%the caverns before you and did not return.  Eventually you can resist
%the yearning to seek out the fantastic place in your dreams no longer.
%After all, when other adventurers came back this way after spending time
%in the caverns, they usually seemed better off than when they passed
%through the first time.  And who was to say that all of those who did
%not return had not just kept going?
最近、あなたは日々の仕事に満たされなく、敬遠しつつある自分に気がついた。
最近数ヶ月、あなたが見る夢は探査、盗み、聖戦、戦闘といったものであったが、
あなたにはその理由がはっきりとはわからなかった。
あなたはこれらの夢を実際には生まれてこのかたずっと見続けていて、
なぜか今までそれらについて忘れようとしていたのではないか、と思った。
ある夜、あなたは飛び起きて、
夢の中の洞窟のあらゆる角の後ろに潜んでいるように見える、
奇妙で強力な怪物の鮮明な記憶にぞっとして大声で叫んだ。
あなたの夢に出没する内容は本当なのだろうか?
夜毎に、廃墟の近くの不思議な洞窟に入るという欲求が強くなった。
それでも、毎朝あなたはそのような考えを、
洞窟に入って帰って来なかった人々の話を思い出すことで頭から追い出した。
しかし、ついにあなたは夢の中に出てくる不思議な場所を捜し求めるという思いを
抑えきれなくなった。
何といっても、冒険者は最初にその道を通っていった時より、
こちらに戻って来る時の方が裕福になっているように見えるのだ。
戻って来ない人たちは皆、単にまだ冒険を続けているだけなのでは?
%%.pg

%Asking around, you hear about a bauble, called the Amulet of Yendor by some,
%which, if you can find it, will bring you great wealth.  One legend you were
%told even mentioned that the one who finds the amulet will be granted
%immortality by the gods.  The amulet is rumored to be somewhere beyond the
%Valley of Gehennom, deep within the Mazes of Menace.  Upon hearing the
%legends, you immediately realize that there is some profound and
%undiscovered reason that you are to descend into the caverns and seek
%out that amulet of which they spoke.  Even if the rumors of the amulet's
%powers are untrue, you decide that you should at least be able to sell the
%tales of your adventures to the local minstrels for a tidy sum, especially
%if you encounter any of the terrifying and magical creatures of
%your dreams along the way.  You spend one last night fortifying yourself
%at the local inn, becoming more and more depressed as you watch the odds
%of your success being posted on the inn's walls getting lower and lower.
あちこちで尋ねるうちに、あなたはある噂を耳にした。
イェンダーの魔除けとかいうものがあり、手に入れることができたなら、
すばらしい財産をもたらすと言うのだ。
あなたが聞いたとある伝説では、
魔除けを見つけたものは神から不死の体を授かるとさえ言われているらしい。
魔除けは恐怖の迷宮の奥深く、ゲヘナの谷を越えたどこかにあるという噂である。
伝説を聞いた瞬間あなたは、自分には洞窟に入って彼らの話している
魔除けを捜し求めるべき深遠で未知のなんらかの理由があると確信した。
あなたはたとえ魔除けの力が真実でなくても、
少なくとも冒険談を地元の吟遊詩人にかなりの金額で売り込めるだろうと考えた。
夢の中で出てきた恐ろしく、不思議な怪物に出会うことができればなおさらである。
あなたは地元の安宿で最後の宿を取り対策を練った。
安宿の壁に掲げられた成功率のオッズが下がれば下がるほど、
あなたは意気消沈した。

%%.pg
%\nd In the morning you awake, collect your belongings, and
%set off for the dungeon.  After several days of uneventful
%travel, you see the ancient ruins that mark the entrance to the
%Mazes of Menace.  It is late at night, so you make camp at the entrance
%and spend the night sleeping under the open skies.  In the morning, you
%gather your gear, eat what may be your last meal outside, and enter the
%dungeon\ldots
\nd 朝起きると、あなたは所持品をかき集め、洞窟に向けて旅立った。
無事何日かの旅を続けて恐怖の迷宮への入口を
示す古代の遺跡へとたどり着いた。夜も更けていたので入口でキャンプをすることにし、
その夜は広々とした空の下で眠りに就いた。翌朝あなたは道具をかき集め、
もしかすると地上での最後になるかもしれない食事を済ませ、
洞窟へと入っていった…

%%.hn 1
%\section{What is going on here?}
\section{あなたを取り巻く状況}

%%.pg
%You have just begun a game of {\it NetHack}.  Your goal is to grab as much
%treasure as you can, retrieve the Amulet of Yendor, and escape the
%Mazes of Menace alive.
こうしてあなたは {\it NetHack\/} というゲームを始めることになった。あなたの目的は
持てる限りの宝を集め、イェンダーの魔除けを見つけ出し、さらにはこの恐怖
の迷宮から生きて脱出することである。

%%.pg
%Your abilities and strengths for dealing with the hazards of adventure
%will vary with your background and training:
冒険におけるいろいろな困難に対応するための各種の能力や力は、
あなたの生い立ちと修行によって変化する。

%%.pg
%%
\blist{}
%\item[\bb{Archeologists}]%
\item[\bb{考古学者(Archeologist)}]%
%understand dungeons pretty well; this enables them
%to move quickly and sneak up on the local nasties.  They start equipped
%with the tools for a proper scientific expedition.
洞窟について多くの知識を持っている。
このため素早く動いたり洞窟での不快な事態を切り抜けることができる。
彼らは学術的調査のための探検旅行に適した道具を持って出発する。
%%.pg
%%
%\item[\bb{Barbarians}]%
\item[\bb{野蛮人(Barbarian)}]%
%are warriors out of the hinterland, hardened to battle.
%They begin their quests with naught but uncommon strength, a trusty hauberk,
%and a great two-handed sword.
奥地からやってきた戦士であり、戦闘のために鍛えられている。
彼らは類まれな力の強さを持ち、丈夫な鎖かたびらと大きな両手持ちの
剣のみを持って探求に出発する。
%%.pg
%%
%\item[\bb{Cavemen {\rm and} Cavewomen}]
\item[\bb{洞窟人(Caveman と Cavewoman)}]
%start with exceptional strength, but unfortunately, neolithic weapons.
とりわけ強い力を持っているが、
新石器時代に使われた武器を持って出発する。
%%.pg
%%
%\item[\bb{Healers}]%
\item[\bb{薬師(Healer)}]%
%are wise in medicine and apothecary.  They know the
%herbs and simples that can restore vitality, ease pain, anesthetize,
%and neutralize
%poisons; and with their instruments, they can divine a being's state
%of health or sickness.  Their medical practice earns them quite reasonable
%amounts of money, with which they enter the dungeon.
医術や薬物について精通している。
生気を回復させたり、苦痛を和らげたり、麻酔をかけたり、
毒を中和させたりする香草や薬草について詳しい。
そして器具を使って生物の健康状態を見抜くことができる。
彼らは開業医としてかなりよい報酬を得て、それを持って洞窟に入っていく。
%%.pg
%%
%\item[\bb{Knights}]%
\item[\bb{騎士(Knight)}]%
%are distinguished from the common skirmisher by their
%devotion to the ideals of chivalry and by the surpassing excellence of
%their armor.
騎士道に忠実であることと防具が驚くほど優れていることという
点で、普通の戦士とは区別される。
%%.pg
%%
%\item[\bb{Monks}]%
\item[\bb{モンク(Monk)}]%
%are ascetics, who by rigorous practice of physical and mental
%disciplines have become capable of fighting as effectively without weapons
%as with.  They wear no armor but make up for it with increased mobility.
修道者であり、厳格な肉体的習慣と精神的訓練によって、
武器なしでも武器を使うのと同じように戦うことが出来る。
モンクは防具をつけないが、機動性を向上させることで埋め合わせている。
%%.pg
%%
%\item[\bb{Priests {\rm and} Priestesses}]%
\item[\bb{僧侶と尼僧(Priest と Priestess)}]%
%are clerics militant, crusaders
%advancing the cause of righteousness with arms, armor, and arts
%thaumaturgic.  Their ability to commune with deities via prayer
%occasionally extricates them from peril, but can also put them in it.
聖職の戦士であり、
武器と防具を持ち魔法の業を駆使して正義を広めんとする聖戦士である。
祈りを通じて神と語る能力により、彼らはしばしば危難から逃れることができるが、
祈りは危難をもたらすこともある。
%%.pg
%%
%\item[\bb{Rangers}]%
\item[\bb{レンジャー(Ranger)}]%
%are most at home in the woods, and some say slightly out
%of place in a dungeon.  They are, however, experts in archery as well
%as tracking and stealthy movement.
ほとんど森を住みかとし、洞窟は少々場違いかもしれない。
しかし、彼らは追跡術や隠密行動と共に、弓術の達人である。
%%.pg
%%
%\item[\bb{Rogues}]%
\item[\bb{盗賊(Rogue)}]%
%are agile and stealthy thieves, with knowledge of locks,
%traps, and poisons.  Their advantage lies in surprise, which they employ
%to great advantage.
敏捷で身を隠す業に長けた泥棒であり、
錠や罠や毒に詳しい。
彼らの長所は奇襲であり、これは非常な長所となる。
%%.pg
%%
%\item[\bb{Samurai}]%
\item[\bb{侍(Samurai)}]%
%are the elite warriors of feudal Nippon.  They are lightly
%armored and quick, and wear the %
%{\it dai-sho}, two swords of the deadliest
%keenness.
封建時代の日本の精鋭の武人である。
防具は軽装で身のこなしに秀でており、
この上なく研ぎ澄まされた {\it 大小 } 2本の刀を帯びている。
%%.pg
%%
%\item[\bb{Tourists}]%
\item[\bb{観光客(Tourist)}]%
%start out with lots of gold (suitable for shopping with),
%a credit card, lots of food, some maps, and an expensive camera.  Most
%monsters don't like being photographed.
多額の金(買物に最適である)とクレジットカード、大量の食料、
地図、そして高価なカメラを持って出発する。
ほとんどの怪物は写真に撮られるのを嫌うものである。
%%.pg
%%
%\item[\bb{Valkyries}]%
\item[\bb{ワルキューレ(Valkyrie)}]%
%are hardy warrior women.  Their upbringing in the harsh
%Northlands makes them strong, inures them to extremes of cold, and instills
%in them stealth and cunning.
勇敢な女戦士である。
過酷なスカンジナビアの国々で育った彼女らは頑強で、厳しい寒さにも耐え、
身を隠す業と巧妙さを身につけている。
%%.pg
%%
%\item[\bb{Wizards}]%
\item[\bb{魔法使い(Wizard)}]%
%start out with a knowledge of magic, a selection of magical
%items, and a particular affinity for dweomercraft.  Although seemingly weak
%and easy to overcome at first sight, an experienced Wizard is a deadly foe.
魔術の知識とに選び抜かれた魔法の道具を準備しており、
いにしえの魔法学に対して特に造詣が深い。
見かけは弱そうで簡単に打ち倒せるように思えるが、
経験を積んだ魔法使いは恐るべき敵である。
\elist

%%.pg
%You may also choose the race of your character (within limits; most
%roles have restrictions on which races are eligible for them):
また、キャラクターの種族も選ぶことができる
(制限はある; ほとんどの職業はどの種族がなれるかに制限がある):

%%.pg
%%
\blist{}
%\item[\bb{Dwarves}]%
\item[\bb{ドワーフ(Dwarf)}]%
%are smaller than humans or elves, but are stocky and solid
%individuals.  Dwarves' most notable trait is their great expertise in mining
%and metalwork.  Dwarvish armor is said to be second in quality not even to the
%mithril armor of the Elves.
人間やエルフよりも小さいが、がっしりした体つきで丈夫である。
ドワーフの注目すべき特性は採掘と金属工作の専門的技術である。
ドワーフの防具はエルフのミスリル防具と同じとまではいかないまでも、
それに次ぐ品質である。
%%.pg
%%
%\item[\bb{Elves}]%
\item[\bb{エルフ(Elf)}]%
%are agile, quick, and perceptive; very little of what goes
%on will escape an Elf.  The quality of Elven craftsmanship often gives
%them an advantage in arms and armor.
機敏さと迅速さ、そして鋭敏な感覚を持っている。
周囲で起きるどんな些細なことも彼らの目を逃れることはできない。
職人気質の彼らの持つ武器と防具は優秀なものである。
%%.pg
%%
%\item[\bb{Gnomes}]%
\item[\bb{ノーム(Gnome)}]%
%are smaller than but generally similar to dwarves.  Gnomes are
%known to be expert miners, and it is known that a secret underground mine
%complex built by this race exists within the Mazes of Menace, filled with
%both riches and danger.
ドワーフよりもさらに小さいが、ドワーフと似ている。
ノームは熟練した採掘師として知られ、
富と危険で満ちている秘密の地下坑道を恐怖の迷宮に作っていることで知られる。
%%.pg
%%
%\item[\bb{Humans}]%
\item[\bb{人間(Human)}]%
%are by far the most common race of the surface world, and
%are thus the norm to which other races are often compared.  Although
%they have no special abilities, they can succeed in any role.
地上で最も一般的な種族で、ゆえに他の種族と比較される時の基準となっている。
特別な能力は持っていないが、あらゆる職業で成功することができる。
%%.pg
%%
%\item[\bb{Orcs}]%
\item[\bb{オーク(Orc)}]%
%are a cruel and barbaric race that hate every living thing
%(including other orcs).  Above all others, Orcs hate Elves with a passion
%unequalled, and will go out of their way to kill one at any opportunity.
%The armor and weapons fashioned by the Orcs are typically of inferior quality.
残酷で野蛮な種族で、(他のオークを含む)あらゆる生物を嫌っている。
特に、オークはエルフを感性の違いゆえに嫌い、わざわざあらゆる機会を見つけて殺そうとする。
オークによって作られた防具と武器は一般的に品質が劣る。
\elist

%%.hn 1
%\section{What do all those things on the screen mean?}
\section{画面に表示されるものの意味}
%%.pg
%On the screen is kept a map of where you have been and what you have
%seen on the current dungeon level; as you explore more of the level,
%it appears on the screen in front of you.
画面上にはあなたが現在いる洞窟の階
ですでに探索を終えた部分の地図と、そこで目にした物が示されている。さら
に深く探検を進めるに連れて洞窟はその姿を次第にあなたの前の画面に明らか
にしていく。

%%.pg
%When {\it NetHack\/}'s ancestor {\it rogue\/} first appeared, its screen
%orientation was almost unique among computer fantasy games.  Since
%then, screen orientation has become the norm rather than the
%exception; {\it NetHack\/} continues this fine tradition.  Unlike text
%adventure games that accept commands in pseudo-English sentences and
%explain the results in words, {\it NetHack\/} commands are all one or two
%keystrokes and the results are displayed graphically on the screen.  A
%minimum screen size of 24 lines by 80 columns is recommended; if the
%screen is larger, only a $21\times80$ section will be used for the map.
{\it NetHack\/} の祖先である {\it rogue\/} が最初に現れたとき、その
スクリーン指向はコンピュータファンタジーゲームの中でも全く独特のものであった。
それ以来スクリーン指向は例外的なものというよりむしろ標準となった。
{\it NetHack\/} はこのすばらしい伝統を受け継いでいる。疑似的な英語で
コマンドを入力し、結果が文章で表示されるテキストアドベンチャーゲームとは
異なり、{\it NetHack\/} のコマンドはすべて 1 文字か 2 文字のキー入力で
与えられ、その結果は画面上にグラフィカルに表示される。
画面は最低 80 桁および 24 行の大きさが推奨される。
それ以上の大きさがあっても $21\times80$ の領域だけが地図の表示に使われる。

%%.pg
%{\it NetHack\/} can even be played by blind players, with the assistance of
%Braille readers or speech synthesisers.  Instructions for configuring
%{\it NetHack\/} for the blind are included later in this document.
{\it NetHack\/} は点字や音声合成の助けを借りて、目の不自由な人でもプレイできる。
目の不自由な人のための{\it NetHack\/}の設定方法については、このドキュメントで
後述する。

%%.pg
%{\it NetHack\/} generates a new dungeon every time you play it; even the
%authors still find it an entertaining and exciting game despite
%having won several times.
{\it NetHack\/} はプレイするたびに新しい洞窟を作成する。
このため何回かゲームに勝利したことのある作者たちにも、
いまだこのゲームは楽しく興奮に満ちたものと感じられる。

%%.pg
%{\it NetHack\/} offers a variety of display options.  The options available to
%you will vary from port to port, depending on the capabilities of your
%hardware and software, and whether various compile-time options were
%enabled when your executable was created.  The three possible display
%options are: a monochrome character interface, a color character interface,
%and a graphical interface using small pictures called tiles.  The two
%character interfaces allow fonts with other characters to be substituted,
%but the default assignments use standard ASCII characters to represent
%everything.  There is no difference between the various display options
%with respect to game play.  Because we cannot reproduce the tiles or
%colors in the Guidebook, and because it is common to all ports, we will
%use the default ASCII characters from the monochrome character display
%when referring to things you might see on the screen during your game.
{\it NetHack\/} はさまざまな表示のオプションを提供する。
オプションはあなたのハードウェアやソフトウェアの能力に依存し、
プラットフォーム毎にさまざまである。
さまざまなコンパイル時のオプションは実行ファイルが作られたときにのみ可能となる。
モノクロキャラクタインタフェース、
カラーキャラクタインタフェース、
タイルと呼ばれる小さな絵を用いたグラフィカルインタフェースの 3 つの表示方法がある。
2 つのキャラクタインタフェースはフォントを他の文字に置き換えることもできるが、
標準設定では 標準の ASCII キャラクタを用いて全てを表現する。
さまざまな表示のオプションはゲーム内容には影響を与えない。
タイルや色はガイドブックで表記できないし、
それは全てのプラットフォームに共通ではないので、
ゲーム中に表示されるものの説明としては
モノクロキャラクタディスプレイで表示される標準の ASCII キャラクタを用いる。
%%.pg
%In order to understand what is going on in {\it NetHack}, first you must
%understand what {\it NetHack\/} is doing with the screen.  The {\it NetHack\/}
%screen replaces the ``You see \ldots'' descriptions of text adventure games.
%Figure 1 is a sample of what a {\it NetHack\/} screen might look like.
%The way the screen looks for you depends on your platform.
{\it NetHack\/} で何が起きているのかを理解するには、まず {\it NetHack\/} では画面が
どうなっているのかを理解しなくてはならない。{\it NetHack\/} の画面は、テキストアド
ベンチャーゲームにおける「あなたは…を見た」のような文章の代わりである。
図 1 は {\it NetHack\/} では画面に何が表示されるかの一例である。
画面がどのように表示されるかはプラットホームに依存する。
%%.BR 2

%\vbox{
\begin{verbatim}
        The bat bites!

                ------
                |....|    ----------
                |.<..|####...@...$.|
                |....-#   |...B....+
                |....|    |.d......|
                ------    -------|--



        Player the Rambler    St:12 Dx:7 Co:18 In:11 Wi:9 Ch:15 Neutral
        Dlvl:1 $:0 HP:9(12) Pw:3(3) AC:10 Exp:1/19 T:257 Weak
\end{verbatim}
\begin{center}
%Figure 1
図1
\end{center}
%}

%%.hn 2
%\subsection*{The status lines (bottom)}
\subsection*{ステータス行(画面の下部)}

%%.pg
%The bottom two lines of the screen contain several cryptic pieces of
%information describing your current status.  If either status line
%becomes longer than the width of the screen, you might not see all of
%it.  Here are explanations of what the various status items mean
%(though your configuration may not have all the status items listed
%below):
画面の下 2 行には、あなたの現在の状態を表す暗号のような情報の断片が含まれている。
どちらかの行が画面の幅よりも長くなるとその全部を見ることはできない
かもしれない。以下に状態を示すいろいろな項目が何を意味するのかを述べる
(コンフィグレーションによっては、以下に一覧されている状態を示す項目の
すべてが表示されるとは限らない)。

%%.lp
\blist{}
%\item[\bb{Rank}]
\item[\bb{ランク(Rank)}]
%Your character's name and professional ranking (based on the
%experience level, see below).
あなたのキャラクタ名と(後述の経験レベルに基づく)職業別の等級。
%%.lp
%\item[\bb{Strength}]
\item[\bb{強さ(Strength)}]
%A measure of your character's strength; one of your six basic
%attributes.  A human character's attributes can range from 3 to 18 inclusive;
%non-humans may exceed these limits
%(occasionally you may get super-strengths of the form 18/xx, and magic can
%also cause attributes to exceed the normal limits).  The
%higher your strength, the stronger you are.  Strength affects how
%successfully you perform physical tasks, how much damage you do in
%combat, and how much loot you can carry.
あなたのキャラクタの力の強さの値であり、6 つの基本的な属性の 1
つである。人間のキャラクターの場合,力の強さは 3 から 18 の間の値をとる。
その他の場合,この制限を越えることもある
(時折 18/xx のような特別な力を得ることもあり、魔法の力も普通の制限を越える
能力を与えることがある)。
力の強さの値が大きいほど、あなたの力は強い。
力の強さは力仕事をどれだけうまくやれるか、戦いでどれだけの
ダメージを与えるか、どれくらい荷物を運べるかに影響する。
%%.lp
%\item[\bb{Dexterity}]
\item[\bb{素早さ(Dexterity)}]
%Dexterity affects your chances to hit in combat, to avoid traps, and
%do other tasks requiring agility or manipulation of objects.
素早さは戦いにおける命中率、罠を逃れる可能性、そしてその他の
敏捷さを必要とする仕事や物をいかに器用に扱えるかなどのことがら
に影響を与える。
%%.lp
%\item[\bb{Constitution}]
\item[\bb{耐久力(Constitution)}]
%Constitution affects your ability to recover from injuries and other
%strains on your stamina.
耐久力は怪我やその他の圧迫によるスタミナの消耗から回復する能力に影響する。
%When strength is low or modest, constitution also affects how much you
%can carry.  With sufficiently high strength, the contribution to
%carrying capacity from your constitution no longer matters.
強さが低いか中くらいの場合、どれだけのものを運べるかに耐久力も影響する。
十分に強さが高い場合、運搬能力に関して耐久力は関係なくなる。
%%.lp
%\item[\bb{Intelligence}]
\item[\bb{知力(Intelligence)}]
%Intelligence affects your ability to cast spells and read spellbooks.
知力は呪文を唱える能力や魔法書を読む能力に影響を与える。
%%.lp
%\item[\bb{Wisdom}]
\item[\bb{賢さ(Wisdom)}]
%Wisdom comes from your practical experience (especially when dealing with
%magic).  It affects your magical energy.
賢さは(特に魔法を扱う)実戦的な経験により得られる。
魔法のエネルギーに影響する。
%%.lp
%\item[\bb{Charisma}]
\item[\bb{魅力(Charisma)}]
%Charisma affects how certain creatures react toward you.  In
%particular, it can affect the prices shopkeepers offer you.
魅力はある種の生物があなたに対してとる態度に影響を与える。
特に店主が示す価格に影響する。
%%.lp
%\item[\bb{Alignment}]
\item[\bb{属性(Alignment)}]
%%
%{\it Lawful}, {\it Neutral\/} or {\it Chaotic}.  Often, Lawful is
%taken as good and Chaotic as evil, but legal and ethical do not always
%coincide.  Your alignment influences how other
%monsters react toward you.  Monsters of a like alignment are more likely
%to be non-aggressive, while those of an opposing alignment are more likely
%to be seriously offended at your presence.
属性には {\it 秩序 (Lawful)}, {\it 中立 (Neutral)\/},
{\it 混沌(Chaotic)} がある。
しばしば、秩序は善、混沌は悪であるとされるが、法と道徳が
常に一致するとは限らない。
あなたの属性は他の怪物のあなたに対する反応に影響を与える。
属性が似ている怪物は好戦的ではなく、
反対の属性に対してはあなたの存在に対して極めて好戦的であるらしい。
%%.lp
%\item[\bb{Dungeon Level}]
\item[\bb{階数(Dungeon Level)}]
%How deep you are in the dungeon.  You start at level one and the number
%increases as you go deeper into the dungeon.  Some levels are special,
%and are identified by a name and not a number.  The Amulet of Yendor is
%reputed to be somewhere beneath the twentieth level.
どれだけ洞窟の奥深くまで行ったかを示す。
初期値は 1 で、洞窟を奥深く行くほど大きな値になる。
いくつかの階は特別で、数字ではなく名前で識別される。
イェンダーの魔除けは地下 20 階より深くのどこかにあると考えられている。
%%.lp
%\item[\bb{Gold}]
\item[\bb{金(Gold)}]
%The number of gold pieces you are openly carrying.  Gold which you have
%concealed in containers is not counted.
大っぴらに所持している金貨の数である。箱などに隠したお金は勘定に入っていない。
%%.lp
%\item[\bb{Hit Points}]
\item[\bb{体力(Hit Points)}]
%Your current and maximum hit points.  Hit points indicate how much
%damage you can take before you die.  The more you get hit in a fight,
%the lower they get.  You can regain hit points by resting, or by using
%certain magical items or spells.  The number in parentheses is the maximum
%number your hit points can reach.
体力の現在値と最大値である。
体力はあとどれだけのダメージを受けると死ぬかを示す。
戦いで傷つくほど値は減少する。体力は休息や魔法のアイテムや呪文により回復できる。
括弧の中の数字は体力のとり得る最大値である。
%%.lp
%\item[\bb{Power}]
\item[\bb{魔力(Power)}]
%Spell points.  This tells you how much mystic energy ({\it mana\/})
%you have available for spell casting.  Again, resting will regenerate the
%amount available.
呪文に使う魔力である。これは呪文を唱えるのに必要な神秘の力
({\it mana: マナ\/}) がどれほどあるかを示す。
魔力もまた休息することにより回復する。
%%.lp
%\item[\bb{Armor Class}]
\item[\bb{防御値(Armor Class)}]
%A measure of how effectively your armor stops blows from unfriendly
%creatures.  The lower this number is, the more effective the armor; it
%is quite possible to have negative armor class.
非友好的な生物からの攻撃を防具がどれほど食い止められるかを示す値である。
数字が小さいほど防具は有効であり、
防御値が負の値になることさえ十分あり得る。
%%.lp
%\item[\bb{Experience}]
\item[\bb{経験値(Experience)}]
%Your current experience level and experience points.  As you
%adventure, you gain experience points.  At certain experience point
%totals, you gain an experience level.  The more experienced you are,
%the better you fight and withstand magical attacks.  Many dungeons
%show only your experience level here.
現在の経験レベルと経験値である。冒険が進むと経験値を得ること
ができる。経験値の合計がある一定値まで達すると、経験レベルが上
がる。経験を積むにつれ、戦い方が上達し魔法の攻撃にも耐えられるよ
うになる。多くのバージョンでは経験レベルだけが表示される。
%%.lp
%\item[\bb{Time}]
\item[\bb{時間(Time)}]
%The number of turns elapsed so far, displayed if you have the
%{\it time\/} option set.
経過したターン数である。
{\it time\/}
オプションがオンのときに表示される。
%%.lp
%\item[\bb{Status}]
\item[\bb{状態(Status)}]
%Hunger:
%your current hunger status.
%Values are {\it Satiated}, {\it Not~Hungry\/} (or {\it Normal\/}),
%{\it Hungry}, {\it Weak}, and {\it Fainting}.
%%.\" not mentioned: Fainted
%Not shown when {\it Normal}.
腹具合:
現在の腹具合状態。
値は {\it Satiated}, {\it Not~Hungry\/} (or {\it Normal\/}),
{\it Hungry}, {\it Weak}, {\it Fainting}。

%%.lp ""
%Encumbrance:
%an indication of how what you are carrying affects your ability to move.
%Values are {\it Unencumbered}, {\it Encumbered}, {\it Stressed},
%{\it Strained}, {\it Overtaxed}, and {\it Overloaded}.
%Not shown when {\it Unencumbered}.
荷物の重さ:
運んでいる物があなたの移動能力にどれくらい影響を与えているかを示す。
値は {\it Unencumbered} (通常)、{\it Encumbered} (よろめき)、
{\it Stressed} (圧迫)、{\it Strained} (限界)、
{\it Overtaxed} (過重)、 {\it Overloaded} (超過)。
{\it Unencumbered} のときには表示されない。

%%.lp ""
%Fatal~conditions:
%{\it Stone\/} (aka {\it Petrifying}, turning to stone),
%{\it Slime\/} (turning into green slime),
%{\it Strngl\/} (being strangled),
%{\it FoodPois\/} (suffering from acute food poisoning),
%{\it TermIll\/} (suffering from a terminal illness).
致命的な状態:
{\it Stone\/} (石化)、
{\it Slime\/} (どろどろ; スライム化)、
{\it Strngl\/} (窒息)、
{\it FoodPois\/} (食毒)、
{\it TermIll\/} (病気)。

%%.lp ""
%Non-fatal~conditions:
%{\it Blind\/} (can't see), {\it Deaf\/} (can't hear),
%{\it Stun\/} (stunned), {\it Conf\/} (confused), {\it Hallu\/} (hallucinating).
致命的でない状態:
{\it Blind\/} (盲目)、{\it Deaf\/} (耳聾)、
{\it Stun\/} (眩暈)、{\it Conf\/} (混乱)、{\it Hallu\/} (幻覚)。

%%.lp ""
%Movement~modifiers:
%{\it Lev\/} (levitating), {\it Fly\/} (flying), {\it Ride\/} (riding).
移動状態変化:
{\it Lev\/} (浮遊)、{\it Fly\/} (飛行)、{\it Ride\/} (騎乗)。

%%.lp ""
%Other conditions and modifiers exist, but there isn't enough room to
%display them with the other status fields.  The `{\tt \^{}X}' command shows
%all relevant status conditions.
その他の状態も存在するが、それらの状態を表示するための空きがない。
`{\tt \^{}X}' コマンドは関係する全ての状態を表示する。
\elist

%%.hn 2
%\subsection*{The message line (top)}
\subsection*{メッセージ行(画面の上部)}

%%.pg
%The top line of the screen is reserved for messages that describe
%things that are impossible to represent visually.  If you see a
%``{\tt --More--}'' on the top line, this means that {\it NetHack\/} has
%another message to display on the screen, but it wants to make certain
%that you've read the one that is there first.  To read the next message,
%just press the space bar.
画面の最上行は視覚的に表現できないことを説明するメッセージのために使われる。
最上行に``{\tt --More--}''が表示されたときは、
このあとにさらにメッセージが続いているけれども
現在のメッセージをまず確実に読むよう促しているということである。
次のメッセージを読むためには単にスペースキーを押せばよい。

%%.pg
%To change how and what messages are shown on the message line,
%see ``{\it Configuring Message Types\/}`` and the {\it verbose\/}
%option.
メッセージ行にどんなメッセージをどのように表示するかを変更するには、
``{\it Configuring Message Types\/}`` と
{\it verbose\/} オプションを参照のこと。

%%.hn 2
%\subsection*{The map (rest of the screen)}
\subsection*{地図(画面のその他の部分)}

%%.pg
%The rest of the screen is the map of the level as you have explored it
%so far.  Each symbol on the screen represents something.  You can set
%various graphics
%options to change some of the symbols the game uses; otherwise, the
%game will use default symbols.  Here is a list of what the default
%symbols mean:
画面のその他の部分は、あなたが現在いる階で
それまでに探索し終わった部分の地図である。
画面上のシンボルはそれぞれ何かを表している。
表示されるシンボルのうちいくつかは、
さまざまなグラフィックオプションを設定することによって変えることができる。
そうでなければ標準設定のシンボルが用いられる。
次に標準設定のシンボルの意味の一覧を示す。

\blist{}
%%.lp
\item[\tb{"- と |}]
%The walls of a room, or an open door.  Or a grave ({\tt |}).
部屋の壁、または開いた扉、または墓 ({\tt |})。
%%.lp
\item[\tb{.}]
%The floor of a room, ice, or a doorless doorway.
部屋の床、氷、または扉のない出入口。
%%.lp
\item[\tb{\#}]
%A corridor, or iron bars, or a tree, or possibly a kitchen sink (if
%your dungeon has sinks), or a drawbridge.
通路、鉄の棒、木、あるいは(あなたの迷宮にあるなら)台所の流し台、
または跳ね橋
%%.lp
\item[\tb{>}]
%Stairs down: a way to the next level.
上り階段:上の階への通路。
%%.lp
\item[\tb{<}]
%Stairs up: a way to the previous level.
下り階段:下の階への通路。
%%.lp
\item[\tb{+}]
%A closed door, or a spellbook containing a spell you may be able to learn.
閉じた扉、または学習できるかもしれない呪文の書かれた魔法書。
%%.lp
\item[\tb{@}]
%Your character or a human.
あなたのキャラクター、または人間。
%%.lp
\item[\tb{\$}]
%A pile of gold.
金貨の山。
%%.lp
\item[\tb{\^}]
%A trap (once you have detected it).
罠(ひとたび発見すれば表示される)。
%%.lp
\item[\tb{)}]
%A weapon.
武器。
%%.lp
\item[\tb{[}]
%A suit or piece of armor.
服またはなんらかの防具。
%%.lp
\item[\tb{\%}]
%Something edible (not necessarily healthy).
食料(衛生的であるとは限らない)。
%%.lp
\item[\tb{?}]
%A scroll.
巻物。
%%.lp
\item[\tb{/}]
%A wand.
杖。
%%.lp
\item[\tb{=}]
%A ring.
指輪。
%%.lp
\item[\tb{!}]
%A potion.
水薬。
%%.lp
\item[\tb{(}]
%A useful item (pick-axe, key, lamp \ldots).
便利なアイテム(つるはし、鍵、ランプ…)。
%%.lp
\item[\tb{"}]
%An amulet or a spider web.
魔除けまたはくもの巣。
%%.lp
\item[\tb{*}]
%A gem or rock (possibly valuable, possibly worthless).
宝石または岩(貴重なものかもしれないし、価値のないものかもしれない)。
%%.lp
\item[\tb{\`}]
%A boulder or statue.
岩、または彫像。
%%.lp
\item[\tb{0}]
%An iron ball.
鉄球。
%%.lp
\item[\tb{_}]
%An altar, or an iron chain.
祭壇、または鉄の鎖。
%%.lp
\item[\tb{\{}]
%A fountain.
泉
%%.lp
\item[\tb{\}}]
%A pool of water or moat or a pool of lava.
水たまり、または堀、または溶岩の海。
%%.lp
\item[\tb{$\backslash$}]
%An opulent throne.
豪華な王座
%%.lp
\item[\tb{a-zA-Z およびその他のシンボル}]
%Letters and certain other symbols represent the various inhabitants
%of the Mazes of Menace.  Watch out, they can be nasty and vicious.
%Sometimes, however, they can be helpful.
これらの文字とその他のいくつかのシンボルは恐怖の迷宮のいろいろな住人たちを表す。
彼らは不機嫌で悪意に満ちているかも知れないので警戒しなければならない。
が、ときには何かの手助けになることもある。
%%.lp
\item[\tb{I}]
%This marks the last known location of an invisible or otherwise unseen
%monster.  Note that the monster could have moved.
%The `{\tt F}' and `{\tt m}' commands may be useful here.
このマークは透明、あるいはその他の見えない怪物が最後にいたらしい場所に表示される。
怪物はすでに移動しているかもしれないことに注意すること。
ここでは`{\tt F}'と`{\tt m}'のコマンドが便利だろう。

\elist
%%.pg
%You need not memorize all these symbols; you can ask the game what any
%symbol represents with the `{\tt /}' command (see the next section for
%more info).
これらのシンボルをすべて記憶する必要はない。`{\tt /}' コマンドでどのシンボ
ルが何を表すか知ることができる(詳細は次の項を参照のこと)。

%%.hn 1
%\section{Commands}
\section{コマンド}

%%.pg
%Commands can be initiated by typing one or two characters to which
%the command is bound to, or typing the command name in the extended
%commands entry.  Some commands,
%like ``{\tt search}'', do not require that any more information be collected
%by {\it NetHack\/}.  Other commands might require additional information, for
%example a direction, or an object to be used.  For those commands that
%require additional information, {\it NetHack\/} will present you with either
%a menu of choices, or with a command line prompt requesting information.
%Which you are presented with will depend chiefly on how you have set the
%`{\it menustyle\/}'
%option.
コマンドは、そのコマンドに割り当てられている 1 文字か 2 文字のキー入力を
行うか、拡張コマンドエントリのコマンド名を入力することで実行できる。
いくつかのコマンド、例えば``{\tt search}''はそれ以上 {\it NetHack\/} に
情報を与える必要がない。
他のコマンドは例えば方向や使用する物などの情報をさらに与える必要がある。
これらさらなる情報を必要とするコマンドに対しては、
{\it NetHack\/} は選択メニューもしくはコマンドラインプロンプトのいずれかを表示する。
どちらが表示されるかは主として
`{\it menustyle\/}'
オプションをどのように設定したかによる。

%%.pg
%For example, a common question in the form ``{\tt What do you want to
%use? [a-zA-Z\ ?*]}'', asks you to choose an object you are carrying.
%Here, ``{\tt a-zA-Z}'' are the inventory letters of your possible choices.
%Typing `{\tt ?}' gives you an inventory list of these items, so you can see
%what each letter refers to.  In this example, there is also a `{\tt *}'
%indicating that you may choose an object not on the list, if you
%wanted to use something unexpected.  Typing a `{\tt *}' lists your entire
%inventory, so you can see the inventory letters of every object you're
%carrying.  Finally, if you change your mind and decide you don't want
%to do this command after all, you can press the `ESC' key to abort the
%command.
例えば
``{\tt What do you want to use? [a-zA-Z\ ?*]}'' という形式の質問がよくあるが、
これはあなたに持っている物のうちどれを選ぶかを尋ねるものである。
この質問において「a-zA-Z」はあなたが選べる持ち物の目録記号である。
`{\tt ?}' を入力するとこれらの物の目録一覧が得られ、
個々の記号が何を表しているのか知ることができる。
この例では、`{\tt *}' もある。これは一覧には表示されておらず、
通常そのコマンドでの使用を想定していない物でも、使おうと思えばそれを
選ぶことができることを示している。`{\tt *}' を入力するとすべての持ち物の目録が
表示され、個々の持ち物すべてについての目録記号を知ることができる。
考え直して結局このコマンドを使わないことにしたときには `ESC' キーを押せば
コマンドを終了することができる。

%%.pg
%You can put a number before some commands to repeat them that many
%times; for example, ``{\tt 10s}'' will search ten times.  If you have the
%{\it number\verb+_+pad\/}
%option set, you must type `{\tt n}' to prefix a count, so the example above
%would be typed ``{\tt n10s}'' instead.  Commands for which counts make no
%sense ignore them.  In addition, movement commands can be prefixed for
%greater control (see below).  To cancel a count or a prefix, press the
%`ESC' key.
コマンドの中には、その前に数字を入力することで何回も繰り返すこと
ができるものもある。例えば「10s」は 10 回の探索を表している。もし
{\it number\verb+_+pad\/}
オプションがオンのときには数字の前に `{\tt n}' を入力しなければならない。
つまり上の例では代わりに「n 10s」と入力しなければならない。
複数回実行することが無意味なコマンドではこれは無視される。
さらに、移動コマンドの前に特定の文字(プレフィックス)を付けることによって
さまざまな移動方法をとることができる(後述)。
繰り返し回数やプレフィックスを取り消すためには `ESC' キーを押せばよい。

%%.pg
%The list of commands is rather long, but it can be read at any time
%during the game through the `{\tt ?}' command, which accesses a menu of
%helpful texts.  Here are the default key bindings for your reference:
コマンドの一覧はかなり長いが、ゲーム中に `{\tt ?}' コマンドを使用して
ヘルプメニューを見ることにより、いつでもコマンドの一覧を見ることができる。
以下は標準キー配置の一覧である。

\blist{}
%%.lp
\item[\tb{?}]
%Help menu:  display one of several help texts available.
ヘルプメニュー: 表示可能なヘルプ画面のうちの 1 つを表示する。
%%.lp
\item[\tb{/}]
%The {\tt whatis} command, to
%tell what a symbol represents.  You may choose to specify a location
%or type a symbol (or even a whole word) to explain.
{\tt これは何} コマンド;
シンボルが何を意味するかを示す。指定するには場所を示すか、
またはシンボル(あるいは単語全体)を入力するかを選択することができる。
%Specifying a location is done by moving the cursor to a particular spot
%on the map and then pressing one of `{\tt .}', `{\tt ,}', `{\tt ;}',
%or `{\tt :}'.  `{\tt .}' will explain the symbol at the chosen location,
%conditionally check for ``{\tt More info?}'' depending upon whether the
%`{\it help\/}'
%option is on, and then you will be asked to pick another location;
%`{\tt ,}' will explain the symbol but skip any additional
%information, then let you pick another location;
%`{\tt ;}' will skip additional info and also not bother asking
%you to choose another location to examine; `{\tt :}' will show additional
%info, if any, without asking for confirmation.  When picking a location,
%pressing the {\tt ESC} key will terminate this command, or pressing `{\tt ?}'
%will give a brief reminder about how it works.
場所の指定は、適切な位置までカーソルを移動させてから
`{\tt .}'、`{\tt ,}'、`{\tt ;}'、`{\tt :}'のどれかを押すことによって行なう。
`{\tt .}'を押すと、選んだ場所で表示されているシンボルの説明が表示され、
もし `{\it help\/}'
オプションがオンなら場合によっては``{\tt More info?}''(詳細を見る?)と質問され、
それからさらに他の場所を指定することができる。
`{\tt ,}'を押すと、シンボルの説明が表示されるが、詳細情報は表示されない;
それから新しい場所を指定する。
`{\tt ;}'を押すと、詳細情報は表示されず、さらに他の場所の指定もできない。
`{\tt :}'を押すと、(もしあるなら)確認なしで詳細情報が表示される。
場所を選ぶ時に {\tt ESC} キーを押すことによってこのコマンドを中断でき、
`{\tt ?}'を押すとどのように動作するかの簡単な説明が表示される。

%%.lp ""
%If the
%{\it autodescribe\/}
%option is on, a short description of what you see at each location is
%shown as you move the cursor.  Typing `{\tt \#}' while picking a location will
%toggle that option on or off.
%The
%{\it whatis\verb+_+coord\/}
%option controls whether the short description includes map coordinates.
{\it autodescribe\/} がオンの場合、カーソルを動かす毎にその場所で見える物の
短い説明が表示される。
場所を選ぶ際に `{\tt \#}' を押すとこのオプションのオンオフが切り替わる。
{\it whatis\verb+_+coord\/}
オプションは短い説明にマップの座標を含むかどうかを制御する。

%%.lp ""
%Specifying a name rather than a location
%always gives any additional information available about that name.
場所でなく名前を指定した場合は、
常にその名前に対する追加情報が表示される。

%%.lp ""
%You may also request a description of nearby monsters,
%all monsters currently displayed, nearby objects, or all objects.
%The
%{\it whatis\verb+_+coord\/}
%option controls which format of map coordinate is included with their
%descriptions.
近くの怪物、現在表示されている全ての怪物、近くの物、全ての物の説明を
要求することもできる。
{\it whatis\verb+_+coord\/}
オプションはその説明にマップの座標を含むかどうかを制御する。
%%.lp
\item[\tb{\&}]
%Tell what a command does.
コマンドの動作を示す。
%%.lp
\item[\tb{<}]
%Go up to the previous level (if you are on a staircase or ladder).
上の階へ行く(階段やはしごにいるとき)。
%%.lp
\item[\tb{>}]
%Go down to the next level (if you are on a staircase or ladder).
下の階へ行く(階段やはしごにいるとき)。
%%.lp
\item[\tb{[yuhjklbn]}]
%Go one step in the direction indicated (see Figure 2).  If you sense
%or remember
%a monster there, you will fight the monster instead.  Only these
%one-step movement commands cause you to fight monsters; the others
%(below) are ``safe.''
指定した方向へ一歩進む(図 2 参照)。その方向に怪物がいると感じているか
覚えているときは、
移動するのではなく怪物と戦うことになる。
怪物と戦うことになるのはこれらの一歩移動のコマンドだけである;
その他のコマンド(後述)は「安全」である。
%%.sd
\begin{center}
\begin{tabular}{cc}
%\verb+   y  k  u   + & \verb+   7  8  9   +\\
%\verb+    \ | /    + & \verb+    \ | /    +\\
%\verb+   h- . -l   + & \verb+   4- . -6   +\\
%\verb+    / | \    + & \verb+    / | \    +\\
%\verb+   b  j  n   + & \verb+   1  2  3   +\\
%                     & (if {\it number\verb+_+pad\/} set)
\verb+   y  k  u   + & \verb+   7  8  9   +\\
\verb+    \ | /    + & \verb+    \ | /    +\\
\verb+   h- . -l   + & \verb+   4- . -6   +\\
\verb+    / | \    + & \verb+    / | \    +\\
\verb+   b  j  n   + & \verb+   1  2  3   +\\
                     & ({\it number\verb+_+pad\/} がオンなら)
\end{tabular}
\end{center}
%%.ed
\begin{center}
%Figure 2
図2
\end{center}
%%.lp
\item[\tb{[YUHJKLBN]}]
%Go in that direction until you hit a wall or run into something.
壁に突き当たるか何かに衝突するまで指定した方向に進む。
%%.lp
\item[\tb{m[yuhjklbn]}]
%Prefix:  move without picking up objects or fighting (even if you remember
%a monster there).\\
プレフィックス: 物を拾ったり、
(たとえそこに怪物がいると思っていても)戦ったりせずに移動する。\\
%%.lp ""
%A few non-movement commands use the `{\tt m}' prefix to request operating
%via menu (to temporarily override the
%{\it menustyle:Traditional\/}
%option).
%Primarily useful for `{\tt ,}' (pickup) when there is only one class of
%objects present (where there won't be any ``what kinds of objects?'' prompt,
%so no opportunity to answer `{\tt m}' at that prompt).\\
いくつかの移動しないコマンドは、(一時的に
{\it menustyle:Traditional\/}
オプションを上書きして)メニュー経由での操作を要求するために
`{\tt m}'接頭辞を使う。
一つの種類だけの物が存在している
(``what kinds of objects?'' と聞かれないので、ここで `{\tt m}' と
答える機会がない)場合に`{\tt ,}' (拾う) を使うときに有用である。
%%.lp ""
%A few other commands (eat food, offer sacrifice, apply tinning-kit) use
%the `{\tt m}' prefix to skip checking for applicable objects on the floor
%and go straight to checking inventory,
%or (for ``{\tt \#loot}'' to remove a saddle),
%skip containers and go straight to adjacent monsters. The prefix will
%make ``{\tt \#travel}'' command show a menu of interesting targets in sight.
%In debug mode (aka ``wizard mode''), the `{\tt m}' prefix may also be
%used with the ``{\tt \#teleport}'' and ``{\tt \#wizlevelport}'' commands.
いくつかのその他のコマンド(食料を食べる、生け贄を捧げる、缶詰を作る)では、
`{\tt m}' 接頭辞で床の上にある適用可能な物のチェックを飛ばして、
すぐに持ち物のチェックに向かったり、
(鞍を外すために ``{\tt \#loot}'' を使う場合)
入れ物を飛ばしてすぐに隣接する怪物に向かったりする。
``{\tt \#travel}'' コマンドではこの接頭辞で
視界内の興味深い目標のメニューを表示する。
デバッグモード(またの名を``ウィザードモード'')では、
`{\tt m}' 接頭辞は
``{\tt \#teleport}'' と ``{\tt \#wizlevelport}'' コマンドにも使える。
%%.lp
\item[\tb{F[yuhjklbn]}]
%Prefix:  fight a monster (even if you only guess one is there).
プレフィックス: (そこに怪物がいると予想しているだけでも)怪物と戦う。
%%.lp
\item[\tb{M[yuhjklbn]}]
%Prefix:  Move far, no pickup.
プレフィックス: 物を拾わずに遠くへ移動する。
%%.lp
\item[\tb{g[yuhjklbn]}]
%Prefix:  Move until something interesting is found.
プレフィックス: 何かが見つかるまで移動する。
%%.lp
\item[\tb{G[yuhjklbn] {\rm or} <CONTROL->[yuhjklbn]}]
%Prefix:  Same as `{\tt g}', but forking of corridors is not considered
%interesting.
プレフィックス: 'g' と同じ。ただし通路の分岐点では止まらない。
%%.lp
\item[\tb{_}]
%Travel to a map location via a shortest-path algorithm.\\
指定された位置まで、最短距離アルゴリズムを用いて移動する。\\
%%.lp ""
%The shortest path
%is computed over map locations the hero knows about (e.g. seen or
%previously traversed).  If there is no known path, a guess is made instead.
最短距離は、あなたが知っている(見た、または以前に通った)地図配置を
基に計算される。
知っている道がない場合、代わりに推測が行われる。
%Stops on most of
%the same conditions as the `G' command, but without picking up
%objects, similar to the `M' command.  For ports with mouse
%support, the command is also invoked when a mouse-click takes place on a
%location other than the current position.
`G'コマンドとほぼ同じ条件で停止するが、
`M'コマンドと同様に物を拾わないで移動する。
マウスに対応している版では、現在位置以外の位置を
マウスでクリックしたときにもこのコマンドが発動する。
%%.lp
\item[\tb{.}]
%Wait or rest, do nothing for one turn.
待つ、または休憩する; 1 ターン何もしない。
%%.lp
\item[\tb{a}]
%Apply (use) a tool (pick-axe, key, lamp \ldots).\\
道具(つるはし、鍵、ランプ…)を用いる(使う)。\\
%%.lp ""
%If used on a wand, that wand will be broken, releasing its magic in the
%process.  Confirmation is required.
杖に対して使うと、杖を壊し、その過程で魔力が開放される。
確認が必要である。
%%.lp
\item[\tb{A}]
%Remove one or more worn items, such as armor.\\
1 つまたは複数の身につけているもの(防具など)をはずす。\\
%%.lp ""
%Use `{\tt T}' (take off) to take off only one piece of armor
%or `{\tt R}' (remove) to take off only one accessory.
防具を1つだけはずすときには 'T'(take off: 防具をはずす)を、アクセサリを
1 つだけはずすときは 'R'(remove: アクセサリをはずす)を使用せよ。
%%.lp
\item[\tb{\^{}A}]
%Redo the previous command.
1 つ前のコマンドを繰り返す。
%%.lp
\item[\tb{c}]
%Close a door.
扉を閉じる。
%%.lp
\item[\tb{C}]
%Call (name) a monster, an individual object, or a type of object.\\
怪物、個々のオブジェクト、オブジェクトの種類に名前を付ける。\\
%%.lp ""
%Same as extended command ``{\tt \#name}''.
拡張コマンド``{\tt \#name}''と同じ。
%%.lp
\item[\tb{\^{}C}]
%Panic button.  Quit the game.
パニックボタン。ゲームを放棄する。
%%.lp
\item[\tb{d}]
%Drop something.\\
何かを下に置く。\\
%For example {\tt d7a} --- drop seven items of object
%{\it a}.
例えば {\tt d7a} --- {\it a} という物を 7 個下に置く。
%%.lp
\item[\tb{D}]
%Drop several things.\\
いくつかの物を下に置く。\\
%%.lp ""
%In answer to the question\\
次の質問\\
%``{\tt What kinds of things do you want to drop? [!\%= BUCXaium]}''\\
``{\tt What kinds of things do you want to drop?[!\%= BUCXaium]}''
(どの種類のものを置きますか?)\\
%you should type zero or more object symbols possibly followed by
%`{\tt a}' and/or `{\tt i}' and/or `{\tt u}' and/or `{\tt m}'.
%In addition, one or more of
%the bless\-ed/\-un\-curs\-ed/\-curs\-ed groups may be typed.\\
に対して、0 個以上の物のシンボルを入力することで答える。
さらに`{\tt a}', `{\tt i}', `{\tt m}', `{\tt u}' をその後に入力することもできる。
加えて、祝福された(B)/呪われていない(U)/呪われた(C)アイテムを指定することもできる。\\
%%.sd
%%.si
%{\tt DB}  --- drop all objects known to be blessed.\\
%{\tt DU}  --- drop all objects known to be uncursed.\\
%{\tt DC}  --- drop all objects known to be cursed.\\
%{\tt DX}  --- drop all objects of unknown B/U/C status.\\
%{\tt Da}  --- drop all objects, without asking for confirmation.\\
%{\tt Di}  --- examine your inventory before dropping anything.\\
%{\tt Du}  --- drop only unpaid objects (when in a shop).\\
%{\tt Dm}  --- use a menu to pick which object(s) to drop.\\
%{\tt D\%u} --- drop only unpaid food.
{\tt DB}  --- 祝福されていると判明している物全てを下に置く。\\
{\tt DU}  --- 呪われていないと判明している物全てを下に置く。\\
{\tt DC}  --- 呪われていると判明している物全てを下に置く。\\
{\tt DX}  --- 祝福/呪いが判明していない物全てを下に置く。\\
{\tt Da}  --- 確認なしにすべての物を下に置く。\\
{\tt Di}  --- 物を置く前に持ち物を確認する。\\
{\tt Du}  --- (店にいるとき)代金未払の物だけを下に置く。\\
{\tt Dm}  --- 置く物を選ぶのにメニューを用いる。\\
{\tt D\%u} --- 代金未払の食料だけを下に置く。
%%.ei
%%.ed

%The last example shows a combination.
%There are three categories of object filtering: class (`{\tt !}' for
%potions, `{\tt ?}' for scrolls, and so on), shop status (`{\tt u}' for
%unpaid, in other words, owned by the shop), and bless/curse state
%(`{\tt B}', `{\tt U}', `{\tt C}', and `{\tt X}' as shown above).
%If you specify more than one value in a category (such as ``{\tt !?}'' for
%potions and scrolls or ``{\tt BU}'' for blessed and uncursed), an inventory
%object will meet the criteria if it matches any of the specified
%values (so ``{\tt !?}'' means `{\tt !}' or `{\tt ?}').
%If you specify more than one category, an inventory object must meet
%each of the category criteria (so ``{\tt \%u}'' means class `{\tt \%}' and
%unpaid `{\tt u}').
%Lastly, you may specify multiple values within multiple categories:
%``{\tt !?BU}'' will select all potions and scrolls which are known to be
%blessed or uncursed.
%(In versions prior to 3.6, filter combinations behaved differently.)
最後の例は組み合わせを示している。
物のフィルタリングに三つのカテゴリがある:
種類 (水薬 は `{\tt !}'、巻物は `{\tt ?}' など)、
買い物の状態 (未払い、言い換えると店が持っているものは `{\tt u}')、
祝福/呪い状態(上述の通りの `{\tt B}', `{\tt U}', `{\tt C}', {\tt X}')。
一つのカテゴリで二つ以上指定した場合
(水薬と巻物で ``{\tt !?}'' や、祝福と呪われていないで ``{\tt BU}'' など)、
持ち物は指定された値のどれかに一致した場合に条件を満たす
(従って ``{\tt !?}'' は `{\tt !}' または `{\tt ?}' を意味する)。.
複数のカテゴリを指定すると、持ち物は
それぞれのクラスタの条件に一致した場合に条件を満たす
(従って ``{\tt \%u}'' は種類 `{\tt \%}' と未払い `{\tt u}' を意味する)。
最後に、複数のカテゴリの複数の値を指定できる:
``{\tt !?BU}'' は、祝福または呪われていないと分かっている、
全ての水薬と巻物を選ぶ。
(3.6 より前のバージョンでは、フィルタの組み合わせは異なる振る舞いだった。)
%%.lp
\item[\tb{\^{}D}]
%Kick something (usually a door).
何か(普通は扉)を蹴る。
%%.lp
\item[\tb{e}]
%Eat food.\\
食料を食べる。\\
%%.lp ""
%Normally checks for edible item(s) on the floor, then if none are found
%or none are chosen, checks for edible item(s) in inventory.
%Precede `{\tt e}' with the `{\tt m}' prefix to bypass attempting to eat
%anything off the floor.\\
通常は床にある食べ物が調べられ、何も見つからないかどれも選ばなかった場合には
持ち物の食べ物が調べられる。
`{\tt e}' の前に `{\tt m}' 接頭辞を置くことで床にあるものを食べようとするのを
回避する。
%%.lp ""
%If you attempt to eat while already satiated, you might choke to death.
%If you risk it, you will be asked whether
%to ``continue eating?'' {\it if you survive the first bite\/}.
%You can set the
%{\it paranoid\verb+_+confirmation:eating\/}
%option to require a response of ``{\tt yes}'' instead of just `{\tt y}'.
既に満腹のときに食べようとすると、窒息して死ぬかもしれない。
その危険を冒すなら、{\it 最初の一口を生き残ったなら \/}
``continue eating?'' と聞かれる。
返答に単なる `{\tt y}' ではなく ``{\tt yes}'' が必要なように
{\it paranoid\verb+_+confirmation:eating\/} オプションを設定できる。
%%.lp
%% Make sure Elbereth is not hyphenated below, the exact spelling matters.
%% (Only specified here to parallel Guidebook.mn; use of \tt font implicity
%% prevents automatic hyphenation in TeX and LaTeX.)
%\hyphenation{Elbereth}		%override the deduced syllable breaks
\item[\tb{E}]
%Engrave a message on the floor.\\
床にメッセージを刻み込む。\\
%%.sd
%%.si
%{\tt E-} --- write in the dust with your fingers.\\
{\tt E-} --- ほこりの中に指で書く。\\
%%.ei
%%.ed
%%.lp ""
%Engraving the word ``{\tt Elbereth}'' will cause most monsters to not attack
%you hand-to-hand (but if you attack, you will rub it out); this is
%often useful to give yourself a breather.
``{\tt Elbereth}'' という語を刻み込むと、
たいていの怪物はあなたに白兵戦を挑んで来なくなる
(しかしあなたが攻撃をすると文字は消えてしまう)。
これは、一息いれたいときになかなか便利である。
%%.lp
\item[\tb{f}]
%Fire (shoot or throw) one of the objects placed in your quiver (or
%quiver sack, or that you have at the ready).
%You may select ammunition with a previous `{\tt Q}' command, or let the
%computer pick something appropriate if {\it autoquiver\/} is true.\\
矢筒(または矢筒袋または準備しているもの)に入れてあるものを発射する
(投げるか撃つ)。
発射するものは予め`{\tt Q}'コマンドで選択することも出来るし、
{\it autoquiver\/} オプションが設定されている場合は、コンピューターに
自動的に適当なものを設定させることもできる。\\
%%.lp ""
%See also `{\tt t}' (throw) for more general throwing and shooting.
より一般的に投げたり撃ったりするには `{\tt t}'(投げる) も参照のこと。
%%.lp
\item[\tb{i}]
%List your inventory (everything you're carrying).
持ち物の目録(持っている物すべて)を表示する。
%%.lp
\item[\tb{I}]
%List selected parts of your inventory, usually be specifying the character
%for a particular set of objects, like `{\tt [}' for armor or `{\tt !}'
%for potions.\\
持ち物の目録のうち指定した一部を表示する; 普通は鎧なら`{\tt [}'、薬なら`{\tt !}'というように
オブジェクトの種類の文字で指定する。\\
%%.sd
%%.si
%{\tt I*} --- list all gems in inventory;\\
%{\tt Iu} --- list all unpaid items;\\
%{\tt Ix} --- list all used up items that are on your shopping bill;\\
%{\tt IB} --- list all items known to be blessed;\\
%{\tt IU} --- list all items known to be uncursed;\\
%{\tt IC} --- list all items known to be cursed;\\
%{\tt IX} --- list all items whose bless/curse status is unknown;\\
%{\tt I\$} --- count your money.
{\tt I*} --- 持ち物のうち宝石をすべて表示する。\\
{\tt Iu} --- 代金未払いの物をすべて表示する。\\
{\tt Ix} --- 代金未払いだが使ってしまった物をすべて表示する。\\
{\tt IB} --- 祝福されていると分かっている物を全て表示する。\\
{\tt IU} --- 呪われていないと分かっている物を全て表示する。\\
{\tt IC} --- 呪われていると分かっている物を全て表示する。\\
{\tt IX} --- 祝福/呪いの状態が分からないものを全て表示する。\\
{\tt I\$} --- お金を数える。
%%.ei
%%.ed
%%.lp
\item[\tb{o}]
%Open a door.
扉を開ける。
%%.lp
\item[\tb{O}]
%Set options.\\
オプションを設定する。\\
%%.lp ""
%A menu showing the current option values will be
%displayed.  You can change most values simply by selecting the menu
%entry for the given option (ie, by typing its letter or clicking upon
%it, depending on your user interface).  For the non-boolean choices,
%a further menu or prompt will appear once you've closed this menu.
現在のオプションの値の一覧が表示される。
(見出し文字をタイプするかクリックすることによって)変更したいオプションを選んで、
ほとんどの値を変更することができる。
ブール値でないオプションの場合は、さらにメニューが出るか、プロンプトが出る。
%The available options
%are listed later in this Guidebook.  Options are usually set before the
%game rather than with the `{\tt O}' command; see the section on options below.
設定可能なオプションはこのガイドブックの後ほどに一覧がある。
オプションは通常は `{\tt O}' コマンドではなく、ゲームを始める前に設定する。
後述のオプションの項を参照のこと。
%%.lp
\item[\tb{\^{}O}]
%Show overview.\\
概要を表示する。
%%.lp ""
%Shortcut for ``{\tt \#overview}'':
%list interesting dungeon levels visited.\\
``{\tt \#overview}'' の短縮版:
訪れた興味深い階の一覧。\\
%%.lp ""
%(Prior to 3.6.0, `{\tt \^{}O}' was a debug mode command which listed
%the placement of all special levels.
%Use ``{\tt \#wizwhere}'' to run that command.)
%%.lp
\item[\tb{p}]
%Pay your shopping bill.
代金を払う。
%%.lp
\item[\tb{P}]
%Put on an accessory (ring, amulet, or blindfold).\\
装飾品(指輪、魔除け、目隠し)を身につける。\\
%%.lp ""
%This command may also be used to wear armor.  The prompt for
%which inventory item to use will only list accessories, but choosing
%an unlisted item of armor will attempt to wear it.
%(See the `{\tt W}' command below.  It lists armor as the inventory
%choices but will accept an accessory and attempt to put that on.)
このコマンドは防具を身につけるのにも使える。
どれを選ぶかの一覧では装飾品しか表示されないが、一覧に表示されていない
防具を選択することでそれを身につけようとすることができる。
(後述する `{\tt W}' コマンドを参照のこと。
これは一覧として防具が表示されるが、装飾品も選択できて、それを
身につけようとする)。
%%.lp
\item[\tb{\^{}P}]
%Repeat previous message.\\
1 つ前のメッセージをもう一度表示する。\\
%%.lp ""
%Subsequent {\tt \^{}P}'s repeat earlier messages.
%For some interfaces, the behavior can be varied via the
%{\it msg\verb+_+window\/} option.
続けて {\tt \^{}P} を入力するとさらに前のメッセージが順に表示される。
一部のインターフェースでは、
この振る舞いは {\it msg\verb+_+windows\/} オプションによって変化する。
%%.lp
\item[\tb{q}]
%Quaff (drink) something (potion, water, etc).
何か(水薬,水など)を飲む。
%%.lp
\item[\tb{Q}]
%Select an object for your quiver, quiver sack, or just generally at
%the ready (only one of these is available at a time).  You can then throw
%this (or one of these) using
%the `f' command.\\
矢筒、矢筒袋、あるいた単に一般的に準備するものを選択する(同時に指定できるのは
これらのうち一つのみ)。
ここで選択したものは`f'コマンドで発射することができる。\\
%%.lp ""
%(In versions prior to 3.3 this was the command to quit
%the game, which has been moved to ``{\tt \#quit}''.)
(バージョン 3.3 以前ではこのコマンドはゲームを放棄するものだったが、
その機能は`{\tt \#quit}'に移動した。)
%%.lp
\item[\tb{r}]
%Read a scroll or spellbook.
巻物や魔法書を読む。
%%.lp
\item[\tb{R}]
%Remove a worn accessory (ring, amulet, or blindfold).\\
装飾品(指輪、魔除け、目隠し)をはずす。\\
%%.lp ""
%If you're wearing more than one, you'll be prompted for which one to
%remove.  When you're only wearing one, then by default it will be removed
%without asking, but you can set the
%{\it paranoid\verb+_+confirmation\/}
%option to require a prompt.\\
複数の物を身につけている場合、どれを外すかを指定する。
一つしか身につけていない場合、標準設定では確認なしにその一つを外すが、
確認が必要になるように
{\it paranoid\verb+_+confirmation\/}
オプションをセットできる。\\
%%.lp ""
%This command may also be used to take off armor.  The prompt for which
%inventory item to remove only lists worn accessories, but an item of
%worn armor can be chosen.
%(See the `{\tt T}' command below.  It lists armor as the inventory
%choices but will accept an accessory and attempt to remove it.)
このコマンドは防具を外すのにも使える。
外す物を選ぶプロンプトには身につけている装飾品のみが表示されるが、
身につけている防具も選ぶことが出来る。
(後述の `{\tt T}' コマンドを参照のこと。
これは持ち物の選択時に防具が表示できるが、装飾品も受け付けて外そうとする。)
%%.lp
\item[\tb{\^{}R}]
%Redraw the screen.
画面を描き直す。
%%.lp
\item[\tb{s}]
%Search for secret doors and traps around you.
%It usually takes several tries to find something.\\
周囲の隠し扉や罠を探す。何かを見つけるには普通何回も探す必要がある。\\
%%.lp ""
%Can also be used to figure out whether there is still a monster at
%an adjacent ``remembered, unseen monster'' marker.
隣接する「覚えているが見えていない怪物」マーカーの怪物が
まだそこにいるかを確認するためにも使える。
%%.lp
\item[\tb{S}]
%Save the game (which suspends play and exits the program).
%The saved game will be restored automatically the next time you play
%using the same character name.\\
ゲームをセーブする(プレイを中断してプログラムを終了する)。
セーブされたゲームは、同じキャラクタ名を使った次回プレイ時に
自動的に復旧される。\\
%%.lp ""
%In normal play, once a saved game is restored the file used to hold
%the saved data is deleted.
%In explore mode, once restoration is accomplished you are asked whether
%to keep or delete the file.
%Keeping the file makes it feasible to play for a while then quit
%without saving and later restore again.\\
通常のプレイでは、セーブしたゲームが一旦復元されると、
セーブしたデータを保存するのに使われていたファイルは削除される。
探検モードでは、復元が完了した後ファイルを残すか削除するかを
訊ねられる。
ファイルを残すと、しばらくプレイした後セーブせずに終了し、
後で再び復元することが可能になる。\\
%%.lp ""
%There is no ``save current game state and keep playing'' command, not
%even in explore mode where saved game files can be kept and re-used.
例えセーブしたゲームファイルを残して再利用できる探検モードでも、
``現在のゲームの状態をセーブしてプレイを続ける'' コマンドはない。
%%.lp
\item[\tb{t}]
%Throw an object or shoot a projectile.\\
物を投げる。または矢などを発射する。\\
%%.lp ""
%There's no separate ``shoot'' command.
%If you ``throw'' an arrow while wielding a bow, you are shooting
%that arrow and any weapon skill bonus or penalty for bow applies.
%If you ``throw'' an arrow while not wielding a bow, you are throwing
%it by hand and it will generally be less effective than when shot.\\
独立した ``射撃'' コマンドはない。
弓を装備しているときに矢を ``投げる'' と、矢を撃つことになり、
弓に関する武器スキルのボーナスやペナルティが適用される。
弓を装備せずに矢を ``投げる'' と、手で矢を投げたことになり、
一般的に撃つよりも効率的でない。\\
%%.lp ""
%See also `{\tt f}' (fire) for throwing or shooting an item pre-selected
%via the `{\tt Q}' (quiver) command.
`{\tt Q}' (矢筒) コマンドで事前に選択された物を投げたり撃ったりする
`{\tt f}' (射撃) も参照。
%%.lp
\item[\tb{T}]
%Take off armor.\\
防具をはずす。\\
%%.lp ""
%If you're wearing more than one piece, you'll be prompted for which
%one to take off.  (Note that this treats a cloak covering a suit
%and/or a shirt, or a suit covering a shirt, as if the underlying items
%weren't there.)
%When you're only wearing one, then by default it will
%be taken off without asking, but you can set the
%{\it paranoid\verb+_+confirmation\/}
%option to require a prompt.\\
複数の物を身につけている場合は、どれを外すかのプロンプトが出る。
(例え実際に身につけていなくても、クロークは防具やシャツの上に身につけ、
防具はシャツの上に身につけているものとして扱われる。)
一つしか身につけている場合は、デフォルトでは質問されることなくそれをはずすが、
{\it paranoid\verb+_+confirmation\/}
オプションを設定することでプロンプトを要求できるようにできる。\\
%%.lp ""
%This command may also be used to remove accessories.  The prompt
%for which inventory item to take off only lists worn armor, but a worn
%accessory can be chosen.
%(See the `{\tt R}' command above.  It lists accessories as the inventory
%choices but will accept an item of armor and attempt to take it off.)
このコマンドはアクセサリをはずすのにも使える。
どれを外すかを選ぶためのプロンプトには身につけている防具のみが表示されるが、
身につけているアクセサリも選ぶことができる。
(上述した `{\tt R}' を参照のこと。
選択肢としてはアクセサリのみが表示されるが防具も選ぶことができ、それを
外そうとする。)
%%.lp
\item[\tb{\^{}T}]
%Teleport, if you have the ability.
テレポート能力があればテレポートする。
%%.lp
\item[\tb{v}]
%Display version number.
バージョン番号を表示する。
%%.lp
\item[\tb{V}]
%Display the game history.
このゲームの履歴を表示する。
%%.lp
\item[\tb{w}]
%Wield weapon.\\
武器を持つ。\\
%%.sd
%%.si
%{\tt w-} --- wield nothing, use your bare (or gloved) hands.\\
{\tt w-} --- 武器として何も持たず素手(またはグローブを付けた手)になる。\\
%%.ei
%%.ed
%Some characters can wield two weapons at once; use the `{\tt X}' command
%(or the ``{\tt \#twoweapon}'' extended command) to do so.
一部のキャラクタは二つの武器を同時に持つことが出来る;
そうするには `{\tt X}' コマンド
(または ``{\tt \#twoweapon}'' 拡張コマンド) を使う。
%%.lp
\item[\tb{W}]
%Wear armor.\\
鎧を装備する。\\
%%.lp ""
%This command may also be used to put on an accessory (ring, amulet, or
%blindfold).  The prompt for which inventory item to use will only list
%armor, but choosing an unlisted accessory will attempt to put it on.
%(See the `{\tt P}' command above.  It lists accessories as the inventory
%choices but will accept an item of armor and attempt to wear it.)
このコマンドはアクセサリ(指輪、魔除け、目隠し)を身につけるのにも
使われる。
どれを選ぶかの一覧では防具しか表示されないが、一覧に表示されていない
アクセサリを選択することでそれを身につけようとすることができる。
どれを身につけるかを選ぶためのプロンプトには防具のみが表示されるが、
身につけるためのアクセサリも選ぶことができる。
(前述した `{\tt P}' コマンドを参照のこと。
選択肢としてはアクセサリのみが表示されるが防具も選ぶことができ、それを
身につけようとする。)
%%.lp
\item[\tb{x}]
%Exchange your wielded weapon with the item in your alternate weapon slot.\\
装備している武器を予備の武器と交換する。\\
%%.lp ""
%The latter is used as your secondary weapon when engaging in
%two-weapon combat.  Note that if one of these slots is empty,
%the exchange still takes place.
予備の武器は二刀流攻撃で用いられる。
予備の武器がなくても交換は実行される(今持っているものを予備にして、
素手になる)ことに注意。
%%.lp
\item[\tb{X}]
%Toggle two-weapon combat, if your character can do it.  Also available
%via the ``{\tt \#twoweapon}'' extended command.\\
キャラクタが実行可能なら、二刀流モードを切り替える。
``{\tt \#twoweapon}'' 拡張コマンドも利用可能である。\\
%%.lp ""
%(In versions prior to 3.6 this was the command to switch from normal
%play to ``explore mode'', also known as ``discovery mode'', which has now
%been moved to ``{\tt \#exploremode}''.)
(3.6 より前のバージョンではこれは通常のプレイから
「探索モード」またの名を「発見モード」に切り替えるコマンドであった;
これは ``{\tt \#exploremode}'' に移動した。)
%%.lp
\item[\tb{\^{}X}]
%Display basic information about your character.\\
あなたのキャラクタの基本的な情報を表示する。\\
%%.lp ""
%Displays name, role, race, gender (unless role name makes that
%redundant, such as {\tt Caveman} or {\tt Priestess}), and alignment,
%along with your patron deity and his or her opposition.  It also
%shows most of the various items of information from the status line(s)
%in a less terse form, including several additional things which don't
%appear in the normal status display due to space considerations.\\
名前、職業、(職業名によって冗長にならないなら)性別、属性、従っている神および
敵対している神を表示する。
また、ステータス行の内容を、スペースの問題で通常のステータス表示には現れない
追加のものも含めて、より詳細に表示する。\\
%%.lp ""
%In normal play, that's all that `{\tt \^{}X}' displays.
%In explore mode, the role and status feedback is augmented by the
%information provided by {\it enlightenment\/} magic.
通常のプレイでは、これが `{\tt \^{}X}' が表示する全てである。
探検モードでは、職業と状態は {\it enlightenment\/} (啓蒙)魔法で
提供される情報によって強化される。
%%.lp
\item[\tb{z}]
%Zap a wand.\\
杖を振る。\\
%%.sd
%%.si
%{\tt z.} --- to aim at yourself, use `{\tt .}' for the direction.
{\tt z.} --- 自分に振る場合は、方向に`{\tt .}'を使う。
%%.ei
%%.ed
%%.lp
\item[\tb{Z}]
%Zap (cast) a spell.\\
呪文を唱える。\\
%%.sd
%%.si
%{\tt Z.} --- to cast at yourself, use `{\tt .}' for the direction.
{\tt Z.} --- 自分を狙いたいときは、方向に`{\tt .}'を使う。
%%.ei
%%.ed
%%.lp
\item[\tb{\^{}Z}]
%Suspend the game (UNIX versions with job control only).
ゲームを一時中止する(ジョブコントロール機能のある UNIX バージョンのみ)。
%%.lp
\item[\tb{:}]
%Look at what is here.
足元に何があるか見る。
%%.lp
\item[\tb{;}]
%Show what type of thing a visible symbol corresponds to.
見えているシンボルが何を示すかを表示する。
%%.lp
\item[\tb{,}]
%Pick up some things from the floor beneath you.\\
足下にあるものを拾う。\\
%%.lp ""
%May be preceded by `{\tt m}' to force a selection menu.
`{\tt m}'の後に押すことによって、設定に関わらず選択メニューを表示する。
%%.lp
\item[\tb{@}]
%Toggle the {\it autopickup\/} option on and off.
{\it autopickup\/} オプションのオン・オフを切り替える。
%%.lp
\item[\tb{\^{}}]
%Ask for the type of an adjacent trap you found earlier.
今までに発見した隣接した罠の種類を調べる。
%%.lp
\item[\tb{)}]
%Tell what weapon you are wielding.
持っている武器を表示する。
%%.lp
\item[\tb{[}]
%Tell what armor you are wearing.
付けている防具を表示する。
%%.lp
\item[\tb{=}]
%Tell what rings you are wearing.
はめている指輪を表示する。
%%.lp
\item[\tb{"}]
%Tell what amulet you are wearing.
付けている魔除けを表示する。
%%.lp
\item[\tb{(}]
%Tell what tools you are using.
使っている道具を表示する。
%%.lp
\item[\tb{*}]
%Tell what equipment you are using.\\
装備しているものを表示する。\\
%%.lp ""
%Combines the preceding five type-specific
%commands into one.
上の 5 つのコマンドを一つにしたもの。
%%.lp
\item[\tb{\$}]
%Count your gold pieces.
持っている金貨を数える。
%%.lp
\item[\tb{+}]
%List the spells you know.\\
知っている呪文の一覧を表示する。\\
%%.lp ""
%Using this command, you can also rearrange
%the order in which your spells are listed, either by sorting the entire
%list or by picking one spell from the menu then picking another to swap
%places with it.  Swapping pairs of spells changes their casting letters,
%so the change lasts after the current `{\tt +}' command finishes.  Sorting
%the whole list is temporary.  To make the most recent sort order persist
%beyond the current `{\tt +}' command, choose the sort option again and then
%pick ``reassign casting letters''.  (Any spells learned after that will
%be added to the end of the list rather than be inserted into the sorted
%ordering.)
このコマンドを使うと、リスト全体をソートするか、メニューから一つの呪文を
選択して他の呪文と入れ替えることで、呪文が表示される順番を
入れ替えることが出来る。
呪文の組を入れ替えると詠唱の文字が変わるので、変更は今回の
`{\tt +}' コマンドが終了した後もそのままになる。
リスト全体のソートは一時的なものである。
今回の `{\tt +}' コマンド以降も最近のソート順のままにするには、
ソートオプションを再び選んで、``reassign casting letters'' を行う。
(これを行った後に覚えた呪文は、ソート順に挿入されるのではなく、
リストの末尾に追加される。)
%%.lp
\item[\tb{$\backslash$}]
%Show what types of objects have been discovered.
今までにどんな種類の物を見つけたかを表示する。
%%.lp
\item[\tb{\`}]
%Show discovered types for one class of objects.
一つの種類の物の中で見つけた種類を表示する。
%%.lp
\item[\tb{!}]
%Escape to a shell.
シェルに抜ける。
%%.lp
\item[\tb{\#}]
%Perform an extended command.\\
拡張コマンドを実行する。\\
%%.lp ""
%As you can see, the authors of {\it NetHack\/}
%used up all the letters, so this is a way to introduce the less frequently
%used commands.
%What extended commands are available depends on what features
%the game was compiled with.
以上から分かるように {\it NetHack\/} の作者たちはすべての文字を
使い果たしてしまったので、あまり頻繁に使われないコマンドはこのようにして
導入された。
どの拡張コマンドが使用可能かは、
ゲームのコンパイル時にどの機能が有効にされたかによる。
%%.lp
\item[\tb{\#adjust}]
%Adjust inventory letters (most useful when the
%{\it fixinv\/}
%option is ``on''). Autocompletes. Default key is `{\tt M-a}'.\\
持ち物の目録記号を変更する(
{\it fixinv\/}
オプションが``on''の時に非常に便利である)。
補完される。
標準キーは `{\tt M-a}'。\\
%%.lp ""
%This command allows you to move an item from one particular inventory
%slot to another so that it has a letter which is more meaningful for you
%or that it will appear in a particular location when inventory listings
%are displayed.
%You can move to a currently empty slot, or if the destination is
%occupied---and won't merge---the item there will swap slots with the one
%being moved.
%``{\tt \#adjust}'' can also be used to split a stack of objects; when
%choosing the item to adjust, enter a count prior to its letter.\\
このコマンドは、ある特定の持ち物スロットから他のスロットに
変更できる; これによりあなたにとってより意味のある文字にしたり、
一覧表示時に特定の位置に現れるようにしたりできる。
現在空いているスロットに移動させることもできるし、
変更先が既に使われている場合---結合できないなら--そこのいる物は
位置が入れ替わる。
``{\tt \#adjust}'' はまた、物の塊を分割するのにも使える;
調整する物を選択するときに、文字の前に数を入力する。
%%.lp ""
%Adjusting without a count used to collect all compatible stacks when
%moving to the destination.  That behavior has been changed; to gather
%compatible stacks, ``{\tt \#adjust}'' a stack into its own inventory slot.
%If it has a name assigned, other stacks with the same name or with
%no name will merge provided that all their other attributes match.
%If it does not have a name, only other stacks with no name are eligible.
%In either case, otherwise compatible stacks with a different name
%will not be merged.  This contrasts with using ``{\tt \#adjust}'' to move
%from one slot to a different slot.  In that situation, moving (no
%count given) a compatible stack will merge if either stack has a
%name when the other doesn't and give that name to the result, while
%splitting (count given) will ignore the source stack's name when
%deciding whether to merge with the destination stack.
数を指定せずに調整する場合、以前は移動先に移すときに全ての互換性のある
まとまりは集められていた。
この振る舞いは変更された;
互換性のあるまとまりを集めるには、まとまりを新しいスロットに
``{\tt \#adjust}'' する。
名前が割り当てられている場合、その他の属性が同じなら
同じ名前か名前のないまとまりが集められる。
名前がない場合、その他の名前のないまとまりのみが有効である。
どちらの場合も、異なる名前のまとまりは集められない。
これはあるスロットから他のスロットに移すために
``{\tt \#adjust}'' を使った場合と対照的である。
この場合、(数が指定されず)互換性のあるまとまりを移動する場合には、
どちらかに名前がついていてもう片方についていないなら、
これは集められ、その名前が付けられる。
一方、(数が指定されて)分割される場合、移動先とまとまりを集めるかどうかを
決めるときに元のまとまりの名前は無視される。
%%.lp
\item[\tb{\#annotate}]
%Allows you to specify one line of text to associate with the current
%dungeon level.  All levels with annotations are displayed by the
%``{\tt \#overview}'' command. Autocompletes.
%Default key is `{\tt M-A}',
%and also `{\tt \^{}N}' if {\it number\verb+_+pad\/} is on.
現在の階層に結びつけられる 1 行の文を指定できる。
注釈が付けられた全ての階層は
``{\tt \#overview}'' コマンドで表示される。
補完される。
標準キーは `{\tt M-A}' と、
{\it number\verb+_+pad\/} がオンなら `{\tt \^{}N}'。
%%.lp
\item[\tb{\#apply}]
%Apply (use) a tool such as a pick-axe, a key, or a lamp.
%Default key is `{\tt a}'.\\
つるはし、鍵、ランプのような道具を用いる(使う)。
標準キーは `{\tt a}'。\\
%%.lp ""
%If the tool used acts on items on the floor, using the `{\tt m}' prefix
%skips those items.\\
道具が床にある物に使われる場合、`{\tt m}' 接頭辞を使うと
それらの物は飛ばす。\\
%%.lp ""
%If used on a wand, that wand will be broken, releasing its magic in the
%process.  Confirmation is required.
杖に使うと、杖は壊れ、その過程で魔力が解放される。
確認が必要。
%%.lp
\item[\tb{\#attributes}]
%Show your attributes. Default key is `{\tt \^{}X}'.
属性を表示する。
標準キーは `{\tt \^{}X}'。
%%.lp
\item[\tb{\#autopickup}]
%Toggle the {\it autopickup\/} option. Default key is `{\tt @}'.
{\it autopickup\/} オプションを切り替える。
標準キーは `{\tt @}'。
%%.lp
\item[\tb{\#call}]
%Call (name) a monster, or an object in inventory, on the floor,
%or in the discoveries list, or add an annotation for the
%current level (same as ``{\tt \#annotate}''). Default key is `{\tt C}'.
怪物や持ち物や床や発見物一覧にある物に名前を付けたり、
現在の階に注釈を付ける (``{\tt \#annotate}'' と同じ)。
標準キーは `{\tt C}'。
%%.lp
\item[\tb{\#cast}]
%Cast a spell. Default key is `{\tt Z}'.
扉を閉じる。
標準キーは `{\tt Z}'。
%%.lp
\item[\tb{\#chat}]
%Talk to someone. Default key is `{\tt M-c}'.
誰かと話をする。
標準キーは`{\tt M-c}'。
%%.lp
\item[\tb{\#close}]
%Close a door. Default key is `{\tt c}'.
扉を閉じる。
標準キーは `{\tt c}'。
%%.lp
\item[\tb{\#conduct}]
%List voluntary challenges you have maintained. Autocompletes.
%Default key is `{\tt M-C}'.\\
維持している自発的挑戦の一覧を表示する。
補完される。
標準キーは `{\tt M-C}'。
%%.lp ""
%See the section below entitled ``Conduct'' for details.
詳しくは後述する``Conduct''の章を参照すること。
%%.lp
\item[\tb{\#dip}]
%Dip an object into something. Autocompletes. Default key is `{\tt M-d}'.
物を何かに浸す。
補完される。
標準キーは `{\tt M-d}'。
%%.lp
\item[\tb{\#down}]
%Go down a staircase. Default key is `{\tt >}'.
階段を降りる。
標準キーは `{\tt >}'。
%%.lp
\item[\tb{\#drop}]
%Drop an item. Default key is `{\tt d}'.
物を下に置く。
標準キーは `{\tt d}'。
%%.lp
\item[\tb{\#droptype}]
%Drop specific item types. Default key is `{\tt D}'.
特定の酒類の物を下に置く。
標準キーは `{\tt D}'。
%%.lp
\item[\tb{\#eat}]
%Eat something. Default key is `{\tt e}'.
%The `{\tt m}' prefix skips eating items on the floor.
何かを食べる。
標準キーは `{\tt e}'。
`{\tt m}' 接頭辞は床にある物を食べるのを飛ばす。
%%.lp
\item[\tb{\#engrave}]
%Engrave writing on the floor. Default key is `{\tt E}'.
床に何かを刻み込む。
標準キーは `{\tt E}'。
%%.lp
\item[\tb{\#enhance}]
%Advance or check weapon and spell skills. Autocompletes.
%Default key is `{\tt M-e}'.
武器の技量を高めたり、調べたりする。
補完される。
標準キーは `{\tt M-e}'。
%%.lp
\item[\tb{\#exploremode}]
%Enter the explore mode.\\
探検モードに入る。\\
%%.lp ""
%Requires confirmation; default response is `{\tt n}' (no).
%To really switch to explore mode, respond with `{\tt y}'.
%You can set the
%{\it paranoid\verb+_+confirmation:quit\/}
%option to require a response of ``{\tt yes}'' instead.
確認が必要; 標準の返答は`{\tt n}' (ノー)。
本当に探検モードに切り替えるなら、`{\tt y}' と返答する。
代わりに返答に ``{\tt yes}'' を必要とするように
{\it paranoid\verb+_+confirmation:quit\/} オプションを設定できる。
%%.lp
\item[\tb{\#fire}]
%Fire ammunition from quiver. Default key is `{\tt f}'.
矢筒から弾を発射する。
標準キーは `{\tt f}'。
%%.lp
\item[\tb{\#force}]
%Force a lock. Autocompletes. Default key is `{\tt M-f}'.
錠をこじ開ける。
補完される。
標準キーは `{\tt M-f}'。
%%.lp
\item[\tb{\#glance}]
%Show what type of thing a map symbol corresponds to. Default key is `{\tt ;}'.
地図のシンボルが何を示すかを表示する。
標準キーは `{\tt ;}'。
%%.lp
\item[\tb{\#help}]
%Show the help menu.
%Default key is `{\tt ?}',
%and also `{\tt h}' if {\it number\verb+_+pad\/} is on.
ヘルプメニューを表示する。
標準キーは `{\tt ?}' および
{\it number\verb+_+pad\/} がオンの場合 `{\tt h}'。
%%.lp
\item[\tb{\#herecmdmenu}]
%Show a menu of possible actions in your current location.
現在の位置で可能な行動のメニューを表示する。
%%.lp
\item[\tb{\#history}]
%Show long version and game history. Default key is `{\tt V}'.
長いバージョンとゲームの歴史を表示する。
標準キーは `{\tt V}'。
%%.lp
\item[\tb{\#inventory}]
%Show your inventory. Default key is `{\tt i}'.
持ち物の目録を表示する。
標準キーは `{\tt i}'。
%%.lp
\item[\tb{\#inventtype}]
%Inventory specific item types. Default key is `{\tt I}'.
指定した物の種類の一覧を表示する。
標準キーは `{\tt I}'。
%%.lp
\item[\tb{\#invoke}]
%Invoke an object's special powers. Autocompletes. Default key is `{\tt M-i}'.
物が持つ特別な能力を発動する。
補完される。
標準キーは `{\tt M-i}'。
%%.lp
\item[\tb{\#jump}]
%Jump to another location. Autocompletes.
%Default key is `{\tt M-j}',
%and also `{\tt j}' if {\it number\verb+_+pad\/} is on.
別の場所へジャンプする。
補完される。
標準キーは `{\tt M-j}' および
{\it number\verb+_+pad\/} がオンの場合 `{\tt j}'。
%%.lp
\item[\tb{\#kick}]
%Kick something.
%Default key is `{\tt \^{}D}',
%and also `{\tt k}' if {\it number\verb+_+pad\/} is on.
何かを蹴る。
標準キーは `{\tt \^{}D}' および
{\it number\verb+_+pad\/} がオンの場合 `{\tt k}'。
%%.lp
\item[\tb{\#known}]
%Show what object types have been discovered.
%Default key is `{\tt $\backslash$}'.
今までにどんな種類の物を見つけたかを表示する。
標準キーは `{\tt $\backslash$}'。
%%.lp
\item[\tb{\#knownclass}]
%Show discovered types for one class of objects.
%Default key is `{\tt `}'.
一つの種類の物の中で見つけた種類を表示する。
標準キーは `{\tt `}'。
%%.lp
\item[\tb{\#levelchange}]
%Change your experience level.
%Autocompletes.
%Debug mode only.
経験レベルを変更する。
補完される。
デバッグモードのみ。
%%.lp
\item[\tb{\#lightsources}]
%Show mobile light sources.
%Autocompletes.
%Debug mode only.
携帯光源を表示する。
補完される。
デバッグモードのみ。
%%.lp
\item[\tb{\#look}]
%Look at what is here, under you. Default key is `{\tt :}'.
足元に何があるか見る。
標準キーは `{\tt :}'。
%%.lp
\item[\tb{\#loot}]
%Loot a box or bag on the floor beneath you, or the saddle
%from a steed standing next to you. Autocompletes.
%Precede with the `{\tt m}' prefix to skip containers at your location
%and go directly to removing a saddle.
%Default key is `{\tt M-l}',
%and also `{\tt l}' if {\it number\verb+_+pad\/} is on.
あなたの足下の床に置いてある箱や鞄、またはあなたの隣に立っている馬などの
鞍を漁る。
補完される。
`{\tt m}' を前置すると現在の場所にある入れ物を飛ばして
直接鞍を外す。
標準キーは `{\tt M-l}' および
{\it number\verb+_+pad\/} がオンなら `{\tt l}'。
%%.lp
\item[\tb{\#monster}]
%Use a monster's special ability (when polymorphed into monster form).
%Autocompletes. Default key is `{\tt M-m}'.
(怪物の姿に変化している時に)怪物の特殊能力を使う。
補完される。
標準キーは `{\tt M-m}'。
%%.lp
\item[\tb{\#name}]
%Name a monster, an individual object, or a type of object.
%Same as ``{\tt \#call}''.
%Autocompletes.
%Default keys are `{\tt N}', `{\tt M-n}', and `{\tt M-N}'.
怪物、個々のオブジェクト、オブジェクトの種類に名前を付ける。
``{\tt \#call}''と同じ。
補完される。
標準キーは `{\tt N}', `{\tt M-n}', `{\tt M-N}'。
%%.lp
\item[\tb{\#offer}]
%Offer a sacrifice to the gods. Autocompletes. Default key is `{\tt M-o}'.\\
神にいけにえを捧げる。
補完される。
標準キーは `{\tt M-o}' 。\\
%%.lp ""
%You'll need to find an altar to have any chance at success.
%Corpses of recently killed monsters are the fodder of choice.
成功するためには祭壇を見つける必要がある。
最近殺したモンスターの死体がよい選択肢である。
%%.lp ""
%The `{\tt m}' prefix skips offering any items which are on the altar.\\
`{\tt m}' 接頭辞は、床にある物を捧げるのを飛ばす。\\
%%.lp
\item[\tb{\#open}]
%Open a door. Default key is `{\tt o}'.
扉を開ける。
標準キーは `{\tt o}'。
%%.lp
\item[\tb{\#options}]
%Show and change option settings. Default key is `{\tt O}'.
オプション設定を確認、変更する。
標準キーは `{\tt O}'。
%%.lp
\item[\tb{\#overview}]
%Display information you've discovered about the dungeon.  Any visited
%level (unless forgotten due to amnesia) with an annotation is included,
%and many things (altars, thrones, fountains, and so on; extra stairs
%leading to another dungeon branch) trigger an automatic annotation.
%If dungeon overview is chosen during end-of-game disclosure, every visited
%level will be included regardless of annotations. Autocompletes.
%Default keys are `{\tt \^{}O}', and `{\tt M-O}'.
洞窟について発見した情報を表示する。
(記憶喪失で忘れない限り)全ての訪問済みの階と注釈が含まれ、
多くのもの(祭壇、玉座、泉など; 他の洞窟へ分岐する追加の階段)があると
自動的に注釈が追加される。
ゲーム終了時の情報公開中に洞窟の概要が選ばれた場合、注釈のあるなしに関わらず
全ての訪れた階層が含まれる。
補完される。
標準キーは `{\tt \^{}O}' と `{\tt M-O}'。
%% DON'T PANIC!
%%.lp
\item[\tb{\#panic}]
%Test the panic routine.
%Terminates the current game.
%Autocompletes.
%Debug mode only.\\
パニックルーチンをテストする。
現在のゲームを終了する。
補完される。
デバッグモードのみ。\\
%%.lp ""
%Asks for confirmation; default is `{\tt n}' (no); continue playing.
%To really panic, respond with `{\tt y}'.
%You can set the
%{\it paranoid\verb+_+confirmation:quit\/}
%option to require a response of ``{\tt yes}'' instead.
確認される; 標準設定は `{\tt n}' (ノー); プレイを続ける。
本当にパニックさせるには、`{\tt y}' と返答すること。
代わりに  ``{\tt yes}'' との返答が必要にするために
{\it paranoid\verb+_+confirmation:quit\/}
オプションを設定できる。
%%.lp
\item[\tb{\#pay}]
%Pay your shopping bill. Default key is `{\tt p}'.
代金を支払う。
標準キーは `{\tt p}'。
%%.lp
\item[\tb{\#pickup}]
%Pick up things at the current location. Default key is `{\tt ,}'.
%The `{\tt m}' prefix forces use of a menu.
現在の位置にある物を拾う。
標準キーは `{\tt ,}'。
`{\tt m}' 接頭辞はメニューの使用を強制する。
%%.lp
\item[\tb{\#polyself}]
%Polymorph self.
%Autocompletes.
%Debug mode only.
自身を変化させる。
補完される。
デバッグモードのみ。
%%.lp
\item[\tb{\#pray}]
%Pray to the gods for help. Autocompletes. Default key is `{\tt M-p}'.\\
神に祈って助けを求める。
補完される。
標準キーは `{\tt M-p}' 。\\
%%.lp ""
%Praying too soon after receiving prior help is a bad idea.
%(Hint: entering the dungeon alive is treated as having received help.
%You probably shouldn't start off a new game by praying right away.)
%Since using this command by accident can cause trouble, there is an
%option to make you confirm your intent before praying.  It is enabled
%by default, and you can reset the
%{\it paranoid\verb+_+confirmation\/}
%option to disable it.
前回の助けの後すぐに祈るのは良くない考えである。
(ヒント: 洞窟に生きて入るということは助けを受けたものとして扱われる。
新しいゲームを始めてすぐに祈るべきではないだろう。)
このコマンドを間違って使うのは問題を引き起こすので、
祈る前に意思を確認するようにするオプションがある。
これは標準設定ではオンになっており、オフにするために
{\it paranoid\verb+_+confirmation\/} をリセットすることが出来る。
%%.lp
\item[\tb{\#prevmsg}]
%Show previously displayed game messages. Default key is `{\tt \^{}P}'.
一つ前に表示されたゲームメッセージを表示する。
標準キーは `{\tt \^{}P}'。
%%.lp
\item[\tb{\#puton}]
%Put on an accessory (ring, amulet, etc). Default key is `{\tt P}'.
装飾品(指輪、魔除けなど)を身につける。
標準キーは `{\tt P}'。
%%.lp
\item[\tb{\#quaff}]
%Quaff (drink) something. Default key is `{\tt q}'.
何かを飲む。標準キーは `{\tt q}'。
%%.lp
\item[\tb{\#quit}]
%Quit the program without saving your game. Autocompletes.
%Default key is `{\tt M-q}'.\\
ゲームをセーブせずにプログラムを終了する。
補完される。
標準キーは `{\tt M-q}'。\\
%%.lp ""
%Since using this command by accident would throw away the current game,
%you are asked to confirm your intent before quitting.
%Default response is `{\tt n}' (no); continue playing.
%To really quit, respond with `{\tt y}'.
%You can set the
%{\it paranoid\verb+_+confirmation:quit\/}
%option to require a response of ``{\tt yes}'' instead.
このコマンドを間違って使うと現在のゲームを捨ててしまうことになるので、
終了する前に意思を確認される。
標準の返答は `{\tt n}' (ノー); プレイを続ける。
本当に終了するためには、`{\tt y}' と応答する。
{\it paranoid\verb+_+confirmation:quit\/} オプションをオンにすることで、
代わりに ``{\tt yes}'' の入力が必要であるようにすることができる。
%%.lp
\item[\tb{\#quiver}]
%Select ammunition for quiver. Default key is `{\tt Q}'.
矢筒のための弾を選択する。
標準キーは `{\tt Q}'。
%%.lp
\item[\tb{\#read}]
%Read a scroll, a spellbook, or something else. Default key is `{\tt r}'.
巻物、魔法書、あるいはその他の何かを読む。
標準キーは `{\tt r}'。
%%.lp
\item[\tb{\#redraw}]
%Redraw the screen.
%Default key is `{\tt \^{}R}',
%and also `{\tt \^{}L}' if {\it number\verb+_+pad\/} is on.
画面を書き直す。
標準キーは `{\tt \^{}R}' および
{\it number\verb+_+pad\/} がオンなら `{\tt \^{}L}'。
%%.lp
\item[\tb{\#remove}]
%Remove an accessory (ring, amulet, etc). Default key is `{\tt R}'.
装飾品 (指輪、魔除けなど) をはずす。
標準キーは `{\tt R}'。
%%.lp
\item[\tb{\#ride}]
%Ride (or stop riding) a saddled creature. Autocompletes.
%Default key is `{\tt M-R}'.
鞍の付いている生物に乗る(あるいは乗るのをやめる)。
補完される。
標準キーは `{\tt M-R}'。
%%.lp
\item[\tb{\#rub}]
%Rub a lamp or a stone. Autocompletes. Default key is `{\tt M-r}'.
ランプや石をこする。
補完される。
標準キーは `{\tt M-r}'。
%%.lp
\item[\tb{\#save}]
%Save the game and exit the program.
%Default key is `{\tt S}'.
ゲームをセーブしてプログラムを終了する。
標準キーは `{\tt S}'。
%%.lp
\item[\tb{\#search}]
%Search for traps and secret doors around you. Default key is `{\tt s}'.
周囲の隠し扉や罠を探す。
標準キーは `{\tt s}'。
%%.lp
\item[\tb{\#seeall}]
%Show all equipment in use. Default key is `{\tt *}'.
装備している物を表示する。
標準キーは `{\tt *}'。
%%.lp
\item[\tb{\#seeamulet}]
%Show the amulet currently worn. Default key is `{\tt "}'.
付けている魔除けを表示する。
標準キーは `{\tt "}'。
%%.lp
\item[\tb{\#seearmor}]
%Show the armor currently worn. Default key is `{\tt [}'.
付けている防具を表示する。
標準キーは `{\tt [}'。
%%.lp
\item[\tb{\#seegold}]
%Count your gold. Default key is `{\tt \$}'.
持っている金貨を数える。
標準キーは `{\tt \$}'。
%%.lp
\item[\tb{\#seenv}]
%Show seen vectors.
%Autocompletes.
%Debug mode only.
視界ベクターを表示する。
補完される。
デバッグモードのみ。
%%.lp
\item[\tb{\#seerings}]
%Show the ring(s) currently worn. Default key is `{\tt =}'.
はめている指輪を表示する。
標準キーは `{\tt =}'。
%%.lp
\item[\tb{\#seespells}]
%List and reorder known spells. Default key is `{\tt +}'.
知っている呪文の一覧を表示し、並べ替える。
標準キーは `{\tt +}'。
%%.lp
\item[\tb{\#seetools}]
%Show the tools currently in use. Default key is `{\tt (}'.
使っている道具を表示する。
標準キーは `{\tt (}'。
%%.lp
\item[\tb{\#seetrap}]
%Show the type of an adjacent trap. Default key is `{\tt \^{}}'.
隣接した罠の種類を表示する。
標準キーは `{\tt \^{}}'。
%%.lp
\item[\tb{\#seeweapon}]
%Show the weapon currently wielded. Default key is `{\tt )}'.
持っている武器を表示する。
標準キーは  `{\tt )}'。
%%.lp
\item[\tb{\#shell}]
%Do a shell escape. Default key is `{\tt !}'.
シェルに抜ける。
標準キーは `{\tt !}'。
%%.lp
\item[\tb{\#sit}]
%Sit down. Autocompletes. Default key is `{\tt M-s}'.
座る。
補完される。
標準キーは `{\tt M-s}'。
%%.lp
\item[\tb{\#stats}]
%Show memory usage statistics.
%Autocompletes.
%Debug mode only.
メモリ使用状況を表示する。
補完される。
デバッグモードのみ。
%%.lp
\item[\tb{\#suspend}]
%Suspend the game. Default key is `{\tt \^{}Z}'.
ゲームを一時中止する。
標準キーは `{\tt \^{}Z}'。
%%.lp
\item[\tb{\#swap}]
%Swap wielded and secondary weapons. Default key is `{\tt x}'.
装備している武器を予備の武器と交換する。
標準キーは `{\tt x}'。
%%.lp
\item[\tb{\#takeoff}]
%Take off one piece of armor. Default key is `{\tt T}'.
防具を一つ外す。
標準キーは `{\tt T}'。
%%.lp
\item[\tb{\#takeoffall}]
%Remove all armor. Default key is `{\tt A}'.
全ての防具を外す。
標準キーは `{\tt A}'。
%%.lp
\item[\tb{\#teleport}]
%Teleport around the level. Default key is `{\tt \^{}T}'.
その階の中でテレポートする。
標準キーは `{\tt \^{}T}'。
%%.lp
\item[\tb{\#terrain}]
%Show bare map without displaying monsters, objects, or traps.
%Autocompletes.
怪物、物、罠を表示せずに、基の地図を表示する。
補完される。
%%.lp
\item[\tb{\#therecmdmenu}]
%Show a menu of possible actions in a location next to you.
現在の隣の位置に対して可能な行動のメニューを表示する。
%%.lp
\item[\tb{\#throw}]
%Throw something. Default key is `{\tt t}'.
何かを投げる。
標準キーは `{\tt t}'。
%%.lp
\item[\tb{\#timeout}]
%Look at the timeout queue.
%Autocompletes.
%Debug mode only.
タイムアウトキューを見る。
補完される。
デバッグモードのみ。
%%.lp
\item[\tb{\#tip}]
%Tip over a container (bag or box) to pour out its contents.
%Autocompletes. Default key is `{\tt M-T}'.
%The `{\tt m}' prefix makes the command use a menu.
入れ物(鞄や箱)をひっくり返して中の物を出す。
補完される。
標準キーは `{\tt M-T}'。
`{\tt m}' 接頭辞はこのコマンドでメニューを使うようにする。
%%.lp
\item[\tb{\#travel}]
%Travel to a specific location on the map.
%Default key is `{\tt \verb+_+}'.
%Using the ``request menu'' prefix shows a menu of interesting targets in sight
%without asking to move the cursor.
%When picking a target with cursor and the {\it autodescribe\/}
%option is on, the top line will show ``(no travel path)'' if
%your character does not know of a path to that location.
地図の指定された場所まで移動(travel)する。
標準キーは `{\tt \verb+_+}'。
``request menu'' 接頭辞を使うと、カーソル移動を訊ねることなく
視界内の興味深い目標の一覧を表示する。
カーソルで目標を選び、{\it autodescribe\/} オプションがオンなら、
もしあなたのキャラクタがその場所までの道のりを知らない場合には
一番上の行に ``(no travel path)'' と表示される。
%%.lp
\item[\tb{\#turn}]
%Turn undead away. Autocompletes. Default key is `{\tt M-t}'.
不死の怪物を追い払う。
補完される。
標準キーは `{\tt M-t}'。
%%.lp
\item[\tb{\#twoweapon}]
%Toggle two-weapon combat on or off. Autocompletes.
%Default key is `{\tt X}',
%and also `{\tt M-2}' if {\it number\verb+_+pad\/} is off.\\
二刀流戦闘のオン・オフを切り替える。
補完される。
標準キーは `{\tt X}' 、{\it number\verb+_+pad\/} がオフの場合は
`{\tt M-2}' も使える。
%%.lp ""
%Note that you must
%use suitable weapons for this type of combat, or it will
%be automatically turned off.
二刀流に適切な武器を使うこと。
さもなければ自動的にオフになる。
%%.lp
\item[\tb{\#untrap}]
%Untrap something (trap, door, or chest).
%Default key is `{\tt M-u}', and `{\tt u}' if {\it number\verb+_+pad\/} is on.\\
何か(罠、扉、宝箱)の罠をはずす。
標準キーは `{\tt M-u}' と、
{\it number\verb+_+pad\/} がオンなら `{\tt u}'。\\
%%.lp ""
%In some circumstances it can also be used to rescue trapped monsters.
状況によっては罠に掛かった怪物を助けるのにも使えるかも知れない。
%%.lp
\item[\tb{\#up}]
%Go up a staircase. Default key is `{\tt <}'.
階段を上る。
標準キーは `{\tt <}'。
%%.lp
\item[\tb{\#vanquished}]
%List vanquished monsters.
%Autocompletes.
%Debug mode only.
絶滅させた怪物の一覧を表示する。
補完される。
デバッグモードのみ。
%%.lp
\item[\tb{\#version}]
%Print compile time options for this version of {\it NetHack\/}.
%Autocompletes. Default key is `{\tt M-v}'.
このバージョンの {\it NetHack\/} をコンパイルしたときのオプションを表示する。
補完される。
標準キーは `{\tt M-v}'。
%%.lp
\item[\tb{\#versionshort}]
%Show version string. Default key is `{\tt v}'.
バージョン文字列を表示する。
標準キーは `{\tt v}'。
%%.lp
\item[\tb{\#vision}]
%Show vision array.
%Autocompletes.
%Debug mode only.
視界配列を表示する。
補完される。
デバッグモードのみ。
%%.lp
\item[\tb{\#wait}]
%Rest one move while doing nothing.
%Default key is `{\tt .}', and also `{\tt{ }}' if
%{\it rest\verb+_+on\verb+_+space\/} is on.
何もせずに 1 ターン休憩する。
標準キーは `{\tt .}' と、
{\it rest\verb+_+on\verb+_+space\/} がオンなら `{\tt{ }}'。
%%.lp
\item[\tb{\#wear}]
%Wear a piece of armor. Default key is `{\tt W}'.
鎧を一つ装備する。
標準キーは `{\tt W}'。
%%.lp
\item[\tb{\#whatdoes}]
%Tell what a key does. Default key is `{\tt \&}'.
あるキーが何をするかを示す。
標準キーは `{\tt \&}'。
%%.lp
\item[\tb{\#whatis}]
%Show what type of thing a symbol corresponds to. Default key is `{\tt /}'.
シンボルに対応している物が何かを表示する。
標準キーは `{\tt /}'。
%%.lp
\item[\tb{\#wield}]
%Wield a weapon. Default key is `{\tt w}'.
武器を装備する。
標準キーは `{\tt w}'。
%%.lp
\item[\tb{\#wipe}]
%Wipe off your face. Autocompletes. Default key is `{\tt M-w}'.
顔を拭う。
補完される。
標準キーは `{\tt M-w}'。
%%.lp
\item[\tb{\#wizbury}]
%Bury objects under and around you.
%Autocompletes.
%Debug mode only.
あなたの下と周りに物を埋める。
補完される。
デバッグモードのみ。
%%.lp
\item[\tb{\#wizdetect}]
%Search for hidden things (secret doors or traps or unseen monsters)
%within a modest radius.
%Autocompletes.
%Debug mode only.
%Default key is `{\tt \^{}E}'.
ある程度の範囲内に隠されたもの(隠し扉や罠や見えない怪物)を探す。
補完される。
デバッグモードのみ。
標準キーは`{\tt \^{}E}'.
%%.lp
\item[\tb{\#wizgenesis}]
%Create a monster.
%May be prefixed by a count to create more than one.
%Autocompletes.
%Debug mode only.
%Default key is `{\tt \^{}G}'.
怪物を造る。
回数を前置することで複数造ることができる。
補完される。
デバッグモードのみ。
標準キーは`{\tt \^{}G}'.
%%.lp
\item[\tb{\#wizidentify}]
%Identify all items in inventory.
%Autocompletes.
%Debug mode only.
%Default key is `{\tt \^{}I}'.
全ての持ち物を識別する。
補完される。
デバッグモードのみ。
標準キーは `{\tt \^{}I}'。
%%.lp
\item[\tb{\#wizintrinsic}]
%Set one or more intrinsic attributes.
%Autocompletes.
%Debug mode only.
いくつかの特性を設定する。
補完される。
デバッグモードのみ。
%%.lp
\item[\tb{\#wizlevelport}]
%Teleport to another level.
%Autocompletes.
%Debug mode only.
%Default key is `{\tt \^{}V}'.
他の階に瞬間移動する。
補完される。
デバッグモードのみ。
標準キーは `{\tt \^{}V}'。
%%.lp
\item[\tb{\#wizmap}]
%Map the level.
%Autocompletes.
%Debug mode only.
%Default key is `{\tt \^{}F}'.
この階をマッピングする。
補完される。
デバッグモードのみ。
標準キーは `{\tt \^{}F}'。
%%.lp
\item[\tb{\#wizrumorcheck}]
%Verify rumor boundaries.
%Autocompletes.
%Debug mode only.
噂の境界を検証する。
補完される。
デバッグモードのみ。
%%.lp
\item[\tb{\#wizsmell}]
%Smell monster.
%Autocompletes.
%Debug mode only.
怪物の匂いを嗅ぐ。
補完される。
デバッグモードのみ。
%%.lp
\item[\tb{\#wizwhere}]
%Show locations of special levels.
%Autocompletes.
%Debug mode only.
特殊階の場所を表示する。
補完される。
デバッグモードのみ。
%%.lp
\item[\tb{\#wizwish}]
%Wish for something.
%Autocompletes.
%Debug mode only.
%Default key is `{\tt \^{}W}'.
何かを願う。
補完される。
デバッグモードのみ。
標準キーは `{\tt \^{}W}'。
%%.lp
\item[\tb{\#wmode}]
%Show wall modes.
%Autocompletes.
%Debug mode only.
壁のモードを表示する。
補完される。
デバッグモードのみ。
%%.lp
\item[\tb{\#zap}]
%Zap a wand. Default key is `{\tt z}'.
杖を振る。
標準キーは `{\tt z}'。
%%.lp
\item[\tb{\#?}]
%Help menu:  get the list of available extended commands.
ヘルプメニュー: 利用可能な拡張コマンドの一覧を表示する。
\elist

%%.pg
%\nd If your keyboard has a meta key (which, when pressed in combination
%with another key, modifies it by setting the `meta' [8th, or `high']
%bit), you can invoke many extended commands by meta-ing the first
%letter of the command.
\nd もしあなたのキーボードにメタキー
(別のキーと一緒に押すことによって
そのキーの`メタ'[第 8、または`上位']ビットをセットする)があれば、
コマンドの頭文字をメタキーと一緒に押すことによって多くの
拡張コマンドを起動することができる。
%In {\it NT, OS/2, PC\/ {\rm and} ST NetHack},
%the `Alt' key can be used in this fashion;
%on the {\it Amiga}, set the {\it altmeta\/} option to get this behavior.
{\it NT, OS/2, PC, ST NetHack} では
`Alt' キーがこの目的に使われる。
{\it Amiga} では、{\it altmeta\/}
オプションを設定することでこの振る舞いが得られる。
%On other systems, if typing `Alt' plus another key transmits a
%two character sequence consisting of an {\tt Escape}
%followed by the other key, you may set the {\it altmeta\/}
%option to have {\it NetHack\/} combine them into meta\+key.
その他のシステムでは、`Alt' と他のキーをタイプすると {\tt Escape} と
タイプしたキーからなる 2 文字の並びが転送されるので、
{\it NetHack\/} にこれをmeta+キーと組み合わせるために
{\it altmeta\/}
オプションを設定できる。
\blist{}
%%.lp
\item[\tb{M-?}]
%{\tt\#?} (not supported by all platforms)
{\tt\#?} (対応していないプラットホームもある)
%%.lp
\item[\tb{M-2}]
%{\tt\#twoweapon} (unless the {\it number\verb+_+pad\/} option is enabled)
{\tt\#twoweapon} ({\it number\verb+_+pad\/} がオフの場合)
%%.lp
\item[\tb{M-a}]
%{\tt\#adjust}
{\tt\#adjust}
%%.lp
\item[\tb{M-A}]
%{\tt\#annotate}
{\tt\#annotate}
%%.lp
\item[\tb{M-c}]
%{\tt\#chat}
{\tt\#chat}
%%.lp
\item[\tb{M-C}]
%{\tt\#conduct}
{\tt\#conduct}
%%.lp
\item[\tb{M-d}]
%{\tt\#dip}
{\tt\#dip}
%%.lp
\item[\tb{M-e}]
%{\tt\#enhance}
{\tt\#enhance}
%%.lp
\item[\tb{M-f}]
%{\tt\#force}
{\tt\#force}
%%.lp
\item[\tb{M-i}]
%{\tt\#invoke}
{\tt\#invoke}
%%.lp
\item[\tb{M-j}]
%{\tt\#jump}
{\tt\#jump}
%%.lp
\item[\tb{M-l}]
%{\tt\#loot}
{\tt\#loot}
%%.lp
\item[\tb{M-m}]
%{\tt\#monster}
{\tt\#monster}
%%.lp
\item[\tb{M-n}]
%{\tt\#name}
{\tt\#name}
%%.lp
\item[\tb{M-o}]
%{\tt\#offer}
{\tt\#offer}
%%.lp
\item[\tb{M-O}]
%{\tt\#overview}
{\tt\#overview}
%%.lp
\item[\tb{M-p}]
%{\tt\#pray}
{\tt\#pray}
%%.Ip
\item[\tb{M-q}]
%{\tt\#quit}
{\tt\#quit}
%%.lp
\item[\tb{M-r}]
%{\tt\#rub}
{\tt\#rub}
%%.lp
\item[\tb{M-R}]
%{\tt\#ride}
{\tt\#ride}
%%.lp
\item[\tb{M-s}]
%{\tt\#sit}
{\tt\#sit}
%%.lp
\item[\tb{M-t}]
%{\tt\#turn}
{\tt\#turn}
%%.lp
\item[\tb{M-T}]
%{\tt\#tip}
{\tt\#tip}
%%.lp
\item[\tb{M-u}]
%{\tt\#untrap}
{\tt\#untrap}
%%.lp
\item[\tb{M-v}]
%{\tt\#version}
{\tt\#version}
%%.lp
\item[\tb{M-w}]
%{\tt\#wipe}
{\tt\#wipe}
\elist

%%.pg
%\nd If the {\it number\verb+_+pad\/} option is on, some additional letter commands
%are available:
\nd {\it number\verb+_+pad\/}
オプションがオンのときは、これらに加えいくつかの文字コマンドが有効になる。
\blist{}
%%.lp
\item[\tb{h}]
%{\tt\#help}
{\tt\#help}
%%.lp
\item[\tb{j}]
%{\tt\#jump}
{\tt\#jump}
%%.lp
\item[\tb{k}]
%{\tt\#kick}
{\tt\#kick}
%%.lp
\item[\tb{l}]
%{\tt\#loot}
{\tt\#loot}
%%.lp
\item[\tb{N}]
%{\tt\#name}
{\tt\#name}
%%.lp
\item[\tb{u}]
%{\tt\#untrap}
{\tt\#untrap}
\elist

%%.hn 1
%\section{Rooms and corridors}
\section{部屋と通路}

%%.pg
%Rooms and corridors in the dungeon are either lit or dark.
%Any lit areas within your line of sight will be displayed;
%dark areas are only displayed if they are within one space of you.
%Walls and corridors remain on the map as you explore them.
洞窟内の部屋や通路は明りがついていることもあるし、ついていないこともある。
明りのついている部分で自分の視野に入る部分は画面に表示される。
暗いところでは周囲 1 つ分の空間だけが見える。
壁や通路は画面に表示されたままになる。

%%.pg
%Secret corridors are hidden.  You can find them with the `{\tt s}' (search)
%command.
隠し通路は表示されない。これらは `{\tt s}'(search: 探す) コマンドで
発見することができる。

%%.hn 2
%\subsection*{Doorways}
\subsection*{出入口}

%%.pg
%Doorways connect rooms and corridors.  Some doorways have no doors;
%you can walk right through.  Others have doors in them, which may be
%open, closed, or locked.  To open a closed door, use the `{\tt o}' (open)
%command; to close it again, use the `{\tt c}' (close) command.
出入口は部屋と通路を接続するものである。出入口の中には扉のないものがある。
このときにはそのまま通り抜けることができる。その他の出入口には扉があるが、
その扉は開いているか、閉じているか、錠がかかっているかのいずれかである。
閉じている扉を開けるには `{\tt o}'(open: 扉を開ける) コマンドを
用いる。再び扉を閉じるには `{\tt c}'(close: 扉を閉じる) コマンドを用いる。

%%.pg
%You can get through a locked door by using a tool to pick the lock
%with the `{\tt a}' (apply) command, or by kicking it open with the
%`{\tt \^{}D}' (kick) command.
扉に錠がかかっているときは `{\tt a}'(apply: 道具を用いる) コマンドで錠をはずす
道具を使うか、`{\tt \^{}D}'(kick: 蹴る) コマンドで扉を蹴破ることで
通ることができる。

%%.pg
%Open doors cannot be entered diagonally; you must approach them
%straight on, horizontally or vertically.  Doorways without doors are
%not restricted in this fashion.
開いた扉に斜めから入ることはできない。水平あるいは垂直方向から真っ直ぐに
近付かなければならない。扉のない出入口にはこのような制限はない。

%%.pg
%Doors can be useful for shutting out monsters.  Most monsters cannot
%open doors, although a few don't need to (for example, ghosts can walk through
%doors).
扉は怪物を締め出すのに役に立つ。たいていの怪物は扉を開けることができない。
けれどもいくつかの怪物にとっては扉を開ける必要などない(例: 亡霊は
扉を通り抜けることができる)。

%%.pg
%Secret doors are hidden.  You can find them with the `{\tt s}' (search)
%command.  Once found they are in all ways equivalent to normal doors.
隠し扉は表示されない。これらは `{\tt s}'(search: 探す) コマンドで発見する
ことができる。一度発見すると普通の扉と同様になる。

%%.hn 2
%\subsection*{Traps (`{\tt \^{}}')}
\subsection*{罠 (`{\tt \^{}}')}

%%.pg
%There are traps throughout the dungeon to snare the unwary delver.
%For example, you may suddenly fall into a pit and be stuck for a few
%turns trying to climb out.  Traps don't appear on your map until you
%see one triggered by moving onto it, see something fall into it, or you
%discover it with the `{\tt s}' (search) command.  Monsters can fall prey to
%traps, too, which can be a very useful defensive strategy.
そそっかしい冒険者を陥れようとする罠が洞窟のあちこちにある。
例えば落し穴に落ちると上り出るのに数ターンの間その場から動けなくなるであろう。
そこに誰かが足を踏み入れて引っかかるか、`{\tt s}'(search: 探す) コマンドで見つけるか
して初めて罠は表示される。怪物も罠の餌食になることがある。これは非常に有効な
防御戦術である。

%%.pg
%There is a special pre-mapped branch of the dungeon based on the
%classic computer game ``{\tt Sokoban}.''  The goal is to push the boulders
%into the pits or holes.  With careful foresight, it is possible to
%complete all of the levels according to the traditional rules of
%Sokoban.  Some allowances are permitted in case the player gets stuck;
%however, they will lower your luck.
古典的コンピュータゲームである``{\tt 倉庫番}''を基にした特別なダンジョンへの
分かれ道も存在する。
このダンジョンでの目的は大岩を落し穴に入れることである。
慎重に先読みすれば、全ての階は倉庫番の伝統のルールでクリアできる。
行き詰まってしまった冒険者のためにいくつかの助け舟が用意されているが、
これらは運を低下させてしまう。

%%.hn 2
%\subsection*{Stairs and ladders (`{\tt <}', `{\tt >}')}
\subsection*{階段とはしご (`{\tt <}', `{\tt >}')}

%%.pg
%In general, each level in the dungeon will have a staircase going up
%(`{\tt <}') to the previous level and another going down (`{\tt >}')
%to the next
%level.  There are some exceptions though.  For instance, fairly early
%in the dungeon you will find a level with two down staircases, one
%continuing into the dungeon and the other branching into an area
%known as the Gnomish Mines.  Those mines eventually hit a dead end,
%so after exploring them (if you choose to do so), you'll need to
%climb back up to the main dungeon.
一般的に洞窟の各階には、前の階への上り階段('<')と、
次の階への下り階段('>')がひとつずつある。
しかし、例外もある。
例えば、洞窟の比較的浅い階であなたは二つの下り階段がある階を発見するだろう。
ひとつは洞窟の続きであり、
もうひとつはノームの坑道として知られる地帯に続いている。
この坑道は最後には行き止まりになっていて、
ここを探検した(あなたがそうすることを選択した場合)後、
元の洞窟にまで戻って来る必要がある。

%%.pg
%When you traverse a set of stairs, or trigger a trap which sends you
%to another level, the level you're leaving will be deactivated and
%stored in a file on disk.  If you're moving to a previously visited
%level, it will be loaded from its file on disk and reactivated.  If
%you're moving to a level which has not yet been visited, it will be
%created (from scratch for most random levels, from a template for
%some ``special'' levels, or loaded from the remains of an earlier game
%for a ``bones'' level as briefly described below).  Monsters are only
%active on the current level; those on other levels are essentially
%placed into stasis.
%
階段を使うか、あるいは他の階に移動する罠に引っ掛かった場合、
いままでいた階は非活性化されて、ディスク上のファイルに貯えられる。
昔訪れた階に再び訪れた場合、ディスク上のファイルが読み込まれて再活性化される。
初めて訪れる階に移動した場合、その階は(ほとんどの階は一から、
「特別な」階はテンプレートから、「骨の」階(後述)の場合はファイルから読み込んで)
新たに作成される。
怪物は現在の階にいるものだけが行動する。
他の階にいるものは止まっている。
%%.pg
%Ordinarily when you climb a set of stairs, you will arrive on the
%corresponding staircase at your destination.  However, pets (see below)
%and some other monsters will follow along if they're close enough when
%you travel up or down stairs, and occasionally one of these creatures
%will displace you during the climb.  When that occurs, the pet or other
%monster will arrive on the staircase and you will end up nearby.
普通、あなたが階段を使うと、あなたは目的地で対応する階段の上に現れる。
しかし、ペット(後述)といくつかの怪物はあなたが階段を上り降りした時についてきて、
時々それらの一体があなたと入れ替わることがある。
こうなった場合、ペットやその他の怪物が階段の上に現れ、
あなたはそのそばに現れる。

%%.pg
%Ladders serve the same purpose as staircases, and the two types of
%inter-level connections are nearly indistinguishable during game play.
はしごは階段と同じ役目であり、階を繋ぐこの二つの種類は
ゲーム中ほとんど区別ができない。

%%.hn 2
%\subsection*{Shops and shopping}
\subsection*{店と買い物}

%%.pg
%Occasionally you will run across a room with a shopkeeper near the door
%and many items lying on the floor.  You can buy items by picking them
%up and then using the `{\tt p}' command.  You can inquire about the price
%of an item prior to picking it up by using the ``{\tt \#chat}'' command
%while standing on it.  Using an item prior to paying for it will incur a
%charge, and the shopkeeper won't allow you to leave the shop until you
%have paid any debt you owe.
時々ドアのそばに店主がいて、床にたくさんの物が置いてある部屋に
行き当たることがある.物を拾って、`{\tt p}'コマンドを使うことで
物を買うことができる.
物を拾う前に、物の上に立って``{\tt \#chat}''コマンドを使うことで値段を確認できる.
支払いをする前に物を使うと借金になり、店主は債務を支払うまでは
店の外に出してくれなくなる.

%%.pg
%You can sell items to a shopkeeper by dropping them to the floor while
%inside a shop.  You will either be offered an amount of gold and asked
%whether you're willing to sell, or you'll be told that the shopkeeper
%isn't interested (generally, your item needs to be compatible with the
%type of merchandise carried by the shop).
店の中にいるときに物を落すことで物を店主に売ることができる.
売却価格を提示され、売るかどうかを確認されるか、
単に店主が興味がないことを知らされる(一般的に、
店で売っている物と同じ種類の物が買い取り対象である)。

%%.pg
%If you drop something in a shop by accident, the shopkeeper will usually
%claim ownership without offering any compensation.  You'll have to buy
%it back if you want to reclaim it.
偶然で物を店の中に落してしまうと、店主は普通代償なしに所有権を主張する.
物を取り戻すためには買い戻す必要がある.

%%.pg
%Shopkeepers sometimes run out of money.  When that happens, you'll be
%offered credit instead of gold when you try to sell something.  Credit
%can be used to pay for purchases, but it is only good in the shop where
%it was obtained; other shopkeepers won't honor it.  (If you happen to
%find a ``credit card'' in the dungeon, don't bother trying to use it in
%shops; shopkeepers will not accept it.)
店主は資金不足になることがある。こうなった場合、何かを売ろうとすると
現金の代わりに信用貸しを提案される。
信用貸しは物を買うときに使えるが、同じ店でだけ有効である。
他の店主は引き受けない。
(迷宮で「クレジットカード」を手に入れることがあるが、
これを店で使おうとしてはいけない。店主は受け入れない。)

%%.pg
%The {\tt \$} command, which reports the amount of gold you are carrying
%(in inventory, not inside bags or boxes), will also show current shop
%debt or credit, if any.  The {\tt Iu} command lists unpaid items
%(those which still belong to the shop) if you are carrying any.
%The {\tt Ix} command shows an inventory-like display of any unpaid
%items which have been used up, along with other shop fees, if any.
{\tt \$}コマンドを使うと、今持ち運んでいる金貨の数(かばんや箱に入っているものは
除く)とともに、負債や信用貸しがあれば表示される。
{\tt Iu}コマンドは未払いの物(まだ店が所有する物)の一覧(あれば)を表示する。
{\tt Ix}コマンドは使ってしまった物の代金の一覧(あれば)が表示される。

%%.hn 3
%\subsubsection*{Shop idiosyncrasies}
\subsubsection*{店の特殊な性質}

%%.pg
%Several aspects of shop behavior might be unexpected.
店のいくつかの性質は意外なものかもしれない。

\begin{itemize}
%% note: a bullet is the default item label so we could omit [$\bullet$] here
%%.lp \(bu 2
\item[$\bullet$]
%The price of a given item can vary due to a variety of factors.
アイテムの値段は色々な要素によって様々に変化する。
%%.lp \(bu 2
\item[$\bullet$]
%A shopkeeper treats the spot immediately inside the door as if it were
%outside the shop.
店主はドアの一つ内側の場所は店の外として扱う。
%%.lp \(bu 2
\item[$\bullet$]
%While the shopkeeper watches you like a hawk, he will generally ignore
%any other customers.
店主はあなたを鷹のように注意深く見張っているが、
一般的にその他の消費者については無視する。
%%.lp \(bu 2
\item[$\bullet$]
%If a shop is ``closed for inventory,'' it will not open of its own accord.
店が``closed for inventory''(「棚卸のため閉店」)の場合、
待っていても開店はしない。
%%.lp \(bu 2
\item[$\bullet$]
%Shops do not get restocked with new items, regardless of inventory depletion.
店は例え品物が枯渇しても新しいアイテムが入荷することはない。
\end{itemize}

%%.hn 1
%\section{Monsters}
\section{怪物}

%%.pg
%Monsters you cannot see are not displayed on the screen.  Beware!
%You may suddenly come upon one in a dark place.  Some magic items can
%help you locate them before they locate you (which some monsters can do
%very well).
画面上にはあなたから見える怪物しか表示されないが、ひょっとすると闇の
中で怪物とばったり出くわすことになるかもしれないので注意が必要である。
魔法のアイテムの中には、怪物があなたを発見するよりも前にあなたが怪物を
発見するのに役立つ物もある(しかしこの能力に大変優れている怪物もいる。)

%%.pg
%The commands `{\tt /}' and `{\tt ;}' may be used to obtain information
%about those
%monsters who are displayed on the screen.  The command ``{\tt \#name}''
%(by default bound to `{\tt C}'), allows you
%to assign a name to a monster, which may be useful to help distinguish
%one from another when multiple monsters are present.  Assigning a name
%which is just a space will remove any prior name.
`{\tt /}' コマンドと `{\tt ;}' コマンドが画面上に表示されている怪物に関する情報を得るのに
用いられる。
``{\tt \#name}'' コマンド(標準では `{\tt C}' に割り当てられている) は
怪物に名前をつけるのに用いられ、
複数の怪物がいる時に、ある怪物を他のものと見分けたいときに便利である。
名前としてスペースを指定すると、以前につけた名前を消すことになる。

%%.pg
%The extended command ``{\tt \#chat}'' can be used to interact with an adjacent
%monster.  There is no actual dialog (in other words, you don't get to
%choose what you'll say), but chatting with some monsters such as a
%shopkeeper or the Oracle of Delphi can produce useful results.
拡張コマンド「{\tt \#chat}」によって、隣接する怪物と交流できる。
実際の台詞は選択できない(言い換えると、あなたが何をしゃべるかは選べない)が、
店主や Oracle of Delphi(デルファイの神殿)といった相手と話をすることは、
有益な結果をもたらす。

%%.hn 2
%\subsection*{Fighting}
\subsection*{戦闘}

%%.pg
%If you see a monster and you wish to fight it, just attempt to walk
%into it.  Many monsters you find will mind their own business unless
%you attack them.  Some of them are very dangerous when angered.
%Remember:  discretion is the better part of valor.
発見した怪物と戦いたいときには、
単にその怪物に向かって移動するようにすればよい。
多くの怪物はあなたが戦いを挑まない限り、
他のことを気にすることはない。
腹を立てると大変危険な怪物もいる。
「三十六計逃げるにしかず」ということわざを忘れずに。

%%.pg
%In most circumstances, if you attempt to attack a peaceful monster by
%moving into its location, you'll be asked to confirm your intent.  By
%default an answer of `{\tt y}' acknowledges that intent,
%which can be error prone if you're using `{\tt y}' to move.  You can set the
%{\it paranoid\verb+_+confirmation\/}
%option to require a response of ``{\tt yes}'' instead.
ほとんどの場合、友好的な怪物の位置に移動することで攻撃をしようとすると、
意図を確認される。
デフォルトでは `{\tt y}' を押すと確認したことになるが、
移動に `{\tt y}' を使っていると間違いがちである。
代わりに ``{\tt yes}'' の返答が必要なように
{\it paranoid\verb+_+confirmation\/}
オプションを設定できる。
%%.pg

%If you can't see a monster (if it is invisible, or if you are blinded),
%the symbol `I' will be shown when you learn of its presence.
%If you attempt to walk into it, you will try to fight it just like
%a monster that you can see; of course,
%if the monster has moved, you will attack empty air.  If you guess
%that the monster has moved and you don't wish to fight, you can use the `m'
%command to move without fighting; likewise, if you don't remember a monster
%but want to try fighting anyway, you can use the `F' command.
怪物が見えない(怪物が透明、またはあなたが目が見えない)場合、
怪物がいるとあなたが思った位置に`I'の文字が表示される。
その区画に歩こうとすると、見えている時と同じようにあなたは怪物に攻撃しようとする。
もちろん、もし怪物が移動してしまっていたら、あなたの攻撃は空を切ることになる。
もし怪物がすでに移動してしまっていると考え、戦いたくない場合は、
`m'コマンドを使うことで戦うことなく移動できる。
逆に、そこに怪物がいるかどうかわからないけれどもとにかく戦って見たい場合は、
`F'コマンドが使える。

%%.hn 2
%\subsection*{Your pet}
\subsection*{ペット}

%%.pg
%You start the game with a little dog (`{\tt d}'), kitten (`{\tt f}'),
%or pony (`{\tt u}'), which follows
%you about the dungeon and fights monsters with you.
%Like you, your pet needs food to survive.
%Dogs and cats usually feed themselves on fresh carrion and other meats;
%horses need vegetarian food which is harder to come by.
%If you're worried about your pet or want to train it, you
%can feed it, too, by throwing it food.
%A properly trained pet can be very useful under certain circumstances.
あなたは仔犬(`{\tt d}')、子猫(`{\tt f}')、ポニー(`{\tt u}')のいずれかとともにゲームを始める。
あなたのペットはあなたとともに洞窟をさまよい、怪物と戦う。
ペットはあなたと同様に生きのびるための食料を必要とする。
犬と猫は新鮮な死肉やその他の肉をえさにしている;
馬は手に入れるのがより難しい草食の食料が必要である。
もしもペットのことが気がかりであったり、ペットを飼い慣らしておきたいと思うなら、
あなたが直接えさを与えることもできる。
そのためには食料をペットに向かって投げてやればよい。
きちんと訓練されたペットはある種の状況のもとで極めて利用価値が高い。

%%.pg
%Your pet also gains experience from killing monsters, and can grow
%over time, gaining hit points and doing more damage.  Initially, your
%pet may even be better at killing things than you, which makes pets
%useful for low-level characters.
ペットも怪物を倒していくにつれて経験を積んでいく。
また時間とともに成長もし、体力や相手に与えるダメージも増加する。
初めのうちはあなたよりもペットの方が強いだろうから、
レベルの低いキャラクタにとっては役に立つだろう。

%%.pg
%Your pet will follow you up and down staircases if it is next to you
%when you move.  Otherwise your pet will be stranded and may become
%wild.  Similarly, when you trigger certain types of traps which alter
%your location (for instance, a trap door which drops you to a lower
%dungeon level), any adjacent pet will accompany you and any non-adjacent
%pet will be left behind.  Your pet may trigger such traps itself; you
%will not be carried along with it even if adjacent at the time.
あなたが他の階へ移動するとき、隣にペットを連れていればペットもあなた
について移動する。置き去りにすると野生化してしまうかもしれない。
同様に、もしあなたが自分の位置が変わるような罠にかかった場合
(例えば、トラップドアで下の階に落ちた場合)、
隣にいたペットはついていき、隣にいなかったペットは取り残される。
ペットが自分でそのような罠にかかることもある。
あなたはたとえその時にペットの隣にいても一緒に移動することはない。

%%.hn 2
%\subsection*{Steeds}
\subsection*{軍馬}

%%.pg
%Some types of creatures in the dungeon can actually be ridden if you
%have the right equipment and skill.  Convincing a wild beast to let
%you saddle it up is difficult to say the least.  Many a dungeoneer
%has had to resort to magic and wizardry in order to forge the alliance.
ダンジョンにいるある種の生物は、あなたが正しい装備と能力を
持っていれば、乗ることができる。
野生の生物にあなたが乗ることを納得させるのは困難である。
多くの洞窟の主は協力関係をでっち上げるのに魔法の力に頼らなければならない。
%Once you do have the beast under your control however, you can
%easily climb in and out of the saddle with the ``{\tt \#ride}'' command.  Lead
%the beast around the dungeon when riding, in the same manner as
%you would move yourself.  It is the beast that you will see displayed
%on the map.
しかし、一旦動物をあなたの支配下に置いたなら、
``{\tt \#ride}''コマンドで乗り降りすることができる。
動物に乗って洞窟を移動する方法は、自分自身が移動するのと同じである。
地図に表示されるのはあなたが乗っている動物である。

%%.pg
%Riding skill is managed by the ``{\tt \#enhance}'' command.  See the section
%on Weapon proficiency for more information about that.
乗馬技術は`{\tt \#enhance}'コマンドで管理することができる。
詳しくは後述する武器の技術の章を参照すること。

%%.pg
%Use the `{\tt a}' (apply) command and pick a saddle in your inventory to
%attempt to put that saddle on an adjacent creature.  If successful,
%it will be transferred to that creature's inventory.
鞍を隣にいる生物に付けようとするには、
`{\tt a}' (apply) コマンドを使って持物から鞍を選ぶ。
成功すると、それはその生物の持物に移される。

%%.pg
%Use the ``{\tt \#loot}'' command while adjacent to a saddled creature to
%try to remove the saddle from that creature.  If successful, it will
%be transferred to your inventory.
隣にいる鞍の付いている生物から鞍を外そうとするには、
``{\tt \#loot}'' コマンドを使う。
成功すると、それはあなたの持物に移される。

%%.hn 2
%\subsection*{Bones levels}
\subsection*{骨の階}

%%.pg
%You may encounter the shades and corpses of other adventurers (or even
%former incarnations of yourself!) and their personal effects.  Ghosts
%are hard to kill, but easy to avoid, since they're slow and do little
%damage.  You can plunder the deceased adventurer's possessions;
%however, they are likely to be cursed.  Beware of whatever killed the
%former player; it is probably still lurking around, gloating over its
%last victory.
冒険者の幽霊とその死体(あるいはそれは生まれ変わる前のあなた自身であ
ることさえある)や、その所持品に出くわすかもしれない。
幽霊を殺すことは難しいが、
動きがのろくほとんどダメージを受けないので簡単に逃げることができる。
あなたは死んだ冒険者の所持品を奪うこともできる。
しかしそれらは呪われていることが多い。
以前の冒険者を死に到らしめたものにはすべて注意を払わねばならない。
それはあたりを徘徊し、その直前の勝利に酔いしれているからだ。

%%.hn 2
%\subsection*{Persistence of Monsters}
\subsection*{怪物の永続性}

%%.pg
%Monsters (a generic reference which also includes humans and pets) are only
%shown while they can be seen or otherwise sensed.
%Moving to a location where you can't see or sense a monster any more
%will result in it disappearing from your map, similarly if it is the
%one who moved rather than you.
怪物 (一般に人間とペットを含む表現) は、それが見えるかその他の方法で
感知できる場合にのみ表示される。
怪物があなたが見えなかったり感知できない場所に移動すると、
それがあなた以外の誰かの場合と同様、地図から消える。

%%.pg
%However, if you encounter a monster which you can't see or
%sense---perhaps it is invisible and has just tapped you on the
%noggin---a special ``remembered, unseen monster'' marker will be displayed at
%the location where you think it is.
%That will persist until you have
%proven that there is no monster there, even if the unseen monster
%moves to another location or you move to a spot where the marker's
%location ordinarily wouldn't be seen any more.
しかし、見えたり感知できたりしない怪物と遭遇した--おそらく見えず、
出会い頭に出会った--場合、
特別な「覚えているが見えない怪物」マーカが、それがいるとあなたが考えている
場所に表示される。
これは、例え見えない怪物が他の場所に移動していたり、あなたが通常マーカの
位置が見えない位置に移動したとしても、
あなたがそこに怪物がいないと確信するまで表示され続ける。

%%.hn 1
%\section{Objects}
\section{物}

%%.pg
%When you find something in the dungeon, it is common to want to pick
%it up.  In {\it NetHack}, this is accomplished automatically by walking over
%the object (unless you turn off the {\it autopickup\/}
%option (see below), or move with the `{\tt m}' prefix (see above)), or
%manually by using the `{\tt ,}' command.
洞窟の中で何かを見つけた場合、それを拾いたいと思うのはよくあることである。
{\it NetHack\/} ではその物の上を通ることによって自動的に拾うことができ(
{\it autopickup\/} オプション(後述)がオフになっているときや
`{\tt m}' プレフィックス(前述)を用いて移動するときはこの限りではない)、
または `{\tt ,}' コマンドを使って手動で拾うことができる。
%%.pg
%If you're carrying too many items, {\it NetHack\/} will tell you so and you
%won't be able to pick up anything more.  Otherwise, it will add the object(s)
%to your pack and tell you what you just picked up.
持ち物が多すぎるときには、{\it NetHack\/} はあなたにそのように告げ、
それ以上拾うことはできなくなる。
そうでなければ {\it NetHack\/} はその物をあなたの荷物に加え、何を拾ったかが表示される。
%%.pg
%As you add items to your inventory, you also add the weight of that object
%to your load.  The amount that you can carry depends on your strength and
%your constitution.  The
%stronger and sturdier
%you are, the less the additional load will affect you.  There comes
%a point, though, when the weight of all of that stuff you are carrying around
%with you through the dungeon will encumber you.  Your reactions
%will get slower and you'll burn calories faster, requiring food more frequently
%to cope with it.  Eventually, you'll be so overloaded that you'll either have
%to discard some of what you're carrying or collapse under its weight.
持ち物が増えるに連れて、荷物の重さが重くなる。
どれだけの重さの物を運べるかは筋力と耐久力による。
力が強く丈夫なほど、荷物の重さの影響は小さくなる。
しかしそれでも、洞窟の中を持ち歩いている荷物の重さがあなたに負担になるときがくる。
反応は鈍くなり、カロリー消費が早くなってより多くの食料が必要となる。
そして最終的には荷物が重すぎて何かを捨てないと動けなくなってしまう。
%%.pg
%{\it NetHack\/} will tell you how badly you have loaded yourself.
%If you are encumbered, one of the conditions
%``{\it Burdened\/}'', ``{\it Stressed\/}'', ``{\it Strained\/}'',
%``{\it Overtaxed\/}'' or ``{\it Overloaded\/}'' will be
%shown on the bottom line status display.
NetHack では荷物の重さがどれくらい悪影響を与えるかを教えてくれる。
荷物が邪魔になってきた場合、状態
``{\it Burdened\/}'', ``{\it Stressed\/}'', ``{\it Strained\/}'',
``{\it Overtaxed\/}'', ``{\it Overloaded\/}'' が
の一つが最下行に表示される。

%%.pg
%When you pick up an object, it is assigned an inventory letter.  Many
%commands that operate on objects must ask you to find out which object
%you want to use.  When {\it NetHack\/} asks you to choose a particular object
%you are carrying, you are usually presented with a list of inventory
%letters to choose from (see Commands, above).
物を拾ったとき、その物には目録記号が割り当てられる。
物に関する多くのコマンドはあなたがどの物を使いたいのかを尋ねてくる。
持ち物のうち特定の物を選ぶように {\it NetHack\/} が尋ねてきたときは、
普通目録記号の一覧が表示されてその中から選ぶ(前述のコマンドの項を参照のこと)。

%%.pg
%Some objects, such as weapons, are easily differentiated.  Others, like
%scrolls and potions, are given descriptions which vary according to
%type.  During a game, any two objects with the same description are
%the same type.  However, the descriptions will vary from game to game.
いくつかの物、例えば武器類のような物は、どんなものか簡単に区別が付けられる。
この他の物、例えば巻物や水薬はその種類に応じていろいろな名前が付けられている。
1 回のゲームの間は同じ名前を持った物は同じ種類の物である。
しかし物に付けられる名前はゲームごとに違うものになる。

%%.pg
%When you use one of these objects, if its effect is obvious, {\it NetHack\/}
%will remember what it is for you.  If its effect isn't extremely
%obvious, you will be asked what you want to call this type of object
%so you will recognize it later.  You can also use the ``{\tt \#name}''
%command, for the same purpose at any time, to name
%all objects of a particular type or just an individual object.
%When you use ``{\tt \#name}'' on an object which has already been named,
%specifying a space as the value will remove the prior name instead
%of assigning a new one.
このような物を使った時、もしその効果が明らかな場合は {\it NetHack\/} がそれが
何であるかを憶えていてくれる。
その効果があまり明らかでない場合はその物の種類を何と名付けるか尋ねてくる。
このため後になってそれを思い出すことができる。
またいつでも``{\tt \#name}''コマンドを使って
同様のことができ、
ある種類の物すべてに名前を付けたり個々の物に名前を付けたりできる。
すでに名前をつけているものに``{\tt \#name}''を使う場合、
新しい名前としてスペースを指定することによって、
以前つけていた名前を消すことができる。

%%.hn 2
%\subsection*{Curses and Blessings}
\subsection*{呪いと祝福}

%%.pg
%Any object that you find may be cursed, even if the object is
%otherwise helpful.  The most common effect of a curse is being stuck
%with (and to) the item.  Cursed weapons weld themselves to your hand
%when wielded, so you cannot unwield them.  Any cursed item you wear
%is not removable by ordinary means.  In addition, cursed arms and armor
%usually, but not always, bear negative enchantments that make them
%less effective in combat.  Other cursed objects may act poorly or
%detrimentally in other ways.
あなたが発見したいろいろな物は、たとえそれが役に立つものであったとしても
呪いがかけられているかも知れない。呪いがかけられている物を使うと
それが張り付いてしまうことは最もよく見られる結果である。
呪いがかけられている武器を持つとそれは手に張り付いてしまって取れなくなる。
呪いがかけられている物を身につけると普通の方法でははずすことができない。
さらに、呪いがかけられている武器や防具には必ずと言うわけではないが
たいていの場合負の魔力が与えられていて、そうでない物よりも戦闘時の効力が劣る。
その他の呪いがかけられている物は、あまり役に立たなかったり
またはその他の点で害を及ぼしたりするだろう。

%%.pg
%Objects can also be blessed.  Blessed items usually work better or
%more beneficially than normal uncursed items.  For example, a blessed
%weapon will do more damage against demons.
また祝福されている物もある。祝福されている物は
呪いがかけられていない普通の物に比べて具合良く働き役に立つ。
例えば祝福された武器は悪魔たちに一層のダメージを与えることができるだろう。

%%.pg
%Objects which are neither cursed nor blessed are referred to as uncursed.
%They could just as easily have been described as unblessed, but the
%uncursed designation is what you will see within the game.  A ``glass
%half full versus glass half empty'' situation; make of that what you will.
呪われてるわけでも祝福されているわけでもない物は「呪われていない(uncursed)」と
呼ばれる。
これは「祝福されていない(unblessed)」と表現することもできるが、
ゲームの中では「呪われていない」という表現が使われる。
これは「コップに半分だけ入っている/半分しか入っていない」問題である;
解釈はあなた次第である。

%%.pg
%There are magical means of bestowing or removing curses upon objects,
%so even if you are stuck with one, you can still have the curse
%lifted and the item removed.  Priests and Priestesses have an innate
%sensitivity to this property in any object, so they can more easily avoid
%cursed objects than other character roles.
魔法を使うと物に呪いをかけたり呪いを解いたりすることができる。
このためたとえ物が張り付いてしまっても、
呪いを解いてはずすことができる。
僧侶は生来呪いや祝福に敏感なので、
他の職業の冒険者よりも容易に呪われているものを避けることができる。

%%.pg
%An item with unknown status will be reported in your inventory with no prefix.
%An item which you know the state of will be distinguished in your inventory
%by the presence of the word ``cursed'', ``uncursed'' or ``blessed'' in the
%description of the item.
あなたの持ち物の目録の中で呪いがかけられているかどうか分からない物は
目録のなかで特に何の説明もない。どんな状態にあるか知っている物は
目録一覧で「cursed(呪われた)」「uncursed(呪われていない)」「blessed(祝福された)」などの
言葉が物の説明に与えられる。
%In some cases ``uncursed'' will be omitted as being redundant when
%enough other information is displayed.
%The
%{\it implicit\verb+_+uncursed\/}
%option can be used to control this; toggle it off to have ``uncursed''
%be displayed even when that can be deduced from other attributes.
場合によっては、その他の情報が十分に表示されている場合は
「uncursed(呪われていない)」は省略される。
{\it implicit\verb+_+uncursed\/}
オプションはこれを制御するのに使われる;
これをオフにすると、例えその他の属性から推測される場合でも
「uncursed(呪われていない)」を表示する。

%%.hn 2
%\subsection*{Weapons (`{\tt )}')}
\subsection*{武器 (`{\tt )}')}

%%.pg
%Given a chance, most monsters in the Mazes of Menace will gratuitously try to
%kill you.
%You need weapons for self-defense (killing them first).
%Without a
%weapon, you do only 1--2 hit points of damage (plus bonuses, if any).
%Monk characters are an exception; they normally do more damage with
%bare (or gloved) hands than they do with weapons.
恐怖の迷宮に住むほとんどすべての怪物は、チャンスと見れば見境なくあなたを
殺そうとするだろう。自分の身を守る(先に怪物を殺してしまう)ために
あなたは武器を必要とする。武器なしではポイントにして 1 ~ 2 のダメージ
(いくらかの加算があるかも知れないが)を与えることしかできない。
但し、モンクは例外である。モンクは武器を使って攻撃するよりも
素手(あるいはグローブをつけた手)で攻撃した方が大きなダメージを
与えることができる。

%%.pg
%There are wielded weapons, like maces and swords, and thrown weapons,
%like arrows and spears.  To hit monsters with a weapon, you must wield it and
%attack them, or throw it at them.  You can simply elect to throw a spear.
武器には鎚矛や剣のような振り回すためのものと、矢や槍のような投げつけるためのものがある。
武器で怪物に打撃を与えるためには、武器を手にもって怪物を攻撃するか
武器を怪物に投げなければならない。投げるには、単純に槍を投げるように選択するだけでよい。
%To shoot an arrow, you should first wield a bow, then throw the arrow.
%Crossbows shoot crossbow bolts.  Slings hurl rocks and (other) stones
%(like gems).
弓から矢を発射するには、まず弓を手に持ってから矢を投げればよい。クロスボウは
crossbow bolt(クロスボウ用の太矢)を発射するためのものである。
投石器は岩や(宝石のような)その他の石を投げつけるために用いる。

%%.pg
%Enchanted weapons have a ``plus'' (or ``to hit enhancement'' which can be
%either positive or negative) that adds to your chance to
%hit and the damage you do to a monster.  The only way to determine a weapon's
%enchantment is to have it magically identified somehow.
魔力のある武器には「追加能力」(攻撃能力と言うことで正と負の両方を取り得る)が
付けられており、
攻撃が怪物に当たる可能性と与えるダメージに追加される。
武器の魔力を測定するには、なんらかの方法で魔法を使って鑑定するしかない。
%Most weapons are subject to some type of damage like rust.  Such
%``erosion'' damage can be repaired.
多くの武器は錆のようなある種のダメージを受けやすい。このような
「腐食」ダメージは修復可能である。

%%.pg
%The chance that an attack will successfully hit a monster, and the amount
%of damage such a hit will do, depends upon many factors.  Among them are:
%type of weapon, quality of weapon (enchantment and/or erosion), experience
%level, strength, dexterity, encumbrance, and proficiency (see below).  The
%monster's armor class---a general defense rating, not necessarily due to
%wearing of armor---is a factor too; also, some monsters are particularly
%vulnerable to certain types of weapons.
攻撃がうまく怪物に命中するかと、命中した時にどれくらいのダメージを与えるかは、
多くの要素によって決定される。それらには以下のものがある:
武器の種類、武器の品質(魔力や腐食)、経験レベル、筋力、敏捷性、荷物の重さ、
技量(後述)。
怪物の防御値(一般的な防御率。防具を着ることによるとは限らない)も要素の一つである。
また、ある種の武器に対して特に耐性を持つ怪物もいる。

%%.pg
%Many weapons can be wielded in one hand; some require both hands.
%When wielding a two-handed weapon, you can not wear a shield, and
%vice versa.  When wielding a one-handed weapon, you can have another
%weapon ready to use by setting things up with the `{\tt x}' command, which
%exchanges your primary (the one being wielded) and alternate weapons.
多くの武器は片手持ちであるが、両手が必要な武器もある。
両手持ちの武器と盾を同時に持つことはできない。
片手持ちの武器を持っている場合、
もうひとつの武器を予備として準備しておいて、
`{\tt x}'コマンドで今使っている武器と交換することができる。
%And if you have proficiency in the ``two weapon combat'' skill, you
%may wield both weapons simultaneously as primary and secondary; use the
%`{\tt X}' command to engage or disengage that.
%Only some types of characters (barbarians, for instance) have the necessary
%skill available.  Even with that skill, using two weapons at once incurs
%a penalty in the chance to hit your target compared to using just one
%weapon at a time.
さらに「二刀流戦闘」技能の技量がある場合、二つの武器を同時に使うことができる。
`{\tt X}' コマンドで二刀流戦闘をするかどうかを切り替えられる。
一部のキャラクター(例えば野蛮人)だけがこの技能を持っている。
たとえこの技能を持っていても、一度に二つの武器を使うと
ひとつしか使わない時に比べて敵に命中する確率にペナルティを受ける。

%%.pg
%There might be times when you'd rather not wield any weapon at all.
%To accomplish that, wield `{\tt -}', or else use the `{\tt A}' command which
%allows you to unwield the current weapon in addition to taking off
%other worn items.
全く何の武器も装備したくない時もあるかもしれない。
そうするためには、装備するものを選ぶ時に`{\tt -}'を選ぶか、あるいは`{\tt A}'コマンドで
その他の身につけているものを外すのと同時に現在使っている武器を外すことができる。

%%.pg
%Those of you in the audience who are AD\&D players, be aware that each
%weapon which existed in AD\&D does roughly the same damage to monsters in
%{\it NetHack}.  Some of the more obscure weapons (such as the %
%{\it aklys}, {\it lucern hammer}, and {\it bec-de-corbin\/}) are defined
%in an appendix to {\it Unearthed Arcana}, an AD\&D supplement.
AD\&D をプレイしたことのある読者はお気付きだろうが、AD\&D に登場した武器は
{\it NetHack\/} においても怪物にだいたい同じダメージを与える。
あまり良く知られていない武器(
{\it aklys}(アキリス), {\it lucern hammer}(ルッツェンハンマー),
{\it bec-de-corbin\/}(ベッグ・デ・コルビン) など)のいくつかは AD\&D の
追加ルールである {\it Unearthed Arcana} の付録で詳しく説明されている。

%%.pg
%The commands to use weapons are `{\tt w}' (wield), `{\tt t}' (throw),
%`{\tt f}' (fire, an alternate way of throwing), `{\tt Q}' (quiver),
%`{\tt x}' (exchange), `{\tt X}' (twoweapon), and ``{\tt \#enhance}''
%(see below).
武器を使うコマンドは `{\tt w}'(wield: 武器を持つ)、 `{\tt t}'(throw: 投げる)、
`{\tt f}' (fire: 武器を発射するもうひとつの方法), `{\tt Q}'(quiver: 矢筒)、
`{\tt x}' (exchange: 武器を交換する)、`{\tt X}' (二刀流)、`{\tt \#enhance}'(後述) である。

%%.hn 3
%\subsection*{Throwing and shooting}
\subsection*{ものを投げることと発射すること}

%%.pg
%You can throw just about anything via the `{\tt t}' command.  It will prompt
%for the item to throw; picking `{\tt ?}' will list things in your inventory
%which are considered likely to be thrown, or picking `{\tt *}' will list
%your entire inventory.  After you've chosen what to throw, you will
%be prompted for a direction rather than for a specific target.  The
%distance something can be thrown depends mainly on the type of object
%and your strength.  Arrows can be thrown by hand, but can be thrown
%much farther and will be more likely to hit when thrown while you are
%wielding a bow.
`{\tt t}'コマンドを使うことによって、どんな物でも投げることができる。
このコマンドを指定すると、投げる物を尋ねてくる。
`{\tt ?}'を押すと、持ち物の中で投げるのに向いていそうな物の一覧が表示され、
`{\tt *}'を押すと、全ての持ち物の一覧が表示される。
何を投げるかを指定した後、(どの目標に投げるかではなく)どの方向に投げるかを尋ねられる。
投げることができる距離は、主に物の種類と筋力による。
矢は手で投げることもできるが、弓を持って投げた方が遥かに遠くまで飛び、
目標にも当たりやすい。

%%.pg
%You can simplify the throwing operation by using the `{\tt Q}' command to
%select your preferred ``missile'', then using the `{\tt f}' command to
%throw it.  You'll be prompted for a direction as above, but you don't
%have to specify which item to throw each time you use `{\tt f}'.  There is
%also an option,
%{\it autoquiver},
%which has {\it NetHack\/} choose another item to automatically fill your
%quiver (or quiver sack, or have at the ready) when the inventory slot used
%for `{\tt Q}' runs out.
`{\tt Q}'コマンドで予め好みの「発射物」を選択しておき、
`{\tt f}'コマンドで投げることによって、投げる操作を簡単にすることが出来る。
先程と同様に投げる方向は尋ねられるが、
`{\tt f}'コマンドを使うたびに何を投げるかを指定する必要はない。
{\it autoquiver\/}
オプションをオンにすると、
`{\tt Q}'コマンドで設定した物がなくなった時に、
{\it NetHack\/} が自動的に矢筒(または矢筒袋または準備している物)に他の物を
入れてくれる。

%%.pg
%Some characters have the ability to fire a volley of multiple items in a
%single turn.  Knowing how to load several rounds of ammunition at
%once---or hold several missiles in your hand---and still hit a
%target is not an easy task.  Rangers are among those who are adept
%at this task, as are those with a high level of proficiency in the
%relevant weapon skill (in bow skill if you're wielding one to
%shoot arrows, in crossbow skill if you're wielding one to shoot bolts,
%or in sling skill if you're wielding one to shoot stones).
%The number of items that the character has a chance to fire varies from
%turn to turn.  You can explicitly limit the number of shots by using a
%numeric prefix before the `{\tt t}' or `{\tt f}' command.
一度に複数の物を投げることができるキャラクターもいる。
一度に複数の矢を装填し(あるいは一度に複数の物を持ち)、
それを目標に当てることはたやすい仕事ではない。
レンジャーはこの技能に優れているし、適切な武器の技能
(弓を持って矢を射るなら弓の技能、クロスボゥを使うならクロスボゥの技能、
スリングを持って石を投げるならスリングの技能)
において高いレベルの技量があれば他の職業でも可能である。
あなたがいくつのアイテムを同時に投げることができるかは
毎ターン変化する。
`{\tt t}'コマンドや`{\tt f}'コマンドの前に数字を指定することによって、
発射する数を明示的に制限することもできる。
%For example, ``{\tt 2f}'' (or ``{\tt n2f}'' if using
%{\it number\verb+_+pad\/}
%mode) would ensure that at most 2 arrows are shot
%even if you could have fired 3.  If you specify
%a larger number than would have been shot (``{\tt 4f}'' in this example),
%you'll just end up shooting the same number (3, here) as if no limit
%had been specified.  Once the volley is in motion, all of the items
%will travel in the same direction; if the first ones kill a monster,
%the others can still continue beyond that spot.
例えば、``{\tt 2f}''(
{\it number\verb+_+pad\/}
オプションがオンの時には``{\tt n2f}'')と指定すると、
たとえ 3 本発射可能でも最大 2 本しか発射されない。
もし、実際に発射できる数よりも大きい数を指定した場合
(この例では ``{\tt 4f}'')は、
特に数字を指定しなかった場合と同様、
実際に発射可能な数(ここでは 3 本)しか発射されない。
一旦動作を開始したなら、全てのアイテムは同じ方向に飛ぶ。
もし 1 本目で怪物を倒したなら、残りはその先まで飛んでいくことになる。

%%.hn 3
%\subsection*{Weapon proficiency}
\subsection*{武器の技量}

%%.pg
%You will have varying degrees of skill in the weapons available.
%Weapon proficiency, or weapon skills, affect how well you can use
%particular types of weapons, and you'll be able to improve your skills
%as you progress through a game, depending on your role, your experience
%level, and use of the weapons.
利用できる武器の技能はさまざまに異なっている。
武器の技量(技能)はある種の武器をどれくらいうまく扱えるかに影響する。
技能はゲームを通じて向上させることが出来るが、
それは職業や経験レベルや武器の使用回数による。

%%.pg
%For the purposes of proficiency, weapons have
%been divided up into various groups such as daggers, broadswords, and
%polearms.  Each role has a limit on what level of proficiency a character
%can achieve for each group.  For instance, wizards can become highly
%skilled in daggers or staves but not in swords or bows.
技量を示すために、武器は daggers 、 broadswords 、 polearms といった風に
いくつかのグループに分けられている。
それぞれのグループで技量レベルをどこまで高めることが出来るかは職業毎に決まっている。
例えば魔法使いは daggers や staves に対しては高いレベルに達することが出来るが、
swords や bows に対してはそうではない。

%%.pg
%The ``{\tt \#enhance}'' extended command is used to review current weapons
%proficiency
%(also spell proficiency) and to choose which skill(s) to improve when
%you've used one or more skills enough to become eligible to do so.  The
%skill rankings are ``none'' (sometimes also referred to as ``restricted'',
%because you won't be able to advance), ``unskilled'', ``basic'', ``skilled'',
%and ``expert''.  Restricted skills simply will not appear in the list
%shown by ``{\tt \#enhance}''.
%(Divine intervention might unrestrict a particular
%skill, in which case it will start at unskilled and be limited to basic.)
%Some characters can enhance their barehanded combat or martial arts skill
%beyond expert to ``master'' or ``grand master''.
``{\tt \#enhance}''拡張コマンドで現在の武器(と呪文)の技量を見ることができる。
さらに、ひとつまたは複数の技能を向上させることができる状態なら、
どの技能を向上させるかを選択することができる。
技能のランクは``none'' (技能を向上させることができないという意味で
``restricted''(制限された) と呼ばれることもある)、``unskilled''(初心者),
``basic''(入門者), ``skilled''(熟練者), ``expert''(エキスパート)である。
制限された技能は単に``{\tt \#enhance}''コマンドで一覧に表示されない。
(神の介入によりある種の技能の制限が解除されることもある。
その時には技能は"unskilled"となり、限界は"basic"までとなる)
キャラクターによっては、素手での戦闘やマーシャルアーツの技能を
``master''(マスター)や``grand master''(グランドマスター)にまで向上させる
ことができる。

%%.pg
%Use of a weapon in which you're restricted or unskilled
%will incur a modest penalty in the chance to hit a monster and also in
%the amount of damage done when you do hit; at basic level, there is no
%penalty or bonus; at skilled level, you receive a modest bonus in the
%chance to hit and amount of damage done; at expert level, the bonus is
%higher.  A successful hit has a chance to boost your training towards
%the next skill level (unless you've already reached the limit for this
%skill).  Once such training reaches the threshold for that next level,
%you'll be told that you feel more confident in your skills.  At that
%point you can use ``{\tt \#enhance}'' to increase one or more skills.
%Such skills
%are not increased automatically because there is a limit to your total
%overall skills, so you need to actively choose which skills to enhance
%and which to ignore.
技能レベルが"restricted"や"unskilled"の武器を使うと、怪物への命中率や
命中した時に与えるダメージにペナルティがある。
"basic"ではペナルティもボーナスもない。
"skilled"では命中率と与えるダメージにささやかなボーナスがある。
"expert"ではボーナスは大きくなる。
攻撃が命中すると、使っている武器に関する技能を(もし最大に達していないなら)
向上させるための訓練度を増やすチャンスがある。
この訓練度が次のレベルに必要な値に達すると、
自分の技能により自信を持てるように感じたことが告げられる。
この時点で``{\tt \#enhance}''コマンドを使うことによってひとつまたは複数の技能を
向上させることができる。
技能は自動的には向上しない。
なぜなら全ての技能レベルの合計には制限があるので、
どの技能を向上させて、どれを無視するかを決める必要があるからである。

%%.hn 3
%\subsection*{Two-Weapon combat}
\subsection*{二刀流}

%%.pg
%Some characters can use two weapons at once.  Setting things up to
%do so can seem cumbersome but becomes second nature with use.
%To wield two weapons, you need to use the ``{\tt \#twoweapon}'' command.
%But first you need to have a weapon in each hand.
%(Note that your two weapons are not fully equal; the one in the
%hand you normally wield with is considered primary and the other
%one is considered secondary.  The most noticeable difference is
%after you stop---or before you begin, for that matter---wielding
%two weapons at once.  The primary is your wielded weapon and the
%secondary is just an item in your inventory that's been designated
%as alternate weapon.)
一部のキャラクタは二つの武器を一度に使うことができる。
そうするための準備は厄介だが、使うことで身についていく。
二つの武器を装備するには、``{\tt \#twoweapon}'' コマンドを使う必要がある。
しかし、まずそれぞれの手に一つの武器を手にする必要がある。
(二つの武器は完全に同じではないことに注意すること; 通常装備している手で
手にしている武器は主要武器として扱われ、もう一方は補助武器として扱われる。
もっとも注意するべき違いは、二つの武器を手にするのを止めた後---
または始める前---である。
優先武器は手にしている武器であり、補助武器は単に、
予備の武器として指定された、持ち物の一つである。)

%%.pg
%If your primary weapon is wielded but your off hand is empty or has
%the wrong weapon, use the sequence `{\tt x}', `{\tt w}', `{\tt x}' to
%first swap your
%primary into your off hand, wield whatever you want as secondary
%weapon, then swap them both back into the intended hands.
%If your secondary or alternate weapon is correct but your primary
%one is not, simply use `{\tt w}' to wield the primary.
%Lastly, if neither hand holds the correct weapon,
%use `{\tt w}', `{\tt x}', `{\tt w}'
%to first wield the intended secondary, swap it to off hand, and then
%wield the primary.
主要武器を手にしているけれどももう片方の手が空いていたり間違った武器を
持っていたりする場合、`{\tt x}', `{\tt w}', `{\tt x}' とすると、
まず主要武器をもう片方の手に持ち替え、補助武器として使いたいものを装備し、
再び意図通りの手に入れ替えられる。
補助武器が正しいけれども主要武器が間違っている場合、単に
`{\tt w}' を使って主要武器を手にすることができる。
最後に、どちらの手にも正しい武器を手にしていない場合、
`{\tt w}', `{\tt x}', `{\tt w}' として、最初に補助武器を手にし、
それを入れ替え、主要武器を手にする。

%%.pg
%The whole process can be simplified via use of the
%{\it pushweapon\/}
%option.  When it is enabled, then using `{\tt w}' to wield something
%causes the currently wielded weapon to become your alternate weapon.
%So the sequence `{\tt w}', `{\tt w}' can be used to first wield the weapon you
%intend to be secondary, and then wield the one you want as primary
%which will push the first into secondary position.
処理全体は
{\it pushweapon\/}
オプションを使うことで単純化できる。
これを有効にして、何かを装備するのに 'w' を使うと、
現在装備している武器を予備の武器にする。
従って 'w', 'w' と続けると、最初に予備にしたい武器を装備し、
それから優先したい武器を装備すると、最初の武器が予備の武器に押し出される。

%%.pg
%When in two-weapon combat mode, using the `{\tt X}' command
%toggles back to single-weapon mode.
%Throwing or dropping either of the
%weapons or having one of them be stolen or destroyed will also make you
%revert to single-weapon combat.
二刀流モードのとき、`{\tt X}' コマンドを使うと、
単一武器モードに戻る。
どちらかの武器を投げたり落としたりするか、
どちらかが盗まれたり壊されたりしても、
単一武器モードに戻る。

%%.hn 2
%\subsection*{Armor (`{\tt [}')}
\subsection*{防具 (`{\tt [}')}

%%.pg
%Lots of unfriendly things lurk about; you need armor to protect
%yourself from their blows.  Some types of armor offer better
%protection than others.  Your armor class is a measure of this
%protection.  Armor class (AC) is measured as in AD\&D, with 10 being
%the equivalent of no armor, and lower numbers meaning better armor.
%Each suit of armor which exists in AD\&D gives the same protection in
%{\it NetHack}.  Here is an (incomplete) list of the armor classes provided by
%various suits of armor:
多くの非友好的なものが洞窟には潜んでいる。
そういったものからの攻撃から我が身を守るには防具が必要である。
防具の中には他の種類の防具よりも防御効果に優れた物がある。
Armor class(AC)(防御値)はこの防御効果の尺度である。
防御値 (AC)は AD\&D での場合と同じように評価される。
10 が防具なしの状態と等しく、数値が小さいほど優れた防具であることを示す。
AD\&D に存在する鎧はいずれも {\it NetHack\/} においても同じ防御効果を示す。
以下に(不完全ではあるが) 各種の鎧ごとに規定される防御値の一覧を示す。

\begin{center}
\begin{tabular}{lllll}
%dragon scale mail      & 1 & \makebox[20mm]{}  & plate mail            & 3\\
%crystal plate mail     & 3 &                   & bronze plate mail     & 4\\
%splint mail            & 4 &                   & banded mail           & 4\\
%dwarvish mithril-coat  & 4 &                   & elven mithril-coat    & 5\\
%chain mail             & 5 &                   & orcish chain mail     & 6\\
%scale mail             & 6 &                   & dragon scales         & 7\\
%studded leather armor  & 7 &                   & ring mail             & 7\\
%orcish ring mail       & 8 &                   & leather armor         & 8\\
%leather jacket         & 9 &                   & no armor              & 10\\
dragon scale mail(ドラゴンの鱗鎧)  & 1 & plate mail(鋼鉄の鎧)         & 3\\
crystal plate mail(水晶の鎧)       & 3 & bronze plate mail(青銅の鎧)  & 4\\
splint mail(鉄片の鎧)              & 4 & banded mail(帯金の鎧)        & 4\\
dwarvish mithril-coat(ドワーフのミスリル服) & 4 & elven mithril-coat(エルフのミスリル服) & 5\\
chain mail(鎖かたびら)             & 5 & orcish chain mail(オークの鎖かたびら) & 6\\
scale mail(鱗の鎧)                 & 6 & dragon scales (ドラゴンの鱗) & 7\\
studded leather armor(鋲付き皮鎧)  & 7 & ring mail(鉄環の鎧)          & 7\\
orcish ring mail(オークの鉄環の鎧) & 8 & leather armor(皮鎧)          & 8\\
leather jacket(皮ジャケット)       & 9 & no armor(鎧なし)             & 10\\
\end{tabular}
\end{center}

%%.pg
%\nd You can also wear other pieces of armor (for example, helmets, boots,
%shields, cloaks)
%to lower your armor class even further, but you can only wear one item
%of each category (one suit of armor, one cloak, one helmet, one
%shield, and so on) at a time.
\nd さらに他の防具(例: 兜、靴、楯、クローク) を身につけ、
防御値の値を小さくすることもできる。
ただし同じ範疇に入る物は 1 つしか身につけることはできない
(鎧は一式、クロークは 1 着、兜は 1 個、楯は 1 枚など)。

%%.pg
%If a piece of armor is enchanted, its armor protection will be better
%(or worse) than normal, and its ``plus'' (or minus) will subtract from
%your armor class.  For example, a +1 chain mail would give you
%better protection than normal chain mail, lowering your armor class one
%unit further to 4.  When you put on a piece of armor, you immediately
%find out the armor class and any ``plusses'' it provides.  Cursed
%pieces of armor usually have negative enchantments (minuses) in
%addition to being unremovable.
防具に魔力があればその防具の防御効果は通常の物よりも良く(もしくは悪く)なっており、
それに付いている「+記号」(あるいは-記号)の分だけ防御値の値が小さくなる。
例えば +1 鎖かたびらは、通常の鎖かたびらよりも防御効果が高く、
防御値は 1 単位分小さくなって 4 になる。
防具を身につけると直ちに防御値と「+記号」の値が分かる。
呪いがかけられた防具は通常負の魔力(-記号)を持っており、取りはずすことはできない。

%%.pg
%Many types of armor are subject to some kind of damage like rust.  Such
%damage can be repaired.  Some types of armor may inhibit spell casting.
多くの鎧は錆のようなある種のダメージを受けやすい。
このようなダメージは修復可能である。
魔法を唱えるのを妨げる防具もある。

%%.pg
%The commands to use armor are `{\tt W}' (wear) and `{\tt T}' (take off).
鎧を使うためのコマンドは `{\tt W}'(wear: 防具を付ける) と
`{\tt T}'(take off: 防具をはずす) である。
%The `{\tt A}' command can also be used to take off armor as well as other
%worn items.
`{\tt A}'(remove all: 全てをはずす)もまた他の身につけるもの同様、
防具をはずすのに使用できる。

%%.hn 2
%\subsection*{Food (`{\tt \%}')}
\subsection*{食料 (`{\tt \%}')}

%%.pg
%Food is necessary to survive.  If you go too long without eating you
%will faint, and eventually die of starvation.
%Some types of food will spoil, and become unhealthy to eat,
%if not protected.
%Food stored in ice boxes or tins (``cans'')
%will usually stay fresh, but ice boxes are heavy, and tins
%take a while to open.
生きのびるには食料が不可欠である.長い間食料を口にしないままでいると
やがて昏倒し、徐々に餓死への道をたどることになる。
保存の処置を取ってないと傷んでしまい、食べるには不衛生になる食料もある。
アイスボックスや缶に入っている食料は通常はいつまでも傷まないが、
アイスボックスは重く、缶は開けるのに少しばかり時間が必要である。

%%.pg
%When you kill monsters, they usually leave corpses which are also
%``food.''  Many, but not all, of these are edible; some also give you
%special powers when you eat them.  A good rule of thumb is ``you are
%what you eat.''
怪物を殺すと通常はその死体が残るが、これは「食料」にもなる。
すべてと言うわけではないが多くは食べられるし、
中には食べると特別な力がつくものもある。
大ざっぱに言えば「食べたものになる」ということである。

%%.pg
%Some character roles and some monsters are vegetarian.  Vegetarian monsters
%will typically never eat animal corpses, while vegetarian players can,
%but with some rather unpleasant side-effects.
菜食主義の職業や怪物もいる。
菜食主義の怪物は決して動物の死体を食べない。
一方菜食主義のプレイヤーは動物の死体を食べることができるが、
なんらかのありがたくない副作用がある。

%%.pg
%You can name one food item after something you like to eat with the
%{\it fruit\/} option.
{\it fruit\/} オプションによってあなたの好物の食料にちなんで
その名前を 1 つ設定することができる。

%%.pg
%The command to eat food is `{\tt e}'.
食料を食べるためのコマンドは `{\tt e}' である。

%%.hn 2
%\subsection*{Scrolls (`{\tt ?}')}
\subsection*{巻物 (`{\tt ?}')}

%%.pg
%Scrolls are labeled with various titles, probably chosen by ancient wizards
%for their amusement value (for example, ``READ ME,'' or ``THANX MAUD'' backwards).
%Scrolls disappear after you read them (except for blank ones, without
%magic spells on them).
巻物にはいろいろな表題が付けられているが、
これはおそらくいにしえの魔術師たちが暇潰しとして選んだのであろう
(例:``READ ME'' とか ``THANX MAUD'' の逆さ読みとか)。
巻物は読むと消滅する(ただし魔法の呪文の書かれていない白紙は消滅しない)。

%%.pg
%One of the most useful of these is the %
%{\it scroll of identify}, which
%can be used to determine what another object is, whether it is cursed or
%blessed, and how many uses it has left.  Some objects of subtle
%enchantment are difficult to identify without these.
これらの中で最も利用価値の高いものは{\it scroll of identify(識別の巻物)}である。
これは他の物についてそれが何であるか、
呪いがかけられているか祝福されているか、
あと何回効力を発揮できるかを確定できる。
得体の知れない魔力を持つ物の中にはこの巻物なしでは何であるか鑑定し難いものもある。

%%.pg
%A mail daemon may run up and deliver mail to you as a %
%{\it scroll of mail} (on versions compiled with this feature).
メイルデーモンが走ってきてあなたに{\it scroll of mail(手紙の巻物)}を渡してくれることもある
(この機能を有効にしてコンパイルされているバージョンの場合)。
%To use this feature on versions where {\it NetHack\/}
%mail delivery is triggered by electronic mail appearing in your system mailbox,
%you must let {\it NetHack\/} know where to look for new mail by setting the
%``MAIL'' environment variable to the file name of your mailbox.
{\it NetHack\/} のメール配達機能はシステムのメールボックスに電子メールが
やってきた場合に起動される。
この機能を使用するには環境変数``MAIL''にあなたのメールボックスの
ファイル名を設定して {\it NetHack\/} に新しいメールを探す場所を知らせる必要が
ある。
%You may also want to set the ``MAILREADER'' environment variable to the
%file name of your favorite reader, so {\it NetHack\/} can shell to it when you
%read the scroll.
また望むならば環境変数``MAILREADER''に使いたいメール受信プログラ
ムのファイル名を設定することもできる。
このとき {\it NetHack\/} からそのプログラムを起動してその巻物を読むことができる。
%On versions of {\it NetHack\/} where mail is randomly
%generated internal to the game, these environment variables are ignored.
メールがゲーム内においてランダムに生成されるバージョンの {\it NetHack\/} では
これらの環境変数は無視される。
%You can disable the mail daemon by turning off the
%{\it mail\/} option.
メイルデーモンは
{\it mail\/}
オプションでオフにできる。

%%.pg
%The command to read a scroll is `{\tt r}'.
巻物を読むためのコマンドは `{\tt r}' である。

%%.hn 2
%\subsection*{Potions (`{\tt !}')}
\subsection*{水薬 (`{\tt !}')}

%%.pg
%Potions are distinguished by the color of the liquid inside the flask.
%They disappear after you quaff them.
水薬は小びんに入っている液体の色によって区別される。
水薬は飲むと消滅してしまう。

%%.pg
%Clear potions are potions of water.  Sometimes these are
%blessed or cursed, resulting in holy or unholy water.  Holy water is
%the bane of the undead, so potions of holy water are good things to
%throw (`{\tt t}') at them.  It is also sometimes very useful to dip
%(``{\tt \#dip}'') an object into a potion.
透明な水薬は水である。これらはしばしば祝福されていたり呪いがかけられていたりして、
聖水や不浄な水になったりする。不死の怪物にとって聖水は有害なので、
聖水を不死の怪物に投げつける(`{\tt t}')と有効である。
聖水に他の物を浸す(``{\tt \#dip}'')のもたいへん有益である。

%%.pg
%The command to drink a potion is `{\tt q}' (quaff).
水薬を飲むためのコマンドは `{\tt q}'(quaff: 飲む) である。

%%.hn 2
%\subsection*{Wands (`{\tt /}')}
\subsection*{杖 (`{\tt /}')}

%%.pg
%Wands usually have multiple magical charges.
%Some types of wands require a direction in which to zap them.
%You can also
%zap them at yourself (just give a `{\tt .}' or `{\tt s}' for the direction).
%Be warned, however, for this is often unwise.
%Other types of wands
%don't require a direction.  The number of charges in a
%wand is random and decreases by one whenever you use it.
杖は通常複数回魔法効果を発揮する。
一部の杖は杖を振る際に方向を指示する必要がある。
杖を自分に向けて振ることもできる(方向として `{\tt .}' か `{\tt s}' を入力する)が、
これはしばしば愚かな行為となる。
その他の杖は方向は必要でない。
杖が効力を発揮する回数は杖ごとに不定で、杖を使う度にその回数は 1 ずつ減る。

%%.pg
%When the number of charges left in a wand becomes zero, attempts to use the
%wand will usually result in nothing happening.  Occasionally, however, it may
%be possible to squeeze the last few mana points from an otherwise spent wand,
%destroying it in the process.  A wand may be recharged by using suitable
%magic, but doing so runs the risk of causing it to explode.  The chance
%for such an explosion starts out very small and increases each time the
%wand is recharged.
杖の魔法の量がなくなると、通常は杖を使用しても何も起きない。
しかしながら時折、最後の一握りの魔力を使いきった杖から搾り取ることが可能である。
しかしこれをすると杖は壊れてしまう。
杖は適切な魔法によって再充填することができるが、
そうすると杖が爆発してしまう可能性がある。
爆発する可能性は最初は非常に小さく、同じ杖に何度も再充填するごとに大きくなる。

%%.pg
%In a truly desperate situation, when your back is up against the wall, you
%might decide to go for broke and break your wand.  This is not for the faint
%of heart.  Doing so will almost certainly cause a catastrophic release of
%magical energies.
自暴自棄な行為ではあるが、どうにもならなくなったときには杖を壊すという手もある。
これは弱気な行為ではない。それを行うことにより、
破壊的な魔法のエネルギーが解放されるからである。

%%.pg
%When you have fully identified a particular wand, inventory display will
%include additional information in parentheses: the number of times it has
%been recharged followed by a colon and then by its current number of charges.
ある杖を完全に鑑定した場合、持ち物の一覧では括弧の中に追加の情報が表示される。
それは再充填された回数、コロン、現在の使用可能な回数である。
%A current charge count of {\tt -1} is a special case indicating that the wand
%has been cancelled.
杖が無効化された場合には使用可能な回数は {\tt -1} になる。

%%.pg
%The command to use a wand is `{\tt z}' (zap).  To break one, use the `{\tt a}'
%(apply) command.
杖を使うためのコマンドは `{\tt z}'(zap: 杖を振る) である。壊すには `{\tt a}'
(apply: 道具を使う)である。

%%.hn 2
%\subsection*{Rings (`{\tt =}')}
\subsection*{指輪 (`{\tt =}')}

%%.pg
%Rings are very useful items, since they are relatively permanent
%magic, unlike the usually fleeting effects of potions, scrolls, and
%wands.
指輪は大変役に立つ物である。
というのも水薬や巻物や杖のように魔力が一過性にしか働かないものと違い、
指輪の魔力は比較的恒久的であるからだ。

%%.pg
%Putting on a ring activates its magic.  You can wear only two
%rings, one on each ring finger.
指輪をはめることによってその魔力は発揮される。
指輪は両手の薬指に 1 つずつ、計 2 つしかはめることはできない。

%%.pg
%Most rings also cause you to grow hungry more rapidly, the rate
%varying with the type of ring.
またたいていの指輪は身につけると腹の減り方が速くなる。
減り方は指輪の種類によって異なる。

%%.pg
%The commands to use rings are `{\tt P}' (put on) and `{\tt R}' (remove).
指輪を使うためのコマンドは `{\tt P}'(put on: 指輪をはめる) と
`{\tt R}'(remove: 指輪をはずす) である。

%%.hn 2
%\subsection*{Spellbooks (`{\tt +}')}
\subsection*{魔法書 (`{\tt +}')}

%%.pg
%Spellbooks are tomes of mighty magic.  When studied with the `{\tt r}' (read)
%command, they transfer to the reader the knowledge of a spell (and
%therefore eventually become unreadable)---unless the attempt backfires.
魔法書は強力な魔法を記した大きな本である。`{\tt r}'(read: 読む) コマンドで
学ぶと呪文の知識が読み手に転送される(したがって最後には読めなくなる)か、
さもなくばその試みは不測の結果に終る。
%Reading a cursed spellbook or one with mystic runes beyond
%your ken can be harmful to your health!
呪いがかけられた魔法書や知識の及ばない神秘の古代文字で記された魔法書を
読むと健康状態に害が及ぶ可能性がある!

%%.pg
%A spell (even when learned) can also backfire when you cast it.  If you
%attempt to cast a spell well above your experience level, or if you have
%little skill with the appropriate spell type, or cast it at
%a time when your luck is particularly bad, you can end up wasting both the
%energy and the time required in casting.
呪文を唱えたとき(学んでいる時も)にも魔力が逆流することもある。
あなたの経験レベルでは遠く及ばない高度な呪文を唱えようと試みたり、
対応する呪文タイプに関する技能が低かったり、
ひどくついてないときに呪文を唱えたりすると、
呪文を唱えるのに必要なエネルギーと時間を浪費しただけに終わってしまうこともある。

%%.pg
%Casting a spell calls forth magical energies and focuses them with
%your naked mind.  Some of the magical energy released comes from within
%you.
%Casting temporarily drains your magical power, which will slowly be
%recovered, and causes you to need additional food.
%Casting of spells also requires practice.  With practice, your
%skill in each category of spell casting will improve.  Over time, however,
%your memory of each spell will dim, and you will need to relearn it.
呪文を唱えることは、魔法のエネルギーを呼び起こしてそれを精神そのもの
に集中させることである。
あなた自身の内部から出る魔法のエネルギーが開放される。
唱えると一時的に魔法の力が吸い出される;
これはゆっくりと回復し、その際に追加の食料が必要になる。
呪文を唱えるには練習が必要である。
練習していれば、それぞれの呪文領域の技能は向上していく。
しかしながら、時間が経つにつれてそれぞれの呪文に関する記憶は薄れ、
もう一度呪文を学ぶ必要がでてくるだろう。

%%.pg
%Some spells require a direction in which to cast them, similar to wands.
%To cast one at yourself, just give a `{\tt .}' or `{\tt s}' for the direction.
%A few spells require you to pick a target location rather than just specify
%a particular direction.
%Other spells don't require any direction or target.
一部の呪文は、杖と同様、呪文を唱える方向を指定する必要がある。
自分自身に対して唱えるには、方向として`{\tt .}'または`{\tt s}'を指定する。
いくつかの呪文は単なる方向ではなく場所を指定する必要がある。
その他の呪文は方向や場所を指定する必要はない。

%%.pg
%Just as weapons are divided into groups in which a character can become
%proficient (to varying degrees), spells are similarly grouped.
%Successfully casting a spell exercises its skill group; using the
%``{\tt \#enhance}'' command to advance a sufficiently exercised skill
%will affect all spells within the group.  Advanced skill may increase the
%potency of spells, reduce their risk of failure during casting attempts,
%and improve the accuracy of the estimate for how much longer they will
%be retained in your memory.
武器が習熟度の面でグループ分けされているのと同様、呪文もグループ分けされている。
呪文を唱えるのに成功すると、そのグループを訓練したことになる;
十分に訓練したスキルを向上させる `{\tt \#enhance}' コマンドはそのグループの全ての
呪文に影響する。
スキルが向上すると呪文の性能が向上し、詠唱時の失敗のリスクが減り、
どれくらい長い間呪文が記憶に残っているかの推定の精度が向上する。
%Skill slots are shared with weapons skills.  (See also the section on
%``Weapon proficiency''.)
技能スロットは武器スキルと共有する。(``武器の技量''の項も参照のこと)

%%.pg
%Casting a spell also requires flexible movement, and wearing various types
%of armor may interfere with that.
魔法を唱えるには自由に移動できなければならない。また、いくつかの種類の鎧を
着ていると魔法を唱えるのを妨げるだろう。

%%.pg
%The command to read a spellbook is the same as for scrolls, `{\tt r}' (read).
%The `{\tt +}' command lists each spell you know along with its level, skill
%category, chance of failure when casting, and an estimate of how strongly
%it is remembered.
%The `{\tt Z}' (cast) command casts a spell.
魔法書を読むためのコマンドは巻物を読むときと同じく `{\tt r}'(read: 読む) である。
`{\tt +}' コマンドにより、知っている呪文毎に、レベル,スキル分野,成功率、どれくらい
強く記憶しているかが一覧表示される。
`{\tt Z}'(cast: 呪文を唱える) コマンドにより呪文を唱える。

%%.hn 2
%\subsection*{Tools (`{\tt (}')}
\subsection*{道具 (`{\tt (}')}

%%.pg
%Tools are miscellaneous objects with various purposes.  Some tools
%have a limited number of uses, akin to wand charges.  For example, lamps burn
%out after a while.  Other tools are containers, which objects can
%be placed into or taken out of.
道具はいろいろな目的に使う種々雑多な物である。杖などと同じように使用
回数に制限のある物もある。例えばランプはしばらくたつと燃え尽きてしまう。
その他の道具の中には容器も含まれており、物を出し入れすることができる。

%%.pg
%The command to use tools is `{\tt a}' (apply).
道具を使うためのコマンドは `{\tt a}'(apply: 道具を用いる) である。

%%.hn 3
%\subsection*{Containers}
\subsection*{箱}

%%.pg
%You may encounter bags, boxes, and chests in your travels.  A tool of
%this sort can be opened with the ``{\tt \#loot}'' extended command when
%you are standing on top of it (that is, on the same floor spot),
%or with the `{\tt a}' (apply) command when you are carrying it.  However,
%chests are often locked, and are in any case unwieldy objects.
冒険の途中で鞄や箱やひつに出くわすこともあるであろう。
これらは置いてある場所に立っている場合には
拡張コマンド「{\tt \#loot}」によって、
また持っているときには `{\tt a}'(apply: 道具を用いる) コマンドで開けることができる。
しかしながらひつにはしばしば錠がかかっており、大抵は重くて運びにくい物体である。
%You must set one down before unlocking it by
%using a key or lock-picking tool with the `{\tt a}' (apply) command,
%by kicking it with the `{\tt \^{}D}' command,
%or by using a weapon to force the lock with the ``{\tt \#force}''
%extended command.
ひつは手に持つことはできない物なので、
錠をはずす(`{\tt a}'(apply: 道具を用いる) コマンドで鍵や錠をはずす道具を使ったり、
`{\tt \^{}D}'コマンドで蹴とばしたり、
拡張コマンド``{\tt \#force}''により武器を使ってこじ開けるなどの方法による)には
床に置かなくてはならない。

%%.pg
%Some chests are trapped, causing nasty things to happen when you
%unlock or open them.  You can check for and try to deactivate traps
%with the ``{\tt \#untrap}'' extended command.
ひつには罠が仕掛けられているものもあり、錠をはずしたりふたを開けたり
したときに不快な出来事が起きる。拡張コマンド``{\tt \#untrap}''によってチェックを
して罠が無効になるよう試みることができる。

%%.hn 2
%\subsection*{Amulets (`{\tt "}')}
\subsection*{魔除け (`{\tt "}')}

%%.pg
%Amulets are very similar to rings, and often more powerful.  Like
%rings, amulets have various magical properties, some beneficial,
%some harmful, which are activated by putting them on.
魔除けは指輪と大変よく似ており、しばしばもっと大きな効力を持っている。
指輪と同じように魔除けにはいろいろな魔法の特性があり、有益なものも有害
なものもあって、身につけることにより効力を発揮する。

%%.pg
%Only one amulet may be worn at a time, around your neck.
魔除けは一つだけしか首にかけることができない。

%%.pg
%The commands to use amulets are the same as for rings, `{\tt P}' (put on)
%and `{\tt R}' (remove).
魔除けを使うためのコマンドは指輪の場合と同じく `{\tt P}'(put on: 指輪を
はめる) と `{\tt R}'(remove: 指輪をはずす) である。

%%.hn 2
%\subsection*{Gems (`{\tt *}')}
\subsection*{宝石 (`{\tt *}')}

%%.pg
%Some gems are valuable, and can be sold for a lot of gold.  They are also
%a far more efficient way of carrying your riches.  Valuable gems increase
%your score if you bring them with you when you exit.
宝石の中には価値のあるものもあり、高値で売れる。
宝石は財産を持ち歩く方法としてとして極めて効果的なやり方である。
価値のある宝石を脱出時に所持していれば得点に加算される。

%%.pg
%Other small rocks are also categorized as gems, but they are much less
%valuable.  All rocks, however, can be used as projectile weapons (if you
%have a sling).  In the most desperate of cases, you can still throw them
%by hand.
他の小さな石も宝石に分類されるが、その価値はほとんどない。
しかし、全ての石は(もしスリングを持っているなら)弾として有効である。
窮余に陥いったら手で投げることも可能である。

%%.hn 2
%\subsection*{Large rocks (`{\tt `}')}
\subsection*{大きな岩 (`{\tt `}')}
%%.pg
%Statues and boulders are not particularly useful, and are generally
%heavy.  It is rumored that some statues are not what they seem.
彫像や巨石は特に有用ではないし、一般的に重いものである。
見かけとは異なった彫像もあるという噂である。

%%.pg
%Very large humanoids (giants and their ilk) have been known to use boulders
%as weapons.
巨大なヒューマノイド(巨人やその仲間)は巨石を武器として用いることができる。

%%.pg
%For some configurations of the program, statues are no longer shown
%as `{\tt `}'
%but by the letter representing the monster they depict instead.
一部のプログラム設定では、彫像はもはや
`{\tt `}'
ではなく、表現されている怪物を意味する文字で表される。

%%.hn 2
%\subsection*{Gold (`{\tt \$}')}
\subsection*{金 (`{\tt \$}')}

%%.pg
%Gold adds to your score, and you can buy things in shops with it.
金は得点に加算され、また店では金で物を買うことができる。
%There are a number
%of monsters in the dungeon that may be influenced by the amount of gold
%you are carrying (shopkeepers aside).
迷宮にはあなたの持つお金に影響されるかもしれない数多くの怪物が
いる(店主は別として)。

%%.hn 2
%\subsection*{Persistence of Objects}
\subsection*{物の永続性}

%%.pg
%Normally, if you have seen an object at a particular map location and
%move to another location where you can't directly see that object any
%more, it will continue to be displayed on your map.
%That remains the case even if it is not actually there any
%more---perhaps a monster has picked it up or it has rotted
%away---until you can see or feel that location again.
%One notable exception is that if the object gets covered by the
%``remembered, unseen monster'' marker.
通常、あなたが地図上のある特定の位置である物を見て、
物がもう直接見えない位置に移動した場合、
それは地図上に表示され続ける。
これは、--- おそらく怪物が拾ったか腐ってしまったかして --- 実際には
すでにそこになくても、あなたがその場所をもう一度見るか感じるかするまで
残り続ける。
注意するべき例外の一つは、その物が「覚えているけれども見えない怪物」
マーカで覆われた場合である。
%When that marker is later removed
%after you've verified that no monster is there, you will forget that
%there was any object there regardless of whether the unseen monster
%actually took the object.
%If the object is still there, then once you see or feel that location
%again you will re-discover the object and resume remembering it.
あなたがそこに怪物がいないことを確認してマーカが消えると、
見えない怪物が実際にその物を拾ったかどうかに関わらず、
あなたはそこにあった物を忘れる。
物がそこにあり続けている場合、再びその場所を見たり感知したりすると
物を再び発見してそれを記憶し続ける。

%%.pg
%The situation is the same for a pile of objects, except that only the
%top item of the pile is displayed.
%The
%{\it hilite\verb+_+pile\/}
%option can be enabled in order to show an item differently when it is
%the top one of a pile.
この状況は物の山でも同様だが、山の一番上の物のみが表示される。
山の上にいる敵に物を違う方法で表示するために、
{\it hilite\verb+_+pile\/} コマンドを有効に出来る。

%%.hn 1
%\section{Conduct}
\section{制限プレイ}

%%.pg
%As if winning {\it NetHack\/} were not difficult enough, certain players
%seek to challenge themselves by imposing restrictions on the
%way they play the game.  The game automatically tracks some of
%these challenges, which can be checked at any time with the {\tt \#conduct}
%command or at the end of the game.  When you perform an action which
%breaks a challenge, it will no longer be listed.  This gives
%players extra ``bragging rights'' for winning the game with these
%challenges.  Note that it is perfectly acceptable to win the game
%without resorting to these restrictions and that it is unusual for
%players to adhere to challenges the first time they win the game.
単に {\it NetHack\/} に勝利するだけでは満足できない一部のプレイヤーは、
プレイに制限を設けることに挑戦している。
これらの挑戦の一部はゲームによって自動的に記録され、
ゲーム中いつでも {\tt \#conduct} コマンドで確認できる。
またゲーム終了時にも表示される。
挑戦に反するような行動を取ると、もはやその挑戦は表示されない。l
これらの挑戦に勝利することによって、プレイヤーは追加の``自慢する権利''を
得られる。
なお、これらの制限に従うことなくゲームに勝利することも
完全に有効であり、最初にゲームに勝利するときは普通これらの挑戦は関係ない。

%%.pg
%Several of the challenges are related to eating behavior.  The most
%difficult of these is the foodless challenge.  Although creatures
%can survive long periods of time without food, there is a physiological
%need for water; thus there is no restriction on drinking beverages,
%even if they provide some minor food benefits.
%Calling upon your god for help with starvation does
%not violate any food challenges either.
挑戦のいくつかは食べ物に関するものである。
最も困難な挑戦は食料なしの挑戦である。
生物は食料なしでも長い間生存できるが、水は必要である。
従って、飲み物に対しては何の制限もない。例え栄養分があってもである。
神に祈って飢えをしのぐことはいかなる食べ物に関する挑戦にも違反しない。

%%.pg
%A strict vegan diet is one which avoids any food derived from animals.
%The primary source of nutrition is fruits and vegetables.  The
%corpses and tins of blobs (`b'), jellies (`j'), and fungi (`F') are
%also considered to be vegetable matter.  Certain human
%food is prepared without animal products; namely, lembas wafers, cram
%rations, food rations (gunyoki), K-rations, and C-rations.
%Metal or another normally indigestible material eaten while polymorphed
%into a creature that can digest it is also considered vegan food.
%Note however that eating such items still counts against foodless conduct.
厳密な菜食主義者は動物から作られた食べ物を食べない。
基本的な栄養源は果物と野菜である。
ブロッブ(`b')、ゼリー(`j')、細菌(`F')も野菜とみなされる。
人間の食料にも動物を使っていないものがある。
レンバス、クラム、「食料」(丸薬)、Kレーション、Cレーションがそうである。
他の生物に変化しているときは、金属などの普通消化できない物質も
厳密な菜食主義者用の食料として扱われる。
しかし、これらの物も食料なしの挑戦には反することに注意すること。

%%.pg
%Vegetarians do not eat animals;
%however, they are less selective about eating animal byproducts than vegans.
%In addition to the vegan items listed above, they may eat any kind
%of pudding (`P') other than the black puddings,
%eggs and food made from eggs (fortune cookies and pancakes),
%food made with milk (cream pies and candy bars), and lumps of
%royal jelly.  Monks are expected to observe a vegetarian diet.
菜食主義者は動物を食べない。
しかし、動物からの副生産物を食べることは厳密な菜食主義者ほどは厳しくない。
上記の厳密な菜食主義者が食べられるものに追加して、
ブラックプリン以外のプリン(`P')、卵、卵から作られた食べ物
(占いクッキーとパンケーキ)、牛乳から造られた食べ物
(クリームパイとキャンディーバー)、ロイヤルゼリーを食べることができる。
モンクは菜食主義者とみなされる。

%%.pg
%Eating any kind of meat violates the vegetarian, vegan, and foodless
%conducts.  This includes tripe rations, the corpses or tins of any
%monsters not mentioned above, and the various other chunks of meat
%found in the dungeon.  Swallowing and digesting a monster while polymorphed
%is treated as if you ate the creature's corpse.
%Eating leather, dragon hide, or bone items while
%polymorphed into a creature that can digest it, or eating monster brains
%while polymorphed into a mind flayer, is considered eating
%an animal, although wax is only an animal byproduct.
肉を食べることは厳密な菜食主義者、菜食主義者、食料なしの挑戦に反する。
これにはほし肉、上記以外の怪物の死体や缶詰、迷宮で見つかる
その他の肉のかたまりを含む。
変化中に「飲み込んで消化」攻撃をかけるのはその怪物の死体を
食べたものとみなす。
皮、竜鱗、骨でできた物体を、これらを消化できる怪物に変化して食べたり、
マインドフレイヤーに変化して脳を食べることは動物を食べたとみなす。
ただし、蜜蝋は動物の副生成物とみなす。

%%.pg
%Regardless of conduct, there will be some items which are indigestible,
%and others which are hazardous to eat.  Using a swallow-and-digest
%attack against a monster is equivalent to eating the monster's corpse.
%Please note that the term ``vegan'' is used here only in the context of
%diet.  You are still free to choose not to use or wear items derived
%from animals (e.g. leather, dragon hide, bone, horns, coral), but the
%game will not keep track of this for you.  Also note that ``milky''
%potions may be a translucent white, but they do not contain milk,
%so they are compatible with a vegan diet.  Slime molds or
%player-defined ``fruits'', although they could be anything
%from ``cherries'' to ``pork chops'', are also assumed to be vegan.
挑戦に関わらず、消化不可能な物もあり、食べると危険なものもある。
怪物に対して「飲み込んで消化」攻撃をかけることは、
その怪物の死体を食べたものとみなす。
「厳格な菜食主義者(``vegan'')」という言葉は食べ物に関する文脈でのみ
用いられることに注意してほしい。
動物から作られた物(皮、竜の鱗、骨、角、さんご)を使ったり身につけたり
することは制限に違反しない。
また、「ミルク色の」水薬は不透明な白色ではあるが、牛乳を含んでいるわけ
ではないので、厳密な菜食主義者の制限に違反しない。
スライムモルドやプレイヤーが定義した``fruits''は、
例え「さくらんぼ」であろうが「ポークチョップ」であろうが、
厳密な菜食主義者の制限に違反しない。

%%.pg
%An atheist is one who rejects religion.  This means that you cannot
%{\tt \#pray}, {\tt \#offer} sacrifices to any god,
%{\tt \#turn} undead, or {\tt \#chat} with a priest.
%Particularly selective readers may argue that playing Monk or Priest
%characters should violate this conduct; that is a choice left to the
%player.  Offering the Amulet of Yendor to your god is necessary to
%win the game and is not counted against this conduct.  You are also
%not penalized for being spoken to by an angry god, priest(ess), or
%other religious figure; a true atheist would hear the words but
%attach no special meaning to them.
無神論者は宗教を否定する。つまり、{\tt \#pray}(祈る), {\tt \#offer}
(いけにえを捧げる),
{\tt \#turn} (不死のものを戻す), 僧侶に対する {\tt \#chat}(話す)ことはできない。
一部の読者はモンクや僧侶でプレイするのもこの制限に反していると主張するかもしれないが、
ここではプレイヤーの選択の余地を残してある。
イェンダーの魔除けを神に捧げるのはゲームに勝利するために必要なので
制限には反しないものとする。
また、怒った神や僧侶やその他の宗教的存在が話す言葉を聞いても制限には反しない。
真の無神論者は言葉は聞くが,そこになんら特別な意味を見出さない。

%%.pg
%Most players fight with a wielded weapon (or tool intended to be
%wielded as a weapon).  Another challenge is to win the game without
%using such a wielded weapon.  You are still permitted to throw,
%fire, and kick weapons; use a wand, spell, or other type of item;
%or fight with your hands and feet.
ほとんどのプレイヤーは武器(あるいは武器として装備することを
考慮している道具)を手に戦う。
挑戦の一つはこのような武器を手にして使わずにゲームに勝利することである。
武器を投げたり、発射したり、蹴ったりするのは許される。
また、杖、呪文、あるいはその他のアイテムを使ったり、
素手や脚で戦うことも許される。

%%.pg
%In {\it NetHack}, a pacifist refuses to cause the death of any other monster
%(i.e. if you would get experience for the death).  This is a particularly
%difficult challenge, although it is still possible to gain experience
%by other means.
{\it NetHack} において、平和主義者は他の怪物を殺してはならない(言い換えると、
怪物の死によって経験値を得てはならない)。
これは特に難しい挑戦であるが、他の手段によって経験値を得ることは可能である。

%%.pg
%An illiterate character cannot read or write.  This includes reading
%a scroll, spellbook, fortune cookie message, or t-shirt; writing a
%scroll; or making an engraving of anything other than a single ``X'' (the
%traditional signature of an illiterate person).  Reading an engraving,
%or any item that is absolutely necessary to win the game, is not counted
%against this conduct.  The identity of scrolls and spellbooks (and
%knowledge of spells) in your starting inventory is assumed to be
%learned from your teachers prior to the start of the game and isn't
%counted.
文盲者は読み書きができない。これには以下の行為が含まれる。
巻き物を読む、魔法書を読む、占いクッキーのメッセージを読む、Tシャツの文字を読む、
巻き物を書く、``X''一文字(文盲者の伝統的なサイン)以外の文字を刻む。
刻んである文字を読むことと、ゲームに勝利するために絶対に必要なアイテムを
読むことは制限には反しないものとする。
ゲーム開始時に持っている巻き物と魔法書の内容を知っていることおよび
呪文の知識は、ゲーム開始以前に教師から教わったものとみなし、
制限には反しないものとする。

%%.pg
%There are several other challenges tracked by the game.  It is possible
%to eliminate one or more species of monsters by genocide; playing without
%this feature is considered a challenge.  When the game offers you an
%opportunity to genocide monsters, you may respond with the monster type
%``none'' if you want to decline.  You can change the form of an item into
%another item of the same type (``polypiling'') or the form of your own
%body into another creature (``polyself'') by wand, spell, or potion of
%polymorph; avoiding these effects are each considered challenges.
%Polymorphing monsters, including pets, does not break either of these
%challenges.
%Finally, you may sometimes receive wishes; a game without an attempt to
%wish for any items is a challenge, as is a game without wishing for
%an artifact (even if the artifact immediately disappears).  When the
%game offers you an opportunity to make a wish for an item, you may
%choose ``nothing'' if you want to decline.
その他にいくつかの細かいゲームとして記録される挑戦がある。
怪物は虐殺することが可能である。これをせずにプレイするのも挑戦とみなす。
怪物を虐殺する機会が与えられたとき、
``none'' (``なし'')と答えると、虐殺しないままにすることができる。
あなたはアイテムを同じグループの他のアイテムに変化させたり (``polypiling'')、
変化の杖・呪文・薬で自分の体を他の生物に変えたり(``polyself'')できる。
これらの効果を用いないことはそれぞれ挑戦とみなす。
最後に、あなたはときどき願いをかなえてもらうことがある。
なんのアイテムも願わずにプレイするのは一つの挑戦であり、
聖器を願わない(たとえ直ちに消滅したとしても)でプレイするのも一つの挑戦である。
アイテムを願う機会を与えられたとき、
``nothing'' (``なし'') と答えると、願わないままにすることができる。

%%.hn 1
%\section{Options}
\section{オプション}

%%.pg
%Due to variations in personal tastes and conceptions of how {\it NetHack\/}
%should do things, there are options you can set to change how {\it NetHack\/}
%behaves.
人にはそれぞれいろいろな好みがあり {\it NetHack\/} の遊び方もそれぞれ異なっているの
で、{\it NetHack\/} の振る舞いを変更するため設定できるオプションがある。

%%.hn 2
%\subsection*{Setting the options}
\subsection*{オプションの設定}

%%.pg
%Options may be set in a number of ways.  Within the game, the `{\tt O}'
%command allows you to view all options and change most of them.
%You can also set options automatically by placing them in a configuration
%file, or in the ``NETHACKOPTIONS'' environment variable.
%Some versions of {\it NetHack\/} also have front-end programs that allow
%you to set options before starting the game or a global configuration
%for system administrators.
オプションを設定する方法にはいくつかある。ゲーム中に `{\tt O}' コマンドを
使うことによって全てのオプションを見ることができ、そのほとんどを変更で
きる。また、設定ファイルや環境変数``NETHACKOPTIONS''で自動的に設定する
こともできる。
{\it NetHack\/} のバージョンによってはゲーム開始前にオプションや
システム管理者のためのグローバルな設定をすることができるプログラムが
ついていることもある。

%%.hn 2
%\subsection*{Using a configuration file}
\subsection*{設定ファイルを使う}

%%.pg
%The default name and location of the configuration file varies on different
%operating systems.\\
設定ファイルの標準設定のファイル名と位置は OS によって異なる。\\

%%.lp ""
%On Unix, Linux and Mac OS X it is \mbox{``.nethackrc''} in the user's home
%directory. The file may not exist, but it is a normal ASCII text file and
%can be created with any text editor.\\
Unix, Linux, Mac OS X では、これはユーザーのホームディレクトリの
\mbox{``.nethackrc''} である。
このファイルは存在していないかも知れないが、これは通常の
ASCII テキストファイルで、任意のテキストエディタで作ることができる。\\

%%.lp ""
%On Windows, it is \mbox{``.nethackrc''} in the folder
%\mbox{{``\%USERPROFILE\%\textbackslash NetHack\textbackslash 3.6''}}. The
%file may not exist, but it is a normal ASCII text file and can be created
%with any text editor.
Windowsでは、
\mbox{{``\%USERPROFILE\%\textbackslash NetHack\textbackslash 3.6''}} フォルダの
\mbox{``.nethackrc''} である。
このファイルは存在していないかも知れないが、これは通常の
ASCII テキストファイルで、任意のテキストエディタで作ることができる。
%After runing {\it NetHack\/} for the first time, you should find a default
%template for ths configuration file named \mbox{``.nethackrc.template''} in
%\mbox{{``\%USERPROFILE\%\textbackslash NetHack\textbackslash 3.6''}}.
%If you had not created the configuration file, {\it NetHack\/} will create
%the configuration file for you using the default template file.
{\it NetHack\/} の初回実行後、
\mbox{{``\%USERPROFILE\%\textbackslash NetHack\textbackslash 3.6''}} に
\mbox{``.nethackrc.template''} というファイル名で、
設定ファイルのデフォルトテンプレートを見つけられるはずである。
設定ファイルを作成していない場合、
{\it NetHack\/} はデフォルトテンプレートファイルを使って設定ファイルを
作成する。

%%.lp ""
%On MS-DOS it is \mbox{``defaults.nh''} in the same folder as
%\mbox{{\it nethack.exe\/}}.\\
MS-DOS では、\mbox{{\it nethack.exe\/}} と
同じフォルダの \mbox{``defaults.nh''} である。\\

%%.lp ""
%Any line in the configuration file starting with `{\tt \#}' is treated as a comment.
%Empty lines are ignored. Any line beginning with `{\tt [}' and ending in `{\tt ]}' is considered a section
%marker. The text between the square brackets is the section name.
%Lines after a section marker belong to that section, and are
%ignored unless a CHOOSE -statement was used to select that section.
%Section names are case insensitive.
設定ファイルの中の`{\tt \#}'で始まる行はコメントとして扱われる。
空行は無視される。
`{\tt [}' で始まり `{\tt ]}' で終わる行は節マーカとして扱われる。
大かっこで囲まれたテキストは節の名前である。
節マーカ以降の行はその節に属し、その節を選択するために
CHOOSE 文が使われない限り無視される。
節の名前は大文字小文字を無視する。

%%.pg
%You can use different configuration statements in the file, some
%of which can be used multiple times.
%In general, the statements are
%written in capital letters, followed by an equals sign, followed by
%settings particular to that statement.
ファイル中に異なる設定文を使うことができ、そのいくつかは複数回
使われる。
一般的に、文は英大文字で書かれ、等号が続き、その文に特有の設定が続く。

%%.pg
%Here is a list of allowed statements:
以下は指定できる文の一覧である:

%%.lp
\blist{}
\item[\bb{OPTIONS}]
%There are two types of options, boolean and compound options.
%Boolean options toggle a setting on or off, while compound options
%take more diverse values.
%Prefix a boolean option with `no' or `!' to turn it off.
%For compound options, the option name and value are separated by a colon.
%Some options are persistent, and apply only to new games.
%You can specify multiple OPTIONS statements, and multiple options
%separated by commas in a single OPTIONS statement.
%(Comma separated options are processed from right to left.)
真偽値オプションと複合オプションの2種類のオプションがある。
真偽値オプションは設定のオンとオフを切り替える一方、
複合オプションはより様々な値を取る。
真偽値オプションに `no' または `!' を前置するとオフになる。
複合オプションでは、オプション名と値はコロンで区切られる。
一部のオプションは永続し、ゲーム開始時にのみ適用される。
複数のOPTIONS文を指定したり、一つのOPTIONS文の中でカンマで区切ることで
複数のオプションを指定したりできる。
(カンマで区切られたオプションは右から左に処理される。)

%%.lp ""
%Example:
例:
%%.sd
\begin{verbatim}
    OPTIONS=dogname:Fido
    OPTIONS=!legacy,autopickup,pickup_types:$"=/!?+
\end{verbatim}
%%.ed

%%.lp
\item[\bb{HACKDIR}]
%Default location of files {\it NetHack\/} needs. On Windows HACKDIR
%defaults to the location of the {\it NetHack.exe\/} or {\it NetHackw.exe\/} file
%so setting HACKDIR to override that is not usually necessary or recommended.
{\it NetHack\/} が必要なファイルのデフォルトの位置。
Windows での HACKDIR のデフォルトは {\it NetHack.exe\/} または
{\it NetHackw.exe\/} ファイルの位置なので、
HACKDIR を上書きするように設定するのは通常不要で非推奨である。
%%.lp
\item[\bb{LEVELDIR}]
%The location that in-progress level files are stored. Defaults to HACKDIR,
%must be writable.
進行中のレベルファイルを保管する位置。
デフォルトは HACKDIR で、書き込み可能でなければならない。
%%.lp
\item[\bb{SAVEDIR}]
%The location where saved games are kept. Defaults to HACKDIR, must be
%writable.
保存されたゲームを保持する位置。
デフォルトは HACKDIR で、書き込み可能でなければならない。
%%.lp
\item[\bb{BONESDIR}]
%The location that bones files are kept. Defaults to HACKDIR, must be
%writable.
骨ファイルを保持する位置。
デフォルトは HACKDIR で、書き込み可能でなければならない。
%%.lp
\item[\bb{LOCKDIR}]
%The location that file synchronization locks are stored. Defaults to
%HACKDIR, must be writable.
同期ロックを保管する位置。
デフォルトは HACKDIR で、書き込み可能でなければならない。
%%.lp
\item[\bb{TROUBLEDIR}]
%The location that a record of game aborts and self-diagnosed game problems
%is kept. Defaults to HACKDIR, must be writable.
ゲーム中段の記録とゲームの問題の自己診断を保持する位置。
デフォルトは HACKDIR で、書き込み可能でなければならない。
%%
%% config file entries beyond this point are shown alphabetically
%%
%%.lp
\item[\bb{AUTOCOMPLETE}]
%Enable or disable an extended command autocompletion.
%Autocompletion has no effect for the X11 windowport.
%You can specify multiple autocompletions. To enable
%autocompletion, list the extended command. Prefix the
%command with ``{{\tt !}}'' to disable the autocompletion
%for that command.
拡張コマンドの自動補完をオンまたはオフにする。
自動補完は X11 ウィンドウポートでは無効である。
複数の自動補完を指定できる。
自動補完をオンにするには、拡張コマンドの一覧を書く。
コマンドに ``{{\tt !}}'' を前置すると、そのコマンドの自動補完を
オフにする。

%%.lp ""
%Example:
例:
%%.sd
\begin{verbatim}
   AUTOCOMPLETE=zap,!annotate
\end{verbatim}
%%.ed

%%.lp
\item[\bb{AUTOPICKUP\_EXCEPTION}]
%Set exceptions to the {{\it pickup\_types\/}}
%option. See the ``Configuring Autopickup Exceptions'' section.
{{\it pickup\_types\/}} オプションの例外を設定する。
「自動拾い例外の設定」の節を参照のこと。
%%.lp
\item[\bb{BINDINGS}]
%Change the key bindings of some special keys, menu accelerators, or
%extended commands. You can specify multiple bindings. Format is key
%followed by the command, separated by a colon.
%See the ``Changing Key Bindings`` section for more information.
一部の特殊キー、メニューアクセラレータ、拡張コマンドのキー配置を
変更する。
複数の配置を指定できる。
形式は、キーの後に、転んで区切られたコマンドが続く。
さらなる情報については ``Changing Key Bindings`` 節を参照のこと。

%%.lp ""
%Example:
例:
%%.sd
\begin{verbatim}
   BIND=^X:getpos.autodescribe
\end{verbatim}
%%.ed

%%.lp
\item[\bb{CHOOSE}]
%Chooses at random one of the comma-separated parameters as an active
%section name. Lines in other sections are ignored.
カンマで区切られたパラメータの一つを有効なセクション名としてランダムに選ぶ。
他のセクションの行は無視される。

%%.lp ""
%Example:
例:
%%.sd
\begin{verbatim}
   OPTIONS=color
   CHOOSE=char A,char B
   [char A]
   OPTIONS=role:arc,race:dwa,align:law,gender:fem
   [char B]
   OPTIONS=role:wiz,race:elf,align:cha,gender:mal
\end{verbatim}
%%.ed

%%.lp
\item[\bb{MENUCOLOR}]
%Highlight menu lines with different colors.
%See the ``Configuring Menu Colors`` section.
メニュー行を異なった色でハイライトする。
``Configuring Menu Colors`` 節を参照のこと。
%%.lp
\item[\bb{MSGTYPE}]
%Change the way messages are shown in the top status line.
%See the ``Configuring Message Types`` section.
上部ステータス行に表示されるメッセージがどのように表示されるかを変更する。
``Configuring Message Types`` 節を参照のこと。
%%.lp
\item[\bb{ROGUESYMBOLS}]
%Custom symbols for for the rogue level's symbol set.
%See {\it SYMBOLS} below.
ローグレベルのシンボル集合のためのカスタムシンボル。
後述する {\it SYMBOLS} を参照のこと。
%%.lp
\item[\bb{SOUND}]
%Define a sound mapping.
%See the ``Configuring User Sounds'' section.
効果音のマッピングを定義する。
``Configuring User Sounds'' 節を参照のこと。
%%.lp
\item[\bb{SOUNDDIR}]
%Define the directory that contains the sound files.
%See the ``Configuring User Sounds'' section.
効果音ファイルがあるディレクトリを定義する。
``Configuring User Sounds'' 節を参照のこと。
%%.lp
\item[\bb{SYMBOLS}]
%Override one or more symbols in the symbol set used for all dungeon
%levels except for the special rogue level.
%See the ``Modifying {\it NetHack\/} Symbols'' section.
特殊なローグレベル以外の全てのダンジョンレベルで使われるシンボル集合の
いくつかのシンボルを上書きする。
``Modifying {\it NetHack\/} Symbols'' 節を参照のこと。
%%.pg

%%.lp ""
%Example:
例:
%%.sd
\begin{verbatim}
   # replace small punctuation (tick marks) with digits
   SYMBOLS=S_boulder:0,S_golem:7
\end{verbatim}
%%.ed

%%.lp
\item[\bb{WIZKIT}]
%Debug mode only:  extra items to add to initial inventory.
%Value is the name of a text file containing a list of item names,
%one per line, up to a maximum of 128 lines.
%Each line is processed by the function that handles wishing.
デバッグモードのみ: 初期持物に追加される物。
値は物の名前を 1 行一つ(最大 128 行)で書いたテキストファイルの名前である。
各行は願いを扱う機能によって処理される。

%%.lp ""
%Example:
例:
%%.sd
\begin{verbatim}
   WIZKIT=~/wizkit.txt
\end{verbatim}
%%.ed
\elist

%%.lp ""
%%.pg
%Here is an example of configuration file contents:
以下は設定ファイルの中身の例である:
%%.sd
\begin{verbatim}
    # Set your character's role, race, gender, and alignment.
    OPTIONS=role:Valkyrie, race:Human, gender:female, align:lawful

    # Turn on autopickup, set automatically picked up object types
    OPTIONS=autopickup,pickup_types:$"=/!?+

    # Map customization
    OPTIONS=color           # Display things in color if possible
    OPTIONS=lit_corridor    # Show lit corridors differently
    OPTIONS=hilite_pet,hilite_pile
    # Replace small punctuation (tick marks) with digits
    SYMBOLS=S_boulder:0,S_golem:7

    # No startup splash screen. Windows GUI only.
    OPTIONS=!splash_screen
\end{verbatim}
%%.ed
%%.BR 2

%%.hn 2
%\subsection*{Using the NETHACKOPTIONS environment variable}
\subsection*{環境変数 NETHACKOPTIONS を使う}

%%.pg
%The NETHACKOPTIONS variable is a comma-separated list of initial
%values for the various options.  Some can only be turned on or off.
%You turn one of these on by adding the name of the option to the list,
%and turn it off by typing a `{\tt !}' or ``{\tt no}'' before the name.
%Others take a
%character string as a value.  You can set string options by typing
%the option name, a colon or equals sign, and then the value of the string.
%The value is terminated by the next comma or the end of string.
環境変数 NETHACKOPTIONS にはいろいろなオプションの初期値をカンマで区
切って列挙し設定する。オプションのうちのいくつかのものは単にオンかオフ
の選択ができるだけである。そのオプション名をリストに入れるとオンになり、
オプション名の前に `{\tt !}' か``{\tt no}''をつけるとオフになる。その他のオプショ
ンでは設定値として文字列が必要である。これらのオプションを設定するには
オプション名、コロンまたはイコール記号、設定値の順で入力すればよい。
設定値は次のカンマもしくは文字列の最後までとなる。

%%.pg
%For example, to set up an environment variable so that
%{\it color\/} is {\tt on},
%{\it legacy\/} is {\tt off},
%character {\it name\/} is set to ``{\tt Blue Meanie}'',
%and named {\it fruit\/} is set to ``{\tt lime}'',
%you would enter the command
例えば {\it color\/} をオン、{\it legacy\/} をオフ、
{\it name\/}が``{\tt Blue Meanie}'' に、
{\it fruit\/}が``{\tt lime}''になるように
環境変数を設定するには {\it csh} では次のコマンドを入力すればよい。
(`!'は特殊文字なのでエスケープしてやる必要があることに注意)
%%.SD i
\begin{verbatim}
    setenv NETHACKOPTIONS "color,\!leg,name:Blue Meanie,fruit:lime"
\end{verbatim}
%%.ED

%\nd in {\it csh}
%(note the need to escape the `!' since it's special
%to that shell), or the pair of commands
\nd {\it sh}, {\it ksh}, {\it bash} では
%%.SD i
\begin{verbatim}
    NETHACKOPTIONS="color,!leg,name:Blue Meanie,fruit:lime"
    export NETHACKOPTIONS
\end{verbatim}
%%.ED

%\nd in {\it sh}, {\it ksh}, or {\it bash}.
とすればよい。

%%.pg
%The NETHACKOPTIONS value is effectively the same as a single OPTIONS
%statement in a configuration file.
%The ``OPTIONS='' prefix is implied and comma separated options are
%processed from right to left.
%Other types of configuration statements such as BIND or MSGTYPE are
%not allowed.
NETHACKOPTIONS の値は事実上設定ファイル内の
1 行の OPTIONS 文と同じである。
``OPTIONS='' 接頭辞が暗黙に含まれ、
カンマ区切りのオプションは右から左に処理される。
BIND や MSGTYPE のようなその他の設定文は使えない。

%%.pg
%Instead of a comma-separated list of options,
%NETHACKOPTIONS can be set to the full name of a configuration file you
%want to use.
%If that full name doesn't start with a slash, precede it with `{\tt @}'
%(at-sign) to let NetHack know that the rest is intended as a file name.
%If it does start with `{\tt /}', the at-sign is optional.
NETHACKOPTIONS は、カンマ区切りにオプションのリストの代わりに、
使いたい設定ファイルのフルネームを設定できる。
フルネームがスラッシュで始まっていない場合、
`{\tt @}' (アットマーク) を前置することで、
残りがファイル名を意図していることを NetHack に伝える。
`{\tt /}' で始まっているなら、アットマークはオプションである。

%%.hn 2
%\subsection*{Customization options}
\subsection*{カスタマイズオプション}

%%.pg
%Here are explanations of what the various options do.
%Character strings that are too long may be truncated.
%Some of the options listed may be inactive in your dungeon.
以下にいろいろなオプションの役割を説明する。
あまりに長すぎる文字列の部分は無視される。
以下のオプションのうち実装によっては無効となっているオプションもある。

%%.pg
%Some options are persistent, and are saved and reloaded along with
%the game.  Changing a persistent option in the configuration file
%applies only to new games.
一部のオプションは永続し、ゲーム中保存および再設定される。
設定ファイルで永続オプションを変更した場合、新しいゲームにのみ適用される。

\blist{}
%%.lp
\item[\ib{acoustics}]
%Enable messages about what your character hears (default on).
%Note that this has nothing to do with your computer's audio capabilities.
%Persistent.
あなたのキャラクターが聞いたものに関するメッセージを表示する(標準設定はオン)。
これはコンピューターの音声出力機能とは何の関係もないことに注意すること。
永続する。
%%.lp
\item[\ib{align}]
%Your starting alignment ({\tt align:lawful}, {\tt align:neutral},
%or {\tt align:chaotic}).  You may specify just the first letter.
%The default is to randomly pick an appropriate alignment.
%If you prefix the value with `{\tt !}' or ``{\tt no}'', you will
%exclude that alignment from being picked randomly.
%Cannot be set with the `{\tt O}' command.  Persistent.
スタート時の属性({\tt align:lawful}, {\tt align:neutral},
{\tt align:chaotic})。
最初の一文字で指定することもできる。
標準設定ではランダムに選択される。
値に `{\tt !}' または ''no'' を前置すると、ランダムに選択した属性からその属性を
除外できる。
`{\tt O}' コマンドで設定することはできない。
永続する。
%%.lp
\item[\ib{autodescribe}]
%Automatically describe the terrain under cursor when asked to get a location
%on the map (default true).
%The {\it whatis\verb+_+coord\/}
%option controls whether the description includes map coordinates.
地図上の位置を指定するときにカーソルの下の地形を自動的に表現する
(標準設定はオン)。
{\it whatis\verb+_+coord\/} オプションは、
表現に位置の座標を含むかどうかを制御する。
%%.lp
\item[\ib{autodig}]
%Automatically dig if you are wielding a digging tool and moving into a place
%that can be dug (default false).  Persistent.
掘るための道具を手にしていて、掘ることができる地形に移動しようとしたとき、
自動的に掘る(標準設定はオフ)。
永続する。
%%.lp
\item[\ib{autoopen}]
%Walking into a door attempts to open it (default true).  Persistent.
扉に向かって歩くとそれを開けようとする(標準設定はオン)。
永続する。
%%.lp
\item[\ib{autopickup}]
%Automatically pick up things onto which you move (default on).  Persistent.
%See ``{\it pickup\verb+_+types\/}'' to refine the behavior.
移動先にあるものを自動的に拾う(標準設定はオン)。
永続する。
この振る舞いをカスタマイズするためには
``{\it pickup\verb+_+types\/}'' を参照すること。
%%.lp
\item[\ib{autoquiver}]
%This option controls what happens when you attempt the `f' (fire)
%command when nothing is quivered or readied (default false).
%When true, the computer will fill
%your quiver or quiver sack or make ready some suitable weapon.
%Note that it will not take
%into account the blessed/cursed status, enchantment, damage, or
%quality of the weapon; you are free to manually fill your quiver
%or quiver sack or make ready
%with the `Q' command instead.
%If no weapon is found or the option is
%false, the `t' (throw) command is executed instead.  Persistent.
このオプションは何も準備していない時に`f'
(fire:矢筒の中にあるものを発射する)コマンドを実行した時にどうするかを決める
(標準設定はオフ)。
オンなら、コンピューターは適当な武器を矢筒または矢筒袋に入れるか、
適切な武器を準備する。
この時、呪い、祝福、魔法、劣化、武器の数などは考慮されないので注意すること。
代わりに`{\tt Q}'コマンドを使って自分の好きな物を矢筒または矢筒袋に入れるか準備する。
入れる武器がないか、このオプションがオフの時は
代わりに`{\tt t}'(throw:物を投げる)コマンドが実行される。
永続する。
%%.lp
\item[\ib{blind}]
%Start the character permanently blind (default false).  Persistent.
恒久的に盲目の状態でゲームを開始する(標準設定はオフ)。
永続する。
%%.lp
\item[\ib{bones}]
%Allow saving and loading bones files (default true).  Persistent.
骨ファイルの読み書きを行う(標準設定はオン)。
永続する。
%%.lp
\item[\ib{boulder}]
%Set the character used to display boulders (default is the ``large rock''
%class symbol, `{\tt `}').
巨岩を表示するのに用いるキャラクタ(標準設定は``large rock''クラス
文字である `{\tt `}')。
%%.lp
\item[\ib{catname}]
%Name your starting cat (for example, ``{\tt catname:Morris}'').
%Cannot be set with the `{\tt O}' command.
スタート時の猫の名前(例: catname:Morris)。
`{\tt O}' コマンドで設定することはできない。
%%.lp character
\item[\ib{character}]
%Synonym for ``{\tt role}'' to pick the type of your character
%(for example ``{\tt character:Monk}'').  See {\it role\/} for more details.
キャラクターのタイプを設定するための ``{\tt role}'' の別名。
(例:``{\tt character:Monk}'')。
更なる詳細については {\it role\/} を参照のこと。
%%.lp
\item[\ib{checkpoint}]
%Save game state after each level change, for possible recovery after
%program crash (default on).  Persistent.
プログラムがクラッシュした時に復旧できるように
階を移動する毎に状態をセーブする(標準設定はオン)。
永続する。
%%.lp
\item[\ib{checkspace}]
%Check free disk space before writing files to disk (default on).
%You may have to turn this off if you have more than 2 GB free space
%on the partition used for your save and level files
%(because too much space might overflow the calculation and end up
%looking like insufficient space).
%Only applies when MFLOPPY was defined during compilation.
ファイルを書き込む前にディスクの空き容量をチェックする(標準設定はオン)。
セーブファイルやレベルファイルのために使うパーティションに2GB以上の
空き容量がある場合、このオプションをオフをしなければならないかもしれない
(なぜなら空き容量が大きすぎる場合、計算がオーバーフローして
容量が足りないように見えるかもしれない)。
このオプションはコンパイル時にMFLOPPYが定義されているときのみ適用される。
%%.lp
\item[\ib{clicklook}]
%Allows looking at things on the screen by navigating the mouse
%over them and clicking the right mouse button (default off).
マウスの右クリックで、画面上にあるものを見ることができるようにする
(標準設定はオフ)。
%%.lp
\item[\ib{cmdassist}]
%Have the game provide some additional command assistance for new
%players if it detects some anticipated mistakes (default on).
ありがちなミスを検出したときに新規プレイヤーのために
追加のコマンド補助メッセージを表示する(標準設定はオン)。
%%.lp
\item[\ib{confirm}]
%Have user confirm attacks on pets, shopkeepers, and other
%peaceable creatures (default on).  Persistent.
%%.lp
ペット、店主、その他の攻撃してこない生物を
攻撃しようとしたときに確認をする(標準設定はオン)。
永続する。
%%.lp
\item[\ib{dark\verb+_+room}]
%Show out-of-sight areas of lit rooms (default on).  Persistent.
明るい部屋の視界外のエリアを表示する(標準設定はオン)。
永続する。
%%.lp
\item[\ib{disclose}]
%Controls what information the program reveals when the game ends.
%Value is a space separated list of prompting/category pairs
%(default is `{\tt ni na nv ng nc no}',
%prompt with default response of `{\tt n}' for each candidate).
%Persistent.
%The possibilities are:
ゲーム終了時にプログラムがどの情報を公開するかを制御する。
値は指示/カテゴリの組のスペース区切りのリストである
(標準設定は `{\tt ni na nv ng nc no}' で、全ての候補に対してデフォルトを
`{\tt n}' にする)。
永続する。
設定できるものは以下の通り。

%%.sd
%%.si
%{\tt i} --- disclose your inventory;\\
%{\tt a} --- disclose your attributes;\\
%{\tt v} --- summarize monsters that have been vanquished;\\
%{\tt g} --- list monster species that have been genocided;\\
%{\tt c} --- display your conduct;\\
%{\tt o} --- display dungeon overview.
{\tt i} --- "持ち物の公開"\\
{\tt a} --- "属性の公開"\\
{\tt v} --- "退治した怪物の一覧"\\
{\tt g} --- "抹殺した怪物種の一覧"\\
{\tt c} ---"挑戦の表示"\\
{\tt o} --- "迷宮の概要の表示"
%%.ei
%%.ed

%Each disclosure possibility can optionally be preceded by a prefix which
%lets you refine how it behaves.  Here are the valid prefixes:
それぞれの表示に対して、その前にどのように振舞うかを指示する文字を
つけることができる。文字の意味は以下のとおり。

%%.sd
%%.si
%{\tt y} --- prompt you and default to yes on the prompt;\\
%{\tt n} --- prompt you and default to no on the prompt;\\
%{\tt +} --- disclose it without prompting;\\
%{\tt -} --- do not disclose it and do not prompt.
{\tt y} --- "確認するが、標準設定はイエスにする。"\\
{\tt n} --- "確認するが、標準設定はノーにする。"\\
{\tt +} --- "確認なしに表示する。"\\
{\tt -} --- "確認なしで非表示とする。"
%%.ei
%%.ed

%The listing of vanquished monsters can be sorted,
%so there are two additional choices for `{\tt v}':
絶滅した怪物の一覧はソートできるので、`{\tt v}' には二つの追加の
選択肢がある:
%%.sd
%%.si
%{\tt ?} --- prompt you and default to ask on the prompt;\\
%{\tt\#} --- disclose it without prompting, ask for sort order.
{\tt ?} --- "確認して、標準設定は問い合わせる。"\\
{\tt\#} --- "確認なしで表示し、ソート順を問い合わせる。"
%%.ei
%%.ed

%Asking refers to picking one of the orderings from a menu.
%The `{\tt +}' disclose without prompting choice,
%or being prompted and answering `{\tt y}' rather than `{\tt a}',
%will default to showing monsters in the traditional order,
%from high level to low level.\\
問い合わせはメニューから順序の一つを選択する。
確認なしの `{\tt +}' の公開か、
問い合わせに対して `{\tt a}' ではなく `{\tt y}' と答えた場合、
伝統的な順序である、レベルの高いものから低いものの順で怪物を表示する。
%%.lp ""
%Omitted categories are implicitly added with `{\tt n}' prefix.
%Specified categories with omitted prefix implicitly use `{\tt +}' prefix.
%Order of the disclosure categories does not matter, program display for
%end-of-game disclosure follows a set sequence.
省略されたカテゴリには暗黙に `{\tt n}' 接頭辞が追加される。
接頭辞が省略されて指定されたカテゴリには暗黙に `{\tt +}' 接頭辞が使われる。
表示カテゴリの順序は関係なく、プログラムは設定された順序に従って
ゲーム終了時の表示を行う。
%%.lp ""
%(for example, ``{\tt disclose:yi na +v -g o}'')
%The example sets
%{\tt inventory} to {\it prompt\/} and default to {\it yes\/},
%{\tt attributes} to {\it prompt\/} and default to {\it no\/},
%{\tt vanquished} to {\it disclose without prompting\/},
%{\tt genocided} to {\it not disclose\/} and {\it not prompt\/},
%{\tt conduct} to implicitly {\it prompt\/} and default to {\it no\/},
%{\tt overview} to {\it disclose without prompting\/}.
(例: ``{\tt disclose:yi na +v -g o}'')
このように設定した場合、
{\tt 持ち物}は{\it 確認する\/}(標準設定は{\it イエス\/})。
{\tt 属性}は{\it 確認する\/}(標準設定は{\it ノー\/})。
{\tt 退治した怪物}は{\it 確認なしに表示する\/}。
{\tt 抹殺した怪物}は{\it 確認せずに非表示とする\/}。
{\tt 挑戦}は暗黙に{\it 確認する\/}(標準設定は{\it ノー\/})。
{\tt 概要}は{\it 確認せずに表示する\/}。

%%.lp ""
%Note that the vanquished monsters list includes all monsters killed by
%traps and each other as well as by you.
%And the dungeon overview shows all levels you had visited but does not
%reveal things about them that you hadn't discovered.
退治した怪物の一覧は罠や同士討ちによって死んだ怪物も含むことに注意。
そして迷宮の概要は、訪れた全てのレベルを表示するが、発見しなかった
ものについては開示されない。
%%.lp
\item[\ib{dogname}]
%Name your starting dog (for example, ``{\tt dogname:Fang}'').
%Cannot be set with the `{\tt O}' command.
スタート時の犬の名前(例: dogname:Fang)。
`{\tt O}' コマンドで設定することはできない。
%%.lp
\item[\ib{extmenu}]
%Changes the extended commands interface to pop-up a menu of available commands.
%It is keystroke compatible with the traditional interface except that it does
%not require that you hit Enter.
%It is implemented for the tty interface (default off).
拡張コマンドインターフェースを、有効なコマンドの一覧メニューが出るように変更する。
キー入力は、最後に Enter を押す必要がないこと以外は伝統的なインターフェースと互換性がある。
tty インターフェースに実装されている。(標準設定はオフ)
%.lp ""
%For the X11 interface, which always uses a menu for choosing an extended
%command, it controls whether the menu shows all available commands (on)
%or just the subset of commands which have traditionally been considered
%extended ones (off).
拡張コマンドを選ぶために常にメニューが表示される X11 インターフェースでは、
これはメニューに利用可能な全てのコマンドを表示する(オン)か、
伝統的に拡張コマンドと考えられているコマンドの部分集合のみを表示する
(オフ)かを制御する。
%%.lp
\item[\ib{female}]
%An obsolete synonym for ``{\tt gender:female}''.  Cannot be set with the
%`{\tt O}' command.
``{\tt gender:female}'' の古い別名。
`{\tt O}' コマンドで設定することはできない。
%%.lp
\item[\ib{fixinv}]
%An object's inventory letter sticks to it when it's dropped (default on).
%If this is off, dropping an object shifts all the remaining inventory letters.
%Persistent.
持ち物の目録記号と物との対応はその物を下に置いても
変化しない(標準設定はオン)。
オフにしたときは物を下に置くとその物の目録記号以降の記号と
物との対応が 1 つずつずれる。
永続する。
%%.lp
\item[\ib{force\_invmenu}]
%Commands asking for an inventory item show a menu instead of
%a text query with possible menu letters. Default is off.
持物を選ぶコマンドでは、可能なメニュー文字付きのテキスト問い合わせではなく、
メニューを表示する。
標準設定はオフ。
%%.lp
\item[\ib{fruit}]
%Name a fruit after something you enjoy eating (for example, ``{\tt fruit:mango}'')
%(default ``{\tt slime mold}''). Basically a nostalgic whimsy that
%{\it NetHack\/} uses from time to time.  You should set this to something you
%find more appetizing than slime mold.  Apples, oranges, pears, bananas, and
%melons already exist in {\it NetHack\/}, so don't use those.
あなたの好物の果物にちなんでその名前を設定する(例:fruit:mango)
(標準設定は「slime mold」)。
元来 {\it NetHack\/} で時々使われる懐古趣味的な妙な言葉である。
slime mold(ネバネバかび)よりももっと食欲のわく食べ物にするべきであろう。
apples(リンゴ)、oranges(オレンジ)、 pears(洋なし)、 bananas(バナナ)、
melons (メロン) は {\it NetHack\/} にはもう存在しているので、設定してはならない。
%%.lp
\item[\ib{gender}]
%Your starting gender ({\tt gender:male} or {\tt gender:female}).
%You may specify just the first letter.  Although you can
%still denote your gender using the ``{\tt male}'' and ``{\tt female}''
%options, the ``{\tt gender}'' option will take precedence.
%The default is to randomly pick an appropriate gender.
%If you prefix the value with `{\tt !}' or ``{\tt no}'', you will
%exclude that gender from being picked randomly.
開始時の性別({\tt gender:male} または {\tt gender:female}).
最初の一文字で指定することもできる。
``{\tt male}'' と ``{\tt female}'' オプションを指定していても、
``{\tt gender}'' オプションが優先される。
標準設定では適切な性別がランダムに選択される。
値に `{\tt !}' または ``{\tt no}'' を前置すると、ランダムに選択した
性別からその性別を除外できる。
%Cannot be set with the `{\tt O}' command.  Persistent.
`{\tt O}' コマンドで設定することはできない。
永続する。
%%.lp
\item[\ib{goldX}]
%When filtering objects based on bless/curse state (BUCX), whether to
%treat gold pieces as {\tt X} (unknown bless/curse state, when `on')
%or {\tt U} (known to be uncursed, when `off', the default).
%Gold is never blessed or cursed, but it is not described as ``uncursed''
%even when the {\it implicit\verb+_+uncursed\/} option is `off'.
物を祝呪状態(BUCX)を基に絞り込むとき、
金貨を
{\tt X} (祝呪状態不明、`オン'のとき)、あるいは
{\tt U} (呪われていないとわかっている、`オフ'のとき、標準設定) の
どちらとして扱うかを設定する。
金貨は決して祝福されたり呪われたりしないが、例え
{\it implicit\verb+_+uncursed\/} オプションが `オフ' でも
``呪われていない'' とは表示されない。
%%.lp
\item[\ib{help}]
%If more information is available for an object looked at
%with the `{\tt /}' command, ask if you want to see it (default on).
%Turning help off makes just looking at things faster, since you aren't
%interrupted with the ``{\tt More info?}'' prompt, but it also means that you
%might miss some interesting and/or important information.  Persistent.
`{\tt /}' コマンドを使って調べている物について何らかの情報がある場合、
それを見るかどうかを尋ねる(標準設定はオン)。
ヘルプが出ないようにすれば ``{\tt More info?}'' というプロンプトに煩わされなくなるので、
素早く物を調べることができる。
しかしそれは興味深く重要な情報を見逃してしまうかも知れないことを意味する。
永続する。
%%.lp
\item[\ib{herecmd\verb+_+menu}]
%When using a windowport that supports mouse and clicking on yourself or
%next to you, show a menu of possible actions for the location.
%Same as ``{\tt \#herecmdmenu}'' and ``{\tt \#therecmdmenu}'' commands.
マウスに対応しているウィンドウポートを使っていて、自分自身か
自分の隣をクリックしたとき、その位置に対して可能な公道のメニューを表示する。
``{\tt \#herecmdmenu}'' および ``{\tt \#therecmdmenu}'' コマンドと同じ。
%%.lp
\item[\ib{hilite\verb+_+pet}]
%Visually distinguish pets from similar animals (default off).
%The behavior of this option depends on the type of windowing you use.
%In text windowing, text highlighting or inverse video is often used;
%with tiles, generally displays a heart symbol near pets.
ペットを同種の動物から視覚的に区別する(標準設定はオフ)。
このオプションの振る舞いは使っているウィンドウシステムに依存する。
テキストウィンドウ表示では、ハイライトや反転がしばしば使われる。
タイルを使用するときは、一般的にはハートの記号をペットにつける。
%%.lp ""
%With the curses interface, the {\it petattr\/}
%option controls how to highlight pets and setting it will turn the
%{\it hilite\verb+_+pet\/} option on or off as warranted.
cursesインターフェースでは、{\it petattr\/} オプションは
どのようにペットをハイライトするかを制御し、
これを設定すると {\it hilite\verb+_+pet\/} オプションもこれに合わせて
オンオフされる。
%%.lp
\item[\ib{hilite\verb+_+pile}]
%Visually distinguish piles of objects from individual objects (default off).
%The behavior of this option depends on the type of windowing you use.
%In text windowing, text highlighting or inverse video is often used;
%with tiles, generally displays a small plus-symbol beside the object
%on the top of the pile.
オブジェクトの山と個々のオブジェクトで表示を変える(標準設定はオフ)。
このオプションの振る舞いは使用するウィンドウの種類に依存する。
テキスト表示の場合、ハイライトや反転がしばしば使われる;
タイルの場合、一般的には山の一番上のオブジェクトの横に小さいプラスマークが
表示される。
%%.lp
\item[\ib{hitpointbar}]
%Show a hit point bar graph behind your name and title.
%Only available for TTY and Windows GUI, and only when statushilites is on.
名前とランクの後ろにヒットポイントバーを表示する。
TTYとWindows GUI版のみで、ステータスハイライトがオンの場合のみ。
%%.lp
\item[\ib{horsename}]
%Name your starting horse (for example, ``{\tt horsename:Trigger}'').
開始時の馬の名前をつける (例:``{\tt horsename:Trigger}'').
%Cannot be set with the `{\tt O}' command.
`{\tt O}' コマンドで設定することはできない。
%%.lp
\item[\ib{ignintr}]
%Ignore interrupt signals, including breaks (default off).  Persistent.
ブレークを含む割り込み信号を無視する(標準設定はオフ)。
永続する。
%%.lp
\item[\ib{implicit\verb+_+uncursed}]
%Omit ``uncursed'' from inventory lists, if possible (default on).
可能なら、持ち物一覧での「呪われていない」表示を抑制する(標準設定はオン)。
%%.lp
\item[\ib{legacy}]
%Display an introductory message when starting the game (default on).
%Persistent.
ゲーム開始時に説明メッセージを表示する(標準設定はオン)。
永続する。
%%.lp
\item[\ib{lit\verb+_+corridor}]
%Show corridor squares seen by night vision or a light source held by your
%character as lit (default off).  Persistent.
夜目やあなたのキャラクタが持っている光源によって見える通路を
表示する (標準設定はオフ)。
永続する。
%%.lp
\item[\ib{lootabc}]
%When using a menu to interact with a container,
%use the old `{\tt a}', `{\tt b}', and `{\tt c}' keyboard shortcuts
%rather than the mnemonics `{\tt o}', `{\tt i}', and `{\tt b}' (default off).
%Persistent.
箱を扱うときに`{\tt o}', `{\tt i}', `{\tt b}'ではなく
古い`{\tt a}', `{\tt b}', `{\tt c}'のキーを使う
(標準設定はオフ)。
永続する。
%%.lp
\item[\ib{mail}]
%Enable mail delivery during the game (default on).  Persistent.
ゲーム中にメールを配達するようにする(標準設定はオン)。
永続する。
%%.lp
\item[\ib{male}]
%An obsolete synonym for ``{\tt gender:male}''.  Cannot be set with the
%`{\tt O}' command.
``{\tt gender:male}'' の古くなった別名。
`{\tt O}' コマンドで設定することはできない。
%%.lp
\item[\ib{mention\verb+_+walls}]
%Give feedback when walking against a wall (default off).
壁に向かって歩くときにフィードバックを与える(標準設定はオフ)。
%%.lp
\item[\ib{menucolors}]
%Enable coloring menu lines (default off).
%See ``{\it Configuring Menu Colors\/}'' on how to configure the colors.
メニュー行に色を付ける(標準設定はオフ)。
色の設定方法については ``{\it Configuring Menu Colors\/}'' を参照のこと。
%%.lp
\item[\ib{menustyle}]
%Controls the interface used when you need to choose various objects (in
%response to the Drop command, for instance).  The value specified should
%be the first letter of one of the following:  traditional, combination,
%full, or partial.
%Traditional was the only interface available for
%early versions; it consists of a prompt for object class characters,
%followed by an object-by-object prompt for all items matching the selected
%object class(es).
%Combination starts with a prompt for object class(es)
%of interest, but then displays a menu of matching objects rather than
%prompting one-by-one.
%Full displays a menu of
%object classes rather than a character prompt, and then a menu of matching
%objects for selection.
%Partial skips the object class filtering and
%immediately displays a menu of all objects.
%Persistent.
いろいろなオブジェクトを指定するとき(たとえば Dropコマンド)に用
いるインタフェースの制御。値は tradional, combination, full, partial
の 4 つのタイプの最初の一文字を指定する。
Traditional は古いバージョンで唯一利用できたもので、
物の種類を示す文字の入力、次に選んだ種類に当てはまる全ての物を一つずつ
確認する。
Combination は選択したい物の種類を示す文字を入力するが、
次に当てはまる物を一つずつ確認で選択するのではなく、メニュー形式で選択する。
Full は物の種類の文字入力のかわりに、最初に物の種類のメニューを表示し、
それから選択した種類に当てはまる物のメニューを表示する。
Partial は物の種類を選ぶことをせず、
直ちに全ての物が表示されたメニューを表示する。
永続する。
\item[\ib{menu\verb+_+deselect\verb+_+all}]
%Menu character accelerator to deselect all items in a menu.
%Implemented by the Amiga, Gem, X11 and tty ports.
%Default `-'.
メニューの全ての項目を非選択にするキー。
Amiga, Gem, X11, ttyで実装されている。(標準設定は `-')
\item[\ib{menu\verb+_+deselect\verb+_+page}]
%Menu character accelerator to deselect all items on this page of a menu.
%Implemented by the Amiga, Gem and tty ports.
%Default `\verb+\+'.
メニューのうち、現在表示されている全ての項目を非選択にするキー。
Amiga, Gem, tty で実装されている。(標準設定は `\verb+\+')
\item[\ib{menu\verb+_+first\verb+_+page}]
%Menu character accelerator to jump to the first page in a menu.
%Implemented by the Amiga, Gem and tty ports.
%Default `\verb+^+'.
メニューの最初のページへ移動するキー。
Amiga, Gem, tty で実装されている。(標準設定は `\verb+^+')
\item[\ib{menu\verb+_+headings}]
%Controls how the headings in a menu are highlighted.
%Values are ``{\tt none}'', ``{\tt bold}'', ``{\tt dim}'',
%``{\tt underline}'', ``{\tt blink}'', or ``{\tt inverse}''.
%Not all ports can actually display all types.
メニューのうち注目している部分をどのように表示するかを指定する。
値は ``{\tt none}'', ``{\tt bold}'', ``{\tt dim}'',
``{\tt underline}'', ``{\tt blink}'', ``{\tt inverse}'' のいずれかである。
全てのポートで実際に全てのタイプを表示できるわけではない。
\item[\ib{menu\verb+_+invert\verb+_+all}]
%Menu character accelerator to invert all items in a menu.
%Implemented by the Amiga, Gem, X11 and tty ports.
%Default `@'.
メニューの全ての項目の選択状態を反転するキー。
Amiga, Gem, X11, tty で実装されている。(標準設定は `@')
\item[\ib{menu\verb+_+invert\verb+_+page}]
%Menu character accelerator to invert all items on this page of a menu.
%Implemented by the Amiga, Gem and tty ports.
%Default `\verb+~+'.
メニューの現在のページの全ての項目の選択状態を反転するキー。
Amiga, Gem, tty で実装されている。(標準設定は `\verb+~+')
\item[\ib{menu\verb+_+last\verb+_+page}]
%Menu character accelerator to jump to the last page in a menu.
%Implemented by the Amiga, Gem and tty ports.
%Default `\verb+|+'.
メニューの最後のページへ移動するキー。
Amiga, Gem, tty で実装されている。(標準設定は `\verb+|+')
\item[\ib{menu\verb+_+next\verb+_+page}]
%Menu character accelerator to goto the next menu page.
%Implemented by the Amiga, Gem and tty ports.
%Default `\verb+>+'.
メニューの次のページへ移動するキー。
Amiga, Gem, tty で実装されている。(標準設定は `\verb+>+')
\item[\ib{menu\verb+_+objsyms}]
%Show object symbols in menu headings in menus where
%the object symbols act as menu accelerators (default off).
物の種類の記号がメニューのショートカットキーとして動作する場所では、
メニューの一覧に物の種類の記号を表示する(標準設定はオフ)。
\item[\ib{menu\verb+_+overlay}]
%Do not clear the screen before drawing menus, and align
%menus to the right edge of the screen. Only for the tty port.
%(default on)
メニューを書く前に画面をクリアせず、メニューを画面の右端に寄せる。
tty 版のみ。
(標準設定はオン)
\item[\ib{menu\verb+_+previous\verb+_+page}]
%Menu character accelerator to goto the previous menu page.
%Implemented by the Amiga, Gem and tty ports.
%Default `\verb+<+'.
メニューの前のページへ移動するキー。
Amiga, Gem, tty で実装されている。(標準設定は `\verb+<+')
\item[\ib{menu\verb+_+search}]
%Menu character accelerator to search for a menu item.
%Implemented by the Amiga, Gem, X11 and tty ports.
%Default `:'.
メニューで検索を行うキー。
Amiga, Gem, X11, tty で実装されている。(標準設定は `:')
\item[\ib{menu\verb+_+select\verb+_+all}]
%Menu character accelerator to select all items in a menu.
%Implemented by the Amiga, Gem, X11 and tty ports.
%Default `.'.
メニューの全ての項目を選択にするキー。
Amiga, Gem, X11, tty で実装されている。(標準設定は `.')
\item[\ib{menu\verb+_+select\verb+_+page}]
%Menu character accelerator to select all items on this page of a menu.
%Implemented by the Amiga, Gem and tty ports.
%Default `,'.
メニューのうち、現在表示されている全ての項目を選択にするキー。
Amiga, Gem, tty で実装されている。(標準設定は `,')
%% %.lp
%% \item[\ib{menu\verb+_+tab\verb+_+sep}]
%% Format menu entries using TAB to separate columns (default off).
%% Only applicable to some menus, and only useful to some interfaces.
%% Debug mode only.
%%.lp
\item[\ib{monpolycontrol}]
%Prompt for new form whenever any monster changes shape (default off).
%Debug mode only.
怪物が姿を変えるときに新しい形になるか確認する(標準設定はオフ)。
デバッグモードのみ。
%%.lp
\item[\ib{mouse\verb+_+support}]
%Allow use of the mouse for input and travel.
%Valid settings are:
入力と移動にマウスを使えるようにする。
有効な設定は:

%%.sd
%%.si
%{\tt 0} --- disabled\\
%{\tt 1} --- enabled and make OS adjustments to support mouse use\\
%{\tt 2} --- like {\tt 1}, but does not make any OS adjustments\\
{\tt 0} --- 無効\\
{\tt 1} --- 有効; マウスの使用を支援するために OS の補正を使う\\
{\tt 2} --- {\tt 1} と同様だが、OS の補正は使わない\\
%%.ei
%%.ed

%Omitting a value is the same as specifying {\tt 1}
%and negating
%{\it mouse\verb+_+support\/}
%is the same as specifying {\tt 0}.
値を省略すると {\tt 1} を指定したのと同じになり、
{\it mouse\verb+_+support\/} を否定すると
{\tt 0} を指定したのと同じになる。
%%.lp
\item[\ib{msghistory}]
%The number of top line messages to save (and be able to recall
%with `{\tt \^{}P}') (default 20).
%Cannot be set with the `{\tt O}' command.
後で `{\tt \^{}P}' で呼び出せるように保存しておく最上行のメッセージの数
(標準設定は 20)。
`{\tt O}' コマンドで設定することはできない。
%%.lp
\item[\ib{msg\verb+_+window}]
%Allows you to change the way recalled messages are displayed.
%Currently it is only supported for tty (all four choices) and for curses
%(`{\tt f}' and `{\tt r}' choices, default `{\tt r}').
%The possible values are:
過去のメッセージの表示方法を変更する。現時点では tty 
(四つ全ての選択肢) と curses
(`{\tt f}' と `{\tt r}' の選択肢、標準設定は `{\tt r}')
で実装されている。
指定できる値は以下のとおり。

%%.sd
%%.si
%{\tt s} --- single message (default; only choice prior to 3.4.0);\\
%{\tt c} --- combination, two messages as {\it single\/}, then as {\it full\/};\\
%{\tt f} --- full window, oldest message first;\\
%{\tt r} --- full window reversed, newest message first.
{\tt s} --- "single": 1 メッセージだけ (標準設定。これは 3.4.0 以前での振る舞いである)\\
{\tt c} --- "conbination": 最初の 2 メッセージは 'single'、その後は 'full'。\\
{\tt f} --- "full": 全画面; 古いメッセージが先。\\
{\tt r} --- 全画面; 新しいメッセージが先。
%%.ei
%%.ed

%For backward compatibility, no value needs to be specified (which
%defaults to {\it full\/}), or it can be negated (which defaults
%to {\it single\/}).
過去との互換性のために、値なしにもできる({\it full\/}とみなす)し、
否定もできる({\it single\/}とみなす)。
%%.lp
\item[\ib{name}]
%Set your character's name (defaults to your user name).  You can also
%set your character's role by appending a dash and one or more letters of
%the role (that is, by suffixing one of
%``{\tt -A -B -C -H -K -M -P -Ra -Ro -S -T -V -W}'').
あなたのキャラクタの名前を設定する(標準設定はあなたの ユーザー名)。
-記号とキャラクタの職業の頭文字を付け加える(つまり
``{\tt -A -B -C -H -K -M -P -Ra -Ro -S -T -V -W}''
のどれかを後に付ける)ことによって職業を設定することもできる。
%If ``{\tt -@}'' is used for the role, then a random one will be
%automatically chosen.
職業として
``{\tt -@}''
を使用すると、ランダムなひとつが自動的に選択される。
%Cannot be set with the `{\tt O}' command.
`{\tt O}' コマンドで設定することはできない。
%%.lp
\item[\ib{news}]
%Read the {\it NetHack\/} news file, if present (default on).
%Since the news is shown at the beginning of the game, there's no point
%in setting this with the `{\tt O}' command.
{\it NetHack\/} のニュースファイルがあればそれを読む(標準設定はオン)。
ニュースはゲームの最初に表示されるので、
`{\tt O}' コマンドでこれを設定するのは無意味である。
%%.lp
\item[\ib{nudist}]
%Start the character with no armor (default false).  Persistent.
防具なしのキャラクターで開始する(デフォルトはオフ)。
永続する。
%%.lp
\item[\ib{null}]
%Send padding nulls to the terminal (default on).  Persistent.
端末にパディングのためのナルキャラクタを送る(標準設定はオン)。
永続する。
%%.lp
\item[\ib{number\verb+_+pad}]
%Use digit keys instead of letters to move (default 0 or off).\\
%Valid settings are:
移動のために文字のかわりに数字キーを使用する(標準設定は 0 またはオフ)。\\
設定可能な値は:

%%.sd
%%.si
\newlength{\mwidth}
\settowidth{\mwidth}{\tt -0}
\newcommand{\numbox}[1]{\makebox[\mwidth][r]{{\tt #1}}}
%\numbox{0} --- move by letters; `{\tt yuhjklbn}'\\
\numbox{0} --- 文字で移動する; `{\tt yuhjklbn}'\\
%\numbox{1} --- move by numbers; digit `{\tt 5}' acts as `{\tt G}' movement prefix\\
\numbox{1} --- 数字で移動する; 数字 `{\tt 5}' は `{\tt G}' 移動の接頭辞として振る舞う\\
%\numbox{2} --- like {\tt 1} but `{\tt 5}' works as `{\tt g}' prefix instead of as `{\tt G}'\\
\numbox{2} --- {\tt 1} と同様だが数字 `{\tt 5}' は `{\tt G}' ではなく `{\tt g}' 移動の接頭辞として振る舞う\\
%\numbox{3} --- by numbers using phone key layout; {\tt 123} above, {\tt 789} below\\
\numbox{3} --- 電話のキー配置の数字で移動する; {\tt 123} は上、{\tt 789} は下\\
%\numbox{4} --- combines {\tt 3} with {\tt 2}; phone layout plus MS-DOS compatibility\\
\numbox{4} --- {\tt 3} と {\tt 2} の組み合わせ; 電話配置と MS-DOS 互換\\
%\numbox{-1} --- by letters but use `{\tt z}' to go northwest, `{\tt y}' to zap wands
\numbox{-1} --- 文字で移動するが `{\tt z}' で北西に移動し、`{\tt y}' で杖を振る
%%.ei
%%.ed

%For backward compatibility, omitting a value is the same as specifying {\tt 1}
%and negating
%{\it number\verb+_+pad\/}
%is the same as specifying {\tt 0}.
%(Settings {\tt 2} and {\tt 4} are for compatibility with MS-DOS or old PC Hack;
%in addition to the different behavior for `{\tt 5}', `{\tt Alt-5}' acts as `{\tt G}'
%and `{\tt Alt-0}' acts as `{\tt I}'.
%Setting {\tt -1} is to accommodate some QWERTZ keyboards which have the
%location of the `{\tt y}' and `{\tt z}' keys swapped.)
%When moving by numbers, to enter a count prefix for those commands
%which accept one (such as ``{\tt 12s}'' to search twelve times), precede it
%with the letter `{\tt n}' (``{\tt n12s}'').
後方互換性のために、値を省略するのは {\tt 1} と同じで、
否定する {\it number\verb+_+pad\/} と {\tt 0} を指定するのと同じである。
(設定 {\tt 2} と {\tt 4} は MS-DOS や古い PC Hack との互換性のためのものである;
`{\tt 5}' のための異なった振る舞いに加えて、
`{\tt Alt-5}' は `{\tt G}' として振る舞い、
`{\tt Alt-0}' は `{\tt I}' として振る舞う。
設定 {\tt -1} は、`{\tt y}' と `{\tt z}' の位置が入れ替わっている一部の
QWERTZ キーボードに対応するものである。)
数字で移動するとき、(12 回調べるために ``{\tt 12s}'' とするような)
コマンドに対する回数接頭辞を入力するには、
文字 `{\tt n}' を前置する (``{\tt n12s}'')。
%%.lp
\item[\ib{packorder}]
%Specify the order to list object types in (default
%``\verb&")[%?+!=/(*`0_&''). The value of this option should be a string
%containing the symbols for the various object types.  Any omitted types
%are filled in at the end from the previous order.
物の種類を表示するときの順番を指定する(標準設定は
``\verb&")[%?+!=/(*`0_&'').
このオプションにはいろいろな物の種類を表す文字の列を設定する。
設定されなかった種類は以前の順序で最後に表示される。
%%.lp
\item[\ib{paranoid\verb+_+confirmation}]
%A space separated list of specific situations where alternate
%prompting is desired.  The default is ``{\it paranoid\verb+_+confirmation:pray}''.
プロンプトを変えたい状況の、スペースで区切られた一覧。
標準設定は\\
``{\it paranoid\verb+_+confirmation:pray}''。
%%.sd
%%.si
\newlength{\pcwidth}
\settowidth{\pcwidth}{\tt Were-change}
\addtolength{\pcwidth}{\labelsep}
\blist{\leftmargin \pcwidth \topsep 1mm \itemsep 0mm}
\item[{\tt Confirm}]
%for any prompts which are set to require ``yes''
%rather than `y', also require ``no'' to reject instead
%of accepting any non-yes response as no;
`y' ではなく ``yes'' を必要とする場合、yes以外の何でも
否定したことにするのではなく、否定する場合には ``no'' が必要。
\item[{\tt quit~~~}]
%require ``{\tt yes}'' rather than `{\tt y}' to confirm quitting
%the game or switching into non-scoring explore mode;
ゲームを抜けたりスコアが記録されない探検モードに切り替えるときに
`{\tt y}' ではなく ``{\tt yes}'' が必要;
\item[{\tt die~~~~}]
%require ``{\tt yes}'' rather than `{\tt y}' to confirm dying (not
%useful in normal play; applies to explore mode);
死ぬときに `{\tt y}' ではなく ``{\tt yes}'' が必要
(通常のプレイでは無意味; 探検モードで適用される);
\item[{\tt bones~~}]
%require ``{\tt yes}'' rather than `{\tt y}' to confirm saving
%bones data when dying in debug mode
デバッグモードで死んだときの骨ファイルの保存確認に `{\tt y}' ではなく ``{\tt yes}'' が必要;
\item[{\tt attack~}]
%require ``{\tt yes}'' rather than `{\tt y}' to confirm attacking
%a peaceful monster;
友好的な怪物を攻撃するときの確認に `{\tt y}' ではなく ``{\tt yes}'' が必要;
\item[{\tt wand-break}]
%require ``{\tt yes}'' rather than `{\tt y}' to confirm breaking
%a wand;
杖を折るときの確認に `{\tt y}' ではなく ``{\tt yes}'' が必要;
\item[{\tt eating}]
%require ``{\tt yes}'' rather than `{\tt y}' to confirm whether to
%continue eating;
食べ続けるかどうかの確認に `{\tt y}' ではなく ``{\tt yes}'' が必要;
\item[{\tt Were-change}]
%require ``{\tt yes}'' rather than `{\tt y}' to confirm changing form
%due to lycanthropy
%when hero has polymorph control;
プレイヤーが変化制御を持っているときに獣化病により姿を変えるときの
確認に`{\tt y}' ではなく ``{\tt yes}'' が必要;
\item[{\tt pray~~~}]
%require `{\tt y}' to confirm an attempt to pray rather
%than immediately praying; on by default;
祈るときにすぐに祈るのではなく確認で `{\tt y}' が必要; 標準設定でオン;
\item[{\tt Remove~}]
%require selection from inventory for `{\tt R}'
%and `{\tt T}'
%commands even when wearing just one applicable item.
`{\tt R}' と `{\tt T}' コマンドで有効なアイテムを一つしか身につけていないときでも
持ち物一覧からの選択が必要。
\item[{\tt all~~~~}]
%turn on all of the above.
全て有効にする。
\elist
%%.ei
%%.ed
%By default, the pray choice is enabled, the others disabled.
%To disable it without setting
%any of the other choices, use ``{\it paranoid\verb+_+confirmation:none}''.  To keep
%it enabled while setting any of the others, include it in the list,
%such as ``{\it par\-a\-noid\verb+_+con\-fir\-ma\-tion:attack~pray~Remove}''.
標準設定では、pray 選択がオン、それ以外はオフになっている。
その他の選択を設定することなくこれをオフにするには、
``{\it paranoid\verb+_+confirmation:none}'' を使う。
その他を設定しながらこれをオンのままにするには、
``{\it par\-a\-noid\verb+_+con\-fir\-ma\-tion:attack~pray~Remove}'' のように一覧に含める。
%%.lp
\item[\ib{perm\verb+_+invent}]
%If true, always display your current inventory in a window.  This only
%makes sense for windowing system interfaces that implement this feature.
もしオンならば、現在の持ち物一覧を常にウィンドウに表示しておく。
この機能が実装されているウィンドウシステムでのみ有効。
%%.lp
%%.\" petattr is a wincap option but we'll document it here...
\item[\ib{petattr}]
%Specifies one or more text highlighting attributes to use when showing
%pets on the map.
%Effectively a superset of the {\it hilite\verb+_+pet\/} boolean option.
%Curses interface only; value is one or more of the following letters.
地図上にペットを表示するときに使われるハイライト属性の一つ以上の
テキストを指定する。
事実上 {\it hilite\verb+_+pet\/} オンオフオプションの上位互換である。
curses インターフェースのみ; 値は一つ以上の次の文字である。

%%.sd
%%.si
%{\tt n} --- Normal text (no highlighting)\\
%{\tt i} --- Inverse video (default)\\
%{\tt b} --- Bold text\\
%{\tt u} --- Underlined text\\
%{\tt k} --- blinKing text\\
%{\tt d} --- Dim text\\
%{\tt t} --- iTalic text\\
%{\tt l} --- Left line indicator\\
%{\tt r} --- Right line indicator\\
{\tt n} --- 通常のテキスト (ハイライトなし)\\
{\tt i} --- 反転 (標準設定)\\
{\tt b} --- 太字\\
{\tt u} --- 下線\\
{\tt k} --- 点滅\\
{\tt d} --- 薄い表示\\
{\tt t} --- 斜体\\
{\tt l} --- 左の行指示子\\
{\tt r} --- 右の行指示子\\
%%.ei
%%.ed

%Some of those choices might not work, particularly the final three,
%depending upon terminal hardware or terminal emulation software.
ターミナルハードウェアやターミナルエミュレーションソフトウェアに
依存して、これらの選択の一部、特に最後の三つは動作しないかもしれない。

%%.lp ""
%Currently multiple highlight-style letters can be combined by simply
%stringing them together (for example, ``bk''), but in the future
%they might require being separated by plus signs (such as ``b\verb&+&k'',
%which works already).
%When using the `n' choice, it should be specified on its own,
%not in combination with any of the other letters.
現在の所、複数のハイライトスタイル文字は単に結合されている
(例えば ``bk'') が、将来プラス記号で区切られる必要があるようになるかも
しれない (例えば ``b\verb&+&k'' のように; これは既に動作する)。
`n' の選択を使う場合、その他の文字との組み合わせではなく、
単体で指定される必要がある。

%%.lp
\item[\ib{pettype}]
%Specify the type of your initial pet, if you are playing a character class
%that uses multiple types of pets; or choose to have no initial pet at all.
%Possible values are ``{\tt cat}'', ``{\tt dog}'', ``{\tt horse}''
%and ``{\tt none}''.
プレイするキャラクタクラスが複数のタイプのペットを使用できる時に
初期のペットを指定する、または初期のペットを全く指定しない。
指定出来るのは``{\tt cat}'', ``{\tt dog}'', ``{\tt horse}'', ``{\tt none}'' である。
%If the choice is not allowed for the role you are currently playing,
%it will be silently ignored.  For example, ``{\tt horse}'' will only be
%honored when playing a knight.
選択が現在プレイしている職業で使えない場合、暗黙に無視される。
例えば、``{\tt horse}'' は騎士をプレイしている場合にのみ有効になる。
%Cannot be set with the `{\tt O}' command.
`{\tt O}' コマンドで設定することはできない。
%%.lp
\item[\ib{pickup\verb+_+burden}]
%When you pick up an item that would exceed this encumbrance
%level (Unencumbered, Burdened, streSsed, straiNed, overTaxed,
%or overLoaded), you will be asked if you want to continue.
%(Default `S').  Persistent.
物を拾った時にここで指定した荷物の重さレベル
(Unencumbered, Burdened, streSsed, straiNed, overTaxed, overLoaded)
以上になると、続行するかどうかを確認する。
(標準設定は `S')。
永続する。
%%.lp
\item[\ib{pickup\verb+_+thrown}]
%If this option is on and ``{\it autopickup\/}'' is also on, try to pick up
%things that you threw, even if they aren't in
%``{\it pickup\verb+_+types\/}'' or
%match an autopickup exception.  Default is on.  Persistent.
このオプションがオンで
``{\it autopickup\/}''
もオンの場合、
例え
``{\it pickup\verb+_+types\/}''
にないか自動拾い例外にマッチングしても、
自分が投げた物は拾おうとする。
デフォルトはオン。
永続する。
%%.lp
\item[\ib{pickup\verb+_+types}]
%Specify the object types to be picked up when ``{\it autopickup\/}''
%is on.  Default is all types.  You can use
%``{\it autopickup\verb+_+exception\/}''
%configuration file lines to further refine ``{\it autopickup\/}'' behavior.
%Persistent.
``{\it autopickup\/}''
オプションが設定されている時に拾う物の種類を指定する(標準設定は全ての種類)。
``{\it autopickup\verb+_+exception\/}''
行を使うことで
``{\it autopickup\/}''
の振る舞いをより細かく制御できる。
永続する。
%%.lp
\item[\ib{pile\verb+_+limit}]
%When walking across a pile of objects on the floor, threshold at which
%the message ``there are few/several/many objects here'' is given instead
%of showing a popup list of those objects.  A value of 0 means ``no limit''
%(always list the objects); a value of 1 effectively means ``never show
%the objects'' since the pile size will always be at least that big;
%default value is 5.  Persistent.
床の上の物の山の上を歩くとき、物の一覧を表示する代わりに
``there are few/several/many objects here''
(ここにはいくつかの/たくさんの物がある)を表示する閾値。
値 0 は「無制限」(常に物の一覧を表示する)を意味する;
値 1 は、山のサイズが最低 1 はあるので、事実上「物は一切表示しない」を
意味する; 標準設定は 5。
永続する。
%%.lp
\item[\ib{playmode}]
%Values are {\it normal\/}, {\it explore\/}, or {\it debug\/}.
%Allows selection of explore mode (also known as discovery mode) or debug
%mode (also known as wizard mode) instead of normal play.
%Debug mode might only be allowed for someone logged in under a particular
%user name (on multi-user systems) or specifying a particular character
%name (on single-user systems) or it might be disabled entirely.  Requesting
%it when not allowed or not possible results in explore mode instead.
%Default is normal play.
値は {\it normal\/}, {\it explore\/}, {\it debug\/} のいずれか。
通常のプレイの代わりに(発見モードとも呼ばれる)探検モード、あるいは
(ウィザードモードとも呼ばれる)デバッグモードを選択する。
デバッグモードは、(マルチユーザーシステムでは)特定のユーザー名で
ログインしている必要があるかもしれないし、
(シングルユーザーシステムでは)特定のキャラクタ名を指定している
必要があるかもしれないし、
完全に無効になっているかもしれない。
デバッグモードを要求したけれども許されていないか不可能な場合は
代わりに探検モードになる。
標準設定は通常プレイ。
%%.lp
\item[\ib{pushweapon}]
%Using the `w' (wield) command when already wielding
%something pushes the old item into your alternate weapon slot (default off).
%Likewise for the `a' (apply) command if it causes the applied item to
%become wielded.  Persistent.
既に武器を持っている時に `w' (wield: 武器を持つ) コマンドを使ったとき、
既に持っていた武器を予備の武器に設定する(標準設定はオフ)。
アイテムを使うことで装備したことになる場合は
`a' (apply) コマンドと同様である。
永続する。
%%.Ip
\item[\ib{race}]
%Selects your race (for example, ``{\tt race:human}'').  Default is random.
%If you prefix the value with `{\tt !}' or ``{\tt no}'', you will
%exclude that race from being picked randomly.
%Cannot be set with the `{\tt O}' command.  Persistent.
種族を選択する(例: ``{\tt race:human}'')。標準設定はランダムである。
値に `{\tt !}' または ``{\tt no}'' を前置すると、ランダムに選択した種族からその種族を
除外できる。
`{\tt O}' コマンドで設定することはできない。
永続する。
%%.lp
\item[\ib{rest\verb+_+on\verb+_+space}]
%Make the space bar a synonym for the `{\tt .}' (\#wait) command (default off).
%Persistent.
スペースキーを `{\tt .}'(\#wait) コマンドとして使用する
(標準設定はオフ)。
永続する。
%%.lp
\item[\ib{role}]
%Pick your type of character (for example, ``{\tt role:Samurai}'');
%synonym for ``{\it character\/}''.  See ``{\it name\/}'' for an alternate method
%of specifying your role.  Normally only the first letter of the
%value is examined; `r' is an exception with ``{\tt Rogue}'', ``{\tt Ranger}'',
%and ``{\tt random}'' values.
%If you prefix the value with `{\tt !}' or ``{\tt no}'', you will
%exclude that role from being picked randomly.
%Cannot be set with the `{\tt O}' command.  Persistent.
キャラクターのタイプを設定する (例:``{\tt role:Samurai}'')。
``{\it character\/}''オプションと同じである。
職業を設定するその他の方法については``{\it name\/}''オプションを参照のこと。
普通は先頭の 1 文字だけで判別されるが、
`r'は例外である。``{\tt Rogue}''、``{\tt Ranger}''、``{\tt random}''があるからである。
値に `{\tt !}' または ``{\tt no}'' を前置すると、ランダムに選択した職業からその職業を
除外できる。
`{\tt O}' コマンドで設定することはできない。
永続する。
%%.lp
\item[\ib{roguesymset}]
%This option may be used to select one of the named symbol sets found within
%{\tt symbols} to alter the symbols displayed on the screen on the
%rogue level.
このオプションは、rogue レベルで表示されるシンボルを変更するために、
{\tt symbols} の中にある名前付きシンボル集合の一つを選択するために
使われる。
%%.lp
\item[\ib{rlecomp}]
%When writing out a save file, perform run length compression of the map.
%Not all ports support run length compression. It has no
%effect on reading an existing save file.
セーブファイルを書き出すときに、地図にランレングス圧縮を行う。
ランレングス圧縮に対応していないシステムもある。
既存のセーブファイルを読み込むときには影響しない。
%%.lp
\item[\ib{runmode}]
%Controls the amount of screen updating for the map window when engaged
%in multi-turn movement (running via {\tt shift}+direction
%or {\tt control}+direction
%and so forth, or via the travel command or mouse click).
「まとめて移動」する({\tt shift}+方向や{\tt control}+方向で移動するか、
旅行コマンドやマウスのクリックを使った場合)
ときにどれくらいの頻度で地図を更新するかを制御する。
%The possible values are:
設定可能な値は以下の通り:

%%.sd
%%.si
%{\tt teleport} --- update the map after movement has finished;\\
{\tt teleport} --- 移動が完了してから地図を更新する;\\
%{\tt run} --- update the map after every seven or so steps;\\
{\tt run} --- 7 歩ぐらい毎に地図を更新する;\\
%{\tt walk} --- update the map after each step;\\
{\tt walk} --- 1 歩毎に地図を更新する;\\
%{\tt crawl} --- like {\it walk\/}, but pause briefly after each step.
{\tt crawl} --- {\it walk\/} と同様だが、一歩毎にしばらく停止する。
%%.ei
%%.ed

%This option only affects the game's screen display, not the actual
%results of moving.  The default is {\it run\/}; versions prior to 3.4.1
%used {\it teleport\/} only.  Whether or not the effect is noticeable will
%depend upon the window port used or on the type of terminal.  Persistent.
このオプションは画面表示にのみ影響し、実際の移動結果には影響しない。
標準設定は {\it run\/} である。バージョン 3.4.1 以前は {\it teleport\/} のみであった。
効果が確認できるかどうかは使っている版や端末の種類に依存する。
永続する。
%%.lp
\item[\ib{safe\verb+_+pet}]
%Prevent you from (knowingly) attacking your pets (default on).  Persistent.
ペットを(ペットと知りつつ)攻撃してしまうのを防ぐ(標準設定はオン)。
永続する。
%%.lp
\item[\ib{sanity\verb+_+check}]
%Evaluate monsters, objects, and map prior to each turn (default off).
%Debug mode only.
各ターンの前に怪物、物、地図を評価する(標準設定はオフ)。
デバッグモードのみ。
%%.lp
\item[\ib{scores}]
%Control what parts of the score list you are shown at the end (for example,
%``{\tt scores:5top scores/4around my score/own scores}'').  Only the first
%letter of each category (`{\tt t}', `{\tt a}' or `{\tt o}') is necessary.
%Persistent.
最後にスコアリストのどの部分を見るかを制御する
(例:``{\tt scores:5top scores/4around my score/own scores}'')。
それぞれの分野の最初の文字 (`{\tt t}', `{\tt a}', `{\tt o}')のみが必要である。
永続する。
%%.lp
\item[\ib{showexp}]
%Show your accumulated experience points on bottom line (default off).
%Persistent.
最下行に現在の経験点を表示する(標準設定はオフ)。
永続する。
%%.lp
\item[\ib{showrace}]
%Display yourself as the glyph for your race, rather than the glyph
%for your role (default off).  Note that this setting affects only
%the appearance of the display, not the way the game treats you.
%Persistent.
あなたを表示するときに、職業に対応するマークではなく、
種族に対応するマークで表示する(標準設定はオフ)。
この設定は表示にだけ用いられ、ゲームとしては何も変わらないことに注意。
永続する。
%%.lp
\item[\ib{showscore}]
%Show your approximate accumulated score on bottom line (default off).
%Persistent.
最下行に現在のスコアを表示する(標準設定はオフ)。
永続する。
%%.lp
\item[\ib{silent}]
%Suppress terminal beeps (default on).  Persistent.
端末のビープ音を鳴らさない(標準設定はオン)。
永続する。
%%.lp
\item[\ib{sortloot}]
%Controls the sorting behavior of pickup lists for inventory
%and \#loot commands and some others.  Persistent.
持ち物一覧や \#loot コマンドやその他いくつかの場合でのソートの振る舞いを
制御する。
永続する。

%The possible values are:
可能な値は:

%%.sd
%%.si
%{\tt full} --- always sort the lists;\\
{\tt full} --- 常に一覧をソートする;\\
%{\tt loot} --- only sort the lists that don't use inventory
%       letters, like with the \#loot and pickup commands;\\
{\tt loot} --- \#loot コマンドや広くコマンドのように、持ち物を示す文字を
使わないときにだけソートする;\\
%{\tt none} --- show lists the traditional way without sorting.
{\tt none} --- ソートなしの伝統的な方法で一覧を表示する。
%%.ei
%%.ed
%%.lp
\item[\ib{sortpack}]
%Sort the pack contents by type when displaying inventory (default on).
%Persistent.
持ち物の目録を表示するとき種類毎に荷物の内容を並べ替える
(標準設定はオン)。
永続する。
%%.lp
\item[\ib{sparkle}]
%Display a sparkly effect when a monster (including yourself) is hit by an
%attack to which it is resistant (default on).  Persistent.
怪物(やあなた)が攻撃を受け、それに抵抗した場合に画面効果を表示する
(標準設定はオン)。
永続する。
%%.lp
\item[\ib{standout}]
%Boldface monsters and ``{\tt --More--}'' (default off).  Persistent.
怪物と``{\tt --More--}''を太字で表示する(標準設定はオフ)。
永続する。
%%.lp
\item[\ib{statushilites}]
%Controls how many turns status hilite behaviors highlight
%the field. If negated or set to zero, disables status hiliting.
%See ``{\it Configuring Status Hilites\/}'' for further information.
何ターンステータスハイライトでフィールドをハイライトするかを制御する。
負またはゼロを指定すると、ステータスハイライトを無効にする。
さらなる情報については ``{\it Configuring Status Hilites\/}'' を参照のこと。
%%.lp
\item[\ib{status\verb+_+updates}]
%Allow updates to the status lines at the bottom of the screen (default true).
画面の下部でのステータス行の更新を認める (標準設定はオン)。
%%.lp
\item[\ib{suppress\verb+_+alert}]
%This option may be set to a {\it NetHack\/} version level to suppress
%alert notification messages about feature changes for that
%and prior versions (for example, ``{\tt suppress\verb+_+alert:3.3.1}'')
前のバージョンから変更された機能に対して注意を促すメッセージを
表示しないようにする {\it NetHack\/} のバージョンを設定する。
(例: ``{\tt suppress\verb+_+alert:3.3.1}'')
例のように指定すると 3.3.1 以前に変更された機能に対する注意メッセージは表示されない。
%%.lp
\item[\ib{symset}]
%This option may be used to select one of the named symbol sets found within
%{\tt symbols} to alter the symbols displayed on the screen.
%Use ``{\tt symset:default}'' to explicitly select the default symbols.
このオプションは、画面に表示される文字を変更するために、{\tt symbols} にある
名前付きシンボル集合の一つを選択するために使われる。
標準の文字を選択するには明示的に ``{\tt symset:default}'' を使うこと。
%%.lp
\item[\ib{time}]
%Show the elapsed game time in turns on bottom line (default off).  Persistent.
最下行にゲームで経過した時間をターン数で表示する(標準設定はオフ)。
永続する。
%%.lp
\item[\ib{timed\verb+_+delay}]
%When pausing momentarily for display effect, such as with explosions and
%moving objects, use a timer rather than sending extra characters to the
%screen.  (Applies to ``tty'' interface only; ``X11'' interface always
%uses a timer based delay.  The default is on if configured into the
%program.)  Persistent.
爆発や物体の移動など、効果を表示するためにちょっと時間待ちするとき、
特殊なキャラクタを画面に送るかわりにタイマーを使用する。
(``tty'' インタフェースにのみ適用される。
``X11'' インタフェースは常にタイマーを使用する。
標準設定はタイマーによる時間待ちが組み込まれているならばオン)
永続する。
%%.lp
\item[\ib{tombstone}]
%Draw a tombstone graphic upon your death (default on).  Persistent.
死んだとき墓石を表示する(標準設定はオン)。
永続する。
%%.lp
\item[\ib{toptenwin}]
%Put the ending display in a {\it NetHack\/} window instead of on stdout (default off).
%Setting this option makes the score list visible when a windowing version
%of {\it NetHack\/} is started without a parent window, but it no longer leaves
%the score list around after game end on a terminal or emulating window.
ゲーム終了時の表示を標準出力のかわりに {\it NetHack\/} のウィンドウに表示する
(標準設定はオフ)。
このオプションを設定すると
ウィンドウを使うバージョンの {\it NetHack\/} では
起動した親ウィンドウとは別のウィンドウにスコアリストが表示される。
しかし、スコアリストはゲームが終了したあとターミナルやウィンドウには残らない。
%%.lp
\item[\ib{travel}]
%Allow the travel command (default on).  Turning this option off will
%prevent the game from attempting unintended moves if you make inadvertent
%mouse clicks on the map window.  Persistent.
トラベルコマンドを有効にする(標準設定はオン)。
このオプションをオフにすることによって、うっかり地図ウィンドウを
クリックすることによって、望まない移動をしようとすることを防ぐことができる。
永続する。
%% %.lp
%% \item[ib{travel\verb+_+debug}]
%% Display intended path during each step of travel (default off).
%% Debug mode only.
%%.lp
\item[\ib{verbose}]
%Provide more commentary during the game (default on).  Persistent.
ゲーム中のコメントを詳細に表示する(標準設定はオン)。
永続する。
%%.lp
\item[\ib{whatis\verb+_+coord}]
%When using the `{\tt /}' or `{\tt ;}' commands to look around on the map with
%``{\tt autodescribe}''
%on, display coordinates after the description.
%Also works in other situations where you are asked to pick a location.\\
When using the 
``{\tt autodescribe}'' がオンのときに地図上を見て回るために
`{\tt /}' や `{\tt ;}' コマンドを使うとき、説明の後ろに座標を表示する。
場所を選択するその他の状況でも動作する。\\

%%.lp ""
%The possible settings are:
可能な設定は:

%%.sd
%%.si
%{\tt c} --- \verb#compass ('east' or '3s' or '2n,4w')#;\\
%{\tt f} --- \verb#full compass ('east' or '3south' or '2north,4west')#;\\
%{\tt m} --- \verb#map <x,y> (map column x=0 is not used)#;\\
%{\tt s} --- \verb#screen [row,column] (row is offset to match tty usage)#;\\
%{\tt n} --- \verb#none (no coordinates shown) [default]#.
{\tt c} --- \verb#コンパス ('east', '3s', '2n,4w')#;\\
{\tt f} --- \verb#完全コンパス ('east', '3south', '2north,4west')#;\\
{\tt m} --- \verb#地図 <x,y> (地図列 x=0 は使われない)#;\\
{\tt s} --- \verb#画面 [row,column] (row は tty 利用法に一致するオフセット)#;\\
{\tt n} --- \verb#なし (座標は表示されない) [標準設定]#.
%%.ei
%%.ed

%%.lp ""
%The
%{\it whatis\verb+_+coord\/}
%option is also used with
%the `{\tt /m}', `{\tt /M}', `{\tt /o}', and `{\tt /O}' sub-commands
%of `{\tt /}',
%where the `{\it none\/}' setting is overridden with `{\it map}'.
{\it whatis\verb+_+coord\/}
オプションは、`{\tt /}' の
`{\tt /m}', `{\tt /M}', `{\tt /o}', `{\tt /O}' サブコマンドでも使用できる;
ここで `{\it none\/}' 設定は `{\it map}' で上書きされる。
%%.lp
\item[\ib{whatis\verb+_+filter}]
%When getting a location on the map, and using the keys to cycle through
%next and previous targets, allows filtering the possible targets.
%(default none)\\
地図上の場所を指定するときに、
目標の中を循環して移動するキーのいずれかを使っているとき、
可能性のある目標をフィルタリングする。
(標準設定はなし)\\
%%.lp ""
%The possible settings are:
可能な設定は:

%%.sd
%%.si
%{\tt n} --- \verb#no filtering#;\\
%{\tt v} --- \verb#in view only#;\\
%{\tt a} --- \verb#in same area (room, corridor, etc)#.
{\tt n} --- \verb#フィルタリングなし#;\\
{\tt v} --- \verb#視界内のみ#;\\
{\tt a} --- \verb#同じエリア(部屋、通路など)#.
%%.ei
%%.ed
%%.lp ""
%The area-filter tries to be slightly predictive -- if you're standing
%on a doorway, it will consider the area on the side of the door you
%were last moving towards.\\
エリアフィルタはやや予言的である -- 出入り口に立っている場合、
移動してきたエリア側にいるものとして扱われる。
%%.lp ""
%Filtering can also be changed when getting a location with
%the ``getpos.filter'' key.
フィルタリングは、``getpos.filter'' キーで場所を指定するときにも
変更できる。
%%.lp
\item[\ib{whatis\verb+_+menu}]
%When getting a location on the map, and using a key to cycle through
%next and previous targets, use a menu instead to pick a target.
%(default off)
地図上の場所を指定するときに、
目標の中を循環して移動するキーのいずれかを使っているとき、
代わりにメニューを表示する。
(標準設定はオフ)
%%.lp
\item[\ib{whatis\verb+_+moveskip}]
%When getting a location on the map, and using shifted movement keys or
%meta-digit keys to fast-move, instead of moving 8 units at a time,
%move by skipping the same glyphs.
%(default off)
地図上の場所を指定するときに、
高速移動のためにシフトを押しながらの移動や、
meta-数字 キーでの移動を使っているとき、8 マス毎ではなく同じ記号を飛ばして
移動する。
(標準設定はオフ)
%%.lp
\item[\ib{windowtype}]
%When the program has been built to support multiple interfaces,
%select whichone to use, such as ``{\tt tty}'' or ``{\tt X11}''
%(default depends on build-time settings; use ``{\tt \#version}'' to check).
プログラムが複数のインターフェースに対応してビルドされているとき、
ウインドウシステムを使用するかどうかを「tty」または「X11」で選択する
(標準設定はビルド時の設定による; ``{\tt \#version}'' でチェックできる)。
%Cannot be set with the `{\tt O}' command.
`{\tt O}' コマンドで設定することはできない。

%%.lp ""
%When used, it should be the first option set since its value might
%enable or disable the availability of various other options.
%For multiple lines in a configuration file, that would be the first
%non-comment line.
%For a comma-separated list in NETHACKOPTIONS or an OPTIONS line in a
%configuration file, that would be the {\it rightmost\/} option in the list.
使われる場合は、これは設定される最初のオプションである必要がある;
この値はその他の様々なオプションの可用性を有効にしたり無効にしたり
するかもしれないからである。
設定ファイルに複数行ある場合、これは最初の非コメント行である。
NETHACKOPTIONS や設定ファイル中の OPTIONS 行でのカンマ区切りリストについては、
リストの {\it 最も右\/} のオプションである。
%%.lp
\item[\ib{wizweight}]
%Augment object descriptions with their objects' weight (default off).
%Debug mode only.
物の説明にその物の重さを追加する(標準設定はオフ)。
デバッグモードのみ。
%%.lp
\item[\ib{zerocomp}]
%When writing out a save file, perform zero-comp compression of the
%contents. Not all ports support zero-comp compression. It has no effect
%on reading an existing save file.
セーブファイルを書くときに、内容に zero-comp 圧縮を掛ける。
全てのポートで zero-comp 圧縮に対応しているわけではない。
既存のセーブファイルを読み込むときには効果はない。
\elist

%%.hn 2
%\subsection*{Window Port Customization options}
\subsection*{ウインドウシステム版専用カスタマイズオプション}

%%.pg
%Here are explanations of the various options that are
%used to customize and change the characteristics of the
%windowtype that you have chosen.
以下はあなたが選択したウィンドゥタイプの特性をカスタマイズして
変更するために用いる様々なオプションの説明である。
%Character strings that are too long may be truncated.
%Not all window ports will adjust for all settings listed
%here.  You can safely add any of these options to your
%configuration file, and if the window port is capable of adjusting
%to suit your preferences, it will attempt to do so. If it
%can't it will silently ignore it.  You can find out if an
%option is supported by the window port that you are currently
%using by checking to see if it shows up in the Options list.
%Some options are dynamic and can be specified during the game
%with the `{\tt O}' command.
長すぎる文字列は適当に切り詰められる。
全てのウィンドゥタイプに対してこれら全てのオプションが有効というわけではない。
これらのオプションが設定されたとき、もしウィンドゥタイプがその設定を
受け付けるなら、そうされる。もし受け付けられないなら、単に無視される。
現在のウィンドゥタイプで対応しているオプションは、オプション一覧表示
(`{\tt O}')で表示される。
一部のオプションは`{\tt O}'コマンドで動的に変更できる。

\blist{}
%%.lp
\item[\ib{align\verb+_+message}]
% Where to align or place the message window (top, bottom, left, or right)
メッセージウィンドウをどこに置くか(top, bottom, left, right)
%%.lp
\item[\ib{align\verb+_+status}]
% Where to align or place the status window (top, bottom, left, or right).
ステータスウィンドウをどこに置くか(top, bottom, left, right)
%%.lp
\item[\ib{ascii\verb+_+map}]
%If {\it NetHack\/} can, it should display an ascii map.
可能ならASCIIキャラクタマップを表示する
%%.lp
\item[\ib{color}]
%If {\it NetHack\/} can, it should display color for different monsters,
%objects, and dungeon features.
可能なら怪物、物体、洞窟の構成要素をカラーで表示する。

%%.lp
\item[\ib{eight\verb+_+bit\verb+_+tty}]
%If {\it NetHack\/} can, it should pass eight-bit character values (for example, specified with the
%{\it traps \/} option) straight through to your terminal (default off).
可能なら8ビットキャラクタ(例えば
{\it traps\/}
オプションで指定したもの)をそのままターミナルに表示する(標準設定はオフ)。
%%.lp
\item[\ib{font\verb+_+map}]
%If {\it NetHack\/} can, it should use a font by the chosen name for the
%map window.
可能ならマップウィンドウにここで選んだ名前のフォントを使う。
%%.lp
\item[\ib{font\verb+_+menu}]
%If {\it NetHack\/} can, it should use a font by the chosen name for menu
%windows.
可能ならメニューウィンドウにここで選んだ名前のフォントを使う。
%%.lp
\item[\ib{font\verb+_+message}]
%If {\it NetHack\/} can, it should use a font by the chosen name for the message window.
可能ならメッセージウィンドウにここで選んだ名前のフォントを使う。
%%.lp
\item[\ib{font\verb+_+status}]
%If {\it NetHack\/} can, it should use a font by the chosen name for the status window.
可能ならステータスウィンドウにここで選んだ名前のフォントを使う。
%%.lp
\item[\ib{font\verb+_+text}]
%If {\it NetHack\/} can, it should use a font by the chosen name for text windows.
可能ならテキストウィンドウにここで選んだ名前のフォントを使う。
%%.lp
\item[\ib{font\verb+_+size\verb+_+map}]
%If {\it NetHack\/} can, it should use this size font for the map window.
可能ならマップウィンドウにこのフォントサイズを使う。
%%.lp
\item[\ib{font\verb+_+size\verb+_+menu}]
%If {\it NetHack\/} can, it  should use this size font for menu windows.
可能ならメニューウィンドウにこのフォントサイズを使う。
%%.lp
\item[\ib{font\verb+_+size\verb+_+message}]
%If {\it NetHack\/} can, it should use this size font for the message window.
可能ならメッセージウィンドウにこのフォントサイズを使う。
%%.lp
\item[\ib{font\verb+_+size\verb+_+status}]
%If {\it NetHack\/} can, it should use this size font for the status window.
可能ならステータスウィンドウにこのフォントサイズを使う。
%%.lp
\item[\ib{font\verb+_+size\verb+_+text}]
%If {\it NetHack\/} can, it should use this size font for text windows.
可能ならテキストウィンドウにこのフォントサイズを使う。
%%.lp
\item[\ib{fullscreen}]
%If {\it NetHack\/} can, it should try and display on the entire screen rather than in a window.
可能ならウィンドウでなく画面全体で表示しようと試みる。
%%.lp
\item[\ib{guicolor}]
%Use color text and/or highlighting attributes when displaying some
%non-map data (such as menu selector letters).
%Curses interface only; default is on.
(メニュー選択の文字のような) 一部の非マップデータを表示するときに
テキストの色やハイライト属性を使う。
cursesインターフェースのみ; 標準設定はオン。
%%.lp
\item[\ib{large\verb+_+font}]
%If {\it NetHack\/} can, it should use a large font.
可能なら大きいフォントを使う。
%%.lp
\item[\ib{map\verb+_+mode}]
%If {\it NetHack\/} can, it should display the map in the manner specified.
可能なら指定された方法でマップを表示する。
%%.lp
\item[\ib{player\verb+_+selection}]
%If {\it NetHack\/} can, it should pop up dialog boxes or use prompts for character selection.
可能ならキャラクター選択時にダイアログボックスや確認画面を用いる。
%%.lp
\item[\ib{popup\verb+_+dialog}]
%If {\it NetHack\/} can, it should pop up dialog boxes for input.
可能なら入力時にポップアップダイアログボックスを用いる
%%.lp
\item[\ib{preload\verb+_+tiles}]
%If {\it NetHack\/} can, it should preload tiles into memory.
%For example, in the protected mode MS-DOS version, control whether tiles
%get pre-loaded into RAM at the start of the game.  Doing so
%enhances performance of the tile graphics, but uses more memory. (default on).
%Cannot be set with the `{\tt O}' command.
可能ならタイルをメモリに予め読み込んでおく。
例えば、プロテクトモードMS-DOS版の場合、ゲーム開始時にタイルを
RAM に予め読み込むかどうかを制御する。
予め読み込むとタイルグラフィックの性能は向上するが、より多くのメモリを消費する
(標準設定はオン)。
`{\tt O}' コマンドで設定することはできない。
%%.lp
\item[\ib{scroll\verb+_+amount}]
%If {\it NetHack\/} can, it should scroll the display by this number of cells
%when the hero reaches the scroll\verb+_+margin.
可能なら、scroll\verb+_+margin オプションで指定された位置に来たときに、
どれだけの数スクロールさせるかを指定する。
%%.lp
\item[\ib{scroll\verb+_+margin}]
%If {\it NetHack\/} can, it should scroll the display when the hero or cursor
%is this number of cells away from the edge of the window.
可能なら、ウィンドウの端からここで指定されたマス数にあなたまたはカーソルが
移動したときに、画面をスクロールさせる。
%%.lp
\item[\ib{selectsaved}]
%If {\it NetHack\/} can, it should display a menu of existing saved games for the player to
%choose from at game startup, if it can. Not all ports support this option.
可能なら、ゲーム開始時に、可能ならプレイヤーに選択させるために既存のセーブしたゲームの
メニューを表示する。
全てのポートがこのオプションに対応しているわけではない。
%%.lp
\item[\ib{softkeyboard}]
%If {\it NetHack\/} can, it should display an onscreen keyboard.
%Handhelds are most likely to support this option.
可能なら、ソフトウェアキーボードを表示する。
ハンドヘルドはおそらくこのオプションに対応しているであろう。
%%.lp
\item[\ib{splash\verb+_+screen}]
%If {\it NetHack\/} can, it should display an opening splash screen when
%it starts up (default yes).
可能なら、起動時にスプラッシュスクリーンを表示する(標準設定はオン)
%%.lp
\item[\ib{statuslines}]
%Number of lines for traditional below-the-map status display.
%Acceptable values are 2 and 3 (default is 2).
%Curses and tty interfaces only.
伝統的な地図の下のステータス表示の行数。
設定可能な値は 2 と 3 (標準設定は 2)。
curses と tty インターフェースのみ。
%%.lp
\item[\ib{term\verb+_+cols} {\normalfont and}]
%%.lp
\item[\ib{term\verb+_+rows}]
%Curses interface only.
%Number of columns and rows to use for the display.
%Curses will attempt to resize to the values specified but will settle
%for smaller sizes if they are too big.
%Default is the current window size.
cursesインターフェースのみ。
表示に使う桁数と行数。
cursesは指定された値にリサイズしようとするが、
大きすぎる場合にはより小さいサイズに合わせる。
標準設定は現在のウィンドウサイズ。
%%.lp
\item[\ib{tiled\verb+_+map}]
%If {\it NetHack\/} can, it should display a tiled map if it can.
可能なら、タイルマップを表示する。
%%.lp
\item[\ib{tile\verb+_+file}]
%Specify the name of an alternative tile file to override the default.
標準設定を上書きする別のタイルファイルの名前
%%.lp
\item[\ib{tile\verb+_+height}]
%Specify the preferred height of each tile in a tile capable port.
タイルが表示できる環境でのタイルの高さ
%%.lp
\item[\ib{tile\verb+_+width}]
%Specify the preferred width of each tile in a tile capable port
タイルが表示できる環境でのタイルの幅
%%.lp
\item[\ib{use\verb+_+darkgray}]
%Use bold black instead of blue for black glyphs (TTY only).
黒記号のために青ではなくボールドの黒を使う (TTY のみ)。
%%.lp
\item[\ib{use\verb+_+inverse}]
%If {\it NetHack\/} can, it should display inverse when the game specifies it.
可能なら、ゲームが指定したときに画面を反転する。
%%.lp
\item[\ib{vary\verb+_+msgcount}]
%If {\it NetHack\/} can, it should display this number of messages at a time
%in the message window.
可能なら、メッセージウィンドウに一度に表示するメッセージの数を指定する。
%%.lp
\item[\ib{windowborders}]
%Whether to draw boxes around the map, status area, message area, and
%persistent inventory window if enabled.
%Curses interface only.
%Acceptable values are
マップ、ステータスエリア、メッセージエリア、および有効の場合は
固定持ち物ウィンドウの周りに箱を書くかどうか。
cursesインターフェースのみ。
可能な値は

%%.sd
%%.si
%{\tt 0} --- off, never show borders\\
%{\tt 1} --- on, always show borders\\
%{\tt 2} --- auto, on display is at least
%(\verb&24+2&)x(\verb&80+2&)\ \ (default)\\
{\tt 0} --- オフ、常に境界を表示しない\\
{\tt 1} --- オン、常に境界を表示する\\
{\tt 2} --- 自動、最低
(\verb&24+2&)x(\verb&80+2&) ある場合に表示 \ \ (標準設定)\\
%%.ei
%%.ed

%%.lp "
%(The 26x82 size threshold for `2' refers to number of rows and
%columns of the display.
%A width of at least 110 columns (\verb&80+2+26+2&) is needed for
%{\it align_status\/}
%set to {\tt left} or {\tt right}.)
(`2' の 26x82 というサイズの閾値は、画面の行数と列数から決定されている。
{\it align_status\/} を {\tt left} または {\tt right} に設定するには、
最低 110 桁 (\verb&80+2+26+2&) の幅が必要である。
%%.lp
\item[\ib{windowcolors}]
%If {\it NetHack\/} can, it should display windows with the specified
%foreground/background colors. Windows GUI only. The format is
可能ならウィンドウを指定した前景色/背景色で表示する。
Windows GUIのみ。形式は:
\begin{verbatim}
    OPTION=windowcolors:wintype foreground/background
\end{verbatim}

%%.pg
%where wintype is one of {\it menu}, {\it message}, {\it status}, or {\it text}, and
%foreground and background are colors, either a hexadecimal {\it \#rrggbb},
%one of the named colors ({\it black}, {\it red}, {\it green}, {\it brown},
%{\it blue}, {\it magenta}, {\it cyan}, {\it orange},
%{\it brightgreen}, {\it yellow}, {\it brightblue}, {\it brightmagenta},
%{\it brightcyan}, {\it white}, {\it trueblack}, {\it gray}, {\it purple},
%{\it silver}, {\it maroon}, {\it fuchsia}, {\it lime}, {\it olive},
%{\it navy}, {\it teal}, {\it aqua}), or one of Windows UI colors ({\it activeborder},
%{\it activecaption}, {\it appworkspace}, {\it background}, {\it btnface}, {\it btnshadow},
%{\it btntext}, {\it captiontext}, {\it graytext}, {\it greytext}, {\it highlight},
%{\it highlighttext}, {\it inactiveborder}, {\it inactivecaption}, {\it menu},
%{\it menutext}, {\it scrollbar}, {\it window}, {\it windowframe}, {\it windowtext}).
ここで wintype {\it menu}, {\it message}, {\it status}, {\it text} のいずれか、
foreground と background は色で、16 進 {\it \#rrggbb} または
名前付き色の一つ ({\it black}, {\it red}, {\it green}, {\it brown},
{\it blue}, {\it magenta}, {\it cyan}, {\it orange},
{\it brightgreen}, {\it yellow}, {\it brightblue}, {\it brightmagenta},
{\it brightcyan}, {\it white}, {\it trueblack}, {\it gray}, {\it purple},
{\it silver}, {\it maroon}, {\it fuchsia}, {\it lime}, {\it olive},
{\it navy}, {\it teal}, {\it aqua}),
または Windows UI 色の一つ ({\it activeborder},
{\it activecaption}, {\it appworkspace}, {\it background}, {\it btnface}, {\it btnshadow},
{\it btntext}, {\it captiontext}, {\it graytext}, {\it greytext}, {\it highlight},
{\it highlighttext}, {\it inactiveborder}, {\it inactivecaption}, {\it menu},
{\it menutext}, {\it scrollbar}, {\it window}, {\it windowframe}, {\it windowtext})。

%%.lp
\item[\ib{wraptext}]
%If {\it NetHack\/} can, it should wrap long lines of text if they don't fit
%in the visible area of the window.
可能ならウィンドウに収まらない長いテキストを折りたたむ。
\elist

%%.hn 2
%\subsection*{Platform-specific Customization options}
\subsection*{プラットフォーム固有の設定オプション}

%%.pg
%Here are explanations of options that are used by specific platforms
%or ports to customize and change the port behavior.
以下は特定のプラットフォームでカスタマイズや振る舞いの変更をするために
用いられるオプションの説明である。

\blist{}
%%.lp
\item[\ib{altkeyhandler}]
%Select an alternate keystroke handler dll to load ({\it Win32 tty\/ NetHack\/} only).
%The name of the handler is specified without the .dll extension and without any
%path information.
%Cannot be set with the `{\tt O}' command.
キー入力を扱う DLL を選択する({\it Win32 tty\/ NetHack\/} のみ)。
DLL の名前は、.dll の拡張子やパス抜きで指定する。
`{\tt O}' コマンドで設定することはできない。
%%.lp
\item[\ib{altmeta}]
%On Amiga, this option controls whether typing ``Alt'' plus another key
%functions as a meta-shift for that key (default on).
Amigaでは、このオプションは ``Alt'' と他のキーをタイプしたときに
そのキーの meta-shift として機能するかどうかを制御する(標準設定はオン)。
%%.lp
\item[\ib{altmeta}]
%On other (non-Amiga) systems where this option is available, it can be
%set to tell {\it NetHack\/} to convert a two character sequence beginning with
%ESC into a meta-shifted version of the second character (default off).
その他 (Amiga 以外) のシステムでこのオプションが有効の場合、
ESC で始まる 2 文字並びを 2 番目の文字の meta-shift に変換するかどうかを
指定する (標準設定はオフ)。

%%.lp ""
%This conversion is only done for commands, not for other input prompts.
%Note that typing one or more digits as a count prefix prior to a
%command---preceded by {\tt n} if the {\it number\verb+_+pad\/}
%option is set---is also subject to this conversion, so attempting to
%abort the count by typing ESC will leave {\it NetHack\/} waiting for another
%character to complete the two character sequence.  Type a second ESC to
%finish cancelling such a count.  At other prompts a single ESC suffices.
この変換はコマンドに対してのみ行われ、その他の入力プロンプトでは行われない。
コマンドの前にカウント接頭辞としてのいくつかの数字のタイプも、
---
{\it number\verb+_+pad\/}
オプションが設定されているなら {\tt n} を前置して---、
この変換が行われるので、
ESC をタイプすることでカウントを中断しようとすると、
{\it NetHack\/} は 2 文字並びを完成させるために待つことになる。
2 回 ESC をタイプすることでそのようなカウントのキャンセルが完了する。
その他のプロンプトでは単一の ESC で十分である。
%%.lp
\item[\ib{BIOS}]
%Use BIOS calls to update the screen display quickly and to read the keyboard
%(allowing the use of arrow keys to move) on machines with an IBM PC
%compatible BIOS ROM (default off, {\it OS/2, PC\/ {\rm and} ST NetHack\/} only).
高速に画面を書き換え、移動にカーソルキーを使えるようにキーボードを読むために、
IBM-PC 互換の BIOS ROM を使用しているマシンで BIOS コールを使用する。
(標準設定はオフ。{\it OS/2, PC, ST NetHack\/} のみ)
%%.lp
\item[\ib{flush}]
%(default off, {\it Amiga NetHack \/} only).
(標準設定はオフ。{\it Amiga NetHack \/} のみ)。
%%.lp
\item[\ib{Macgraphics}]
%(default on, {\it Mac NetHack \/} only).
(標準設定はオン。 {\it Mac NetHack \/} のみ)。
%%.lp
\item[\ib{page\verb+_+wait}]
%(default off, {\it Mac NetHack \/} only).
(標準設定はオフ。{\it Mac NetHack \/} のみ)。
%%.lp
\item[\ib{rawio}]
%Force raw (non-cbreak) mode for faster output and more
%bulletproof input (MS-DOS sometimes treats `{\tt \^{}P}' as a printer toggle
%without it) (default off, {\it OS/2, PC\/ {\rm and} ST NetHack\/} only).
%Note:  DEC Rainbows hang if this is turned on.
%Cannot be set with the `{\tt O}' command.
raw(cbreak でない) モードを使用して、
より高速な出力と問題の起こらない入力を実現する
(MS-DOS ではプリンタがないにも関わらず `{\tt \^{}P}' をプリンタ出力の
トグルとみなしてしまうことがある)
(標準設定はオフ; {\it OS/2, PC, ST NetHack\/} のみ)。
注意: DEC Rainbow ではこれがオンのときはハングアップしてしまう。
`{\tt O}' コマンドで設定することはできない。
%%.lp
\item[\ib{soundcard}]
%(default off, {\it PC NetHack \/} only).
%Cannot be set with the `{\tt O}' command.
(標準設定はオン; {\it PC NetHack \/} のみ)。
`{\tt O}' コマンドで設定することはできない。
%%.lp
\item[\ib{subkeyvalue}]
%({\it Win32 tty NetHack \/} only).
%May be used to alter the value of keystrokes that the operating system
%returns to {\it NetHack\/} to help compensate for international keyboard
%issues.
%OPTIONS=subkeyvalue:171/92
%will return 92 to {\it NetHack\/}, if 171 was originally going to be returned.
%You can use multiple subkeyvalue statements in the configuration file
%if needed.
%Cannot be set with the `{\tt O}' command.
({\it Win32 tty NetHack \/} のみ)。
国際キーボードの問題を補正するのを助けるために、
OS が {\it NetHack\/} に返すキー入力の値を変更するために使用する。
OPTIONS=subkeyvalue:171/92
とすると、もともと 171 が返されようとしていた場合、
{\it NetHack\/} に 92 を返す。
必要なら、複数の subkeyvalue 行を設定ファイルに書いてもよい。
`{\tt O}' コマンドで設定することはできない。
%%.lp
\item[\ib{video}]
%Set the video mode used ({\it PC\/ NetHack\/} only).
%Values are {\it autodetect\/}, {\it default\/}, or {\it vga\/}.
%Setting {\it vga\/} (or {\it autodetect\/} with vga hardware present) will
%cause the game to display tiles.
%Cannot be set with the `{\tt O}' command.
使用するビデオモードを設定する({\it PC\/ NetHack\/} のみ)。
値は {\tt autodetect\/}, {\tt default\/}, {\tt vga\/} のいずれかである。
{\tt vga\/} (または VGA ハードウェアがあるときに{\tt autodetect\/})に設定すると、
表示にタイルを用いる。
`{\tt O}' コマンドで設定することはできない。
%%.lp
\item[\ib{videocolors}]
\begin{sloppypar}
%Set the color palette for PC systems using NO\verb+_+TERMS
%(default 4-2-6-1-5-3-15-12-10-14-9-13-11, {\it PC\/ NetHack\/} only).
NO\verb+_+TERMS を使用しているPCシステムのカラーパレットをセットする
(標準設定は「4-2-6-1-5-3-15-12-10-14-9-13-11」{\it PC\/ NetHack\/} のみ)。
%The order of colors is red, green, brown, blue, magenta, cyan,
%bright.white, bright.red, bright.green, yellow, bright.blue,
%bright.magenta, and bright.cyan.
シンボルの順番は、赤・緑・茶色・青・マゼンタ・シアン・輝く白・
輝く赤・輝く緑・黄色・輝く青・輝くマゼンタ・輝くシアン。
%Cannot be set with the `{\tt O}' command.
`{\tt O}' コマンドで設定することはできない。
\end{sloppypar}
%%.lp
\item[\ib{videoshades}]
%Set the intensity level of the three gray scales available
%(default dark normal light, {\it PC\/ NetHack\/} only).
%If the game display is difficult to read, try adjusting these scales;
%if this does not correct the problem, try {\tt !color}.
3 段階の利用可能なグレイスケールを設定する
(標準設定は dark normal light; {\it PC\/ NetHack\/} のみ)。
もしゲーム画面が見にくい場合は、
3つのスケールを調整してみること。
もしこれでうまくいかなかったら、{\tt !color} を試してみること。
%Cannot be set with the `{\tt O}' command.
`{\tt O}' コマンドで設定することはできない。
\elist

%%.nh 2
%\subsection*{Regular Expressions}
\subsection*{正規表現}

%%.pg
%Regular expressions are normally POSIX extended regular expressions. It is
%possible to compile {\it NetHack\/} without regular expression support on
%a platform where
%there is no regular expression library. While this is not true of any modern
%platform, if your {\it NetHack\/} was built this way, patterns are instead glob
%patterns. This applies to Autopickup exceptions, Message types, Menu colors,
%and User sounds.
正規表現は通常 POSIX 拡張正規表現である。
正規表現ライブラリがないプラットフォームでは正規表現対応なしで
{\it NetHack\/} をコンパイルすることも可能である。
これは最近のプラットフォームでは真ではないが、この方法で {\it NetHack\/} が
ビルドされると、パターンは代わりにグロブパターンが使われる。
これは自動拾い例外、メッセージ型、メニュー色、ユーザー音声に適用される。

%%.hn 2
%\subsection*{Configuring Autopickup Exceptions}
\subsection*{自動拾い例外の設定}

%%.pg
%You can further refine the behavior of the ``{\tt autopickup}'' option
%beyond what is available through the ``{\tt pickup\verb+_+types}'' option.
``{\it autopickup\/}'' オプションの振る舞いを
``{\tt pickup\verb+_+types}'' オプションを使ってより洗練させることができる。

%%.pg
%By placing ``{\tt autopickup\verb+_+exception}'' lines in your configuration
%file, you can define patterns to be checked when the game is about to
%autopickup something.
``{\tt autopickup\verb+_+exception}''
行を設定ファイルに書くことで、
何かを拾おうとするときにチェックするべきパターンを定義できる。

\blist{}
%%.lp
\item[\ib{autopickup\verb+_+exception}]
%Sets an exception to the ``{\it pickup\verb+_+types}'' option.
`{\it pickup\verb+_+types\/}' オプションの例外を設定する。
%The {\it autopickup\verb+_+exception\/} option should be followed by a regular
%expression to be used as a pattern to match against the singular form of the
%description of an object at your location.
{\it autopickup\verb+_+exception\/}
オプションはあなたの位置にある物体の説明の単数形に一致するパターンを
指定する。これは正規表現であるべきである。

%In addition, some characters are treated specially if they occur as the first
%character in the pattern, specifically:
さらに、パターンの先頭に置いた場合に特別扱いされる文字もある。

%%.sd
%%.si
%{\tt <} --- always pickup an object that matches rest of pattern;\\
%{\tt >} --- never pickup an object that matches rest of pattern.
{\tt <} --- 以下のパターンに一致する物体は常に拾う。\\
{\tt >} --- 以下のパターンに一致する物体は決して拾わない。
%%.ei
%%.ed

%The {\it autopickup\verb+_+exception\/} rules are processed in the order
%in which they appear in your configuration file, thus allowing a
%later rule to override an earlier rule.
{\it autopickup\verb+_+exception\/} 規則は、
設定ファイルで現れた順に処理されるので、
後の規則で前の規則を上書きできる。

%%.lp ""
%Exceptions can be set with the `{\tt O}' command, but because they are not
%included in your configuration file, they won't be in effect if you save
%and then restore your game.
%{\it autopickup\verb+_+exception\/} rules are not saved with the game.
`{\tt O}' コマンドで設定することもできるが、
これらは設定ファイルに含まれないので、ゲームをセーブして復元すると
適用されない。
{\it autopickup\verb+_+exception\/} ルールはゲームと共に保存されない。
\elist

%%.lp "Here are some examples:"
%Here are some examples:
以下は例である:
\begin{verbatim}
    autopickup_exception="<*arrow"
    autopickup_exception=">*corpse"
    autopickup_exception=">* cursed*"
\end{verbatim}

%%.pg
%The first example above will result in autopickup of any type of arrow.
%The second example results in the exclusion of any corpse from autopickup.
%The last example results in the exclusion of items known to be cursed from
%autopickup.
上記の例の最初のものは、全ての矢を自動的に拾う。
二番目の例は、死体は自動拾いの例外とする。
最後の例は、呪われているとわかっているものは自動拾いの例外とする。

%%.lp

%%.hn 2
%\subsection*{Changing Key Bindings}
\subsection*{キー配置を変更する}

%%.pg
%It is possible to change the default key bindings of some special commands,
%menu accelerator keys, and extended commands, by using BIND stanzas in the
%configuration file. Format is key, followed by the command to bind to,
%separated by a colon. The key can be a single character (``{\tt x}''),
%a control key (``{\tt \^{}X}'', ``{\tt C-x}''), a meta key (``{\tt M-x}''),
%or a three-digit decimal ASCII code.
設定ファイルで BIND 文を使うことにより、一部の特殊コマンド、
メニューアクセラレータキー、拡張コマンドのデフォルトのキーバインディングを
変更できる。
形式は、キーの後に、コロンで区切られたバインド先のコマンドが続く。
キーには、単一文字(「``{\tt x}'')、制御キー
(``{\tt \^{}X}'', ``{\tt C-x}'')、メタキー(``{\tt M-x}'')、または 3 桁の
10 進 ASCII コードを使用できる。

%%.pg
%For example:
例:

\begin{verbatim}
    BIND=^X:getpos.autodescribe
    BIND={:menu_first_page
    BIND=v:loot
\end{verbatim}

\blist{}
%%.lp "Extended command keys"
%\item[\tb{Extended command keys}]
\item[\tb{拡張コマンドキー}]
%You can bind multiple keys to the same extended command. Unbind a key by
%using ``{\tt nothing}'' as the extended command to bind to. You can also bind
%the ``{\tt <esc>}'', ``{\tt <enter>}'', and ``{\tt <space>}'' keys.
複数のキーを一つの拡張コマンドに割り当てることができる。
割り当てられている拡張コマンドとして ``{\tt nothing}'' を使うことで
キーへの割り当てを解除する。
また、``{\tt <esc>}'', ``{\tt <enter>}'', ``{\tt <space>}'' キーにも
割り当てることができる。

%%.lp "Menu accelerator keys"
%\item[\tb{Menu accelerator keys}]
\item[\tb{メニューアクセラレーターキー}]
%The menu control or accelerator keys can also be rebound via OPTIONS lines
%in the configuration file.
%You cannot bind object symbols into menu accelerators.
メニュー制御やアクセラレーターキーも設定ファイルのOPTIONS行を通して
再割り当てできる。
物のシンボルをメニューアクセラレーターに割り当てることはできない。

%%.lp "Special command keys"
%\item[\tb{Special command keys}]
\item[\tb{特殊コマンドキー}]
以下は再割り当てできる特殊コマンドである。
その一部は問題なく同じキーに割り当てることができる;
それ以外は同じ「コンテキスト」にあり、同じキーに割り当てると、
そのうち一つだけが利用可能である。
特殊コマンドは単一のキーにのみ割り当てることができる。
\elist

%%.pg
\blist{\itemindent 10mm \labelwidth 15mm \rightmargin 15mm}
%%.lp
\item[{\bb{count}}]
%Prefix to key start a count, to repeat a command this many times.
%With {\it number\verb+_+pad\/} only. Default is~`{\tt n}'.
あるコマンドを複数回繰り返すための回数を開始するための前置キー。
{\it number\verb+_+pad\/} がオンの場合のみ。
標準設定は`{\tt n}'。
%%.lp
\item[{\bb{doinv}}]
%Show inventory. With {\it number\verb+_+pad\/} only. Default is~`{\tt 0}'.
持物を表示する。
{\it number\verb+_+pad\/} がオンの場合のみ。
標準設定は`{\tt 0}'。
%%.lp
\item[{\bb{fight}}]
%Prefix key to force fight a direction. Default is~`{\tt F}'.
強制的にある方向に戦うための前置キー。
標準設定は`{\tt F}'。
%%.lp
\item[{\bb{fight.numpad}}]
%Prefix key to force fight a direction. With {\it number\verb+_+pad\/} only.
%Default is~`{\tt -}'.
強制的にある方向に戦うための前置キー。
{\it number\verb+_+pad\/} がオンの場合のみ。
標準設定は`{\tt -}'。
%%.lp
\item[{\bb{getdir.help}}]
%When asked for a direction, the key to show the help. Default is~`{\tt ?}'.
方向を訊ねられたとき、 ヘルプを表示するキー。
標準設定は`{\tt ?}'。
%%.lp
\item[{\bb{getdir.self}}]
%When asked for a direction, the key to target yourself. Default is~`{\tt .}'.
方向を訊ねられたとき、 自分自身に向けるキー。
標準設定は`{\tt .}'。
%%.lp
\item[{\bb{getdir.self2}}]
%When asked for a direction, the key to target yourself. Default is~`{\tt s}'.
方向を訊ねられたとき、 自分自身に向けるキー。
標準設定は`{\tt s}'。
%%.lp
\item[{\bb{getpos.autodescribe}}]
%When asked for a location, the key to toggle {\it autodescribe\/}.
%Default is~`{\tt \#}'.
場所を訊ねられたとき、{\it autodescribe\/} を切り替えるキー。
標準設定は`{\tt \#}'。
%%.lp
\item[{\bb{getpos.all.next}}]
%When asked for a location, the key to go to next closest interesting thing.
%Default is~`{\tt a}'.
場所を訊ねられたとき、次に近い興味のあるものに移動するキー。
標準設定は`{\tt a}'。
%%.lp
\item[{\bb{getpos.all.prev}}]
%When asked for a location, the key to go to previous closest interesting thing.
%Default is~`{\tt A}'.
場所を訊ねられたとき、前に近い興味のあるものに移動するキー。
標準設定は`{\tt A}'。
%%.lp
\item[{\bb{getpos.door.next}}]
%When asked for a location, the key to go to next closest door or doorway.
%Default is~`{\tt d}'.
場所を訊ねられたとき、次に近いドアや出入り口に移動するキー。
標準設定は`{\tt d}'。
%%.lp
\item[{\bb{getpos.door.prev}}]
%When asked for a location, the key to go to previous closest door or doorway.
%Default is~`{\tt D}'.
場所を訊ねられたとき、前に近いドアや出入り口に移動するキー。
標準設定は`{\tt D}'。
%%.lp
\item[{\bb{getpos.help}}]
%When asked for a location, the key to show help. Default is~`{\tt ?}'.
場所を訊ねられたとき、ヘルプを表示するキー。
標準設定は`{\tt ?}'。
%%.lp
\item[{\bb{getpos.mon.next}}]
%When asked for a location, the key to go to next closest monster.
%Default is~`{\tt m}'.
場所を訊ねられたとき、次に近い怪物に移動するキー。
標準設定は`{\tt m}'。
%%.lp
\item[{\bb{getpos.mon.prev}}]
%When asked for a location, the key to go to previous closest monster.
%Default is~`{\tt M}'.
場所を訊ねられたとき、前に近い怪物に移動するキー。
標準設定は`{\tt M}'。
%%.lp
\item[{\bb{getpos.obj.next}}]
%When asked for a location, the key to go to next closest object.
%Default is~`{\tt o}'.
場所を訊ねられたとき、次に近い物に移動するキー。
標準設定は`{\tt o}'。
%%.lp
\item[{\bb{getpos.obj.prev}}]
%When asked for a location, the key to go to previous closest object.
%Default is~`{\tt O}'.
場所を訊ねられたとき、前に近い物に移動するキー。
標準設定は`{\tt O}'。
%%.lp
\item[{\bb{getpos.menu}}]
%When asked for a location, and using one of the next or previous keys to
%cycle through targets, toggle showing a menu instead. Default is~`{\tt !}'.
場所を訊ねられて、目標の中を循環して移動するキーのいずれかを使っているとき、
代わりにメニューを表示するかを切り替える。
標準設定は`{\tt !}'。
%%.lp
\item[{\bb{getpos.moveskip}}]
%When asked for a location, and using the shifted movement keys or
%meta-digit keys to fast-move around, move by skipping the same glyphs
%instead of by 8 units.
%Default is~`{\tt *}'.
場所を訊ねられて、高速移動のためにシフトを押しながらの移動や、
meta-数字 キーでの移動を使っているとき、8 マス毎ではなく同じ記号を飛ばして
移動する。
標準設定は`{\tt *}'。
%%.lp
\item[{\bb{getpos.filter}}]
%When asked for a location, change the filtering mode when using one of
%the next or previous keys to cycle through targets. Toggles between no
%filtering, in view only, and in the same area only. Default is~`{\tt "}'.
場所を訊ねられて、目標の中を循環して移動するキーのいずれかを使っているとき、
フィルタモードを切り替える。
フィルタなし、視界内のみ、同じエリアのみを順に切り替える。
標準設定は`{\tt "}'。
%%.lp
\item[{\bb{getpos.pick}}]
%When asked for a location, the key to choose the location, and possibly
%ask for more info. Default is~`{\tt .}'.
場所を訊ねられたとき、場所を選択し、おそらくさらなる情報を訊ねるためのキー。
標準設定は`{\tt .}'。
%%.lp
\item[{\bb{getpos.pick.once}}]
%When asked for a location, the key to choose the location, and skip
%asking for more info. Default is~`{\tt ,}'.
場所を訊ねられたとき、場所を選択し、さらなる情報を訊ねるのを飛ばすキー。
標準設定は`{\tt ,}'。
%%.lp
\item[{\bb{getpos.pick.quick}}]
%When asked for a location, the key to choose the location, skip asking
%for more info, and exit the location asking loop. Default is~`{\tt ;}'.
場所を訊ねられたとき、場所を選択し、さらなる情報を訊ねるのを飛ばし、
場所の問い合わせのループを終了するキー。
標準設定は`{\tt ;}'。
%%.lp
\item[{\bb{getpos.pick.verbose}}]
%When asked for a location, the key to choose the location, and show more
%info without asking. Default is~`{\tt :}'.
場所を訊ねられたとき、場所を選択し、訊ねられることなくさらなる情報を
表示するキー。
標準設定は`{\tt :}'。
%%.lp
\item[{\bb{getpos.self}}]
%When asked for a location, the key to go to your location.
%Default is~`{\tt @}'.
場所を訊ねられたとき、自分の位置に移動するキー。
標準設定は`{\tt @}'。
%%.lp
\item[{\bb{getpos.unexplored.next}}]
%When asked for a location, the key to go to next closest unexplored location.
%Default is~`{\tt x}'.
場所を訊ねられたとき、次に近い未探索の場所に移動するキー。
標準設定は`{\tt x}'。
%%.lp
\item[{\bb{getpos.unexplored.prev}}]
%When asked for a location, the key to go to previous closest unexplored
%location. Default is~`{\tt X}'.
場所を訊ねられたとき、前に近い未探索の場所に移動するキー。
標準設定は`{\tt X}'。
%%.lp
\item[{\bb{getpos.valid}}]
%When asked for a location, the key to go to show valid target locations.
%Default is~`{\tt \$}'.
場所を訊ねられたとき、正当な目標の場所を表示するキー。
標準設定は`{\tt \$}'。
%%.lp
\item[{\bb{getpos.valid.next}}]
%When asked for a location, the key to go to next closest valid location.
%Default is~`{\tt z}'.
場所を訊ねられたとき、次に近い正当な位置に移動するキー。
標準設定は`{\tt z}'。
%%.lp
\item[{\bb{getpos.valid.prev}}]
%When asked for a location, the key to go to previous closest valid location.
%Default is~`{\tt Z}'.
場所を訊ねられたとき、前に近い正当な位置に移動するキー。
標準設定は`{\tt Z}'。
%%.lp
\item[{\bb{nopickup}}]
%Prefix key to move without picking up items. Default is~`{\tt m}'.
物を拾わずに移動する前置キー。
標準設定は`{\tt m}'。
%%.lp
\item[{\bb{redraw}}]
%Key to redraw the screen. Default is~`{\tt \^{}R}'.
画面を再描画するキー。
標準設定は`{\tt \^{}R}'。
%%.lp
\item[{\bb{redraw.numpad}}]
%Key to redraw the screen. With {\it number\verb+_+pad\/} only.
%Default is~`{\tt \^{}L}'.
画面を再描画するキー。
{\it number\verb+_+pad\/} がオンの場合のみ。
標準設定は`{\tt \^{}L}'。
%%.lp
\item[{\bb{repeat}}]
%Key to repeat previous command. Default is~`{\tt \^{}A}'.
前回のコマンドを繰り返すキー。
標準設定は`{\tt \^{}A}'。
%%.lp
\item[{\bb{reqmenu}}]
%Prefix key to request menu from some commands. Default is~`{\tt m}'.
一部のコマンドからメニューを要求する前置キー。
標準設定は`{\tt m}'。
%%.lp
\item[{\bb{run}}]
%Prefix key to run towards a direction. Default is~`{\tt G}'.
ある方向に走るための前置キー。
標準設定は`{\tt G}'。
%%.lp
\item[{\bb{run.nopickup}}]
%Prefix key to run towards a direction without picking up items on the way.
%Default is~`{\tt M}'.
途中にある物を拾わずにある方向に走るための前置キー。
標準設定は`{\tt M}'。
%%.lp
\item[{\bb{run.numpad}}]
%Prefix key to run towards a direction. With {\it number\verb+_+pad\/} only.
%Default is~`{\tt 5}'.
ある方向に走るための前置キー。
{\it number\verb+_+pad\/} がオンの場合のみ。
標準設定は`{\tt 5}'。
%%.lp
\item[{\bb{rush}}]
%Prefix key to rush towards a direction. Default is~`{\tt g}'.
ある方向に突進するための前置キー。
標準設定は`{\tt g}'。
\elist


%%.hn 2
%\subsection*{Configuring Message Types}
\subsection*{メッセージ型の設定}

%%.pg
%You can change the way the messages are shown in the message area, when
%the message matches a user-defined pattern.
メッセージがユーザー定義パターンにマッチングしたとき、メッセージを
メッセージエリアに表示する方法を変更できる。

%%.pg
%In general, the configuration file entries to describe the message types
%look like this:
一般的に、メッセージ型を記述する設定ファイルエントリは次のようなものである:
\begin{verbatim}
    MSGTYPE=type "pattern"
\end{verbatim}
\blist{}
%%.lp
\item[\ib{type}]
%how the message should be shown:
どのようにメッセージを表示するか:
%%.sd
%%.si
\\
%{\tt show}  --- show message normally.\\
%{\tt hide}  --- never show the message.\\
%{\tt stop}  --- wait for user with more-prompt.\\
%{\tt norep} --- show the message once, but not again if no other message is
%shown in between.
{\tt show}  --- 通常通りメッセージを表示する。\\
{\tt hide}  --- メッセージを表示しない。\\
{\tt stop}  --- more プロンプトでユーザーを待つ。\\
{\tt norep} --- 1 度目はメッセージを表示するが、他のメッセージが来るまで再びは表示しない。
%%.ei
%%.ed
%%.lp
\item[\ib{pattern}]
%the pattern to match. The pattern should be a regular expression.
マッチングするパターン。
パターンは正規表現である必要がある。
\elist

%%.lp ""
%Here's an example of message types using {\it NetHack's\/} internal
%pattern matching facility:
次のものは {\it NetHack\/} の内部パターンマッチング機能を使ったメッセージ型の
例である。

\begin{verbatim}
    MSGTYPE=stop "You feel hungry."
    MSGTYPE=hide "You displaced *."
\end{verbatim}

%specifies that whenever a message ``You feel hungry'' is shown,
%the user is prompted with more-prompt, and a message matching
%``You displaced  \verb+<+something\verb+>+'' is not shown at all.
このように指定すると、``You feel hungry'' というメッセージが現れた時は常に
more プロンプトが表示され、
"You displaced (何か)." にマッチングするメッセージは一切表示されない。

%%.lp
%The order of the defined MSGTYPE lines is important; the last matching
%rule is used. Put the general case first, exceptions below them.
MSGTYPE 行の定義順は重要である; 最後にマッチングした規則が使われる。
一般的な場合を先に書き、例外をその後に書くこと。

%%.pg

%%.lp
%%.hn 2
%\subsection*{Configuring Menu Colors}
\subsection*{メニュー色の設定}

%%.pg
%Some platforms allow you to define colors used in menu lines when the
%line matches a user-defined pattern.
%At this time the tty, curses, win32tty and
%win32gui interfaces support this.
一部のプラットフォームでは、メニュー行がユーザー定義パターンに
マッチングしたとき、メニュー行に使う色を定義できる。
現在のところ tty, curses, win32tty, win32gui インターフェースが
これに対応している。

%%.pg
%In general, the configuration file entries to describe the menu color mappings
%look like this:
一般的にメニュー色割り当てを記述する設定ファイルエントリは次のようなものである:
\begin{verbatim}
    MENUCOLOR="pattern"=color&attribute
\end{verbatim}

\blist{}
%%.lp
\item[\ib{pattern}]
%the pattern to match;
マッチングするパターン;
%%.lp
\item[\ib{color}]
%the color to use for lines matching the pattern;
パターンがマッチングした行に使う色;
%%.lp
\item[\ib{attribute}]
%the attribute to use for lines matching the pattern. The attribute is
%optional, and if left out, you must also leave out the preceding ampersand.
%If no attribute is defined, no attribute is used.
パターンがマッチングした行に使う属性。
属性はオプションで、省略するなら、アンド記号も省略しなければならない。
属性が定義されない場合、属性は使われない。
\elist

%%.lp ""
%The pattern should be a regular expression.
パターンは正規表現である必要がある。

%%.lp ""
%Allowed colors are {\it black}, {\it red}, {\it green}, {\it brown},
%{\it blue}, {\it magenta}, {\it cyan}, {\it gray}, {\it orange},
%{\it light-green}, {\it yellow}, {\it light-blue}, {\it light-magenta},
%{\it light-cyan}, and {\it white}.
%And {\it no-color}, the default foreground color, which isn't necessarily
%the same as any of the other colors.
指定できる色は:
{\it black}, {\it red}, {\it green}, {\it brown},
{\it blue}, {\it magenta}, {\it cyan}, {\it gray}, {\it orange},
{\it light-green}, {\it yellow}, {\it light-blue}, {\it light-magenta},
{\it light-cyan}, {\it white}。
そして他の色と同じ必要のない標準の前面色である {\it no-color}。

%%.lp ""
%Allowed attributes are {\it none}, {\it bold}, {\it dim}, {\it underline},
%{\it blink}, and {\it inverse}.
%{\it Normal\/} is a synonym for {\it none}.
%Note that the platform used may interpret the attributes any way it
%wants.
指定できる属性は:
{\it none}, {\it bold}, {\it dim}, {\it underline},
{\it blink}, {\it inverse}。
{\it Normal\/} は {\it none} の別名である。
なお、プラットフォームはこれを好きなように解釈するかも知れない。

%%.lp ""
%Here's an example of menu colors using {\it NetHack's\/} internal
%pattern matching facility:
次のものは {\it NetHack \/} の内部パターンマッチング機能を使ったメニュー色の
例である。

\begin{verbatim}
    MENUCOLOR="* blessed *"=green
    MENUCOLOR="* cursed *"=red
    MENUCOLOR="* cursed *(being worn)"=red&underline
\end{verbatim}

%specifies that any menu line with ``~blessed~'' contained
%in it will be shown in green color, lines with ``~cursed~'' will be
%shown in red, and lines with ``~cursed~'' followed by ``(being worn)''
%on the same line will be shown in red color and underlined.
%You can have multiple MENUCOLOR entries in your configuration file,
%and the last MENUCOLOR line that matches
%a menu line will be used for the line.
このように指定すると、``~blessed~'' が含まれているメニュー行は緑色に、
``~cursed~'' の行は赤色に、``~cursed~'' に引き続いて同じ行に ``(being worn)'' が
ある場合は赤色かつ下線付きで表示する。
設定ファイル中に複数の MENUCOLOR エントリを書くことができ、メニュー行に
マッチングした中で最後の MENUCOLOR 行がその行に使われる。

%%.pg
%Note that if you intend to have one or more color specifications match
%``~uncursed~'', you will probably want to turn the
%{\it implicit\verb+_+uncursed\/}
%option off so that all items known to be uncursed are actually
%displayed with the ``uncursed'' description.
``~uncursed~'' にマッチングするような色指定を意図しているなら、
呪われていないことが分かっている全ての持ち物について ``uncursed'' と
表示するための
{\it implicit\verb+_+uncursed\/}
オプションをおそらく設定した方がよいことに注意すること。

%%.lp
%%.hn 2
%\subsection*{Configuring User Sounds}
\subsection*{音声の設定}

%%.pg
%Some platforms allow you to define sound files to be played when a message
%that matches a user-defined pattern is delivered to the message window.
%At this time the Qt port and the win32tty and win32gui ports support the
%use of user sounds.
メッセージウィンドウに表示されたメッセージが
ユーザー定義のパターンと一致したときに
特定の音声ファイルを鳴らすことができるプラットフォームもある。
現在のところ、Qt, win32tty, win32gui がユーザー音声に対応している。

%%.pg
%The following configuration file entries are relevant to mapping user sounds
%to messages:
ユーザー音声とメッセージの関連付けに関する設定ファイル項目は
以下のとおりである。

\blist{}
%%.lp
\item[\ib{SOUNDDIR}]
%The directory that houses the sound files to be played.
音声ファイルのあるディレクトリ
%%.lp
\item[\ib{SOUND}]
%An entry that maps a sound file to a user-specified message pattern.
%Each SOUND entry is broken down into the following parts:
音声ファイルとユーザーが指定するメッセージパターンを関連付ける項目。
各 SOUND 項目は以下の項目に分解される。

%%.sd
%%.si
%{\tt MESG      } --- message window mapping (the only one supported in 3.6);\\
{\tt MESG      } --- メッセージウィンドウ指定 (3.6 ではただ一つだけに対応している)\\
%{\tt pattern   } --- the pattern to match;\\
{\tt pattern   } --- マッチするパターン\\
%{\tt sound file} --- the sound file to play;\\
{\tt sound file} --- 鳴らす音声ファイル\\
%{\tt volume    } --- the volume to be set while playing the sound file.
{\tt volume    } --- 音声ファイルを鳴らすときの音量
%%.ei
%%.ed
\elist

%%.lp ""
%The pattern should be a POSIX extended regular expression.
このパターンは POSIX 拡張正規表現で指定する。

%%.lp
%%.hn 2
%\subsection*{Configuring Status Hilites}
\subsection*{ステータスハイライトの設定}

%%.pg
%Your copy of {\it NetHack\/} may have been compiled with support
%for {\it Status Hilites}.
%If so, you can customize your game display by setting thresholds to
%change the color or appearance of fields in the status display.
あなたの入手した {\it NetHack\/} は {\it ステータスハイライト} に対応して
コンパイルされているかもしれない。
その場合、ステータスディスプレイのフィールドの色や見た目を
閾値を設定することでカスタマイズできる。

%The format for defining status colors is:\\
ステータス色を定義する形式は:\\
\begin{verbatim}
OPTION=hilite_status:field-name/behavior/color&attributes
\end{verbatim}

%For example, the following line in your configuration file will cause
%the hitpoints field to display in the color red if your hitpoints
%drop to or below a threshold of 30%:\\
例えば、設定ファイルに次の行を書くと、ヒットポイントが30%の閾値以下になると
ヒットポイントが赤色で表示される:\\
\begin{verbatim}
OPTION=hilite_status:hitpoints/<=30%/red/normal
\end{verbatim}
%(That example is actually specifying {\tt red\&normal} for  {\tt <=30\%}
%and {\tt no-color\&normal} for {\tt >30\%}.)\\
(この例は実際には {\tt <=30\%} で {\tt red\&normal} に、
{\tt >30\%} で {\tt no-color\&normal} にする。)\\

%For another example, the following line in your configuration file will cause
%wisdom to be displayed red if it drops and green if it rises:\\
もう一つの例として、次の行を設定ファイルに書くと、賢さが低下すると赤で
表示され、上昇すると緑で表示される。\\
\begin{verbatim}
OPTION=hilite_status:wisdom/down/red/up/green
\end{verbatim}

%Allowed colors are black, red, green, brown, blue, magenta, cyan, gray,
%orange, light-green, yellow, light-blue, light-magenta, light-cyan, and white.
%And {\it no-color}, the default foreground color on the display, which
%is not necessarily the same as black or white or any of the other colors.
指定できる色は:
black, red, green, brown, blue, magenta, cyan, gray,
orange, light-green, yellow, light-blue, light-magenta, light-cyan, white。
および {\it no-color}、ディスプレイの標準前景色;
黒や白やその他の色と同じである必要はない。

%Allowed attributes are none, bold, dim, underline, blink, and inverse.
%``Normal'' is a synonym for ``none''; they should not be used in
%combination with any of the other attributes.
指定できる属性は:
none, bold, dim, underline, blink, inverse。
``Normal'' は ``none'' の別名である;
これらは他の属性と組み合わせて使うべきではない。

%To specify both a color and an attribute, use `\&' to combine them.
%To specify multiple attributes, use `+' to combine those.
色と属性の両方を指定するには、それらを結合するために `\&' を使う。
複数の属性を指定するには、それらを結合するために `+' を使う。

%%.lp ""
%For example: {\tt magenta\&inverse+dim}.
例: {\tt magenta\&inverse+dim}。

%Note that the display may substitute or ignore particular attributes
%depending upon its capabilities, and in general may interpret the
%attributes any way it wants.
%For example, on some display systems a request for bold might yield
%blink or vice versa.
%On others, issuing an attribute request while another is already
%set up will replace the earlier attribute rather than combine with it.
%Since nethack issues attribute requests sequentially (at least with
%the {\it tty} interface) rather than all at once, the only way a
%situation like that can be controlled is to specify just one attribute.
なお、ディスプレイはその能力によっては特定の属性を置き換えたり
無視したりするかもしれず、一般的に、これを好きなように解釈するかも
知れないことに注意すること。
例えば、ディスプレイシステムによっては
太字を点滅にしたりその逆にしたりする。
一方、他の属性が既に設定されている時に
他の属性を要求すると、属性を合成するのではなく、以前の属性を置き換える。
nethackは属性要求を(少なくとも {\it tty}  インターフェースでは)
同時にではなく順々に発行するため、
この様な状況を制御する唯一の方法は一つの属性として指定することである。

%You can adjust the display of the following status fields:
次のステータス項目を調整できる:
%%.sd
\begin{center}
\begin{tabular}{lll}
%%TABLE_START
title & dungeon-level & experience-level\\
strength & gold & experience\\
dexterity & hitpoints & HD\\
constitution & hitpoints-max & time\\
intelligence & power & hunger\\
wisdom & power-max & carrying-capacity\\
charisma & armor-class & condition\\
alignment &  & score\\
%%TABLE_END  Do not delete this line.
\end{tabular}
\end{center}
%%.ed
%%.lp ""
%The pseudo-field `characteristics' can be used to set all six
%of Str, Dex, Con, Int, Wis, and Cha at once.  `HD' is `hit dice',
%an approximation of experience level displayed when polymorphed.
%`experience', `time', and `score' are conditionally displayed
%depending upon your other option settings.
疑似フィールド `characteristics' は Str, Dex, Con, Int, Wis, Cha の
六つ全てを一度に設定するのに使える。
`HD' は `ヒットダイス(hit dice)' で、変化したときに表示される
経験レベルの概算である。
`experience', `time', `score' はその他のオプション設定に依存して
条件によっては表示される。

%%.lp ""
%Instead of a behavior, `condition' takes the following condition flags:
%{\it stone}, {\it slime}, {\it strngl}, {\it foodpois}, {\it termill},
%{\it blind}, {\it deaf}, {\it stun}, {\it conf}, {\it hallu},
%{\it lev}, {\it fly}, and {\it ride}.
%You can use `major\_troubles' as an alias
%for stone through termill, `minor\_troubles' for blind through hallu,
%`movement' for lev, fly, and ride, and `all' for every condition.
振る舞いの代わりに、`condition' は次の状態フラグを取れる:
{\it stone}, {\it slime}, {\it strngl}, {\it foodpois}, {\it termill},
{\it blind}, {\it deaf}, {\it stun}, {\it conf}, {\it hallu},
{\it lev}, {\it fly}, {\it ride}。
stone から termill の別名として `major\_troubles' を、
blind through hallu の別名として `minor\_troubles' を、
lev, fly, ride の別名として `movement' を、
そして全ての別名として `all' を使える。

%%.lp ""
%Allowed behaviors are ``always'', ``up'', ``down'', ``changed'', a
%percentage or absolute number threshold, or text to match against.
指定できる振る舞いは ``always'', ``up'', ``down'', ``changed'',
割合または絶対値の閾値、マッチングするテキストである。

\blist{}
%%.lp "*"
%\item[{\tt always}] will set the default attributes for that field.
\item[{\tt always}] はそのフィールドの標準属性を設定する。
%%.lp "*"
%\item[{\tt up}{\normalfont, }{\tt down}] set the field attributes
%for when the field value changes upwards or downwards. This attribute
%times out after {\tt statushilites} turns.
\item[{\tt up}, {\tt down}] は、
フィールドの値が上昇または下降したときのためのフィールド属性を
設定する。
この属性は {\tt statushilites} ターン後に元に戻る。
%%.lp "*"
%\item[{\tt changed}] sets the field attribute for when the field value
%changes. This attribute times out after {\tt statushilites} turns.
%(If a field has both a ``changed'' rule and an ``up'' or ``down''
%rule which matches a change in the field's value,
%the ``up'' or ``down'' one takes precedence.)
\item[{\tt changed}] は、フィールドの値が変更されたときのための
フィールド属性を設定する。
この属性は {\tt statushilites} ターン後に元に戻る。
(あるフィールドに、フィールドの値の変更に一致する、
``changed'' 規則と ``up'' または ``down'' 規則の両方がある場合、
``up'' または ``down'' が優先される。)
%%.lp "*"
%\item[{\tt percentage}] sets the field attribute when the field value
%matches the percentage.
%It is specified as a number between 0 and 100, followed by `{\tt \%}'
%(percent sign).
\item[{\tt percentage}] は、フィールドの値がその百分率に
一致したときのためのフィールド属性を設定する。
0 から 100 までの数値に引き続いて `{\tt \%}' (パーセント記号) で
指定される。
%If the percentage is prefixed with `{\tt <=}' or `{\tt >=}',
%it also matches when value is below or above the percentage.
%Use prefix `{\tt <}' or `{\tt >}' to match when strictly below or above.
%(The numeric limit is relaxed slightly for those: {\tt >-1\%}
%and {\tt <101\%} are allowed.)
百分率に `{\tt <=}' または `{\tt >=}' が前置されている場合、
値がその百分率よりも小さいまたは大きい場合にも一致する。
厳密に小さいまたは大きい場合にのみ一致するためには、
`{\tt <}' または `{\tt >}' を前置する。
(数値制限はこのために少し緩められている: {\tt >-1\%} と {\tt <101\%} は
許されている。)
%Only four fields support percentage rules.
%Percentages for ``{\it hitpoints\/}'' and ``{\it power\/}'' are
%straightforward; they're based on the corresponding maximum field.
%Percentage highlight rules are also allowed for ``{\it experience level\/}''
%and ``{\it experience points\/}'' (valid when the
%{\it showexp\/}
%option is enabled).
四つのフィールドのみが百分率規則に対応している。
``{\it hitpoints\/}'' と ``{\it power\/}'' の百分率は直感的である;
これらはそれぞれの最大値を基にしている。
百分率によるハイライトは ``{\it experience level\/}'' と
``{\it experience points\/}'' ({\it showexp\/} オプションが有効の場合のみ) も
許されている。
%For those, the percentage is based on the progress from the start of
%the current experience level to the start of the next level.
%So if level 2 starts at 20 points and level 3 starts at 40 points,
%having 30 points is 50\% and 35 points is 75\%.
%100\% is unattainable for experience because you'll gain a level and
%the calculations will be reset for that new level, but a rule for
%{\tt =100\%} is allowed and matches the special case of being
%exactly 1 experience point short of the next level.
これらの場合、百分率は現在の経験レベルの開始から次のレベルの開始までの
進捗を基としている。
従って、レベル 2 が 20 ポイントで開始し、3 が 40 ポイントで開始する場合、
30 ポイントは 50\% であり 35 ポイントは 75\% である。
経験値については 100\% は達成できない; レベルは上昇し、
計算は次のレベルに対してリセットされるからである;
しかし、{\tt =100\%} という規則は許され、次のレベルにちょうど 1 経験
ポイントだけ足りない特殊な場合に一致する。
%% (If you manage to reach level 30, there is no next level and the
%% percentage will remain at 0\% no matter have many additional experience
%% points you earn.)
%%.lp "*"
%\item[{\tt absolute}] value sets the attribute when the field value
%matches that number.
%The number must be 0 or higher, except for ``{\it armor-class\/} which
%allows negative values, and may optionally be preceded by `{\tt =}'.
%If the number is preceded by `{\tt <=}' or `{\tt >=}' instead,
%it also matches when value is below or above.
%If the prefix is `{\tt <}' or `{\tt >}', only match when strictly
%above or below.
\item[{\tt absolute}] 値はフィールド値がこの数値に一致した場合に
属性を設定する。
数値は 0 以上でなければならず (例外は ``{\it armor-class\/} で
負数も許される)、オプションで `{\tt =}' を前置できる。
代わりに数値の前に `{\tt <=}' または `{\tt >=}' が前置された場合、
値がより小さいまたはより大きい場合にも一致する。
`{\tt <}' または `{\tt >}' を前置すると、厳密に小さいまたは大きい場合にのみ
一致する。
%%.lp "*"
%\item[{\tt text}] match sets the attribute when the field value matches the text.
%Text matches can only be used for ``{\it alignment\/}'',
%``{\it carrying-capacity\/}'', ``{\it hunger\/}'', ``{\it dungeon-level\/}'',
%and ``{\it title\/}''.
%For title, only the role's rank title
%is tested; the character's name is ignored.
\item[{\tt text}] はフィールドの値がそのテキストに一致した場合に属性を
設定する。
テキストは ``{\it alignment\/}'',
``{\it carrying-capacity\/}'', ``{\it hunger\/}'', ``{\it dungeon-level\/}'',
``{\it title\/}'' にのみ利用できる。
title の場合、職業のランク名のみがテストされる;
キャラクタ名は無視される。
%%.ei
\elist

%The in-game options menu can help you determine the correct syntax for a
%configuration file.
ゲーム内のオプションメニューは、設定ファイルでの正しい文法を決定する
助けになるだろう。

%The whole feature can be disable by setting option {\it statushilites} to 0.
オプション {\it statushilites} を 0 にすることでこの機能全体を無効にできる。

%Example hilites:
ハイライトの例:
\begin{verbatim}
    OPTION=hilite_status: gold/up/yellow/down/brown
    OPTION=hilite_status: characteristics/up/green/down/red
    OPTION=hilite_status: hitpoints/100%/gray&normal
    OPTION=hilite_status: hitpoints/<100%/green&normal
    OPTION=hilite_status: hitpoints/<66%/yellow&normal
    OPTION=hilite_status: hitpoints/<50%/orange&normal
    OPTION=hilite_status: hitpoints/<33%/red&bold
    OPTION=hilite_status: hitpoints/<15%/red&inverse
    OPTION=hilite_status: condition/major/orange&inverse
    OPTION=hilite_status: condition/lev+fly/red&inverse
\end{verbatim}

%%.lp
%%.hn 2
%\subsection*{Modifying {\it NetHack\/} Symbols}
\subsection*{{\it NetHack\/} のシンボルの変更}

%%.pg
%{\it NetHack\/} can load entire symbol sets from the symbol file.
{\it NetHack\/} は symbol ファイルからシンボル集合全体を読み込むことができる。

%%.pg
%The options that are used to select a particular symbol set from the
%symbol file are:
symbol ファイルから特定のシンボル集合を選択するために使われるオプションは:

\blist{}
%%.lp
\item[\ib{symset}]
%Set the name of the symbol set that you want to load.
%{\it symbols\/}.
読み込みたいシンボル集合の名前を設定する。
{\it symbols\/}。

%%.lp
\item[\ib{roguesymset}]
%Set the name of the symbol set that you want to load for display
%on the rogue level.
rogue レベルで表示するために読み込みたいシンボル集合の名前を設定する。
\elist

%You can also override one or more symbols using the {\it SYMBOLS\/} and
%{\it ROGUESYMBOLS\/} configuration file options.
%Symbols are specified as {\it name:value\/} pairs.
%Note that {\it NetHack\/} escape-processes
%the {\it value\/} string in conventional C fashion.
%This means that `\verb+\+' is a prefix to take the following character
%literally.
%Thus `\verb+\+' needs to be represented as `\verb+\\+'.
%The special prefix
%`\verb+\m+' switches on the meta bit in the symbol value, and the
%`{\tt \^{}}' prefix causes the following character to be treated as a control
%character.
{\it SYMBOLS\/} と {\it ROGUESYMBOLS\/} の
設定ファイルオプションを使っていくつかのシンボルを
上書きすることもできる。
シンボルは {\it name:value\/} の組で指定する。
{\it NetHack\/} は {\it value\/} 文字列を伝統的な C の方法でエスケープする。
つまり、`\verb+\+' は次の文字をリテラルにする接頭辞である。
従って、`\verb+\+' は `\verb+\\+' と表現する必要がある。
特殊接頭辞 `\verb+\m+' はシンボル値のメタビットを切り替え、
`{\tt \^{}}' 接頭辞は引き続く文字を制御文字として扱う。

{
\small
\begin{longtable}{lll}
%\caption[]{NetHack Symbols}\\
\caption[]{NetHack Symbols}\\
%Default                      & Symbol Name                & Description\\
標準設定                     & シンボル名                 & 説明\\
\hline \hline
\endhead
%\verb@ @ & S\verb+_+air                     &	(air)\\
%\_ & S\verb+_+altar                   &	(altar)\\
%\verb@"@ & S\verb+_+amulet                  &	(amulet)\\
%\verb@A@ & S\verb+_+angel                   &	(angelic being)\\
%\verb@a@ & S\verb+_+ant                     &	(ant or other insect)\\
%\verb@^@ & S\verb+_+anti\verb+_+magic\verb+_+trap       &	(anti-magic field)\\
%\verb@[@ & S\verb+_+armor                   &	(suit or piece of armor)\\
%\verb@[@ & S\verb+_+armour                  &	(suit or piece of armor)\\
%\verb@^@ & S\verb+_+arrow\verb+_+trap             &	(arrow trap)\\
%\verb@0@ & S\verb+_+ball                    &	(iron ball)\\
%\# & S\verb+_+bars                    &	(iron bars)\\
%\verb@B@ & S\verb+_+bat                     &	(bat or bird)\\
%\verb@^@ & S\verb+_+bear\verb+_+trap              &	(bear trap)\\
%\verb@-@ & S\verb+_+blcorn                  &	(bottom left corner)\\
%\verb@b@ & S\verb+_+blob                    &	(blob)\\
%\verb@+@ & S\verb+_+book                    &	(spellbook)\\
%\verb@)@ & S\verb+_+boomleft                &	(boomerang open left)\\
%\verb@(@ & S\verb+_+boomright               &	(boomerang open right)\\
%\verb@`@ & S\verb+_+boulder                 &	(boulder)\\
%\verb@-@ & S\verb+_+brcorn                  &	(bottom right corner)\\
%\verb@C@ & S\verb+_+centaur                 &	(centaur)\\
%\verb@_@ & S\verb+_+chain                   &	(iron chain)\\
%\# & S\verb+_+cloud                   &	(cloud)\\
%\verb@c@ & S\verb+_+cockatrice              &	(cockatrice)\\
%\$ & S\verb+_+coin                    &	(pile of coins)\\
%\# & S\verb+_+corr                    &	(corridor)\\
%\verb@-@ & S\verb+_+crwall                  &	(wall)\\
%\verb@-@ & S\verb+_+darkroom                &	(dark room)\\
%\verb@^@ & S\verb+_+dart\verb+_+trap              &	(dart trap)\\
%\verb@&@ & S\verb+_+demon                   &	(major demon)\\
%\verb@*@ & S\verb+_+digbeam                 &	(dig beam)\\
%\verb@>@ & S\verb+_+dnladder                &	(ladder down)\\
%\verb@>@ & S\verb+_+dnstair                 &	(staircase down)\\
%\verb@d@ & S\verb+_+dog                     &	(dog or other canine)\\
%\verb@D@ & S\verb+_+dragon                  &	(dragon)\\
%\verb@;@ & S\verb+_+eel                     &	(sea monster)\\
%\verb@E@ & S\verb+_+elemental               &	(elemental)\\
%\verb@/@ & S\verb+_+explode1                &	(explosion top left)\\
%\verb@-@ & S\verb+_+explode2                &	(explosion top center)\\
%\verb@\@ & S\verb+_+explode3                &	(explosion top right)\\
%\verb@|@ & S\verb+_+explode4                &	(explosion middle left)\\
%\verb@ @ & S\verb+_+explode5                &	(explosion middle center)\\
%\verb@|@ & S\verb+_+explode6                &	(explosion middle right)\\
%\verb@\@ & S\verb+_+explode7                &	(explosion bottom left)\\
%\verb@-@ & S\verb+_+explode8                &	(explosion bottom center)\\
%\verb@/@ & S\verb+_+explode9                &	(explosion bottom right)\\
%\verb@e@ & S\verb+_+eye                     &	(eye or sphere)\\
%\verb@^@ & S\verb+_+falling\verb+_+rock\verb+_+trap     &	(falling rock trap)\\
%\verb@f@ & S\verb+_+feline                  &	(cat or other feline)\\
%\verb@^@ & S\verb+_+fire\verb+_+trap              &	(fire trap)\\
%\verb@!@ & S\verb+_+flashbeam               &	(flash beam)\\
%\% & S\verb+_+food                    &	(piece of food)\\
%\{ & S\verb+_+fountain                &	(fountain)\\
%\verb@F@ & S\verb+_+fungus                  &	(fungus or mold)\\
%\verb@*@ & S\verb+_+gem                     &	(gem or rock)\\
%\verb@ @ & S\verb+_+ghost                   &	(ghost)\\
%\verb@H@ & S\verb+_+giant                   &	(giant humanoid)\\
%\verb@G@ & S\verb+_+gnome                   &	(gnome)\\
%\verb@'@ & S\verb+_+golem                   &	(golem)\\
%\verb@|@ & S\verb+_+grave                   &	(grave)\\
%\verb@g@ & S\verb+_+gremlin                 &	(gremlin)\\
%\verb@-@ & S\verb+_+hbeam                   &	(wall)\\
%\# & S\verb+_+hcdbridge               &	(horizontal raised drawbridge)\\
%\verb@+@ & S\verb+_+hcdoor                  &	(closed door)\\
%\verb@.@ & S\verb+_+hodbridge               &	(horizontal lowered drawbridge)\\
%\verb@|@ & S\verb+_+hodoor                  &	(open door)\\
%\verb\^\ & S\verb+_+hole                    &	(hole)\\
%\verb~@~ & S\verb+_+human                   &	(human or elf)\\
%\verb@h@ & S\verb+_+humanoid                &	(humanoid)\\
%\verb@-@ & S\verb+_+hwall                   &	(horizontal wall)\\
%\verb@.@ & S\verb+_+ice                     &	(ice)\\
%\verb@i@ & S\verb+_+imp                     &	(imp or minor demon)\\
%\verb@I@ & S\verb+_+invisible               &	(invisible monster)\\
%\verb@J@ & S\verb+_+jabberwock              &	(jabberwock)\\
%\verb@j@ & S\verb+_+jelly                   &	(jelly)\\
%\verb@k@ & S\verb+_+kobold                  &	(kobold)\\
%\verb@K@ & S\verb+_+kop                     &	(Keystone Kop)\\
%\verb@^@ & S\verb+_+land\verb+_+mine              &	(land mine)\\
%\verb@}@ & S\verb+_+lava                    &	(molten lava)\\
%\verb@l@ & S\verb+_+leprechaun              &	(leprechaun)\\
%\verb@^@ & S\verb+_+level\verb+_+teleporter       &	(level teleporter)\\
%\verb@L@ & S\verb+_+lich                    &	(lich)\\
%\verb@y@ & S\verb+_+light                   &	(light)\\
%\# & S\verb+_+litcorr                 &	(lit corridor)\\
%\verb@:@ & S\verb+_+lizard                  &	(lizard)\\
%\verb@\@ & S\verb+_+lslant                  &	(wall)\\
%\verb@^@ & S\verb+_+magic\verb+_+portal           &	(magic portal)\\
%\verb@^@ & S\verb+_+magic\verb+_+trap             &	(magic trap)\\
%\verb@m@ & S\verb+_+mimic                   &	(mimic)\\
%\verb@]@ & S\verb+_+mimic\verb+_+def              &	(mimic)\\
%\verb@M@ & S\verb+_+mummy                   &	(mummy)\\
%\verb@N@ & S\verb+_+naga                    &	(naga)\\
%\verb@.@ & S\verb+_+ndoor                   &	(doorway)\\
%\verb@n@ & S\verb+_+nymph                   &	(nymph)\\
%\verb@O@ & S\verb+_+ogre                    &	(ogre)\\
%\verb@o@ & S\verb+_+orc                     &	(orc)\\
%\verb@p@ & S\verb+_+piercer                 &	(piercer)\\
%\verb@^@ & S\verb+_+pit                     &	(pit)\\
%\# & S\verb+_+poisoncloud             &	(poison cloud)\\
%\verb@^@ & S\verb+_+polymorph\verb+_+trap         &	(polymorph trap)\\
%\verb@}@ & S\verb+_+pool                    &	(water)\\
%\verb@!@ & S\verb+_+potion                  &	(potion)\\
%\verb@P@ & S\verb+_+pudding                 &	(pudding or ooze)\\
%\verb@q@ & S\verb+_+quadruped               &	(quadruped)\\
%\verb@Q@ & S\verb+_+quantmech               &	(quantum mechanic)\\
%\verb@=@ & S\verb+_+ring                    &	(ring)\\
%\verb@`@ & S\verb+_+rock                    &	(boulder or statue)\\
%\verb@r@ & S\verb+_+rodent                  &	(rodent)\\
%\verb@^@ & S\verb+_+rolling\verb+_+boulder\verb+_+trap  &	(rolling boulder trap)\\
%\verb@.@ & S\verb+_+room                    &	(floor of a room)\\
%\verb@/@ & S\verb+_+rslant                  &	(wall)\\
%\verb@^@ & S\verb+_+rust\verb+_+trap              &	(rust trap)\\
%\verb@R@ & S\verb+_+rustmonst               &	(rust monster or disenchanter)\\
%\verb@?@ & S\verb+_+scroll                  &	(scroll)\\
%\# & S\verb+_+sink                    &	(sink)\\
%\verb@^@ & S\verb+_+sleeping\verb+_+gas\verb+_+trap     &	(sleeping gas trap)\\
%\verb@S@ & S\verb+_+snake                   &	(snake)\\
%\verb@s@ & S\verb+_+spider                  &	(arachnid or centipede)\\
%\verb@^@ & S\verb+_+spiked\verb+_+pit             &	(spiked pit)\\
%\verb@^@ & S\verb+_+squeaky\verb+_+board          &	(squeaky board)\\
%\verb@0@ & S\verb+_+ss1                     &	(magic shield 1 of 4)\\
%\# & S\verb+_+ss2                     &	(magic shield 2 of 4)\\
%\verb+@+ & S\verb+_+ss3                     &	(magic shield 3 of 4)\\
%\verb@*@ & S\verb+_+ss4                     &	(magic shield 4 of 4)\\
%\verb@^@ & S\verb+_+statue\verb+_+trap            &	(statue trap)\\
%\verb@ @ & S\verb+_+stone                   &	(solid rock or unexplored terrain\\
%         &                                  &   \,or dark part of a room)\\
%\verb@]@ & S\verb+_+strange\verb+_+obj      &	(strange object)\\
%\verb@-@ & S\verb+_+sw\verb+_+bc                  &	(swallow bottom center)\\
%\verb@\@ & S\verb+_+sw\verb+_+bl                  &	(swallow bottom left)\\
%\verb@/@ & S\verb+_+sw\verb+_+br                  &	(swallow bottom right	)\\
%\verb@|@ & S\verb+_+sw\verb+_+ml                  &	(swallow middle left)\\
%\verb@|@ & S\verb+_+sw\verb+_+mr                  &	(swallow middle right)\\
%\verb@-@ & S\verb+_+sw\verb+_+tc                  &	(swallow top center)\\
%\verb@/@ & S\verb+_+sw\verb+_+tl                  &	(swallow top left)\\
%\verb@\@ & S\verb+_+sw\verb+_+tr                  &	(swallow top right)\\
%\verb@-@ & S\verb+_+tdwall                  &	(wall)\\
%\verb@^@ & S\verb+_+teleportation\verb+_+trap     &	(teleportation trap)\\
%\verb@\@ & S\verb+_+throne                  &	(opulent throne)\\
%\verb@-@ & S\verb+_+tlcorn                  &	(top left corner)\\
%\verb@|@ & S\verb+_+tlwall                  &	(wall)\\
%\verb@(@ & S\verb+_+tool                    &	(useful item (pick-axe, key, lamp...))\\
%\verb@^@ & S\verb+_+trap\verb+_+door              &	(trap door)\\
%\verb@t@ & S\verb+_+trapper                 &	(trapper or lurker above)\\
%\verb@-@ & S\verb+_+trcorn                  &	(top right corner)\\
%\# & S\verb+_+tree                    &	(tree)\\
%\verb@T@ & S\verb+_+troll                   &	(troll)\\
%\verb@|@ & S\verb+_+trwall                  &	(wall)\\
%\verb@-@ & S\verb+_+tuwall                  &	(wall)\\
%\verb@U@ & S\verb+_+umber                   &	(umber hulk)\\
%\verb@u@ & S\verb+_+unicorn                 &	(unicorn or horse)\\
%\verb@<@ & S\verb+_+upladder                &	(ladder up)\\
%\verb@<@ & S\verb+_+upstair                 &	(staircase up)\\
%\verb@V@ & S\verb+_+vampire                 &	(vampire)\\
%\verb@|@ & S\verb+_+vbeam                   &	(wall)\\
%\# & S\verb+_+vcdbridge               &	(vertical raised drawbridge)\\
%\verb@+@ & S\verb+_+vcdoor                  &	(closed door)\\
%\verb@.@ & S\verb+_+venom                   &	(splash of venom)\\
%\verb@^@ & S\verb+_+vibrating\verb+_+square       &	(vibrating square)\\
%\verb@.@ & S\verb+_+vodbridge               &	(vertical lowered drawbridge)\\
%\verb@-@ & S\verb+_+vodoor                  &	(open door)\\
%\verb@v@ & S\verb+_+vortex                  &	(vortex)\\
%\verb@|@ & S\verb+_+vwall                   &	(vertical wall)\\
%\verb@/@ & S\verb+_+wand                    &	(wand)\\
%\verb@}@ & S\verb+_+water                   &	(water)\\
%\verb@)@ & S\verb+_+weapon                  &	(weapon)\\
%\verb@"@ & S\verb+_+web                     &	(web)\\
%\verb@w@ & S\verb+_+worm                    &	(worm)\\
%\verb@~@ & S\verb+_+worm\verb+_+tail              &	(long worm tail)\\
%\verb@W@ & S\verb+_+wraith                  &	(wraith)\\
%\verb@x@ & S\verb+_+xan                     &	(xan or other extraordinary insect)\\
%\verb@X@ & S\verb+_+xorn                    &	(xorn)\\
%\verb@Y@ & S\verb+_+yeti                    &	(apelike creature)\\
%\verb@Z@ & S\verb+_+zombie                  &	(zombie)\\
%\verb@z@ & S\verb+_+zruty                   &	(zruty)\\
%\verb@ @ & S\verb+_+pet\verb+_+override     &	(any pet if ACCESSIBILITY=1 is set)\\
%\verb@ @ & S\verb+_+hero\verb+_+override    &	(hero if ACCESSIBILITY=1 is set)
\verb@ @ & S\verb+_+air                     &	(air)\\
\_ & S\verb+_+altar                   &	(altar)\\
\verb@"@ & S\verb+_+amulet                  &	(amulet)\\
\verb@A@ & S\verb+_+angel                   &	(angelic being)\\
\verb@a@ & S\verb+_+ant                     &	(ant or other insect)\\
\verb@^@ & S\verb+_+anti\verb+_+magic\verb+_+trap       &	(anti-magic field)\\
\verb@[@ & S\verb+_+armor                   &	(suit or piece of armor)\\
\verb@[@ & S\verb+_+armour                  &	(suit or piece of armor)\\
\verb@^@ & S\verb+_+arrow\verb+_+trap             &	(arrow trap)\\
\verb@0@ & S\verb+_+ball                    &	(iron ball)\\
\# & S\verb+_+bars                    &	(iron bars)\\
\verb@B@ & S\verb+_+bat                     &	(bat or bird)\\
\verb@^@ & S\verb+_+bear\verb+_+trap              &	(bear trap)\\
\verb@-@ & S\verb+_+blcorn                  &	(bottom left corner)\\
\verb@b@ & S\verb+_+blob                    &	(blob)\\
\verb@+@ & S\verb+_+book                    &	(spellbook)\\
\verb@)@ & S\verb+_+boomleft                &	(boomerang open left)\\
\verb@(@ & S\verb+_+boomright               &	(boomerang open right)\\
\verb@`@ & S\verb+_+boulder                 &	(boulder)\\
\verb@-@ & S\verb+_+brcorn                  &	(bottom right corner)\\
\verb@C@ & S\verb+_+centaur                 &	(centaur)\\
\verb@_@ & S\verb+_+chain                   &	(iron chain)\\
\# & S\verb+_+cloud                   &	(cloud)\\
\verb@c@ & S\verb+_+cockatrice              &	(cockatrice)\\
\$ & S\verb+_+coin                    &	(pile of coins)\\
\# & S\verb+_+corr                    &	(corridor)\\
\verb@-@ & S\verb+_+crwall                  &	(wall)\\
\verb@-@ & S\verb+_+darkroom                &	(dark room)\\
\verb@^@ & S\verb+_+dart\verb+_+trap              &	(dart trap)\\
\verb@&@ & S\verb+_+demon                   &	(major demon)\\
\verb@*@ & S\verb+_+digbeam                 &	(dig beam)\\
\verb@>@ & S\verb+_+dnladder                &	(ladder down)\\
\verb@>@ & S\verb+_+dnstair                 &	(staircase down)\\
\verb@d@ & S\verb+_+dog                     &	(dog or other canine)\\
\verb@D@ & S\verb+_+dragon                  &	(dragon)\\
\verb@;@ & S\verb+_+eel                     &	(sea monster)\\
\verb@E@ & S\verb+_+elemental               &	(elemental)\\
\verb@/@ & S\verb+_+explode1                &	(explosion top left)\\
\verb@-@ & S\verb+_+explode2                &	(explosion top center)\\
\verb@\@ & S\verb+_+explode3                &	(explosion top right)\\
\verb@|@ & S\verb+_+explode4                &	(explosion middle left)\\
\verb@ @ & S\verb+_+explode5                &	(explosion middle center)\\
\verb@|@ & S\verb+_+explode6                &	(explosion middle right)\\
\verb@\@ & S\verb+_+explode7                &	(explosion bottom left)\\
\verb@-@ & S\verb+_+explode8                &	(explosion bottom center)\\
\verb@/@ & S\verb+_+explode9                &	(explosion bottom right)\\
\verb@e@ & S\verb+_+eye                     &	(eye or sphere)\\
\verb@^@ & S\verb+_+falling\verb+_+rock\verb+_+trap     &	(falling rock trap)\\
\verb@f@ & S\verb+_+feline                  &	(cat or other feline)\\
\verb@^@ & S\verb+_+fire\verb+_+trap              &	(fire trap)\\
\verb@!@ & S\verb+_+flashbeam               &	(flash beam)\\
\% & S\verb+_+food                    &	(piece of food)\\
\{ & S\verb+_+fountain                &	(fountain)\\
\verb@F@ & S\verb+_+fungus                  &	(fungus or mold)\\
\verb@*@ & S\verb+_+gem                     &	(gem or rock)\\
\verb@ @ & S\verb+_+ghost                   &	(ghost)\\
\verb@H@ & S\verb+_+giant                   &	(giant humanoid)\\
\verb@G@ & S\verb+_+gnome                   &	(gnome)\\
\verb@'@ & S\verb+_+golem                   &	(golem)\\
\verb@|@ & S\verb+_+grave                   &	(grave)\\
\verb@g@ & S\verb+_+gremlin                 &	(gremlin)\\
\verb@-@ & S\verb+_+hbeam                   &	(wall)\\
\# & S\verb+_+hcdbridge               &	(horizontal raised drawbridge)\\
\verb@+@ & S\verb+_+hcdoor                  &	(closed door)\\
\verb@.@ & S\verb+_+hodbridge               &	(horizontal lowered drawbridge)\\
\verb@|@ & S\verb+_+hodoor                  &	(open door)\\
\verb\^\ & S\verb+_+hole                    &	(hole)\\
\verb~@~ & S\verb+_+human                   &	(human or elf)\\
\verb@h@ & S\verb+_+humanoid                &	(humanoid)\\
\verb@-@ & S\verb+_+hwall                   &	(horizontal wall)\\
\verb@.@ & S\verb+_+ice                     &	(ice)\\
\verb@i@ & S\verb+_+imp                     &	(imp or minor demon)\\
\verb@I@ & S\verb+_+invisible               &	(invisible monster)\\
\verb@J@ & S\verb+_+jabberwock              &	(jabberwock)\\
\verb@j@ & S\verb+_+jelly                   &	(jelly)\\
\verb@k@ & S\verb+_+kobold                  &	(kobold)\\
\verb@K@ & S\verb+_+kop                     &	(Keystone Kop)\\
\verb@^@ & S\verb+_+land\verb+_+mine              &	(land mine)\\
\verb@}@ & S\verb+_+lava                    &	(molten lava)\\
\verb@l@ & S\verb+_+leprechaun              &	(leprechaun)\\
\verb@^@ & S\verb+_+level\verb+_+teleporter       &	(level teleporter)\\
\verb@L@ & S\verb+_+lich                    &	(lich)\\
\verb@y@ & S\verb+_+light                   &	(light)\\
\# & S\verb+_+litcorr                 &	(lit corridor)\\
\verb@:@ & S\verb+_+lizard                  &	(lizard)\\
\verb@\@ & S\verb+_+lslant                  &	(wall)\\
\verb@^@ & S\verb+_+magic\verb+_+portal           &	(magic portal)\\
\verb@^@ & S\verb+_+magic\verb+_+trap             &	(magic trap)\\
\verb@m@ & S\verb+_+mimic                   &	(mimic)\\
\verb@]@ & S\verb+_+mimic\verb+_+def              &	(mimic)\\
\verb@M@ & S\verb+_+mummy                   &	(mummy)\\
\verb@N@ & S\verb+_+naga                    &	(naga)\\
\verb@.@ & S\verb+_+ndoor                   &	(doorway)\\
\verb@n@ & S\verb+_+nymph                   &	(nymph)\\
\verb@O@ & S\verb+_+ogre                    &	(ogre)\\
\verb@o@ & S\verb+_+orc                     &	(orc)\\
\verb@p@ & S\verb+_+piercer                 &	(piercer)\\
\verb@^@ & S\verb+_+pit                     &	(pit)\\
\# & S\verb+_+poisoncloud             &	(poison cloud)\\
\verb@^@ & S\verb+_+polymorph\verb+_+trap         &	(polymorph trap)\\
\verb@}@ & S\verb+_+pool                    &	(water)\\
\verb@!@ & S\verb+_+potion                  &	(potion)\\
\verb@P@ & S\verb+_+pudding                 &	(pudding or ooze)\\
\verb@q@ & S\verb+_+quadruped               &	(quadruped)\\
\verb@Q@ & S\verb+_+quantmech               &	(quantum mechanic)\\
\verb@=@ & S\verb+_+ring                    &	(ring)\\
\verb@`@ & S\verb+_+rock                    &	(boulder or statue)\\
\verb@r@ & S\verb+_+rodent                  &	(rodent)\\
\verb@^@ & S\verb+_+rolling\verb+_+boulder\verb+_+trap  &	(rolling boulder trap)\\
\verb@.@ & S\verb+_+room                    &	(floor of a room)\\
\verb@/@ & S\verb+_+rslant                  &	(wall)\\
\verb@^@ & S\verb+_+rust\verb+_+trap              &	(rust trap)\\
\verb@R@ & S\verb+_+rustmonst               &	(rust monster or disenchanter)\\
\verb@?@ & S\verb+_+scroll                  &	(scroll)\\
\# & S\verb+_+sink                    &	(sink)\\
\verb@^@ & S\verb+_+sleeping\verb+_+gas\verb+_+trap     &	(sleeping gas trap)\\
\verb@S@ & S\verb+_+snake                   &	(snake)\\
\verb@s@ & S\verb+_+spider                  &	(arachnid or centipede)\\
\verb@^@ & S\verb+_+spiked\verb+_+pit             &	(spiked pit)\\
\verb@^@ & S\verb+_+squeaky\verb+_+board          &	(squeaky board)\\
\verb@0@ & S\verb+_+ss1                     &	(magic shield 1 of 4)\\
\# & S\verb+_+ss2                     &	(magic shield 2 of 4)\\
\verb+@+ & S\verb+_+ss3                     &	(magic shield 3 of 4)\\
\verb@*@ & S\verb+_+ss4                     &	(magic shield 4 of 4)\\
\verb@^@ & S\verb+_+statue\verb+_+trap            &	(statue trap)\\
\verb@ @ & S\verb+_+stone                   &	(solid rock or unexplored terrain\\
         &                                  &   \,or dark part of a room)\\
\verb@]@ & S\verb+_+strange\verb+_+obj      &	(strange object)\\
\verb@-@ & S\verb+_+sw\verb+_+bc                  &	(swallow bottom center)\\
\verb@\@ & S\verb+_+sw\verb+_+bl                  &	(swallow bottom left)\\
\verb@/@ & S\verb+_+sw\verb+_+br                  &	(swallow bottom right	)\\
\verb@|@ & S\verb+_+sw\verb+_+ml                  &	(swallow middle left)\\
\verb@|@ & S\verb+_+sw\verb+_+mr                  &	(swallow middle right)\\
\verb@-@ & S\verb+_+sw\verb+_+tc                  &	(swallow top center)\\
\verb@/@ & S\verb+_+sw\verb+_+tl                  &	(swallow top left)\\
\verb@\@ & S\verb+_+sw\verb+_+tr                  &	(swallow top right)\\
\verb@-@ & S\verb+_+tdwall                  &	(wall)\\
\verb@^@ & S\verb+_+teleportation\verb+_+trap     &	(teleportation trap)\\
\verb@\@ & S\verb+_+throne                  &	(opulent throne)\\
\verb@-@ & S\verb+_+tlcorn                  &	(top left corner)\\
\verb@|@ & S\verb+_+tlwall                  &	(wall)\\
\verb@(@ & S\verb+_+tool                    &	(useful item (pick-axe, key, lamp...))\\
\verb@^@ & S\verb+_+trap\verb+_+door              &	(trap door)\\
\verb@t@ & S\verb+_+trapper                 &	(trapper or lurker above)\\
\verb@-@ & S\verb+_+trcorn                  &	(top right corner)\\
\# & S\verb+_+tree                    &	(tree)\\
\verb@T@ & S\verb+_+troll                   &	(troll)\\
\verb@|@ & S\verb+_+trwall                  &	(wall)\\
\verb@-@ & S\verb+_+tuwall                  &	(wall)\\
\verb@U@ & S\verb+_+umber                   &	(umber hulk)\\
\verb@u@ & S\verb+_+unicorn                 &	(unicorn or horse)\\
\verb@<@ & S\verb+_+upladder                &	(ladder up)\\
\verb@<@ & S\verb+_+upstair                 &	(staircase up)\\
\verb@V@ & S\verb+_+vampire                 &	(vampire)\\
\verb@|@ & S\verb+_+vbeam                   &	(wall)\\
\# & S\verb+_+vcdbridge               &	(vertical raised drawbridge)\\
\verb@+@ & S\verb+_+vcdoor                  &	(closed door)\\
\verb@.@ & S\verb+_+venom                   &	(splash of venom)\\
\verb@^@ & S\verb+_+vibrating\verb+_+square       &	(vibrating square)\\
\verb@.@ & S\verb+_+vodbridge               &	(vertical lowered drawbridge)\\
\verb@-@ & S\verb+_+vodoor                  &	(open door)\\
\verb@v@ & S\verb+_+vortex                  &	(vortex)\\
\verb@|@ & S\verb+_+vwall                   &	(vertical wall)\\
\verb@/@ & S\verb+_+wand                    &	(wand)\\
\verb@}@ & S\verb+_+water                   &	(water)\\
\verb@)@ & S\verb+_+weapon                  &	(weapon)\\
\verb@"@ & S\verb+_+web                     &	(web)\\
\verb@w@ & S\verb+_+worm                    &	(worm)\\
\verb@~@ & S\verb+_+worm\verb+_+tail              &	(long worm tail)\\
\verb@W@ & S\verb+_+wraith                  &	(wraith)\\
\verb@x@ & S\verb+_+xan                     &	(xan or other extraordinary insect)\\
\verb@X@ & S\verb+_+xorn                    &	(xorn)\\
\verb@Y@ & S\verb+_+yeti                    &	(apelike creature)\\
\verb@Z@ & S\verb+_+zombie                  &	(zombie)\\
\verb@z@ & S\verb+_+zruty                   &	(zruty)\\
\verb@ @ & S\verb+_+pet\verb+_+override     &	(any pet if ACCESSIBILITY=1 is set)\\
\verb@ @ & S\verb+_+hero\verb+_+override    &	(hero if ACCESSIBILITY=1 is set)
\end{longtable}%
}

\hyphenation{sysconf}	%no syllable breaks => don't hyphenate file name
%%.lp
%Notes:
注意:

%%.lp "*"
%Several symbols in this table appear to be blank.
%They are the space character, except for S\verb+_+pet\verb+_+override
%and S\verb+_+hero\verb+_+override which don't have any default value
%and can only be used if enabled in the ``sysconf'' file.
この表のいくつかのシンボルは空白に見える。
これらはスペース文字である;
例外は、標準設定値がなく、``sysconf'' ファイルで有効にされた場合にのみ
使われる S\verb+_+pet\verb+_+override と
S\verb+_+hero\verb+_+override である。

%%.lp "*"
%S\verb+_+rock is misleadingly named; rocks and stones use S\verb+_+gem.
%Statues and boulders are the rock being referred to, but since
%version 3.6.0, statues are displayed as the monster they depict.
%So S\verb+_+rock is only used for boulders and not used at all if
%overridden by the more specific S\verb+_+boulder.
S\verb+_+rock は語弊のある名前である; 岩と石は S\verb+_+gem である。
石像と巨岩は参照される岩であるが、バージョン 3.6.0 から、石像は
モチーフとなっている怪物として表示される。
従って、S\verb+_+rock は巨岩のためだけに使われ、これがより限定的な
S\verb+_+boulder で上書きされると、全く使われないことになる。

%%.pg
%%.hn 2
%\subsection*{Configuring {\it NetHack\/} for Play by the Blind}
\subsection*{視覚障害者が {\it NetHack\/} をプレイするための設定}

%%.pg
%{\it NetHack\/} can be set up to use only standard ASCII characters for making
%maps of the dungeons. This makes the MS-DOS versions of {\it NetHack\/}
%completely
%accessible to the blind who use speech and/or Braille access technologies.
{\it NetHack\/} は洞窟の地図を作るのに標準の ASCII キャラクターだけを使うように
設定することもできる。これにより、MS-DOS 版の {\it NetHack\/} は
音声合成または点字技術を用いれば目の不自由な人でも完全にプレイ可能である。
%Players will require a good working knowledge of their screen-reader's
%review features, and will have to know how to navigate horizontally and
%vertically character by character. They will also find the search
%capabilities of their screen-readers to be quite valuable. Be certain to
%examine this Guidebook before playing so you have an idea what the screen
%layout is like. You'll also need to be able to locate the PC cursor. It is
%always where your character is located. Merely searching for an @-sign will
%not always find your character since there are other humanoids represented
%by the same sign. Your screen-reader should also have a function which
%gives you the row and column of your review cursor and the PC cursor.
%These co-ordinates are often useful in giving players a better sense of the
%overall location of items on the screen.
プレイヤーはこれらの画面読み取りシステムに関する十分な知識が必要で、
上下左右に一文字ずつ移動する方法を知らなければならないだろう。
画面読み取りシステムの検索能力を知ることもかなり価値がある。
プレイする前にこのガイドブックを十分に読めば、
画面のレイアウトがどんな感じかがわかるだろう。
また、 PC カーソルの位置を突き止めることができる必要がある。
これは自分のキャラクターの位置を示している。
`{\tt @}'は他のヒューマノイドを表わすのにも用いられるので、
自分の位置を探すのに`{\tt @}'を検索するだけでは不十分である。
さらに、あなたが見ている位置と PC カーソルの位置の物理的な行と桁を知らせる機能が
必要だろう。この機能はスクリーン全体にどのようなアイテムがあるかを判別するのに
便利である。
%%.pg
%{\it NetHack\/} can also be compiled with support for sending the game messages
%to an external program, such as a text-to-speech synthesizer. If the
%``{\tt \#version}'' extended command shows ``external program as a
%message handler'', your {\it NetHack\/}
%has been compiled with the capability. When compiling {\it NetHack\/}
%from source
%on Linux and other POSIX systems, define {\it MSGHANDLER\/} to enable it.
%To use
%the capability, set the environment variable {\it NETHACK\_MSGHANDLER\/} to an
%executable, which will be executed with the game message as the program's
%only parameter.
{\it NetHack\/} は、ゲームメッセージをテキスト/スピーチシンセサイザーなどの
外部プログラムに送信するサポート付きでコンパイルすることもできる。
拡張コマンド ``{\tt \#version}'' に
``external program as a message handler''
と表示されている場合、{\it NetHack\/} はこの機能付きでコンパイルされている。
Linux やその他の POSIX システムのソースから{\it NetHack\/}をコンパイルする
際には、これを有効にするには {\it MSGHANDLER\/} を定義する。
この機能を使用するには、環境変数 {\it NETHACK\_MSGHANDLER\/} を
実行可能ファイルに設定する; 実行可能ファイルは、ゲームメッセージを
プログラムの唯一の引数として実行される。
%%.pg
%While it is not difficult for experienced users to edit the {\it defaults.nh\/}
%file to accomplish this, novices may find this task somewhat daunting.
慣れたユーザーなら {\it defaults.nh\/} を修正することは難しくないだろうが、
初心者は少しひるんでしまう仕事かもしれない。
%Included within the symbol file of all official distributions of {\it NetHack\/}
%is a symset called {\it NHAccess\/}.  Selecting that symset in your
%configuration file will cause the game to run in a manner accessible
%to the blind. After you have gained some experience with the game
%and with editing files, you may want to alter settings via {\it SYMBOLS=\/}
%and {\it ROGUESYMBOLS=\/}
%in your configuration file to better suit your preferences.
%See the previous section for the special symbols S\verb+_+pet\verb+_+override
%to force a consistent symbol for all pets and S\verb+_+hero\verb+_+override
%to force a unique symbol for the player character if {\it accessibility\/}
%is enabled in the {\it sysconf\/} file.
{\it NetHack\/} の全ての公式配布のシンボルファイルには
{\it NHAccess\/} と呼ばれる symset が含まれている。
設定ファイルでこの symset を選択することにより、目の不自由な人が
プレイできるような設定になる。
ファイルを修正することやゲームそのものに慣れたなら、好みに合うように
設定ファイルの {\it SYMBOLS=\/} と {\it ROGUESYMBOLS=\/} で
設定を変更するとよいだろう。
全てのペットに一貫したシンボルを強制する特殊シンボル
S\verb+_+pet\verb+_+override と、
{\it sysconf\/} ファイルで {\it accessibility\/} が有効の時に
プレイヤーキャラクターに特別なシンボルを強制する
特殊シンボル S\verb+_+hero\verb+_+override については
前述の節を参照のこと。

%%.pg
%The most crucial settings to make the game more accessible are:
ゲームのアクセシビリティの改善に最も影響を与える設定は以下のものである:
%%.pg
\blist{}
%%.lp
\item[\ib{symset:NHAccess}]
%Load a symbol set appropriate for use by blind players.
視力障害者プレイヤーに適したシンボル集合を読み込む。
%%.lp
\item[\ib{roguesymset:NHAccess}]
%Load a symbol set for the rogue level that is appropriate for
%use by blind players.
視力障害者プレイヤーに適した rogue レベルのためのシンボル集合を読み込む。
%%.lp
\item[\ib{menustyle:traditional}]
%This will assist in the interface to speech synthesizers.
これは音声合成システムを利用する助けになる。
%%.lp
\item[\ib{nomenu\verb+_+overlay}]
%Show menus on a cleared screen and aligned to the left edge.
クリアされた画面に左寄せでメニューを表示する。
%%.lp
\item[\ib{number\verb+_+pad}]
%A lot of speech access programs use the number-pad to review the screen.
%If this is the case, disable the number\verb+_+pad option and use the
%traditional Rogue-like commands.
多くの発声プログラムは number-pad を画面の確認に用いる。
この場合、number\verb+_+pad オプションをオフにして、伝統的な
Rogue 風コマンドを使うこと。
%%.lp
\item[\ib{autodescribe}]
%Automatically describe the terrain under the cursor when targeting.
目標を決めるときにカーソルの下の地形を自動的に表現する。
%%.lp
\item[\ib{mention\verb+_+walls}]
%Give feedback messages when walking towards a wall or when travel command
%was interrupted.
壁に向かって歩いたり長距離移動コマンドが中断されたりしたときに
メッセージを表示する。
%%.lp
\item[\ib{whatis\verb+_+coord:compass}]
%When targeting with cursor, describe the cursor position with coordinates
%relative to your character.
カーソルで目標を決めるとき、自分の位置からのカーソルの相対座標を表現する。
%%.lp
\item[\ib{whatis\verb+_+filter:area}]
%When targeting with cursor, filter possible locations so only those in
%the same area (eg. same room, or same corridor) are considered.
カーソルで目標を決めるとき、同じエリア(例えば同じ部屋、同じ通路)のみに
なるように可能な場所をフィルタリングする。
%%.lp
\item[\ib{whatis\verb+_+moveskip}]
%When targeting with cursor and using fast-move, skip the same glyphs instead
%of moving 8 units at a time.
カーソルで目標を決めて高速移動を使うとき、一度に 8 マスずつ動くのではなく、
同じ記号を飛ばす。
%%.lp
\item[\ib{nostatus\verb+_+updates}]
%Prevent updates to the status lines at the bottom of the screen, if
%your screen-reader reads those lines. The same information can be
%seen via the {\tt \#attributes} command.
スクリーンリーダーが画面下部のステータス行を読むとき、
その更新を抑制する。
同じ情報は {\tt \#attributes} コマンドでも得られる。
\elist

%%.hn2
%\subsection*{Global Configuration for System Administrators}
\subsection*{システム管理者のためのグローバルな設定}

%%.pg
%If {\it NetHack\/} is compiled with the SYSCF option, a system administrator
%should set up a global configuration; this is a file in the
%same format as the traditional per-user configuration file (see above).
{\it NetHack\/} が SYSCF オプション付きでコンパイルされている場合、システム管理者は
グローバル設定を設定するべきである; これは(前述した)伝統的なユーザー毎の
設定ファイルと同じ形式のファイルである。

%This file should be named sysconf and placed in the same directory as
%the other {\it NetHack\/} support files.
%The options recognized in this file are listed below. Any option not
%set uses a compiled-in default (which may not be appropriate for your
%system).
このファイルは sysconf という名前で、その他の {\it NetHack\/} 補助ファイルと
同じディレクトリに置かれる。
このファイルで認識されるオプションは以下の通りである。
設定されてなかったオプションは(あなたのシステムでは適切ではないかもしれない)
コンパイル時のデフォルトが使われる。

%%.pg
\blist{}
%%.lp
\item[\ib{WIZARDS}]
%A space-separated list of user name who are allowed to
%play in debug mode (commonly referred to as wizard mode).
%A value of a single
%asterisk (*) allows anyone to start a game in debug mode.
(一般的にウィザードモードと呼ばれる)デバッグモードでプレイすることを
許されたユーザー名の空白区切りのリスト。
値が単一のアスタリスク (*) の場合は誰でもデバッグモードでゲームを
開始できる。
%%.lp
\item[\ib{SHELLERS}]
%A list of users who are allowed to use the shell escape command (`{\tt !}').
%The syntax is the same as WIZARDS.
シェルエスケープコマンド (`{\tt !}') の使用を許されたユーザーのリスト。
文法は WIZARDS と同じ。
%%.lp
\item[\ib{EXPLORERS}]
%A list of users who are allowed to use the explore mode.
%The syntax is the same as WIZARDS.
探検モードを使うことを許されたユーザーのリスト。
文法は WIZARDS と同じ。
%%.lp
\item[\ib{MAXPLAYERS}]
%Limit the maximum number of games that can be running at the same time.
同時にプレイ出来るゲームの数。
%%.lp
\item[\ib{SUPPORT}]
%A string explainign how to get local support (no default value).
ローカルサポートを受ける方法を示す文字列 (標準設定値はなし)。
%%.lp
\item[\ib{RECOVER}]
%A string explaining how to recover a game on this system (no default value).
このシステムでゲームを復旧する方法を示す文字列 (標準設定値はなし)。
%%.lp
\item[\ib{SEDUCE}]
%0 or 1 to disable or enable, respectively, the SEDUCE option.
%When disabled, incubi and succubi behave like nymphs.
0 と 1 でそれぞれ SEDUCE オプションを無効/有効にする。
無効にすると、インキュバスとサキュバスはニンフのように振る舞う。
%%.lp
\item[\ib{CHECK\verb+_+PLNAME}]
%Setting this to 1 will make the EXPLORERS, WIZARDS, and SHELLERS check
%for the player name instead of the user's login name.
これを 1 にすると、EXPLORERS, WIZARDS, SHELLERS は
ユーザーのログイン名ではなくプレイヤー名をチェックする。
%%.lp
\item[\ib{CHECK\verb+_+SAVE\verb+_+UID}]
%0 or 1 to disable or enable, respectively, the UID
%(used identification number) checking for save files (to verify that the
%user who is restoring is the same one who saved).
0 と 1 でそれぞれ、(復元するユーザーがセーブしたユーザーと同じかを
検証するために) セーブファイルの UID (識別番号として使われる)
チェックを無効/有効にする。
\elist

%%.pg
%The following options affect the score file:
以下のオプションは score ファイルに影響する:
\blist {}
%%.pg
%%.lp
\item[\ib{PERSMAX}]
%Maximum number of entries for one person.
一人のエントリの最大数。
%%.lp
\item[\ib{ENTRYMAX}]
%Maximum number of entries in the score file.
score ファイルのエントリの最大数。
%%.lp
\item[\ib{POINTSMIN}]
%Minimum number of points to get an entry in the score file.
score ファイルにエントリを記録する最低得点。
%%.lp
\item[\ib{PERS\verb+_+IS\verb+_+UID}]
%0 or 1 to use user names or numeric userids, respectively, to identify
%unique people for the score file
score ファイル上でプレイヤーを認識するのに、0 ではユーザー名、1 では
数値のユーザー ID を使う。
%%.lp
\item[\ib{ACCESSIBILITY}]
%0 or 1 to disable or enable, respectively, the ability for players
%to set S\verb+_+pet\verb+_+override and S\verb+_+hero\verb+_+override 
%symbols in their configuration file.
プレイヤーが自分自身の設定ファイルで
S\verb+_+pet\verb+_+override and S\verb+_+hero\verb+_+override
シンボルを設定することを、0 では無効に、1 では有効にする。
%%.lp
\item[\ib{PORTABLE\verb+_+DEVICE\verb+_+PATHS}]
%0 or 1 Windows OS only, the game will look for all of its external
%files, and write to all of its output files in one place 
%rather than at the standard locations.
Windows OS のみ; 全ての外部ファイルを検索し、全ての出力ファイルを
書き込むディレクトリを、標準の位置ではなく一ヶ所にする。
%%.lp
\item[\ib{DUMPLOGFILE}]
%A filename where the end-of-game dumplog is saved.
%Not defining this will prevent dumplog from being created. Only available
%if your game is compiled with DUMPLOG. Allows the following placeholders:
ゲーム終了時のダンプログを保存するファイル名。
これを定義していないとダンプログは作成されない。
ゲームが DUMPLOG 付きでコンパイルされている場合にのみ利用可能。
次のプレースフォルダが利用できる:
%%.sd
%%.si
%{\tt \%\%}  --- literal `{\tt \%}'\\
%{\tt \%v}  --- version (eg. ``{\tt 3.6.3-0}'')\\
%{\tt \%u}  --- game UID\\
%{\tt \%t}  --- game start time, UNIX timestamp format\\
%{\tt \%T}  --- current time, UNIX timestamp format\\
%{\tt \%d}  --- game start time, YYYYMMDDhhmmss format\\
%{\tt \%D}  --- current time, YYYYMMDDhhmmss format\\
%{\tt \%n}  --- player name\\
%{\tt \%N}  --- first character of player name
{\tt \%\%}  --- リテラルな `{\tt \%}'\\
{\tt \%v}  --- バージョン (例: ``{\tt 3.6.3-0}'')\\
{\tt \%u}  --- ゲーム UID\\
{\tt \%t}  --- ゲーム開始時刻、UNIX タイムスタンプ形式\\
{\tt \%T}  --- 現在時刻、UNIX タイムスタンプ形式\\
{\tt \%d}  --- ゲーム開始時刻、YYYYMMDDhhmmss 形式\\
{\tt \%D}  --- 現在時刻、YYYYMMDDhhmmss 形式\\
{\tt \%n}  --- プレイヤー名\\
{\tt \%N}  --- プレイヤー名の最初の文字
%%.ei
%%.ed
\elist

%%.hn 1
%\section{Scoring}
\section{得点}

%%.pg
%{\it NetHack\/} maintains a list of the top scores or scorers on your machine,
%depending on how it is set up.  In the latter case, each account on
%the machine can post only one non-winning score on this list.  If
%you score higher than someone else on this list, or better your
%previous score, you will be inserted in the proper place under your
%current name.  How many scores are kept can also be set up when
%{\it NetHack\/} is compiled.
{\it NetHack\/} は設定によって、あなたの機械における高得点の一覧または高得点
者の一覧を作成する。後者の設定になっている場合、ゲームを最後まで終えた
場合を除きそのマシンの 1 つのアカウントにつき 1 つだけしかこの一覧には
載らない。あなたがこの一覧に載っている他の誰かより高い得点を取ったとき、
もしくはあなたの前の得点より高い得点を取ったときにのみ、その一覧のしか
るべき位置にあなたの名前が載る。得点が何個まで載るかもコンパイル時に設
定できる。

%%.pg
%Your score is chiefly based upon how much experience you gained, how
%much loot you accumulated, how deep you explored, and how the game
%ended.  If you quit the game, you escape with all of your gold intact.
%If, however, you get killed in the Mazes of Menace, the guild will
%only hear about 90\,\% of your gold when your corpse is discovered
%(adventurers have been known to collect finder's fees).  So, consider
%whether you want to take one last hit at that monster and possibly
%live, or quit and stop with whatever you have.  If you quit, you keep
%all your gold, but if you swing and live, you might find more.
得点は主に経験値、戦利品、到達した階とゲームの終わり方に基づいている。
ゲームを放棄した場合には持っている金を全額手にして脱出できる。
しかし恐怖の迷宮の中で殺された場合には、亡骸が発見されると持っている金の 90\,\%
だけがギルドに伝えられる(冒険者は亡骸を発見した場合には手数料を取るこ
とが知られている)。従って怪物にさらに一撃を加えて運あらば生き残らんと
するか、放棄してその時点で持っている物で終了するかはあなたが決めること
である。放棄した場合はすべての金を手にすることができる。しかし戦って生
き残ればさらに多くを得ることができるかも知れない。

%%.pg
%If you just want to see what the current top players/games list is, you
%can type
単に現時点での高得点者・高得点の一覧を見たい場合にはほとんどの
バージョンでは
\begin{verbatim}
    nethack -s all
\end{verbatim}
%on most versions.
と入力すればよい。

%%.hn 1
%\section{Explore mode}
\section{探検モード}

%%.pg
%{\it NetHack\/} is an intricate and difficult game.  Novices might falter
%in fear, aware of their ignorance of the means to survive.  Well, fear
%not.  Your dungeon comes equipped with an ``explore'' or ``discovery''
%mode that enables you to keep old save files and cheat death, at the
%paltry cost of not getting on the high score list.
{\it NetHack\/} は複雑で難しいゲームである。初心者は生き残るすべを知らないこ
とに気づき、恐怖におじけづいてしまうかもしれない。しかし恐れることはな
い。冒険する洞窟に「探検」もしくは「発見」モードがある場合には、高得点
の一覧に載らないというとるに足らないデメリットと引き替えに前のセーブファ
イルを残しておけたり不死身となれたりするのである。

%%.pg
%There are two ways of enabling explore mode.  One is to start the game
%with the {\tt -X}
%command-line switch or with the
%{\it playmode:explore\/}
%option.  The other is to issue the `{\tt \#exploremode}' extended command while
%already playing the game.  Starting a new game in explore mode provides your
%character with a wand of wishing in initial inventory; switching
%during play does not.  The other benefits of explore mode are left for
%the trepid reader to discover.
探検モードに入るには 2 通りの方法がある。1 つはゲームの開始時に
{\tt -X} スイッチを付けるか
{\it playmode:explore\/}
オプションを指定することである。
もう 1 つはゲームのプレイ中に `{\tt \#exploremode}' 拡張コマンドを
入力することである。
探検モードで新しいゲームを開始すると、初期装備として願いの杖が追加される;
プレイ中に切り替えた場合は追加されない。
探検モードで得られる他の利点は大胆な読者諸君自ら確かめて欲しい。

%%.pg
%%.hn 2
%\subsection*{Debug mode}
\subsection*{デバッグモード}

%%.pg
%Debug mode, also known as wizard mode, is undocumented aside from this
%brief description and the various ``debug mode only'' commands listed
%among the command descriptions.
%It is intended for tracking down problems within the
%program rather than to provide god-like powers to your character, and
%players who attempt debugging are expected to figure out how to use it
%themselves.
%It is initiated by starting the game with the
%{\tt -D}
%command-line switch or with the
%{\it playmode:debug\/}
%option.
ウィザードモードとも呼ばれるデバッグモードは、この概説と、
コマンド説明で挙げられている様々な「デバッグモード専用」コマンドの他には
文書化されていない。
これはキャラクターに神のような力を与えるためではなく、プログラムの問題を
調べることを意図したもので、デバッグを試みるプレイヤーは
これをどのように使うかが分かることを想定している。
これは
{\tt -D}
コマンドラインスイッチか
{\it playmode:debug\/}
オプションでゲームを始めることで開始される。

%%.pg
%For some systems, the player must be logged in
%under a particular user name to be allowed to use debug mode; for others,
%the hero must be given a particular character name (but may be any role;
%there's no connection between ``wizard mode'' and the {\it Wizard\/} role).
%Attempting to start a game in debug mode when not allowed
%or not available will result in falling back to explore mode instead.
システムによっては、プレイヤーは、デバッグモードを使うことを許されている
特定のユーザー名でログインしなければならない; それ以外では、
キャラクターは特定のキャラクター名を使わなければならない
(しかし職業はどれでもよい; ``ウィザード'' という名前と
職業の {\it Wizard\/} は何の関係もない)。
デバッグモードを許されていなかったり有効でない場合にデバッグモードを
使おうとすると、代わりに探検モードにフォールバックする。

%%.hn
%\section{Credits}
\section{クレジット}
%%.pg
%The original %
%{\it hack\/} game was modeled on the Berkeley
%%.ux
%UNIX
%{\it rogue\/} game.  Large portions of this paper were shamelessly
%cribbed from %
%{\it A Guide to the Dungeons of Doom}, by Michael C. Toy
%and Kenneth C. R. C. Arnold.  Small portions were adapted from
%{\it Further Exploration of the Dungeons of Doom}, by Ken Arromdee.
最初の {\it hack\/} というゲームは、バークレイ版
UNIX のゲーム {\it rogue\/} をモデルとしている。
このドキュメントの大部分は Michael C. Toy と Kenneth C. R. C. Arnold の手になる
{\it A Guide to the Dungeons of Doom(運命の洞窟への招待)} を図々しくも盗用したものである。
また一部は Ken Arromdee による
{\it Further Exploration of the Dungeons of Doom(運命の洞窟の深部への探検)}
から採っている。

%%.pg
%{\it NetHack\/} is the product of literally dozens of people's work.
%Main events in the course of the game development are described below:
{\it NetHack\/} は文字どおり何十人もの人々の手によって完成された。
以下はゲームの開発の過程での主な出来事である。

%%.pg
\bigskip
%\nd {\it Jay Fenlason\/} wrote the original {\it Hack\/} with help from {\it
%Kenny Woodland}, {\it Mike Thome}, and {\it Jon Payne}.
\nd {\it Jay Fenlason\/} は {\it Kenny Woodland}, {\it Mike Thome}, {\it Jon Payne}
らの助けを得て、最初の {\it Hack\/} を書いた。

%%.pg
\medskip
%\nd {\it Andries Brouwer\/} did a major re-write, transforming {\it Hack\/}
%into a very different game, and published (at least) three versions (1.0.1,
%1.0.2, and 1.0.3) for UNIX machines to the Usenet.
\nd {\it Andries Brouwer\/} はプログラムを大幅に書き換えて {\it Hack\/} を
元とはかなり異なったゲームに仕立て上げ、UNIX
マシン用の(少なくとも)3 バージョン(1.0.1、1.0.2、1.0.3)を Usenet で発表した。

%%.pg
\medskip
%\nd {\it Don G. Kneller\/} ported {\it Hack\/} 1.0.3 to Microsoft C and MS-DOS,
%producing {\it PC Hack\/} 1.01e, added support for DEC Rainbow graphics in
%version 1.03g, and went on to produce at least four more versions (3.0, 3.2,
%3.51, and 3.6).
\nd {\it Don G. Kneller\/} は {\it Hack\/} 1.0.3 を Microsoft C を使用して MS-DOS へ移植
し {\it PC Hack\/} 1.01e を作成した。バージョン 1.03g では DEC Rainbow の
グラフィックをサポートし、続いて少なくともあと 4 つのバージョン(3.0、3.2、
3.51、3.6)を作成した。

%%.pg
\medskip
%\nd {\it R. Black\/} ported {\it PC Hack\/} 3.51 to Lattice C and the Atari
%520/1040ST, producing {\it ST Hack\/} 1.03.
\nd {\it R. Black\/} は {\it PC Hack\/} 3.51 を Lattice C を使用して Atari 520/1040ST へ
移植し {\it ST Hack\/} 1.03 を作成した。

%%.pg
\medskip
%\nd {\it Mike Stephenson\/} merged these various versions back together,
%incorporating many of the added features, and produced {\it NetHack\/} version
%1.4.  He then coordinated a cast of thousands in enhancing and debugging
%{\it NetHack\/} 1.4 and released {\it NetHack\/} versions 2.2 and 2.3.
\nd {\it Mike Stephenson\/} はこれらのいろいろなバージョンを再び 1 つに統合し、
多くの追加要素と合わせて {\it NetHack\/} バージョン 1.4 を作成した。さらに彼は
何千人もの人々に作業を分担して {\it NetHack\/} 1.4 の拡張とデバッグを行い、
{\it NetHack\/} バージョン 2.2 と 2.3 を発表した。

%%.pg
\medskip
%\nd Later, Mike coordinated a major rewrite of the game, heading a team which
%included {\it Ken Arromdee}, {\it Jean-Christophe Collet}, {\it Steve Creps},
%{\it Eric Hendrickson}, {\it Izchak Miller}, {\it Eric S. Raymond}, {\it John
%Rupley}, {\it Mike Threepoint}, and {\it Janet Walz}, to produce {\it
%NetHack\/} 3.0c.
\nd さらにその後、Mike は {\it Ken Arromdee}, {\it Jean-Christophe Collet},
{\it Steve Creps}, {\it Eric Hendrickson}、{\it Izchak Miller}、
{\it Eric S. Raymond}、{\it John Rupley}、
{\it Mike Threepoint}、{\it Janet Walz}
を含むチームを率いてゲームの大幅な書き直しを行ない、
{\it NetHack\/} 3.0c を作成した。

%%.pg
\medskip
%\nd {\it NetHack\/} 3.0 was ported to the Atari by {\it Eric R. Smith}, to OS/2 by
%{\it Timo Hakulinen}, and to VMS by {\it David Gentzel}.  The three of them
%and {\it Kevin Darcy\/} later joined the main {\it NetHack Development Team} to produce
%subsequent revisions of 3.0.
\nd {\it NetHack\/} 3.0 は {\it Eric R. Smith} によって Atari へ、
{\it Timo Hakulinen} によって OS/2 へ、
{\it David Gentzel} によって VMS へ移植された。
この 3 人と {\it Kevin Darcy\/} はその後主力開発チームに加わり、
3.0 のその後のいくつかの改訂版を作成した。

%%.pg
\medskip
%\nd {\it Olaf Seibert\/} ported {\it NetHack\/} 2.3 and 3.0 to the Amiga.  {\it
%Norm Meluch}, {\it Stephen Spackman\/} and {\it Pierre Martineau\/} designed
%overlay code for {\it PC NetHack\/} 3.0.  {\it Johnny Lee\/} ported {\it
%NetHack\/} 3.0 to the Macintosh.  Along with various other Dungeoneers, they
%continued to enhance the PC, Macintosh, and Amiga ports through the later
%revisions of 3.0.
\nd {\it Olaf Seibert\/} は {\it NetHack\/} 2.3 と 3.0 を Amiga へ移植した。
{\it Norm Meluch}、{\it Stephen Spackman\/}、{\it Pierre Martineau\/} は
{\it PC NetHack\/} 3.0 のためのオーバーレイルーチンを設計した。
{\it Johnny Lee\/} は {\it NetHack\/} 3.0 を Macintosh へ移植した。
彼らは他のさまざまな洞窟の主たちとともに PC、Macintosh、Amiga
の移植版の拡張を続け、 3.0 のその後のいくつかの改訂版を作成した。

%%.pg
\medskip
%\nd Headed by {\it Mike Stephenson\/} and coordinated by {\it Izchak Miller\/} and
%{\it Janet Walz}, the {\it NetHack Development Team} which now included {\it Ken Arromdee},
%{\it David Cohrs}, {\it Jean-Christophe Collet}, {\it Kevin Darcy},
%{\it Matt Day}, {\it Timo Hakulinen}, {\it Steve Linhart}, {\it Dean Luick},
%{\it Pat Rankin}, {\it Eric Raymond}, and {\it Eric Smith\/} undertook a radical
%revision of 3.0.  They re-structured the game's design, and re-wrote major
%parts of the code.  They added multiple dungeons, a new display, special
%individual character quests, a new endgame and many other new features, and
%produced {\it NetHack\/} 3.1.
\nd {\it Mike Stephenson\/} をリーダーとし、
{\it Izchak Miller\/} と {\it Janet Walz} の助けによって
{\it Ken Arromdee},
{\it David Cohrs}, {\it Jean-Christophe Collet}, {\it Kevin Darcy},
{\it Matt Day}, {\it Timo Hakulinen}, {\it Steve Linhart}, {\it Dean Luick},
{\it Pat Rankin}, {\it Eric Raymond}, {\it Eric Smith\/} を含む制作チームが
3.0の徹底的な見直しを行なった。
彼らはゲームデザインを再構築し、コードの大部分を書き直した。
複合ダンジョン、新しいディスプレイ、それぞれのキャラクター毎の特別のクエスト、
新しいエンドゲームとその他の多くの新しい要素を追加し、
{\it NetHack\/} 3.1を制作した。

%%.pg
\medskip
%\nd {\it Ken Lorber}, {\it Gregg Wonderly\/} and {\it Greg Olson}, with help
%from {\it Richard Addison}, {\it Mike Passaretti}, and {\it Olaf Seibert},
%developed {\it NetHack\/} 3.1 for the Amiga.
\nd {\it Ken Lorber}, {\it Gregg Wonderly\/}, {\it Greg Olson} は
{\it Richard Addison}, {\it Mike Passaretti}, {\it Olaf Seibert} の助けを借りて
Amiga 用 {\it NetHack\/} 3.1 を作成した。

%%.pg
\medskip
%\nd {\it Norm Meluch\/} and {\it Kevin Smolkowski}, with help from
%{\it Carl Schelin}, {\it Stephen Spackman}, {\it Steve VanDevender},
%and {\it Paul Winner}, ported {\it NetHack\/} 3.1 to the PC.
\nd {\it Norm Meluch} と {\it Kevin Smolkowski} は
{\it Carl Schelin}, {\it Stephen Spackman}, {\it Steve VanDevender},
{\it Paul Winner} の助けを借りて {\it NetHack\/} 3.1 を PC に移植した。

%%.pg
\medskip
%\nd {\it Jon W\{tte} and {\it Hao-yang Wang},
%with help from {\it Ross Brown}, {\it Mike Engber}, {\it David Hairston},
%{\it Michael Hamel}, {\it Jonathan Handler}, {\it Johnny Lee},
%{\it Tim Lennan}, {\it Rob Menke}, and {\it Andy Swanson},
%developed {\it NetHack\/} 3.1 for the Macintosh, porting it for MPW.
%Building on their development, {\it Bart House} added a Think C port.
\nd {\it Jon W\{tte} と {\it Hao-yang Wang} は
{\it Ross Brown}, {\it Mike Engber}, {\it David Hairston},
{\it Michael Hamel}, {\it Jonathan Handler}, {\it Johnny Lee},
{\it Tim Lennan}, {\it Rob Menke}, {\it Andy Swanson} の助けを借りて
MPW 用の Macintosh 用 {\it NetHack\/} 3.1 を作成した。
その作成中に、{\it Bart House} は Think C 用を加えた。

%%.pg
\medskip
%\nd {\it Timo Hakulinen\/} ported {\it NetHack\/} 3.1 to OS/2.
%{\it Eric Smith\/} ported {\it NetHack\/} 3.1 to the Atari.
%{\it Pat Rankin}, with help from {\it Joshua Delahunty},
%was responsible for the VMS version of {\it NetHack\/} 3.1.
%{\it Michael Allison} ported {\it NetHack\/} 3.1 to Windows NT.
\nd {\it Timo Hakulinen} は {\it NetHack\/} 3.1 を OS/2 に移植した。
{\it Eric Smith} は {\it NetHack\/} 3.1 を Atari に移植した。
{\it Pat Rankin} は {\it Joshua Delahunty} の助けを借りて VMS 版
{\it NetHack\/} 3.1 を作成した。
{\it Michael Allison} は {\it NetHack\/} 3.1 を Windows NT に移植した。

%%.pg
\medskip
%\nd {\it Dean Luick}, with help from {\it David Cohrs}, developed {\it NetHack\/}
%3.1 for X11.
%{\it Warwick Allison} wrote a tiled version of {\it NetHack\/} for the Atari;
%he later contributed the tiles to the {\it NetHack Development Team} and tile support was
%then added to other platforms.
\nd {\it Dean Luick} は {\it David Cohrs} の助けを借りて X11 用
{\it NetHack\/}3.1 を作成した。
{\it Warwick Allison} は Atari 用に {\it NetHack\/} のタイル版を作成した。
彼は後にタイルを開発チームに寄贈し、タイルサポートは他のプラットフォームにも
追加された。

%%.pg
\medskip
%\nd The 3.2 {\it NetHack Development Team}, comprised of {\it Michael Allison}, {\it Ken
%Arromdee}, {\it David Cohrs}, {\it Jessie Collet}, {\it Steve Creps}, {\it
%Kevin Darcy}, {\it Timo Hakulinen}, {\it Steve Linhart}, {\it Dean Luick},
%{\it Pat Rankin}, {\it Eric Smith}, {\it Mike Stephenson}, {\it Janet Walz},
%and {\it Paul Winner}, released version 3.2 in April of 1996.
\nd {\it Michael Allison}, {\it Ken Arromdee},
{\it David Cohrs}, {\it Jessie Collet}, {\it Steve Creps},
{\it Kevin Darcy}, {\it Timo Hakulinen}, {\it Steve Linhart}, {\it Dean Luick},
{\it Pat Rankin}, {\it Eric Smith}, {\it Mike Stephenson}, {\it Janet Walz},
{\it Paul Winner} からなる 3.2の開発チームは バージョン 3.2 を
1996 年 4 月にリリースした。

%%.pg
\medskip
%\nd Version 3.2 marked the tenth anniversary of the formation of the development
%team.  In a testament to their dedication to the game, all thirteen members
%of the original {\it NetHack Development Team} remained on the team at the start of work on
%that release.  During the interval between the release of 3.1.3 and 3.2,
%one of the founding members of the {\it NetHack Development Team}, {\it Dr. Izchak Miller},
%was diagnosed with cancer and passed away.  That release of the game was
%dedicated to him by the development and porting teams.
\nd バージョン 3.2 は開発チームが結成されてから 10 周年のものであった。
彼らのゲームへの貢献について記すと、13 人の初期の開発チーム全員は
このリリースの開始時の最初の時点までチームに残っていた。
が、 3.1.3 から 3.2 の間に、開発チームの創始者の一人である
{\it Izchak Miller博士}が癌と診断され、亡くなった。
このリリースは開発チームおよび移植チームから彼に捧げられた。

%%.pg
\medskip
%During the lifespan of {\it NetHack\/} 3.1 and 3.2, several enthusiasts
%of the game added
%their own modifications to the game and made these ``variants'' publicly
%available:
{\it NetHack\/} 3.1 と 3.2 の時代に、ゲームに熱狂した何人かが
ゲームに自分自身の変更を加えた「亜種」を公に発表した。

%%.pg
\medskip
%{\it Tom Proudfoot} and {\it Yuval Oren} created {\it NetHack++},
%which was quickly renamed {\it NetHack$--$}.
{\it Tom Proudfoot} と {\it Yuval Oren} は {\it NetHack++} を作成し、
これはすぐに {\it NetHack$--$} に改名された。
%Working independently, {\it Stephen White} wrote {\it NetHack Plus}.
これとは独立に、{\it Stephen White} は {\it NetHack Plus} を作成した。
%{\it Tom Proudfoot} later merged {\it NetHack Plus}
%and his own {\it NetHack$--$} to produce {\it SLASH}.
{\it Tom Proudfoot} は後に {\it NetHack Plus} と {\it NetHack$--$} を統合し、
{\it SLASH} を作成した。
%{\it Larry Stewart-Zerba} and {\it Warwick Allison} improved the spell
%casting system with the Wizard Patch.
{\it Larry Stewart-Zerba} と {\it Warwick Allison} は呪文詠唱システムを改良した
Wizard Patch を作成した。
%{\it Warwick Allison} also ported {\it NetHack\/} to use the Qt interface.
{\it Warwick Allison} は {\it NetHack\/} を Qt インターフェースを使うように
変更した。

%%.pg
\medskip
%{\it Warren Cheung} combined {\it SLASH} with the Wizard Patch
%to produce {\it Slash'EM\/}, and
%with the help of {\it Kevin Hugo}, added more features.
{\it Warren Cheung} は {\it SLASH} と Wizard Patch を統合して {\it Slash'EM\/} を作成し、
{\it Kevin Hugo} の力を借りて、多くの要素を追加した。
%Kevin later joined the
%{\it NetHack Development Team} and incorporated the best of these ideas into {\it NetHack\/} 3.3.
Kevin は後に開発チームに加わり、これらのアイデアを {\it NetHack\/} 3.3 に
統合した。

%%.pg
\medskip
%The final update to 3.2 was the bug fix release 3.2.3, which was released
%simultaneously with 3.3.0 in December 1999 just in time for the Year 2000.
3.2 の最終バージョンはバグ修正版である 3.2.3 である。
これは 2000 年問題対策のために 3.3.0 と同時に 1999 年 12 月に発表された。

%%.pg
\medskip
%The 3.3 {\it NetHack Development Team}, consisting of {\it Michael Allison}, {\it Ken Arromdee},
%{\it David Cohrs}, {\it Jessie Collet}, {\it Steve Creps}, {\it Kevin Darcy},
%{\it Timo Hakulinen}, {\it Kevin Hugo}, {\it Steve Linhart}, {\it Ken Lorber},
%{\it Dean Luick}, {\it Pat Rankin}, {\it Eric Smith}, {\it Mike Stephenson},
%{\it Janet Walz}, and {\it Paul Winner}, released 3.3.0 in
%December 1999 and 3.3.1 in August of 2000.
{\it Michael Allison}, {\it Ken Arromdee},
{\it David Cohrs}, {\it Jessie Collet}, {\it Steve Creps}, {\it Kevin Darcy},
{\it Timo Hakulinen}, {\it Kevin Hugo}, {\it Steve Linhart}, {\it Ken Lorber},
{\it Dean Luick}, {\it Pat Rankin}, {\it Eric Smith}, {\it Mike Stephenson},
{\it Janet Walz}, {\it Paul Winner} からなる 3.3 の開発チームは
1999 年 12 月に 3.3.0を、2000 年 8 月に 3.3.1 をリリースした。

%%.pg
\medskip
%Version 3.3 offered many firsts. It was the first version to separate race
%and profession. The Elf class was removed in preference to an elf race,
%and the races of dwarves, gnomes, and orcs made their first appearance in
%the game alongside the familiar human race.  Monk and Ranger roles joined
%Archeologists, Barbarians, Cavemen, Healers, Knights, Priests, Rogues, Samurai,
%Tourists, Valkyries and of course, Wizards.  It was also the first version
%to allow you to ride a steed, and was the first version to have a publicly
%available web-site listing all the bugs that had been discovered.  Despite
%that constantly growing bug list, 3.3 proved stable enough to last for
%more than a year and a half.
バージョン 3.3 は多くの新要素を提供した。
まず、種族と職業を分離した最初のバージョンである。
職業としてのエルフは削除され、種族の一つとなった。
また、ドワーフ、ノーム、オークが種族として追加された。
モンクとレンジャーの二つが職業として追加された。
さらに、動物に乗ることが出来る最初のバージョンでもあり、
発見された全てのバグをウェブサイトで公表した最初のバージョンでもある。
バグリストが順調に伸びていったにも関わらず、
3.3 は 1 年半続いたことで十分安定していたことを証明した。

%%.pg
\medskip
%The 3.4 {\it NetHack Development Team} initially consisted of
%{\it Michael Allison}, {\it Ken Arromdee},
%{\it David Cohrs}, {\it Jessie Collet}, {\it Kevin Hugo}, {\it Ken Lorber},
%{\it Dean Luick}, {\it Pat Rankin}, {\it Mike Stephenson},
%{\it Janet Walz}, and {\it Paul Winner}, with {\it  Warwick Allison} joining
%just before the release of {\it NetHack\/} 3.4.0 in March 2002.
3.4 開発チームは {\it Michael Allison}, {\it Ken Arromdee},
{\it David Cohrs}, {\it Jessie Collet}, {\it Kevin Hugo}, {\it Ken Lorber},
{\it Dean Luick}, {\it Pat Rankin}, {\it Mike Stephenson},
{\it Janet Walz}, and {\it Paul Winner} で開始され、
{\it NetHack\/} 3.4.0 リリース直前の 2002 年 3 月に
{\it Warwick Allison} が加わった。

%%.pg
\medskip
%As with version 3.3, various people contributed to the game as a whole as
%well as supporting ports on the different platforms that {\it NetHack\/}
%runs on:
バージョン 3.3 と同様に、様々な人々が {\it NetHack\/} の色々な
プラットフォームへの移植を行なってくれた。

%%.pg
\medskip
%\nd{\it Pat Rankin} maintained 3.4 for VMS.
\nd {\it Pat Rankin} は 3.4 の VMS への移植を担当した。

%%.pg
\medskip
%\nd {\it Michael Allison} maintained {\it NetHack\/} 3.4 for the MS-DOS
%platform.
%{\it Paul Winner} and {\it Yitzhak Sapir} provided encouragement.
\nd {\it Michael Allison} は MS-DOS 版の {\it NetHack\/} 3.4 を管理している。
{\it Paul Winner} と {\it Yitzhak Sapir} が手伝っている。

%%.pg
\medskip
%\nd {\it Dean Luick}, {\it Mark Modrall}, and {\it Kevin Hugo} maintained and
%enhanced the Macintosh port of 3.4.
\nd {\it Dean Luick}, {\it Mark Modrall}, {\it Kevin Hugo} は
3.4 の Macintosh への移植と拡張を担当した。

%%.pg
\medskip
%\nd {\it Michael Allison}, {\it David Cohrs}, {\it Alex Kompel},
%{\it Dion Nicolaas}, and
%{\it Yitzhak Sapir} maintained and enhanced 3.4 for the Microsoft Windows
%platform.
%{\it Alex Kompel} contributed a new graphical interface for the Windows port.
%{\it Alex Kompel} also contributed a Windows CE port for 3.4.1.
\nd {\it Michael Allison}, {\it David Cohrs}, {\it Alex Kompel}, {\it Dion Nicolaas},
{\it Yitzhak Sapir} は 3.4 の Microsoft Windows プラットフォームへの
移植と拡張を担当した。
{\it Alex Kompel} は Windows 版への新しいグラフィックインターフェースを
提供した。
{\it Alex Kompel} はまた、3.4.1 の Windows CE 版を提供した。

%%.pg
\medskip
%\nd {\it Ron Van Iwaarden} was the sole maintainer of {\it NetHack\/} for
%OS/2 the past
%several releases. Unfortunately Ron's last OS/2 machine stopped working in
%early 2006. A great many thanks to Ron for keeping {\it NetHack\/} alive on
%OS/2 all these years.
\nd {\it Ron Van Iwaarden} は過去いくつかのリリースの OS/2 版 {\it NetHack\/} の
唯一の管理者だった。
残念ながら Ron の最後の OS/2 マシンは 2006 年初頭に動かなくなった。
近年まで OS/2 で {\it NetHack\/} を生き続けさせたことにことについて Ron に
非常に感謝する。

%%.pg
\medskip
%\nd {\it Janne Salmij\"{a}rvi} and {\it Teemu Suikki} maintained
%and enhanced the Amiga port of 3.4 after {\it Janne Salmij\"{a}rvi} resurrected
%it for 3.3.1.
\nd Amiga への移植は {\it Janne Salmij\"{a}rvi} が 3.3.1 で復活させた後、
{\it Janne Salmij\"{a}rvi} と {\it Teemu Suikki} が 3.4 への保守と
拡張を行った。

%%.pg
\medskip
%\nd {\it Christian ``Marvin'' Bressler} maintained 3.4 for the Atari after he
%resurrected it for 3.3.1.
\nd 3.3.1 で復活した Atari 版は
{\it Christian ``Marvin'' Bressler} が 3.4 を保守した。

%%.pg
\medskip
%The release of {\it NetHack\/} 3.4.3 in December 2003 marked the beginning of
%a long release hiatus. 3.4.3 proved to be a remarkably stable version that
%provided continued enjoyment by the community for more than a decade. The
%{\it NetHack Development Team} slowly and quietly continued to work on the game behind the scenes
%during the tenure of 3.4.3. It was during that same period that several new
%variants emerged within the {\it NetHack\/} community. Notably sporkhack by
%Derek S. Ray, {\it unnethack\/} by Patric Mueller, {\it nitrohack\/} and its
%successors originally by Daniel Thaler and then by Alex Smith, and
%{\it Dynahack\/} by Tung Nguyen. Some of those variants continue to be
%developed, maintained, and enjoyed by the community to this day.
2003 年 12 月の {\it NetHack\/} 3.4.3 のリリースは、長いリリース中断の始まりとなった。
3.4.3 は、コミュニティによって 10 年以上楽しまれ続けられたことで
非常に安定したバージョンであることが示された。
3.4.3 の間、The {\it NetHack Development Team} は水面下でゆっくりと静かにゲームに対して作業を
続けていた。
同じ頃、{\it NetHack\/} コミュニティにいくつかの新しいバリアントが出現した。
特に、Derek S. Ray による sporkhack、Patric Mueller による {\it unnethack\/}、
元々は Daniel Thaler で、それからは Alex Smith による
{\it nitrohack\/} およびその後継、Tung Nguyen による {\it Dynahack\/} である。
これらのバリアントの一部は開発と保守が続けられ、今でもコミュニティで
楽しまれている。

%%.pg
\medskip
%In September 2014, an interim snapshot of the code under development was
%released publicly by other parties. Since that code was a work-in-progress
%and had not gone through the process of debugging it as a suitable release,
%it was decided that the version numbers present on that code snapshot would
%be retired and never used in an official {\it NetHack\/} release. An
%announcement was posted on the {\it NetHack Development Team}'s official {\it nethack.org\/} website
%to that effect, stating that there would never be a 3.4.4, 3.5, or 3.5.0
%official release version.
2014 年 9 月に、開発中のコードの中間的なスナップショットが他のグループによって
公にされた。
コードは作業中のもので、適切なリリースのためのデバッグプロセスを経ていない
ものであったので、コードスナップショットに使われていたバージョン番号は
欠番として、公式な {\it NetHack\/} リリースでは使わないことに決定した。
{\it NetHack Development Team} の公式 {\it nethack.org\/} web サイトにアナウンスが投稿され、
公式リリースバージョンとして 3.4.4, 3.5, 3.5.0 は使わないことを示した。

%%.pg
\medskip
%In January 2015, preparation began for the release of NetHack 3.6.
2015 年 1 月、NetHack 3.6 のリリースのための準備が開始された。

%%.pg
\medskip
%At the beginning of development for what would eventually get released
%as 3.6.0, the {\it NetHack Development Team} consisted of {\it Warwick Allison},
%{\it Michael Allison}, {\it Ken Arromdee},
%{\it David Cohrs}, {\it Jessie Collet},
%{\it Ken Lorber}, {\it Dean Luick}, {\it Pat Rankin},
%{\it Mike Stephenson}, {\it Janet Walz}, and {\it Paul Winner}.
%In early 2015, ahead of the release of 3.6.0, new members
%{\it Sean Hunt}, {\it Pasi Kallinen}, and {\it Derek S. Ray}
%joined the {\it NetHack\/} development team.
最終的に 3.6.0 としてリリースされるものの開発の初期では、
開発チームは {\it Warwick Allison},
{\it Michael Allison}, {\it Ken Arromdee},
{\it David Cohrs}, {\it Jessie Collet},
{\it Ken Lorber}, {\it Dean Luick}, {\it Pat Rankin},
{\it Mike Stephenson}, {\it Janet Walz}, {\it Paul Winner} で構成されていた。
2015 年初頭、3.6.0 リリースに先だって、新しいメンバーである
{\it Sean Hunt}, {\it Pasi Kallinen}, {\it Derek S. Ray} が
NetHack 開発チームに加わった。

%%.pg
\medskip
%Near the end of the development of 3.6.0, one of the significant inspirations
%for many of the humorous and fun features found in the game,
%author Terry Pratchett, passed away. {\it NetHack\/} 3.6.0 introduced
%a tribute to him.
3.6.0 の開発の終了近くに、one of the significant inspirations
for many of the humorous and fun features found in the game,
author Terry Pratchett, passed away.
{\it NetHack\/} 3.6.0 は彼へ捧げられた。

%%.pg
\medskip
%3.6.0 was released in December 2015, and merged work done by the development
%team since the release of 3.4.3 with some of the beloved community
%patches.  Many bugs were fixed and some code was restructured.
3.6.0 は 2015 年 12 月にリリースされ、
3.4.3 リリースから開発チームによってなされた作業と、
いくつかの愛好者コミュニティのパッチがマージされた。
多くのバグが修正され、一部のコードが再構成された。

%%.pg
\medskip
%The {\it NetHack Development Team}, as well as {\it Steve VanDevender} and
%{\it Kevin Smolkowski}, ensured that {\it NetHack\/} 3.6 continued to
%operate on various Unix flavors and maintained the X11 interface.
{\it NetHack Development Team} 及び {\it Steve VanDevender} と
{\it Kevin Smolkowski} は NetHack 3.6 が様々な Unix の亜種で
動作し続けるようにし、また X11 インターフェースを管理している。

%%.pg
\medskip
%{\it Ken Lorber}, {\it Haoyang Wang}, {\it Pat Rankin}, and {\it Dean Luick}
%maintained the port of {\it NetHack\/} 3.6 for Mac OSX.
{\it Ken Lorber}, {\it Haoyang Wang}, {\it Pat Rankin}, {\it Dean Luick} は
{\it NetHack\/} 3.6 の Mac OSX版を管理している。

%%.pg
\medskip
%{\it Michael Allison}, {\it David Cohrs}, {\it Bart House},
%{\it Pasi Kallinen}, {\it Alex Kompel}, {\it Dion Nicolaas},
%{\it Derek S. Ray} and  {\it Yitzhak Sapir}
%maintained the port of  {\it NetHack\/} 3.6 for Microsoft Windows.
{\it Michael Allison}, {\it David Cohrs}, {\it Bart House},
{\it Pasi Kallinen}, {\it Alex Kompel}, {\it Dion Nicolaas},
{\it Derek S. Ray} and  {\it Yitzhak Sapir} は {\it NetHack\/} 3.6 の
Microsoft Windows 版を管理している。

%%.pg
\medskip
%{\it Pat Rankin} attempted to keep the VMS port running for NetHack 3.6,
%hindered by limited access.  {\it Kevin Smolkowski} has updated and tested it
%for the most recent version of OpenVMS (V8.4 as of this writing) on Alpha
%and Integrity (aka Itanium aka IA64) but not VAX.
{\it Pat Rankin} は、アクセスが制限されながら VMS 版を NetHack 3.6 で
動作させ続けようとさせた。
{\it Kevin Smolkowski} はこれを、
Alpha と Integrity (またの名を Itanium またの名を IA64) での
最新版の OpenVMS (記述時点で V8.4) に更新、テストしたが、VAX 版ではない。

%%.pg
\medskip
%{\it Ray Chason}  resurrected the msdos port for 3.6 and contributed the
%necessary updates to the community at large.
{\it Ray Chason} は 3.6 の msdos 版を復活させ、必要な更新をコミュニティに
寄贈した。

%%.pg
\medskip
%In late April 2018, several hundred bug fixes for 3.6.0 and some new features
%were assembled and released as NetHack 3.6.1. The {\it NetHack Development Team} at the
%time of release of 3.6.1 consisted of
%{\it Warwick Allison}, {\it Michael Allison}, {\it Ken Arromdee},
%{\it David Cohrs}, {\it Jessie Collet},
%{\it Pasi Kallinen}, {\it Ken Lorber}, {\it Dean Luick},
%{\it Patric Mueller}, {\it Pat Rankin}, {\it Derek S. Ray},
%{\it Alex Smith}, {\it Mike Stephenson}, {\it Janet Walz}, and
%{\it Paul Winner}.
2018 年 4 月下旬、3.6.0 からの数百のバグ修正と、いくつかの新しい機能が
まとめられ、NetHack 3.6.1 としてリリースされた。
3.6.1 リリース時点での {\it NetHack Development Team} は、
{\it Warwick Allison}, {\it Michael Allison}, {\it Ken Arromdee},
{\it David Cohrs}, {\it Jessie Collet},
{\it Pasi Kallinen}, {\it Ken Lorber}, {\it Dean Luick},
{\it Patric Mueller}, {\it Pat Rankin}, {\it Derek S. Ray},
{\it Alex Smith}, {\it Mike Stephenson}, {\it Janet Walz}, 
{\it Paul Winner} からなる。

%%.pg
\medskip
%In early May 2019, another 320 bug fixes along with some enhancements and
%the adopted curses window port, were released as 3.6.2.
2019 年 5 月初旬、更なる 320 のバグ修正といくつかの拡張と、
寄贈された curses ウィンドウ対応を含んで、3.6.2 としてリリースされた。

%%.pg
\medskip
%{\it Bart House}, who had contributed to the game as a porting team participant
%for decades, joined the {\it NetHack Development Team} in late May 2019.
数十年にわたって移植チームとしてゲームに貢献してきた {\it Bart House} は、
2019 年 5 月に {\it NetHack Development Team} に加わった。

%%.pg
\medskip
%NetHack 3.6.3 was released on December 5, 2019 containing over 190 bug
%fixes to NetHack 3.6.2.
NetHack 3.6.3 は、NetHack 3.6.2 から 190 以上のバグ修正を含んで、
2019 年 12 月 5 日にリリースされた。

%%.pg
\medskip
%NetHack 3.6.4 was released on December 18, 2019 containing a security fix and
%a few bug fixes.
NetHack 3.6.4 は、セキュリティ修正といくつかのバグ修正を含んで、
2019 年 12 月 18 日にリリースされた。

%%.pg
\medskip
%NetHack 3.6.5 was released on January 27, 2020 containing some security fixes
%and a small number of bug fixes.
NetHack 3.6.5 は、いくつかのセキュリティ修正とバグ修正を含んで、
2020 年 1 月 27 日にリリースされた。

%%.pg
\medskip
%NetHack 3.6.6 was released on March 8, 2020 containing a security fix and
%some bug fixes.
NetHack 3.6.6 は、セキュリティ修正といくつかのバグ修正を含んで、
2020 年 3 月 8 日にリリースされた。

%%.pg
\medskip
%\nd The official {\it NetHack\/} web site is maintained by {\it Ken Lorber} at
\nd {\it Ken Lorber} によって管理されている {\it NetHack\/} 公式 web サイトは:
{\catcode`\#=11
\special{html:<a href="https://www.nethack.org/">}}
https:{\tt /}{\tt /}www.nethack.org{\tt /}.
{\catcode`\#=11
\special{html:</a>}}

%%.pg
%%.hn 2

%\subsection*{Special Thanks}
\subsection*{謝辞}
%\nd On behalf of the {\it NetHack\/} community, thank you very much once
%again to {\it M. Drew Streib} and {\it Pasi Kallinen} for providing a
%public NetHack server at nethack.alt.org. Thanks to {\it Keith Simpson}
%and {\it Andy Thomson} for hardfought.org. Thanks to all those
%unnamed dungeoneers who invest their time and effort into annual
%{\it NetHack\/} tournaments such as {\it Junethack},
%{\it The November NetHack Tournament} and in days past,
%{\it devnull.net\/} (gone for now, but not forgotten).
\nd {\it NetHack\/} コミュニティに成り代わって、
nethack.alt.org で公開 NetHack サーバーを提供してくれていることについて
再度 {\it M. Drew Streib} と {\it Pasi Kallinen} に感謝する。
hardfought.org に関して {\it Keith Simpson} と
{\it Andy Thomson} に感謝する。
{\it Junethack}, {\it The November NetHack Tournament} や以前の
(今はないが忘れられない) {\it devnull.net\/} のような、
年次 {\it NetHack\/} トーナメントに時間と労力を使ってくれた
無名の洞窟の主全てに感謝する。
\clearpage

%%.hn
%\section*{Dungeoneers}
\section*{洞窟の主}
%%.pg
%\nd From time to time, some depraved individual out there in netland sends a
%particularly intriguing modification to help out with the game.  The
%{\it NetHack Development Team} sometimes makes note of the names of the worst
%of these miscreants in this, the list of Dungeoneers:
\nd 時々ネットワークの世界のどうしようもない連中が、
ゲームの改良の手助けをしようとしてとりわけ興味をそそるような修正を送ってよこす。
{\it NetHack Development Team} は、ときにはこういった悪党のうちでも最も邪悪な連中の名前を
洞窟の主たちの一覧としてここに記すのだ。
%%.sd
\begin{center}
\begin{tabular}{llll}
%TABLE_START
Adam Aronow & J. Ali Harlow & Mikko Juola\\
Alex Kompel & Janet Walz & Nathan Eady\\
Alex Smith & Janne Salmij\"{a}rvi & Norm Meluch\\
Andreas Dorn & Jean-Christophe Collet & Olaf Seibert\\
Andy Church & Jeff Bailey & Pasi Kallinen\\
Andy Swanson & Jochen Erwied & Pat Rankin\\
Andy Thomson & John Kallen & Patric Mueller\\
Ari Huttunen & John Rupley & Paul Winner\\
Bart House & John S. Bien & Pierre Martineau\\
Benson I. Margulies & Johnny Lee & Ralf Brown\\
Bill Dyer & Jon W\{tte & Ray Chason\\
Boudewijn Waijers & Jonathan Handler & Richard Addison\\
Bruce Cox & Joshua Delahunty & Richard Beigel\\
Bruce Holloway & Karl Garrison & Richard P. Hughey\\
Bruce Mewborne & Keizo Yamamoto & Rob Menke\\
Carl Schelin & Keith Simpson & Robin Bandy\\
Chris Russo & Ken Arnold & Robin Johnson\\
David Cohrs & Ken Arromdee & Roderick Schertler\\
David Damerell & Ken Lorber & Roland McGrath\\
David Gentzel & Ken Washikita & Ron Van Iwaarden\\
David Hairston & Kevin Darcy & Ronnen Miller\\
Dean Luick & Kevin Hugo & Ross Brown\\
Del Lamb & Kevin Sitze & Sascha Wostmann\\
Derek S. Ray & Kevin Smolkowski & Scott Bigham\\
Deron Meranda & Kevin Sweet & Scott R. Turner\\
Dion Nicolaas & Lars Huttar & Sean Hunt\\
Dylan O'Donnell & Leon Arnott & Stephen Spackman\\
Eric Backus & M. Drew Streib & Stefan Thielscher\\
Eric Hendrickson & Malcolm Ryan & Stephen White\\
Eric R. Smith & Mark Gooderum & Steve Creps\\
Eric S. Raymond & Mark Modrall & Steve Linhart\\
Erik Andersen & Marvin Bressler & Steve VanDevender\\
Fredrik Ljungdahl & Matthew Day & Teemu Suikki\\
Frederick Roeber & Merlyn LeRoy & Tim Lennan\\
Gil Neiger & Michael Allison & Timo Hakulinen\\
Greg Laskin & Michael Feir & Tom Almy\\
Greg Olson & Michael Hamel & Tom West\\
Gregg Wonderly & Michael Sokolov & Warren Cheung\\
Hao-yang Wang & Mike Engber & Warwick Allison\\
Helge Hafting & Mike Gallop & Yitzhak Sapir\\
Irina Rempt-Drijfhout & Mike Passaretti\\
Izchak Miller & Mike Stephenson
%TABLE_END  Do not delete this line.
\end{tabular}
\end{center}
%%.ed
\clearpage
%%\vfill
%%\begin{flushleft}
%%\small
%%Microsoft and MS-DOS are registered trademarks of Microsoft Corporation.\\
%%%%Don't need next line if a UNIX macro automatically inserts footnotes.
%%UNIX is a registered trademark of AT\&T.\\
%%Lattice is a trademark of Lattice, Inc.\\
%%Atari and 1040ST are trademarks of Atari, Inc.\\
%%Amiga is a trademark of Commodore-Amiga, Inc.\\
%%%.sm
%%Brand and product names are trademarks or registered trademarks
%%of their respective holders.
%%\end{flushleft}

\end{document}
